%

\documentclass{beamer}
    
    %    \usepackage[english]{babel}
        %\usepackage[latin1]{inputenc}
        %\usepackage[T1]{fontenc}
    
    \mode<presentation>{
      \setbeamertemplate{background canvas}[vertical shading]
      \usetheme{Berkeley}
      \useoutertheme{himinfolines}
    }
      
    \usepackage{ucs}
    \usepackage[utf8]{inputenc}
    \usepackage[english,polutonikogreek,italian,UKenglish,british]{babel}
    \usepackage{graphicx}
    \usepackage{colortbl}
    \usepackage{multicol}
    \usepackage{ulem}
    \usepackage{verbatim}
    \usepackage{alltt}
    \usepackage{ccicons}
    \usepackage{MnSymbol,wasysym}
    \usepackage{tikzsymbols}
    \usepackage{textcomp}
    \usepackage{xmpincl}
    
    \usepackage{parskip}
    \setcounter{nframes}{115}
    \setcounter{nframe}{1}
    \setbeamercovered{dynamic}
    \newenvironment{grcenv}{\begin{otherlanguage}{greek}}{\end{otherlanguage}}
    \newcommand{\g}[1]{\textgreek{#1}}
    \definecolor{darkgreen}{rgb}{0,0.5,0}
    \definecolor{darkblue}{rgb}{0,0,0.5}
    \definecolor{grey}{rgb}{0.5,0.5,0.5}
    \setcounter{tocdepth}{5}
    
    \makeatletter
    
    \makeatother
    %\includexmp{LicencesAndLicensing}
    
    %frame00 metadata
        \title{Codifica TEI - Elementi editoriali e facsimile}
        \author[A.M. Del Grosso]{Angelo Mario Del Grosso}
        \institute{\texttt{angelo.delgrosso@ilc.cnr.it} \\\bigskip\textit{CNR-ILC-LicoLab} \\\bigskip\url{http://licolab.ilc.cnr.it/}}
        \date{Istituto di Linguistica Computazionale ``A. Zampolli'', \today}
        \AtBeginSection[]{
        \begin{frame}<beamer>
        \addtocounter{nframe}{1}
        \footnotesize
        \frametitle{Progress status}
        \tableofcontents[currentsection,hideothersubsections]
        \end{frame}
        }
    
    \begin{document}
    
    \begin{frame}
        \maketitle
    \end{frame}
    
    \begin{frame}
        \frametitle{Sommario della Lezione}
        \tableofcontents
    \end{frame}
    
    \section{Introduzione agli elementi editoriali e facsimile TEI}
    
    \begin{frame}
        \frametitle{Introduzione}
        \addtocounter{nframe}{1}
        
        %\begin{center}
        %    \includegraphics[width=.2\textwidth]{../imgs/tei-r.pdf}
        %\end{center}
    
        \begin{block}{Elementi editoriali}
            \emph{in parte già disponibili nei moduli TEI di base}
            
            \textit{Per la critica testuale indispensabili i moduli}
            \begin{itemize}
                \item \emph{msdescription} descrizione del manoscritto 
                \item \emph{trans} trascrizione di fonti primarie 
                \item \emph{textcrit} apparato critico
                \item \emph{gaiji} caratteri non standard
            \end{itemize}
        \end{block}
        
    \end{frame}
    
    \begin{frame}
        \frametitle{Introduzione}
        \addtocounter{nframe}{1}
        
        \begin{block}{Qual è l'obiettivo della TEI}
           \textbf{ A seconda del tipo di marcatura utili anche}: 
           \begin{itemize}
               \item \emph{analysis}
               \item \emph{certainty}
               \item \emph{figures}
               \item \emph{namesdates}
               \item \emph{verse}
           \end{itemize}
        \end{block}
        
    \end{frame}
    
    \begin{frame}
        \frametitle{Modularità della TEI}
        \addtocounter{nframe}{1}
        
       % \begin{center}
        % \includegraphics[width=.2\textwidth]{../imgs/tei-r.pdf}
        % \end{center}
    
        \begin{itemize}
            
            \item<1-> parleremo del sistema Modulare della TEI
                \begin{itemize}
                    \item<1-> Moduli
                    \item<1-> Classi
                    \item<1-> Macro
                    \item<1-> Datatype
                \end{itemize} 
            \item<2-> parleremo degli elementi basilari
                \begin{itemize}
                    \item<2-> Intestazione TEI (TEIHeader)
                    \item<2-> Elementi e attributi presenti in tutti i documenti TEI
                    \item<2-> Esempi di codifica
                \end{itemize} 
        \end{itemize}
        
    \end{frame}
    
    \section{Elementi editoriali TEI}
    \begin{frame}
    \frametitle{Introduzione}
    \addtocounter{nframe}{1}
    
    %\begin{center}
    %    \includegraphics[width=.2\textwidth]{../imgs/tei-r.pdf}
    %\end{center}

    \begin{block}{Cos'è la TEI}
        la TEI - \textit{acronimo di Text Encoding Initiative} - rappresenta un punto di riferimento per tutte le iniziative il cui scopo principale è quello di digitalizzare risorse testuali in ambito umanistico per fini di ricerca e di conservazione.
    \end{block}
    
\end{frame}

%%
<add> una o più parole aggiunte nel testo
questa parola è <add place="supralinear">stata</add> aggiunta in un secondo momento
<del> una o più parole cancellate nel testo
questa invece era <del rend="overstrike">era</del>
 di troppo e l’ho cancellata
<gap/> parte di testo omessa, mancante o illeggibile
questa <gap reason="illegible" extent="6" unit="chars"/> è illeggibile (forse “parola”?)

%%
<damage> testo danneggiato nel documento originale
   
per qualche goccia d’acqua questa parola si è
   <damage agent="water">scolorita</damage> molto

   <unclear> parte di testo interpretabile con difficoltà
<unclear reason="faded">questa</unclear> si legge
   ancora ma con difficoltà

   <supplied> testo inserito dal curatore perché illeggibile nell’originale o assente (fa parte del modulo transcr)
qui <supplied>mancava qualcosa</supplied> nel testo

è possibile combinare questi elementi (compreso <gap/>)

%%
Dal Vercelli Book: manoscritto in inglese antico, dialetto tardo sassone occidentale, X secolo ca.
   %\begin{center}
    %    \includegraphics[width=.9\textwidth]{/imgs/vercelli.png}
    %\end{center}

%%
- può succedere che una parola non sia semplicemente cancellata, ma che sia anche sostituita da un altro termine
- in questo caso è possibile usare l’elemento <subst> per collegare la sequenza cancellazione → nuovo testo

%%
questa parola è stata <subst>
<del rend="overstrike">scritta</del>
<add place="supralinear">aggiunta</add>
<subst> in un secondo momento

anche a livello della singola lettera
<subst><del rend="overtype">t<del><add>T</add></subst>i scrivo una
   mail domani mattina

   <subst> fa parte del modulo transcr, per ulteriori informazioni (@seq) v. http://www.tei- c.org/release/doc/tei-p5-doc/en/html/PH.html\#PHSU

%%
Attributi e valori consigliati

- <add> 
place inline, supralinear, margin, etc.
hand author, scribe1, scribe2

- <del> 
rend overstrike, subpunct, overtype, dotted
hand author, scribe1, scribe2

- <gap>
reason illegible, cancelled, irrelevant, missing, omissis, censored 
hand editor
agent water, smoke, hole, missing page
extent chars, words, cm

- <unclear> 
reason parterased, ink blot 
hand author, scribe1, scribe2
agent water, smoke, ink

%%
Correzioni e normalizzazioni

 <sic> parola o frase ritenuta errata, ma riportata “com’è”
questa parola è <sic>statta</sic> sbagliata

<corr> correzione di una parola o frase errata
questa parola è <corr>stata</corr> corretta in un
 secondo momento

<orig> parola o frase ritenuta “non standard” 
Allora, mi dici <orig>’ndo</orig> vai?

<reg> parola o frase normalizzata (regularised) Allora, mi dici <reg>dove</reg> vai?

%%

Abbreviazioni ed espansioni 1 <abbr> parola abbreviata, brevigrafo
 chiedi al <abbr>dott.</abbr> Rossi
 in nomine Patris <abbr>7</abbr> Filii
 <abbr>7</abbr> Spiritus Sancti

<expan> espansione di un’abbreviazione chiedi al <expan>dottor</expan> Rossi
 in nomine Patris <expan>et</expan> Filii
 <expan>et</expan> Spiritus Sancti

 %%
 
<abbr> usare l’attributo type per specificare il tipo chiedi al <abbr type="titolo">dott.</abbr> Rossi
 in nomine Patris <abbr type="brevigrafo">7</abbr> Filii
 <abbr type="brevigrafo">7</abbr> Spiritus

 %%
disponibili ulteriori elementi aggiungendo il modulo transcr:
<am> lettere o segni di abbreviazione
<ex> lettere aggiunte espandendo un’abbreviazione
 <abbr>sanct<am>u~</am></abbr>
 <expan>sanctu<ex>m</ex></expan>

 %%
 alcuni elementi di intervento editoriale sono perfettamente speculari:
<sic> - <corr> <orig> - <reg> <abbr> - <expan>

%%

la relazione fra le coppie è importante per essere sicuri che ciascun elemento sia collegato all’altro

Nella TEI P5 è stato introdotto un nuovo elemento <choice> che comprende ogni coppia

%%
- <sic> - <corr>
questa parola è <choice>

<sic>statta</sic>
<corr>stata</corr> </choice> scritta sbagliata

- <orig> - <reg>
Allora, mi dici <choice><orig>’ndo</orig> <reg>dove</reg></choice> vai?

- <abbr> - <expan>
chiedi al <choice><abbr>dott.</abbr> <expan>dott<ex>or</ex></expan></choice> Rossi

%%
Attributi @resp e @cert
- per <add> e <del> abbiamo l’attributo @hand per specificare, se necessario, l’autore della modifica all’originale
- per qualche intervento editoriale, invece, potrebbe essere importante specificare il responsabile (in particolare per l’elemento <supplied>!) e il grado di certezza
- per fare questo si usano gli attributi @resp e @cert

%%
questa parola è <choice>
<sic>statta</sic>
<corr resp="rrdt">stata</corr> </choice> scritta per errore
- il valore di @resp può rimandare a un <respStmt>: <corr resp="\#rrdt">stata</corr>

%%

Altri elementi utili 1

<gb/> gathering begins: indica l’inizio di un nuovo fascicolo nella trascrizione di un manoscritto

<space> usato per marcare la presenza di spazio significativo (ad esempio spazio lasciato dallo scriba per inserire una iniziale miniata)
normalmente usato come elemento vuoto modulo linking

<pc> punctuation character(s): contiene uno o più caratteri che costituiscono una forma di punteggiatura nel testo

modulo analysis

%% 
- <ab> anonymous block: contenitore di testo simile a un paragrafo ma senza il valore semantico di quest’ultimo
– modulo linking
- <seg> arbitrary segment: usato per marcare qualunque porzione di testo,
usare @type per specificare il contenuto semantico – modulo linking
- <w> word: marca una parola a livello grammaticale, @lemma per indicare il lemma e @lemmaref per stabilire un link con un dizionario online
– modulo analysis
- <c> character: marca un singolo carattere nel testo
– modulo analysis

%% 
Esempio

%%
Esempio

Dal Vercelli Book (inglese antico, dialetto tardo sassone occidentale, X secolo ca.
<del rend="erasure">ff</del>fore wæron
<del rend="dot"><g ref="\#aeligddot"/></del>

%% 
Esempio choice

%% esempi presi con screenshot

    
    \section{Elementi facsimile TEI}
    \begin{frame}
    \frametitle{Introduzione}
    \addtocounter{nframe}{1}
    
    %\begin{center}
    %    \includegraphics[width=.2\textwidth]{../imgs/tei-r.pdf}
    %\end{center}

    \begin{block}{Cos'è la TEI}
        la TEI - \textit{acronimo di Text Encoding Initiative} - rappresenta un punto di riferimento per tutte le iniziative il cui scopo principale è quello di digitalizzare risorse testuali in ambito umanistico per fini di ricerca e di conservazione.
    \end{block}
    
\end{frame}

\begin{frame}
  \frametitle{Introduzione}
  \addtocounter{nframe}{1}
  
  %\begin{center}
  %    \includegraphics[width=.2\textwidth]{../imgs/tei-r.pdf}
  %\end{center}

  \begin{block}{Elementi editoriali}
      \emph{in parte già disponibili nei moduli TEI di base}
      
      \textit{Per la critica testuale indispensabili i moduli}
      \begin{itemize}
          \item \emph{msdescription} descrizione del manoscritto 
          \item \emph{trans} trascrizione di fonti primarie 
          \item \emph{textcrit} apparato critico
          \item \emph{gaiji} caratteri non standard
      \end{itemize}
  \end{block}
  
\end{frame}

%%
\begin{frame}
  \frametitle{Facsimile ed edizione digitale}
  \addtocounter{nframe}{1}
  
  %\begin{center}
  %    \includegraphics[width=.2\textwidth]{../imgs/tei-r.pdf}
  %\end{center}

  \begin{block}{Tipi di edizione digitale}
      
      \begin{itemize}
          \item \textbf{edizione ipertestuale} 
          \item \textbf{facsimile digitale} 
          \item \textbf{image-based} 
      \end{itemize}
  \end{block}
  
\end{frame}

\begin{frame}
  \frametitle{Facsimile ed edizione digitale}
  \addtocounter{nframe}{1}
  
  %\begin{center}
  %    \includegraphics[width=.2\textwidth]{../imgs/tei-r.pdf}
  %\end{center}

  \begin{block}{Edizione ipertestuale}
    Prime a essere prodotte, ancora oggi spesso in formato HTML (derivate anche da elaborazioni di documenti TEI XML). 
    \\Formato ideale per edizioni critiche.
    \\Distribuzione sul Web (es.: Biblioteca Digitale Italiana). 
  \end{block}
  
\end{frame}

\begin{frame}
  \frametitle{Facsimile ed edizione digitale}
  \addtocounter{nframe}{1}
  
  %\begin{center}
  %    \includegraphics[width=.2\textwidth]{../imgs/tei-r.pdf}
  %\end{center}

  \begin{block}{Facsimile digitale}
    Riproduzione del manoscritto basata su scansione digitale.
    \\Distribuzione sul Web (es.: Progetto e-codices)    
  \end{block}
  
\end{frame}

\begin{frame}
  \frametitle{Facsimile ed edizione digitale}
  \addtocounter{nframe}{1}
  
  %\begin{center}
  %    \includegraphics[width=.2\textwidth]{../imgs/tei-r.pdf}
  %\end{center}

  \begin{block}{Image-based digital edition}
    \textit{Edizione basata su immagini.}
    \\Il testo dell’edizione (diplomatica, interpretativa, critica) con le immagini del
manoscritto
  \end{block}
  
\end{frame}


%%
\begin{frame}
  \frametitle{Facsimile ed edizione digitale}
  \addtocounter{nframe}{1}
  
  %\begin{center}
  %    \includegraphics[width=.2\textwidth]{../imgs/tei-r.pdf}
  %\end{center}

  \begin{block}{Edizione basata sulle immagini}
      
      \begin{itemize}
          \item Trascrizione collegata alle immagini del manoscritto
          \item Funzionalità principali collegate all'edizione digitale
          \begin{itemize}
            \item immagini in formati e/o risoluzioni diverse 
            \item lente d’ingrandimento
            \item evidenziazione dettagli 
            \item restauro digitale
            \item motore di ricerca testuale
            \item introduzione paleografica/filologica
            \item commento al testo
            \item bibliografia 
          \end{itemize}
      \end{itemize}
  \end{block}
  
\end{frame}

\begin{frame}
  \frametitle{Facsimile ed edizione digitale}
  \addtocounter{nframe}{1}
  
  %\begin{center}
  %    \includegraphics[width=.2\textwidth]{../imgs/tei-r.pdf}
  %\end{center}

  \begin{block}{Esempio edizione digitale image-based}
    \textit{Electronic Beowulf}
  \end{block}
  
\end{frame}


%% 
\begin{frame}
  \frametitle{Facsimile ed edizione digitale}
  \addtocounter{nframe}{1}
  
  %\begin{center}
  %    \includegraphics[width=.2\textwidth]{../imgs/tei-r.pdf}
  %\end{center}

  \begin{block}{Edizione digitale di un manoscritto}
      
      \begin{itemize}
          \item \textbf{immagini} del manoscritto
          \item \textbf{trascrizione} del/i testo/i
          \item creazione del \textbf{facsimile digitale}
          \item \textbf{collegamento} testo-immagine
          
      \end{itemize}
  \end{block}
  
\end{frame}

\begin{frame}
  \frametitle{Facsimile ed edizione digitale}
  \addtocounter{nframe}{1}
  
  %\begin{center}
  %    \includegraphics[width=.2\textwidth]{../imgs/tei-r.pdf}
  %\end{center}

  \begin{block}{Livelli di edizione}
      
      \begin{itemize}
          \item \textbf{edizione diplomatica}
          \item \textbf{edizione diplomatico-interpretativa}
          \item \textbf{edizione critica}
      \end{itemize}
  \end{block}

  \textit{ In caso di singolo manoscritto possiamo avere edizione diplomatica e/o interpretativa}
  
\end{frame}



%%
\begin{frame}
  \frametitle{Facsimile ed edizione digitale}
  \addtocounter{nframe}{1}
  
  %\begin{center}
  %    \includegraphics[width=.2\textwidth]{../imgs/tei-r.pdf}
  %\end{center}

  \begin{block}{livelli di edizione: diplomatica}
    Trascrizione del testo di un testimone rispettando la disposizione e la grafia originale, senza nessun tipo di correzione (errori manifesti) o altri interventi editoriali (espansione abbreviazioni).
  \end{block}
  
\end{frame}

\begin{frame}
  \frametitle{Facsimile ed edizione digitale}
  \addtocounter{nframe}{1}
  
  %\begin{center}
  %    \includegraphics[width=.2\textwidth]{../imgs/tei-r.pdf}
  %\end{center}

  \begin{block}{livelli di edizione: diplomatico-interpretativa}
    Sempre rispettando il testo originale, vengono corretti gli errori più evidenti, regolarizzate certe particolarità ortografiche (suddivisione delle parole), espanse le abbreviazioni, etc.
  \end{block}
  
\end{frame}

\begin{frame}
  \frametitle{Facsimile ed edizione digitale}
  \addtocounter{nframe}{1}
  
  %\begin{center}
  %    \includegraphics[width=.2\textwidth]{../imgs/tei-r.pdf}
  %\end{center}

  \begin{block}{Livelli di edizione: critica}
    Sulla base della collazione di tutte le trascrizioni dei testimoni viene stabilito lo stemma codicum e si tenta di ricostruire il testo originale confrontando le varianti dei testimoni più validi
  \end{block}
  
\end{frame}

%%
\begin{frame}
  \frametitle{Facsimile ed edizione digitale}
  \addtocounter{nframe}{1}
  
  %\begin{center}
  %    \includegraphics[width=.2\textwidth]{../imgs/tei-r.pdf}
  %\end{center}

  \begin{block}{Esempio livelli di edizione}
    \textit{Vercelli Book}
  \end{block}
  
\end{frame}

%%

\begin{frame}
  \frametitle{Facsimile ed edizione digitale}
  \addtocounter{nframe}{1}
  
  %\begin{center}
  %    \includegraphics[width=.2\textwidth]{../imgs/tei-r.pdf}
  %\end{center}

  \begin{block}{Rapporto testo-immagine}
      
      \begin{itemize}
          \item \textbf{collegamento mirato (hot-spot)}
          \item \textbf{collegamento generalizzato}
      \end{itemize}
  \end{block}

\end{frame}

\begin{frame}
  \frametitle{Facsimile ed edizione digitale}
  \addtocounter{nframe}{1}
  
  %\begin{center}
  %    \includegraphics[width=.2\textwidth]{../imgs/tei-r.pdf}
  %\end{center}

  \begin{block}{Collegamento mirato (hot-spot)}
    Una specifica area dell’immagine viene evidenziata in maniera tale che, interagendo con la stessa, vengono visualizzate delle informazioni quali note editoriali, versione migliorata di un dettaglio, commento al testo, etc.
  \end{block}
  
\end{frame}

\begin{frame}
  \frametitle{Facsimile ed edizione digitale}
  \addtocounter{nframe}{1}
  
  %\begin{center}
  %    \includegraphics[width=.2\textwidth]{../imgs/tei-r.pdf}
  %\end{center}

  \begin{block}{Collegamento generalizzato}
    Tutto il testo dell’edizione viene messo in relazione diretta con le immagini, o parti di immagine, corrispondenti, in modo da poter accedere facilmente alla porzione di immagine corrispondente partendo dal testo, e viceversa.
  \end{block}
  
\end{frame}



%%
\begin{frame}
  \frametitle{Facsimile ed edizione digitale}
  \addtocounter{nframe}{1}
  
  %\begin{center}
  %    \includegraphics[width=.2\textwidth]{../imgs/tei-r.pdf}
  %\end{center}

  \begin{block}{Obiettivo}
    Realizzare un collegamento fra testo e immagine in maniera tale che cliccando sul testo viene visualizzata la parte di immagine corrispondente e viceversa
  \end{block}
  
\end{frame}

\begin{frame}
  \frametitle{Facsimile ed edizione digitale}
  \addtocounter{nframe}{1}
  
  %\begin{center}
  %    \includegraphics[width=.2\textwidth]{../imgs/tei-r.pdf}
  %\end{center}

  \begin{block}{Edizione digitale Facsimile}
    Gli schemi di codifica TEI versione P5 (2007) introducono numerosi miglioramenti per quanto riguarda la gestione e trascrizione di manoscritti
  \end{block}

  \textit{Tra queste la nuova sezione Digital facsimiles nel capitolo 11 Representation of Primary Sources: \url{http://www.tei-c.org/release/doc/tei-p5-doc/en/html/PH.html}}

  
\end{frame}


%%
%modulo per la descrizione di manoscritti (10 Manuscript
%Description http://www.tei-c.org/release/doc/tei-p5-doc/en/html/MS.html)
%nuovo elemento <choice> da usare per le coppie di elementi
%di tipo “editoriale”

%%

\begin{frame}
  \frametitle{Facsimile ed edizione digitale}
  \addtocounter{nframe}{1}
  
  %\begin{center}
  %    \includegraphics[width=.2\textwidth]{../imgs/tei-r.pdf}
  %\end{center}

  \begin{block}{Edizione digitale Facsimile}
    Includendo il modulo transcr nello schema di codifica TEI si rende disponibile un attributo globale \texttt{@facs}
  \end{block}
  \begin{block}{Edizione digitale Facsimile}
    \textit{\texttt{@facs (facsimile)} points to all or part of an image which corresponds with the content of the element}
  \end{block}

\end{frame}

\begin{frame}
  \frametitle{Facsimile ed edizione digitale}
  \addtocounter{nframe}{1}
  
  %\begin{center}
  %    \includegraphics[width=.2\textwidth]{../imgs/tei-r.pdf}
  %\end{center}

  \begin{block}{Edizione digitale Facsimile}
    Questo attributo può essere usato in qualsiasi elemento per associare il contenuto dello stesso a un’immagine
  \end{block}
  \begin{block}{Edizione digitale Facsimile: esempi}
    \texttt{<p n="1" facs="para1.jpg" >}
    \\\texttt{<head facs="head.jpg" >}
    \\\texttt{<pb facs="page1.jpg" />}
  \end{block}

\end{frame}


%%

\begin{frame}
  \frametitle{Facsimile ed edizione digitale}
  \addtocounter{nframe}{1}
  
  %\begin{center}
  %    \includegraphics[width=.2\textwidth]{../imgs/tei-r.pdf}
  %\end{center}

\textbf{oltre a \texttt{@facs} è necessario usare gli altri elementi del modulo transcrption per collegare testo a immagine}
  \begin{block}{Edizione image-based}
      
      \begin{itemize}
          \item \texttt{<facsimile>}
          \item \texttt{<surface>}
          \item \texttt{<zone>}
      \end{itemize}
  \end{block}

\end{frame}


\begin{frame}
  \frametitle{Facsimile ed edizione digitale}
  \addtocounter{nframe}{1}
  
  %\begin{center}
  %    \includegraphics[width=.2\textwidth]{../imgs/tei-r.pdf}
  %\end{center}

\textbf{oltre a \texttt{@facs} è necessario usare i altri elementi dl modulo transcrption per collegare testo a immagine}
  \begin{block}{Elemento Facsimile}
      
      \begin{itemize}
          \item \texttt{<facsimile>} 
          \item[] contains a representation of some written source in the form of a set of images rather than as transcribed or encoded text.
      \end{itemize}
  \end{block}

\end{frame}

\begin{frame}
  \frametitle{Facsimile ed edizione digitale}
  \addtocounter{nframe}{1}
  
  %\begin{center}
  %    \includegraphics[width=.2\textwidth]{../imgs/tei-r.pdf}
  %\end{center}

\textbf{oltre a \texttt{@facs} è necessario usare i altri elementi dl modulo transcrption per collegare testo a immagine}
  \begin{block}{Elemento Facsimile}
      
      \begin{itemize}
          \item \texttt{<surface>}
          \item[] defines a written surface in terms of a rectangular coordinate space, optionally grouping one or more graphic representations of that space, and rectangular zones of interest within it.
      \end{itemize}
  \end{block}

  \textit{L'attributo \texttt{@start} points to an element which encodes the starting position of the text corresponding to the inscribed part of the surface.}


\end{frame}

\begin{frame}
  \frametitle{Facsimile ed edizione digitale}
  \addtocounter{nframe}{1}
  
  %\begin{center}
  %    \includegraphics[width=.2\textwidth]{../imgs/tei-r.pdf}
  %\end{center}

\textbf{oltre a \texttt{@facs} è necessario usare i altri elementi dl modulo transcrption per collegare testo a immagine}
  \begin{block}{Elemento Facsimile}
      
      \begin{itemize}
          \item \texttt{<surface>}
          \item[] defines a written surface in terms of a rectangular coordinate space, optionally grouping one or more graphic representations of that space, and rectangular zones of interest within it.
      \end{itemize}
  \end{block}

\end{frame}

\begin{frame}
  \frametitle{Facsimile ed edizione digitale}
  \addtocounter{nframe}{1}
  
  %\begin{center}
  %    \includegraphics[width=.2\textwidth]{../imgs/tei-r.pdf}
  %\end{center}

\textbf{oltre a \texttt{@facs} è necessario usare i altri elementi dl modulo transcrption per collegare testo a immagine}
  \begin{block}{Elemento Facsimile}
      
      \begin{itemize}
          \item \texttt{<zone>}
          \item[] defines a rectangular area contained within a surface element.          .
      \end{itemize}
  \end{block}

\end{frame}


%%
\begin{frame}
  \frametitle{Facsimile ed edizione digitale}
  \addtocounter{nframe}{1}
  
  %\begin{center}
  %    \includegraphics[width=.2\textwidth]{../imgs/tei-r.pdf}
  %\end{center}

  \begin{block}{Elemento Facsimile}
    L’elemento \texttt{<facsimile>} è di tipo strutturale e si pone allo stesso livello di \texttt{<text>} o addirittura in alternativa a quest’ultimo
  \end{block}

  \begin{block}{Elemento Facsimile}
    La TEI permette una \textbf{grande flessibilità}: facsimile digitale con le immagini del ms, facsimile digitale con trascrizione del testo e facsimile digitale con trascrizione e collegamento all'immagine.
  \end{block}

\end{frame}
 
\begin{frame}
  \frametitle{Facsimile ed edizione digitale}
  \addtocounter{nframe}{1}
  
  %\begin{center}
  %    \includegraphics[width=.2\textwidth]{../imgs/tei-r.pdf}
  %\end{center}

\textit{}{Quando il modulo transcr viene aggiunto allo schema di codifica è possibile scegliere fra:}

  \begin{block}{Elemento Facsimile}
      
      \begin{itemize}
          \item \texttt{un <teiHeader> e un <facsimile>}
          \item \texttt{un <teiHeader> e un <text>}
          \item \texttt{un <teiHeader>, un <facsimile> e un <text>}         .
      \end{itemize}
  \end{block}

\end{frame}


%%
\begin{frame}
  \frametitle{Facsimile ed edizione digitale}
  \addtocounter{nframe}{1}
  
  %\begin{center}
  %    \includegraphics[width=.2\textwidth]{../imgs/tei-r.pdf}
  %\end{center}

  \begin{block}{Elemento Facsimile}
    \texttt{<TEI>}
    \\\texttt{<teiHeader>}
    \\\texttt{<!-- ... -->}
    \\\texttt{</teiHeader>}
    \\\texttt{<facsimile>}
     \\\texttt{ <graphic url="page1.png"/>}
     \\\texttt{ <graphic url="page2.png"/>}
     \\\texttt{ <graphic url="page3.png"/>}
     \\\texttt{ <graphic url="page4.png"/>}
    \\\texttt{</facsimile>}
   \\\texttt{</TEI>}
  \end{block}

\end{frame}


%%
\begin{frame}
  \frametitle{Facsimile ed edizione digitale}
  \addtocounter{nframe}{1}
  
  %\begin{center}
  %    \includegraphics[width=.2\textwidth]{../imgs/tei-r.pdf}
  %\end{center}

  \begin{block}{Elemento Facsimile}
    \texttt{<TEI>}
    \\\texttt{<teiHeader>}
    \\\texttt{<!-- ... -->}
    \\\texttt{</teiHeader>}
    \\\texttt{<text>}
     \\\texttt{ <pb facs="page1.png"/>}
     \\\texttt{ <!-- inserire qui il testo di pagina 1 -->}
     \\\texttt{ <pb facs="page2.png"/>}
     \\\texttt{ <!-- inserire qui il testo di pagina 2 -->}
    \\\texttt{</text>}
   \\\texttt{</TEI>}
  \end{block}

\end{frame}

%%
\begin{frame}
  \frametitle{Facsimile ed edizione digitale}
  \addtocounter{nframe}{1}
  
  %\begin{center}
  %    \includegraphics[width=.2\textwidth]{../imgs/tei-r.pdf}
  %\end{center}

  \begin{block}{Elemento Facsimile}
    Grazie a un foglio di stile XSLT è possibile generare una pagina HMTL divisa in due riquadri (immagine e testo)
  \end{block}
  \begin{block}{Elemento Facsimile}
    Nessun collegamento testo-immagine a livello diverso dalla pagina e non è possibile nemmeno individuare aree particolari delle immagini.  
    \\In fine, i puntatori alle immagini sono sparsi per tutto il documento.
  \end{block}

\end{frame}

%%
\begin{frame}
  \frametitle{Facsimile ed edizione digitale}
  \addtocounter{nframe}{1}
  
  %\begin{center}
  %    \includegraphics[width=.2\textwidth]{../imgs/tei-r.pdf}
  %\end{center}

  \begin{block}{Elemento Facsimile}
    La soluzione più efficace è la parallel transcription basata su \texttt{<facsimile> e <text>} e all'uso di \texttt{<surface> e <zone>} all’interno di \texttt{<facsimile>.}
  \end{block}
  \begin{block}{Elemento Facsimile}
    Possibilità di definire le aree delle immagini e collegare il testo della trascrizione a tali aree e/o immagini secondarie.
  \end{block}

\end{frame}



%%
\begin{frame}
  \frametitle{Facsimile ed edizione digitale}
  \addtocounter{nframe}{1}
  
  %\begin{center}
  %    \includegraphics[width=.2\textwidth]{../imgs/tei-r.pdf}
  %\end{center}

  \begin{block}{Elemento Facsimile}
    Le aree delle immagini sono individuate per mezzo di un sistema di coordinate cartesiane registrate come valori di attributi di \texttt{<surface> e <zone>}.
  \end{block}
  \begin{block}{Coordinate area immagini}
    \begin{itemize}
      \item \textbf{ulx, uly} coordinate x e y dell’angolo superiore sinistro 
      \item \textbf{lrx, lry} coordinate x e y dell’angolo inferiore destro
    \end{itemize}

  \end{block}

\end{frame}



%%
\begin{frame}
  \frametitle{Facsimile ed edizione digitale}
  \addtocounter{nframe}{1}
  
  %\begin{center}
  %    \includegraphics[width=.2\textwidth]{../imgs/tei-r.pdf}
  %\end{center}

  \begin{block}{Elemento Surface}
    \texttt{<surface>} individua la superficie scritta di un’immagine
  \end{block}
  \begin{block}{Elemento Surface: esempio}
    
    \texttt{<surface ulx="0" uly="0" lrx="400" lry="280" >} 
    \\\texttt{ <graphic url="page1.png"/>}
    \\\texttt{</surface>}

  \end{block}

\end{frame}

\begin{frame}
  \frametitle{Facsimile ed edizione digitale}
  \addtocounter{nframe}{1}
  
  %\begin{center}
  %    \includegraphics[width=.2\textwidth]{../imgs/tei-r.pdf}
  %\end{center}

  \begin{block}{Elemento Surface}
    può contenere più di un elemento \texttt{<graphic>}
  \end{block}
  \begin{block}{Elemento Surface: esempio}
    
    \texttt{<surface>}
    \texttt{ <graphic url="page1-highRes.png"/> }
    \texttt{ <graphic url="page1-lowRes.png"/>}
    \texttt{</surface>}

  \end{block}
  \textit{\texttt{<surface>} stesso si trova all’interno di \texttt{<facsimile>}}
\end{frame}



%%

\begin{frame}
  \frametitle{Facsimile ed edizione digitale}
  \addtocounter{nframe}{1}
  
  %\begin{center}
  %    \includegraphics[width=.2\textwidth]{../imgs/tei-r.pdf}
  %\end{center}
\textit{l'elemento facsimile invece di \texttt{<graphic>} può contenere una o più \texttt{<zone>}}

  \begin{block}{Elemento Zone}
    
    \texttt{<zone>} definisce una specifica area dell’immagine usando lo stesso sistema di coordinate di \texttt{<surface>}

  \end{block}
  \begin{block}{Elemento Zone: esempio}
    \texttt{<surface ulx="0" uly="0" lrx="500" lry="321" >}
    \\\texttt{ <zone ulx="50" uly="20" lrx="400" lry="280" >}
    \\\texttt{ <graphic url="scrittura.png"/>}
    \\\texttt{ </zone>}
    \\\texttt{ <note>first page</note>}
    \\\texttt{</surface>}
  \end{block}
  \texttt{<surface>} stesso si trova all’interno di \texttt{<facsimile>}
\end{frame}

%o una porzione più piccola (utile per creare un hot-spot):

%<zone ulx="120" uly="48" lrx="143" lry="56">
% <graphic url="gloss.png"/>
% <note>Scribe gloss</note>
%</zone>

%%
\begin{frame}
  \frametitle{Facsimile ed edizione digitale}
  \addtocounter{nframe}{1}
  
  \begin{center}
    \includegraphics[width=.9\textwidth]{imgs/AreeZoneManoscritto.png}
  \end{center}
\end{frame}

%%
\begin{frame}
  \frametitle{Facsimile ed edizione digitale}
  \addtocounter{nframe}{1}
  
  \begin{block}{Esempio facsimile zone}
    \texttt{<facsimile xml:id="imtAnnotatedImage" > <surface>
    <graphic height="1797px" url="LindisfarneFol27rIncipitMatt.jpg" width="1266px"/>
    <zone lrx="1268" lry="1797" rend="visible" rendition="surface" ulx="0" uly="4" xml:id="imtArea-0"/>
    <zone lrx="1267" lry="450" rend="visible" rendition="zone" ulx="1202" uly="356" xml:id="imtArea-1"/>
    <zone lrx="1050" lry="792" rend="visible" rendition="zone" ulx="81" uly="30" xml:id="imtArea-3"/>
    <zone lrx="1190" lry="154" rend="visible" rendition="zone" ulx="503" uly="48" xml:id="imtArea-4"/>
    <!-- altre zone -->
    </surface></facsimile>}
  \end{block}

\end{frame}


%%
\begin{frame}
  \frametitle{Facsimile ed edizione digitale}
  \addtocounter{nframe}{1}
  
  %\begin{center}
  %    \includegraphics[width=.2\textwidth]{../imgs/tei-r.pdf}
  %\end{center}

  \begin{block}{Collegare il testo alle immagini}
    Per collegare il testo della trascrizione alle aree corrispondenti dell’immagine bisogna \textbf{assegnare un identificatore univoco} a \textit{ciascun elemento del facsimile} usando \textbf{l’attributo xml:id}.  
\end{block}

  \begin{block}{Collegare il testo alle immagini}
    
    Usare l’attributo \textbf{facs} negli elementi testuali per \textbf{specificare l’id} degli elementi \texttt{<surface> e <zone>} corrispondenti

  \end{block}
\end{frame}

\begin{frame}
  \frametitle{Facsimile ed edizione digitale}
  \addtocounter{nframe}{1}
  
  %\begin{center}
  %    \includegraphics[width=.2\textwidth]{../imgs/tei-r.pdf}
  %\end{center}

  \begin{block}{Collegare le immagini al testo}
    Per collegare le aree delle immagini ai corrispondenti elementi di testo bisogna
    assegnare un \textbf{identificatore univoco} a \textit{ciascun elemento della trascrizione} usando l’\textbf{attributo xml:id}
  \end{block}

  \begin{block}{Collegare le immagini al testo}
    Usare l’attributo \textbf{start} negli elementi \texttt{<surface> e <zone>} per \textbf{specificare l’id} degli \textit{elementi testuali corrispondenti}.
  \end{block}
\end{frame}



%%

\begin{frame}
  \frametitle{Facsimile ed edizione digitale}
  \addtocounter{nframe}{1}
  
  \begin{block}{Esempio collegamento completo}
    \texttt{<text>
    <body>
     <div>
   <pb facs="\#page1" n="1" xml:id="page-1"/>
   <p>Lorem ipsum ... <gloss facs="\#det1" >semper</gloss></p> </div>
    </body>
   </text>
   <facsimile>
   <surface xml:id="page1” start="\#page-1" ulx="0" uly="0" lrx="500" lry="321" >
   <graphic url="page1.png”>
   <zone xml:id="line1" ulx="50" uly="80" lrx="200" lry="321" >
   <graphic url="line1.png"/>
   <note>First page.</note> </zone>
   <zone xml:id="det1" ulx="120" uly="48" lrx="143" lry="56" > <graphic url="gloss.png"/>
   <note>Scribe gloss.</note>
     </zone>
    </surface>
   </facsimile>}
  \end{block}

\end{frame}


%%
\begin{frame}
  \frametitle{Facsimile ed edizione digitale}
  \addtocounter{nframe}{1}
  
  %\begin{center}
  %    \includegraphics[width=.2\textwidth]{../imgs/tei-r.pdf}
  %\end{center}

  \begin{block}{Eembedded transcription}
    Un metodo a metà fra facsimile digitale e edizione basata su immagini è quello della embedded transcription.
    \\\texttt{http://www.tei-c.org/release/doc/tei-p5-doc/en/html/PH.html\#PHZLAB}
  \end{block}

  \begin{block}{Differenza con il metodo parallel transcription}
    Il testo è considerato di accompagnamento, il focus infatti e sulle immagini (ad esempio disposizione fisica delle parti).
  \end{block}
\end{frame}




%%

\begin{frame}
  \frametitle{Facsimile ed edizione digitale}
  \addtocounter{nframe}{1}
  
  %\begin{center}
  %    \includegraphics[width=.2\textwidth]{../imgs/tei-r.pdf}
  %\end{center}

  \begin{block}{Embedded transcription}
    Un metodo a metà fra facsimile digitale e edizione basata su immagini è quello della embedded transcription.
    \\\texttt{http://www.tei-c.org/release/doc/tei-p5-doc/en/html/PH.html\#PHZLAB}
  \end{block}

  \begin{block}{Differenza con il metodo parallel transcription}
    Il testo è considerato di accompagnamento, il focus infatti e sulle immagini (ad esempio disposizione fisica delle parti).
  \end{block}
\end{frame}

%%

\begin{frame}
  \frametitle{Facsimile ed edizione digitale}
  \addtocounter{nframe}{1}
  
  %\begin{center}
  %    \includegraphics[width=.2\textwidth]{../imgs/tei-r.pdf}
  %\end{center}

  \begin{block}{Embedded transcription}
    Per implementare il metodo si usa l’elemento \texttt{<sourceDoc>} sullo stesso livello gerarchico e in alternativa a \texttt{<facsimile> e <text>}.
  \end{block}

  \begin{block}{Embedded transcription}
    All'interno di \texttt{<sourceDoc>} gli elementi \texttt{<surface> e <zone>} vengono utilizzati in maniera simile a quanto visto per l'elemento \texttt{<facsimile>}.
  \end{block}
\end{frame}

\begin{frame}
  \frametitle{Facsimile ed edizione digitale}
  \addtocounter{nframe}{1}
  
  %\begin{center}
  %    \includegraphics[width=.2\textwidth]{../imgs/tei-r.pdf}
  %\end{center}

  \begin{block}{Embedded transcription}
    L'elemento \texttt{<zone>} contiene una serie di elementi \texttt{<line>} corrispondenti alle righe di testo
  \end{block}

  \begin{block}{Embedded transcription: esempio}
    \texttt{<zone ulx="20" uly="40" lrx="120" lry="180" >}
   \\\texttt{ <line>prima riga di trascrizione</line>}
   \\\texttt{ <line>seconda riga di trascrizione</line>}
   \\\texttt{</zone>}
  \end{block}
\textit{Il content model di \texttt{<line>} è limitato e con ci sono conflitti di gerarchie
}\end{frame}



%%

\begin{frame}
  \frametitle{Facsimile ed edizione digitale}
  \addtocounter{nframe}{1}
  
  %\begin{center}
  %    \includegraphics[width=.2\textwidth]{../imgs/tei-r.pdf}
  %\end{center}

  \begin{block}{Eembedded transcription}
    \begin{center}
        \includegraphics[width=.9\textwidth]{imgs/EsempioEmbeddedTranscription.png}
    \end{center}
  \end{block}

  \begin{block}{Differenza con il metodo parallel transcription}
    \textbf{Esempio Taccuini di Proust}
  \end{block}
\end{frame}



%%

\begin{frame}
  \frametitle{Facsimile ed edizione digitale}
  \addtocounter{nframe}{1}
  
  %\begin{center}
  %    \includegraphics[width=.2\textwidth]{../imgs/tei-r.pdf}
  %\end{center}

  
    \textit{Esistono numerosi programmi per calcolare le coordinate di aree su immagini facsimile.}
  
  \begin{block}{Come inserire le coordinate}
    Software di disegno in formato bitmap strumenti per programmatori di siti web.
    \\Software progettati per creatori di edizioni digitali (\textit{EPPT}).
  \end{block}

\end{frame}

%%

\begin{frame}
  \frametitle{Facsimile ed edizione digitale}
  \addtocounter{nframe}{1}
  
  %\begin{center}
  %    \includegraphics[width=.2\textwidth]{../imgs/tei-r.pdf}
  %\end{center}

  \begin{block}{Strumenti per per immagini facsimile}
   Un utile strumento per ottenere le coordinate di regioni di interesse in formato TEI è \textbf{TEIzoner}.
  \end{block}
\end{frame}


%%

\begin{frame}
  \frametitle{Facsimile ed edizione digitale}
  \addtocounter{nframe}{1}
  
  %\begin{center}
  %    \includegraphics[width=.2\textwidth]{../imgs/tei-r.pdf}
  %\end{center}

  \begin{block}{Edizioni image-based: Esercizio}
   \textbf{ Codificare lettera Bellini}
  \end{block}

\end{frame}



    
    %\section{Elementi di base TEI}
    %% capitoli 3 e 4 delle linee guida e estratti dal libro what is TEI (The structural organization, Varieties of textual structure, parte della TEI cornucopia part 1)
% fare esempio facsimile e codifica dei fenomeni (vedere anche slide del turco e fiormonte-ciotti)

    
    %\section{Personalizzare TEI}
    %\begin{frame}
	\frametitle{Intro Text Encoding Initiative}
	\framesubtitle{Schemi di codifica TEI – Personalizzare TEI}
	\addtocounter{nframe}{1}

    \begin{block}{Personalizzare TEI}
        Nessun progetto di codifica richiede di utilizzare tutte le specifiche definite dalle linee guida della TEI.
    \end{block}
    \begin{block}{Personalizzare TEI}
        
        La TEI fornisce un insieme specifico di elementi che può essere usato per creare uno schema TEI puntuale e ritagliato sulle specifiche del progetto di codifica in corso.

    \end{block}
    
\end{frame}

% \begin{frame}
% 	\frametitle{Intro Text Encoding Initiative}
% 	\framesubtitle{Schemi di codifica TEI – Personalizzare TEI}
% 	\addtocounter{nframe}{1}

%     \begin{block}{Personalizzare TEI}
%         % The TEI provides a special set of elements which can be used to create such a schema specification.
%         La TEI fornisce un insieme specifico di elementi che può essere usato per creare uno schema TEI puntuale e ritagliato sulle specifiche del progetto di codifica in corso.

%     \end{block}
    
% \end{frame}



\begin{frame}
	\frametitle{Intro Text Encoding Initiative}
	\framesubtitle{Schemi di codifica TEI – Personalizzare TEI}
    \addtocounter{nframe}{1}
    
    \begin{block}{Personalizzare TEI - schemaSpec}
        \texttt{<schemaSpec ident="tei-custom" start="TEI teiCorpus" >}
        \\\texttt{<moduleRef key="analysis" include="interp interpGrp pc s w" />}
        \\\texttt{<moduleRef key="linking" include="anchor seg" />}
        %\texttt{<moduleRef key="tagdocs" include="att code eg gi ident val "/> }
        \\\texttt{<moduleRef key="tei "/> }
        %\texttt{<moduleRef key="textstructure" include="TEI argument back body byline closer dateline div docAuthor docDate docEdition docImprint docTitle epigraph front group imprimatur opener postscript salute signed text titlePage titlePart trailer "/> }
        \\\texttt{</schemaSpec>}
    \end{block}
    \textit{Si selezionano o si escudono gli elementi attraverso gli attributi @\textbf{include} e @\textbf{exclude}}
\end{frame}

\begin{frame}
	\frametitle{Intro Text Encoding Initiative}
	\framesubtitle{Schemi di codifica TEI – Personalizzare TEI}
	\addtocounter{nframe}{1}
    \textit{Modifica agli elementi e attributi dei Moduli}
    \begin{block}{Personalizzare TEI - classSpec}
        \texttt{<classSpec type="atts" ident="att.datable.w3c" module="tei" mode="change" >}
        \\\texttt{<attList> }
        \\\texttt{<attDef ident="notAfter" mode="delete" />}
        \\\texttt{<attDef ident="from" mode="delete" />}
        \\\texttt{<attDef ident="to" mode="delete" /> }
        \\\texttt{</attList>}
        \\\texttt{</classSpec>}
    \end{block}
    
\end{frame}



% ODD document, Selezione dei Moduli per lo schema, nuovi elementi, profilo personalizzato TEI, capitolo 22 delle linee guida (Documentation Elements), capitolo 23 delle linee guida (Using the TEI)


% serious use of the TEI requires careful consideration of exactly which of its elements is appropriate to the of things which the project needs to specify more exactly than the TEI does.


% a document using these elements is provides information for a computer to process along with documentation of that information for a human being to read in a single integrated XML document.

% tei_ all tei_lite epidoc

%% esempio da Exemplars tei_lite.odd
% A quick glance at the XML source code for the TEI Lite ODD shows that it appears to be a typical TEI document

% <schemaSpec ident="tei_lite" start="TEI teiCorpus">
%    <moduleRef key="analysis" include="interp interpGrp pc s w" />
%    <moduleRef key="linking" include="anchor seg" />
%    <moduleRef key=" tagdocs " include="att code eg gi ident val "/> 
%    <moduleRef key="tei "/> 
%    <moduleRef key="textstructure " include="TEI argument back body byline closer dateline div docAuthor docDate docEdition docImprint docTitle epigraph front group imprimatur opener postscript salute signed text titlePage titlePart trailer "/> 
% </schemaSpec>


% TEI currently defines 22 modules
% foto con la tabella

% definizione degli elementi, degli attributi, dei datatype e dei valori di default 




%% esempio epidoc

% <elementSpec   ident="div"   mode="change"   module="textstructure">
%    <attList>
%        <attDef ident="type"     mode="replace"     usage="req">
%            <valList type="closed">
%                <valItem ident="apparatus">
%                    <desc>to contain apparatus criticus or textual notes</desc>
%                    </valItem>
%                    <valItem ident="bibliography">
%                        <desc>to contain bibliographical information, previous publications,            etc.
%                        </desc>
%                        </valItem>
%                        <valItem ident="commentary">
%                            <desc>to contain all editorial commentary, historical/prosopographical            discussion, etc.</desc>
%                            </valItem>
%                            <valItem ident="edition">
%                                <desc>to contain the text of the edition itself; may include multiple            text-parts
%                                </desc>
%                                </valItem>
%                                <valItem ident="textpart">
%                                    <desc>used to divide a div[type=edition] into multiple parts (fragments,            columns,
%                                        faces, etc.)</desc>
%                                    </valItem>
%                                    <valItem ident="translation">
%                                        <desc>to contain a translation of the text into one or more modern            languages
%                                        </desc>
%                                        </valItem>
%                                       </valList>
%        </attDef>
%    </attList>
% </elementSpec>

% esempio aggiunta nuovo elemento: <SpaciesName />

%  Choices can be made explicit in a customized schema, and hence tell us which of the many very different approaches to tagging an individual’s name has been adopted in a given set of documents.


    
    \section{Conclusioni}
    %%
\begin{frame}
    \frametitle{Edizioni Digitali}
    \addtocounter{nframe}{1}
    
    %\begin{center}
    %    \includegraphics[width=.2\textwidth]{../imgs/tei-r.pdf}
    %\end{center}

    \begin{block}{Moduli TEI P5}
        Le nuove caratteristiche degli schemi TEI P5 offrono un’ottima base per edizioni digitali complesse con collegamento testo-immagine
    \end{block}
    
\end{frame}



%%
\begin{frame}
    \frametitle{Conclusioni: Editioni Digitali}
    \addtocounter{nframe}{1}
    
    %\begin{center}
    %    \includegraphics[width=.2\textwidth]{../imgs/tei-r.pdf}
    %\end{center}

    \begin{block}{Edizioni digitali scientifiche}
        \begin{itemize}
            \item elementi contenuti nei \textit{moduli di base}
            \item elementi del modulo di \textit{descrizione dei manoscritti}
            \item elementi del modulo di \textit{trascrizione delle fonti primarie}
            \item elementi del modulo di \textit{apparato critico}
            \item elementi del modulo di \textit{gestione di caratteri non standard}
        \end{itemize}
    \end{block}
    
\end{frame}

%%

\begin{frame}
    \frametitle{Conclusioni: Edizioni Digitali}
    \addtocounter{nframe}{1}
    
    %\begin{center}
    %    \includegraphics[width=.2\textwidth]{../imgs/tei-r.pdf}
    %\end{center}

    \begin{block}{elementi per interventi editoriali:}
        \texttt{<abbr> <expan>, <orig> <reg>, <sic> <corr>, <subst>
        <gap/>, <supplied>, <unclear>, <damage>}
    \end{block}

    \begin{block}{strutturali specifici:}
        \texttt{<gb/>, <line>}
    \end{block}
    
\end{frame}


%%
\begin{frame}
    \frametitle{Conclusioni: Edizioni Digitali}
    \addtocounter{nframe}{1}
    
    %\begin{center}
    %    \includegraphics[width=.2\textwidth]{../imgs/tei-r.pdf}
    %\end{center}

    \begin{block}{Elementi di intervento editoriali}
        \begin{itemize}
            \item \texttt{<damage>} marca la parte di testo danneggiata
            \item[] non proprio “intervento editoriale” ma spesso usato contestualmente con \texttt{<gap/>,  <unclear> e <supplied>}
            \item \texttt{<supplied>} testo inserito dal curatore perché l’originale è mancante o illeggibile
        \end{itemize} 
    \end{block}
\end{frame}

\begin{frame}
    \frametitle{Conclusioni: Edizioni Digitali}
    \addtocounter{nframe}{1}
    
    %\begin{center}
    %    \includegraphics[width=.2\textwidth]{../imgs/tei-r.pdf}
    %\end{center}

    \begin{block}{Elementi di intervento editoriali: esempio}
        \texttt{<l n="1" >Nel mezzo del cammin di nostra vita</l>}
        \\\texttt{<l n="2" ><damage agent="fire" extent="1line" ><unclear>Mi ritrovai</unclear> <supplied reason="illegible" resp="amdg" >per una selva oscura,</supplied></damage></l>}
        \\\texttt{<l n="3" >Ché la diritta via era smarrita</l>}
    \end{block}
\end{frame}

%%
\begin{frame}
    \frametitle{Conclusioni: Edizioni Digitali}
    \addtocounter{nframe}{1}
    
    %\begin{center}
    %    \includegraphics[width=.2\textwidth]{../imgs/tei-r.pdf}
    %\end{center}

    \begin{block}{Elementi di intervento editoriale}
        \begin{itemize}
            \item \texttt{<subst>} raggruppa una cancellazione e un’aggiunta scribale per rendere evidente che si tratta di una sostituzione
            \item[] Stessa semantica funzionale di \texttt{<choice>}
        \end{itemize}
    \end{block}

\end{frame}

%%
\begin{frame}
    \frametitle{Conclusioni: Edizioni Digitali}
    \addtocounter{nframe}{1}
    
    %\begin{center}
    %    \includegraphics[width=.2\textwidth]{../imgs/tei-r.pdf}
    %\end{center}

    \begin{block}{Elementi di intervento editoriale}
        \texttt{<l n="1" >Nel mezzo del cammin di nostra vita</l> }
        \\\texttt{<l n="2" >Mi ritrovai }
        \\\texttt{<subst>}
        \\\texttt{<del>pir</del>}
        \\\texttt{<add>per</add>}
        \\\texttt{</subst>}
        \\\texttt{una selva oscura,</l>}
        \\\texttt{<l n="3" >Ché la diritta via era smarrita</l>}
    \end{block}
    
\end{frame}


%%

\begin{frame}
    \frametitle{Conclusioni: Edizioni Digitali}
    \addtocounter{nframe}{1}
    
    %\begin{center}
    %    \includegraphics[width=.2\textwidth]{../imgs/tei-r.pdf}
    %\end{center}

    \begin{block}{Edizione image-based: Elementi strutturali}
        \begin{itemize}
            \item \texttt{<gb/>} \textbf{gathering begins} 
            \item[] marca il punto in cui si presenta un nuovo fascicolo all’interno di un manoscritto
            \item \texttt{@type:} classificazione in base al tipo
            \item \texttt{@n:} numero progressivo
        \end{itemize}
    \end{block}
\end{frame}

\begin{frame}
    \frametitle{Conclusioni: Edizioni Digitali}
    \addtocounter{nframe}{1}
    
    %\begin{center}
    %    \includegraphics[width=.2\textwidth]{../imgs/tei-r.pdf}
    %\end{center}

    \begin{block}{Edizione image-based: Elementi strutturali}
        \begin{itemize}
            \item \texttt{<line> }
            \item[] trascrizione di una riga del foglio del manoscritto 
            \item[] \textbf{può essere contenuto solo da \texttt{<surface> e <zone>}!}
            \item per le righe di testo da inserire all’interno di \texttt{<text>} è sempre necessario usare \texttt{<lb/>}.
        \end{itemize}
    \end{block}
\end{frame}

%%
\begin{frame}
    \frametitle{Conclusioni: Edizioni Digitali}
    \addtocounter{nframe}{1}
    
    %\begin{center}
    %    \includegraphics[width=.2\textwidth]{../imgs/tei-r.pdf}
    %\end{center}

    \begin{block}{Elementi per interventi editoriali}
        A causa della relativa complessità di codifica di edizioni digitali image-based è preferibile usare strumenti software per facilitare la creazione di un facsimile digitale
    \end{block}

    \begin{block}{strutturali specifici}
        Manca però ancora uno strumento/funzione per collegare le immagini annotate al testo della trascrizione
    \end{block}
\end{frame}


    
    \end{document}