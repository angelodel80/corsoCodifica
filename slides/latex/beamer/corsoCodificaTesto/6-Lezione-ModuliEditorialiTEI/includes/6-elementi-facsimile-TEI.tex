\begin{frame}
    \frametitle{Introduzione}
    \addtocounter{nframe}{1}
    
    %\begin{center}
    %    \includegraphics[width=.2\textwidth]{../imgs/tei-r.pdf}
    %\end{center}

    \begin{block}{Cos'è la TEI}
        la TEI - \textit{acronimo di Text Encoding Initiative} - rappresenta un punto di riferimento per tutte le iniziative il cui scopo principale è quello di digitalizzare risorse testuali in ambito umanistico per fini di ricerca e di conservazione.
    \end{block}
    
\end{frame}

%%
Tipi di edizione digitale

edizione ipertestuale
prime a essere prodotte, ancora oggi spesso in formato HTML (← TEI XML), formato ideale per e. critiche
distribuzione sul Web (Biblioteca Digitale Italiana) 

facsimile digitale
riproduzione del manoscritto basata su scansione digitale
distribuzione sul Web (Bodleian Library, Oxford)

edizione basata su immagini (image-based digital edition)
il testo dell’edizione (diplomatica, critica) con le immagini del
manoscritto

%%
trascrizione collegata alle immagini del manoscritto serie di funzionalità “standard”:
immagini in formati e/o risoluzioni diverse 
lente d’ingrandimento
evidenziazione dettagli (← restauro digitale) 
motore di ricerca testuale
introduzione paleografica/filologica, 
commento al testo, bibliografia, etc.
(Electronic Beowulf, Vita Nova)

%% 
edizione digitale di un manoscritto

immagini del manoscritto trascrizione del/i testo/i creazione del facsimile digitale collegamento testo-immagine livello di edizione:
edizione diplomatica
edizione diplomatico-interpretativa (edizione critica)
singolo manoscritto = edizione diplomatica e/o interpretativa

%%
livelli di edizione

e. diplomatica: trascrizione del testo di un testimone rispettando la disposizione e la grafia originale, senza nessun tipo di correzione (errori manifesti) o altri interventi editoriali (espansione abbreviazioni)

e. diplomatico-interpretativa: sempre rispettando il testo originale, vengono corretti gli errori più evidenti, regolarizzate certe particolarità ortografiche (suddivisione delle parole), espanse le abbreviazioni, etc.

e. critica: sulla base della collazione di tutte le trascrizioni
dei testimoni viene stabilito lo stemma codicum e si tenta
di ricostruire il testo originale confrontando le varianti dei
testimoni più validi

%%
esempi electronic Beowulf, (variorum), lettere, etc

%%
collegamento mirato (hot-spot): una specifica area dell’immagine viene evidenziata in maniera tale che, interagendo con la stessa, vengono visualizzate delle informazioni quali note editoriali, versione migliorata di un dettaglio, commento al testo, etc.

collegamento generalizzato: tutto il testo dell’edizione viene messo in relazione diretta con le immagini, o parti di immagine, corrispondenti, in modo da poter accedere facilmente alla porzione di immagine corrispondente partendo dal testo, e viceversa

%%
obiettivo: realizzare un collegamento fra testo e immagine in maniera tale che cliccando sul testo viene visualizzata la parte di immagine corrispondente e viceversa

%%
gli schemi di codifica TEI versione P5 (2007) introducono numerosi miglioramenti per quanto riguarda la gestione e trascrizione di manoscritti

%%
tra queste la nuova sezione Digital facsimiles nel capitolo 11 Representation of Primary Sources:
http://www.tei-c.org/release/doc/tei-p5-doc/en/html/PH.html

%%
modulo per la descrizione di manoscritti (10 Manuscript
Description http://www.tei-c.org/release/doc/tei-p5-doc/en/html/MS.html)
nuovo elemento <choice> da usare per le coppie di elementi
di tipo “editoriale”

%%
includendo il modulo transcr nello schema di codifica TEI si rende disponibile un nuovo attributo globale:
@facs (facsimile) points to all or part of an image which corresponds with the content of the element
questo attributo può essere usato in qualsiasi elemento per associare il contenuto dello stesso a un’immagine:
<p n="1" facs="para1.jpg">
<head facs="head.jpg">
<pb facs="page1.jpg"/>

%%
oltre a @facs è necessario usare i nuovi elementi per collegare testo a immagine:
<facsimile> contains a representation of some written source in the form of a set of images rather than as transcribed or encoded text.
<surface> defines a written surface in terms of a rectangular coordinate space, optionally grouping one or more graphic representations of that space, and rectangular zones of interest within it.
@start points to an element which encodes the starting position of the text corresponding to the inscribed part of the surface.
<zone> defines a rectangular area contained within a surface 23 element.

%%
l’elemento <facsimile> è di tipo strutturale e si pone allo stesso livello di <text> o addirittura in alternativa a quest’ultimo
quando il modulo transcr viene aggiunto allo schema di codifica è possibile scegliere fra:
un <teiHeader> e un <facsimile>
un <teiHeader> e un <text>
un <teiHeader>, un <facsimile> e un <text>
questo permette una grande flessibilità:
caso più semplice: facsimile digitale con le immagini del ms
facsimile digitale con trascrizione del testo facsimile digitale con trascrizione e collegamento

%%
<TEI>
 <teiHeader>
  <!­­...­­>
 </teiHeader>
 <facsimile>
  <graphic url="page1.png"/>
  <graphic url="page2.png"/>
  <graphic url="page3.png"/>
  <graphic url="page4.png"/>
 </facsimile>
</TEI>

%%
<TEI>
 <teiHeader>
  <!­­...­­>
 </teiHeader>
 <text>
  <pb facs="page1.png"/>
   <!­­ inserire qui il testo di pagina 1 ­­>
  <pb facs="page2.png"/>
   <!­­ inserire qui il testo di pagina 2 ­­>
 </text>
</TEI>

%%
grazie a un foglio di stile XSLT è possibile generare una pagina HMTL divisa in due riquadri (immagine e testo)
in alternativa è possibile usare un elemento <facsimile> sullo stesso livello di <text>
ma allora sarebbe necessario aggiungere i collegamenti svantaggi di questa soluzione:
nessun collegamento testo-immagine a livello diverso dalla pagina
non è possibile individuare aree particolari delle immagini
i puntatori alle immagini sono sparsi per tutto il documento (a
meno che non si usi un elemento <facsimile>)

%%
la soluzione più efficace è la parallel transcription basata su <facsimile> e <text>
uso di <surface> e <zone> all’interno di <facsimile>
per definire le aree delle immagini
per collegare il testo della trascrizione a tali aree e/o immagini secondarie
le aree delle immagini sono individuate per mezzo di un sistema di coordinate cartesiane registrate come valori dei seguenti attributi di <surface> e <zone>:
ulx, uly coordinate x e y dell’angolo superiore sinistro lrx, lry coordinate x e y dell’angolo inferiore destro

%%
<surface> individua la superficie scritta di un’immagine
<surface ulx="0" uly="0" lrx="400" lry="280"> <graphic url="page1.png"/>
</surface>
può contenere più di un elemento <graphic>
<surface>
<graphic url="page1­highRes.png"/> <graphic url="page1­lowRes.png"/>
</surface>
invece di <graphic> può contenere una o più <zone> <surface> stesso si trova all’interno di <facsimile>

%%
L’elemento <zone>
<zone> definisce una specifica area dell’immagine usando lo stesso sistema di coordinate di <surface>
un’area dell’immagine:
<surface ulx="0" uly="0" lrx="500" lry="321"> <zone ulx="50" uly="20" lrx="400" lry="280">
<graphic url="scrittura.png"/></zone> <note>first page</note>
</surface>
o una porzione più piccola (utile per creare un hot-spot): <zone ulx="120" uly="48" lrx="143" lry="56">
 <graphic url="gloss.png"/>
 <note>Scribe gloss</note>
</zone>

%%
Esempio immagine zone

%%
<facsimile xml:id="imtAnnotatedImage"> <surface>
<graphic height="1797px" url="LindisfarneFol27rIncipitMatt.jpg" width="1266px"/>
<zone lrx="1268" lry="1797" rend="visible" rendition="surface" ulx="0" uly="­4" xml:id="imtArea_0"/>
<zone lrx="1267" lry="450" rend="visible" rendition="zone" ulx="1202" uly="356" xml:id="imtArea_1"/>
<zone lrx="1050" lry="792" rend="visible" rendition="zone" ulx="81" uly="30" xml:id="imtArea_3"/>
<zone lrx="1190" lry="154" rend="visible" rendition="zone" ulx="503" uly="48" xml:id="imtArea_4"/>
<zone lrx="1184" lry="412" rend="visible" rendition="zone" ulx="1116" uly="353" xml:id="imtArea__6"/>
<zone lrx="86" lry="606" rend="visible" rendition="zone" ulx="2" uly="478" xml:id="imtArea_5"/>
<zone lrx="995" lry="366" rend="visible" rendition="zone" ulx="694" uly="321" xml:id="imtArea_6"/>
  </surface>
</facsimile>

%%
Collegare il testo a <surface> e <zone>
per collegare il testo della trascrizione alle aree corrispondenti dell’immagine:
assegnare un identificatore univoco a ciascun elemento del facsimile usando l’attributo xml:id
usare l’attributo facs negli elementi testuali per specificare l’id degli elementi <surface> e <zone> corrispondenti
per collegare le aree delle immagini ai corrispondenti elementi di testo:
assegnare un identificatore univoco a ciascun elemento della trascrizione usando l’attributo xml:id
usare l’attributo start negli elementi <surface> e <zone> per
specificare l’id degli elementi testuali

%%
<text>
 <body>
  <div>
Esempio bidirezionale completo
<pb facs="#page1" n="1" xml:id="page_1"/>
<p>Lorem ipsum ... <gloss facs="#det1">semper</gloss></p> </div>
 </body>
</text>
<facsimile>
<surface xml:id="page1” start="#page_1" ulx="0" uly="0" lrx="500" lry="321">
<graphic url="page1.png”>
<zone xml:id="line1" ulx="50" uly="80" lrx="200" lry="321">
<graphic url="line1.png"/>
<note>First page.</note> </zone>
<zone xml:id="det1" ulx="120" uly="48" lrx="143" lry="56"> <graphic url="gloss.png"/>
<note>Scribe gloss.</note>
  </zone>
 </surface>
36
</facsimile>

%%
Facsimile con embedded transcription un metodo a metà fra facsimile digitale e edizione basata
su immagini è quello della embedded transcription: http://www.tei-c.org/release/doc/tei-p5-
doc/en/html/PH.html#PHZLAB
differenze importanti rispetto a trascrizione parallela basata su <text>:
il testo è considerato “di accompagnamento”, focus sulle immagini (ad esempio disposizione fisica delle parti)
infatti il modello di codifica di <line> è limitato nessun problema di conflitti di gerarchie

%%
Facsimile con embedded transcription uso dell’elemento <sourceDoc> sullo stesso livello
gerarchico e in alternativa a <facsimile> e <text>
uso di <surface> e <zone> in maniera simile a quanto visto in precedenza
<zone> contiene una serie di <line> corrispondenti alle righe di testo
<zone ulx="20" uly="40" lrx="120" lry="180">
   <line>prima riga di trascrizione</line>
   <line>seconda riga di trascrizione</line>
</zone>

%%
Esempio proust

%%
Come inserire le coordinate
domanda n. 1: come trovo i valori delle coordinate?
domanda n. 2: li devo inserire manualmente nei miei documenti TEI XML?
esistono numerosi programmi per calcolare le coordinate:
software di disegno in formato bitmap strumenti per programmatori di siti web
gli strumenti del software EPPT permettono di ottenere i valori delle coordinate e di inserirli automaticamente
software potente, progettato per creatori di edizioni digitali ancora non conforme alla TEI P5 (possibile conversione)

%%
Esempio teizoner

%%


