\begin{frame}
    \frametitle{Introduzione}
    \addtocounter{nframe}{1}
    
    %\begin{center}
    %    \includegraphics[width=.2\textwidth]{../imgs/tei-r.pdf}
    %\end{center}

    \begin{block}{Cos'è la TEI}
        la TEI - \textit{acronimo di Text Encoding Initiative} - rappresenta un punto di riferimento per tutte le iniziative il cui scopo principale è quello di digitalizzare risorse testuali in ambito umanistico per fini di ricerca e di conservazione.
    \end{block}
    
\end{frame}

%%
le nuove caratteristiche degli schemi TEI P5 offrono un’ottima base per il collegamento testo-immagine
a causa della relativa complessità è preferibile usare strumenti software per facilitare la creazione di un facsimile digitale
al momento l’IMT costituisce una buona soluzione quello che manca:
uno strumento/funzione per collegare le immagini annotate al testo della trascrizione
software di visualizzazione


%%
elementi contenuti nei moduli di base
elementi del modulo di descrizione dei manoscritti (10 Manuscript
Description http://www.tei-c.org/release/doc/tei-p5-doc/en/html/MS.html) elementi strutturali specifici:
<gb/>, <line>
elementi per interventi editoriali:
<abbr> <expan>, <orig> <reg>, <sic> <corr>, <subst>
<gap/>, <supplied>, <unclear>, <damage>
elementi del modulo gaiji
elementi del m. relativo all’apparato critico (12 Critical Apparatus
http://www.tei-c.org/release/doc/tei-p5-doc/en/html/TC.html)

%%
elementi contenuti nei moduli di base
elementi del modulo di descrizione dei manoscritti (10 Manuscript
Description http://www.tei-c.org/release/doc/tei-p5-doc/en/html/MS.html) elementi strutturali specifici:
<gb/>, <line>
elementi per interventi editoriali:
<abbr> <expan>, <orig> <reg>, <sic> <corr>, <subst>
<gap/>, <supplied>, <unclear>, <damage>
elementi del modulo gaiji
elementi del m. relativo all’apparato critico (12 Critical Apparatus
http://www.tei-c.org/release/doc/tei-p5-doc/en/html/TC.html)

%%
Elementi di intervento editoriale <damage> marca la parte di testo danneggiata
non proprio “intervento editoriale” ma spesso usato contestualmente con <gap/>, <unclear> e <supplied>
<supplied> testo inserito dal curatore perché l’originale è mancante o illeggibile
<l n="1">Nel mezzo del cammin di nostra vita</l> <l n="2"><damage agent="fire" extent="1line"><unclear>Mi ritrovai</unclear> <supplied reason="illegible" resp="rrdt">per una selva oscura,</supplied></damage></l>
<l n="3">Ché la diritta via era smarrita</l>

%%
Elementi di intervento editoriale
<subst> raggruppa una cancellazione e un’aggiunta scribale per rendere evidente che si tratta di una sostituzione
stessa semantica funzione di <choice>
<l n="1">Nel mezzo del cammin di nostra vita</l> <l n="2">Mi ritrovai <subst>
  <del>pir</del>
<add>per</add>
</subst>una selva oscura,</l>
<l n="3">Ché la diritta via era smarrita</l>