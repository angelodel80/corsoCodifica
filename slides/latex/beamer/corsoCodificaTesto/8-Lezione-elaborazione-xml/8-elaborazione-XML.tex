% DOM, Javascript XML, XSL
% capitolo 7 testo e automa di ciotti
% SLIDE Chiara sui fogli di stile
% Capitolo DOM professional Web Dev e XML processing
%

\documentclass{beamer}
    
    %    \usepackage[english]{babel}
        %\usepackage[latin1]{inputenc}
        %\usepackage[T1]{fontenc}
    
    \mode<presentation>{
      \setbeamertemplate{background canvas}[vertical shading]
      \usetheme{Berkeley}
      \useoutertheme{himinfolines}
    }
      
    \usepackage{ucs}
    \usepackage[utf8]{inputenc}
    \usepackage[english,polutonikogreek,italian,UKenglish,british]{babel}
    \usepackage{graphicx}
    \usepackage{colortbl}
    \usepackage{multicol}
    \usepackage{ulem}
    \usepackage{verbatim}
    \usepackage{alltt}
    \usepackage{ccicons}
    \usepackage{MnSymbol,wasysym}
    \usepackage{tikzsymbols}
    \usepackage{textcomp}
    \usepackage{xmpincl}
    
    \usepackage{parskip}
    \setcounter{nframes}{105}
    \setcounter{nframe}{1}
    \setbeamercovered{dynamic}
    \newenvironment{grcenv}{\begin{otherlanguage}{greek}}{\end{otherlanguage}}
    \newcommand{\g}[1]{\textgreek{#1}}
    \definecolor{darkgreen}{rgb}{0,0.5,0}
    \definecolor{darkblue}{rgb}{0,0,0.5}
    \definecolor{grey}{rgb}{0.5,0.5,0.5}
    \setcounter{tocdepth}{5}
    
    \makeatletter
    
    \makeatother
    %\includexmp{LicencesAndLicensing}
    
    %frame00 metadata
        \title{Codifica TEI - Visualizzazione ed Elaborazione: Fogli di Stile}
        \author[A.M. Del Grosso]{Angelo Mario Del Grosso \\ \tiny\textit{(Materiale tratto dalle lezioni di C. Di Pietro)}}
        \institute{\texttt{angelo.delgrosso@ilc.cnr.it} \\\textit{CNR-ILC-LicoLab} \\\url{http://licolab.ilc.cnr.it/}}
        \date{Istituto di Linguistica Computazionale ``A. Zampolli'', \today}
        \AtBeginSection[]{
        \begin{frame}<beamer>
        \addtocounter{nframe}{1}
        \footnotesize
        \frametitle{Progress status}
        \tableofcontents[currentsection,hideothersubsections]
        \end{frame}
        }
    
    \begin{document}
    
    \begin{frame}
        \maketitle
    \end{frame}
    
    \begin{frame}
        \frametitle{Sommario della Lezione}
        \tableofcontents
    \end{frame}
    
    \section{Introduzione ai Fogli di Stile}
    
    \begin{frame}
        \frametitle{Visualizzare ed Elaborare documenti XML}
        \addtocounter{nframe}{1}
        
        %\begin{center}
        %    \includegraphics[width=.2\textwidth]{../imgs/tei-r.pdf}
        %\end{center}
        %\textit{In parte già disponibili nei moduli TEI di base}

         \begin{block}{Perché visualizzare il testo}
        %     \emph{Per la critica testuale indispensabili i moduli}
             \begin{itemize}
                \item Controllare la codifica e correggere i refusi
                \item Assicurarsi che tutto sia stato trascritto correttamente
                \item Mostrare il testo a persone che non conoscono XML-TEI
                \item Disporre di una versione del lavoro fuibile
            \end{itemize}
         \end{block}
        
    \end{frame}
    
    \begin{frame}
        \frametitle{Visualizzare ed Elaborare documenti XML}
        \addtocounter{nframe}{1}
        
        \begin{block}{I fogli di stile (style sheet)}
           \begin{itemize}
               \item Descrive il modo in cui un documento elettronico deve essere visualizzato
               \item Il mezzo di visualizzazione può variare: lo schermo di un computer, la stampa, i sintetizzatori vocali, ecc.
           \end{itemize}
        \end{block}
        
    \end{frame}
    
    \begin{frame}
        \frametitle{Visualizzare ed Elaborare documenti XML}
        \addtocounter{nframe}{1}
        
        \begin{block}{Scopo dei fogli di stile}
           \begin{itemize}
               \item \emph{Separazione forma-contenuto}: \textit{la visualizzazione del documento è un processo indipendente (e successivo)}
               \item \emph{Gestione della resa grafica} per molti documenti contemporaneamente: \textit{massima uniformità dello stile}
               \item \emph{Gestione di mezzi diversi} dal monitor: smartphone, sintetizzatore
               vocale, stampante braille, ecc.
           \end{itemize}
        \end{block}
        
    \end{frame}

    \begin{frame}
        \frametitle{Visualizzare ed Elaborare documenti XML}
        \addtocounter{nframe}{1}
        
        \begin{block}{Metodi e Tecnologie}
            \textbf{Quelli più noti e utilizzati sono standard internazionali definiti dal consorzio W3 \url{(http://www.w3.org/Style/CSS)}.}
           \begin{itemize}
            \item CSS: Cascading Style Sheets
            \item XSL: eXtensible Stylesheet Language
           \end{itemize}
        \end{block}
        
    \end{frame}

    \begin{frame}
        \frametitle{Visualizzare ed Elaborare documenti XML}
        \addtocounter{nframe}{1}
        
        \begin{block}{CSS: Cascading Style Sheets}
           
           \begin{itemize}
               \item Nati per HTML, possono essere utilizzati anche con XML
               \item Mostrano cosa c'è nel file, nell'ordine in cui questo compare
               \item Molto semplici, ma anche limitati
           \end{itemize}
        \end{block}
        
    \end{frame}

    \begin{frame}
        \frametitle{Visualizzare ed Elaborare documenti XML}
        \addtocounter{nframe}{1}
        
        \begin{block}{XSL: eXtensible Stylesheet Language}
           
           \begin{itemize}
               \item Trasforma XML in qualcos'altro: HTML, PDF, ODT, EPUB
               \item Molto potente, ma complesso e difficile da imparare
           \end{itemize}
        \end{block}
        
    \end{frame}


    \begin{frame}
        \frametitle{Visualizzare ed Elaborare documenti XML}
        \addtocounter{nframe}{1}
        
        \begin{block}{Perché due tipologie di fogli di stile}
           
           \begin{itemize}
               \item I documenti XML possono essere resi in tre modi diversi
               \item Usare CSS quando si può e XSL quando è necessario
           \end{itemize}
        \end{block}
        
    \end{frame}


    \begin{frame}
        \frametitle{Visualizzare ed Elaborare documenti XML}
        \addtocounter{nframe}{1}
        
        \begin{center}
            \includegraphics[width=.9\textwidth]{imgs/XML-VisualModality.png}
        \end{center}
        %\textit{In parte già disponibili nei moduli TEI di base}
    
    \end{frame}


    \begin{frame}
        \frametitle{Visualizzare ed Elaborare documenti XML}
        \addtocounter{nframe}{1}
        \begin{center}
            \textbf{CSS e XSL: caratteristiche a confronto}
        \end{center}
       
        \begin{center}
            \includegraphics[width=.9\textwidth]{imgs/css-xsl.png}
        \end{center}
        %\textit{In parte già disponibili nei moduli TEI di base}
    
    \end{frame}

    \begin{frame}
        \frametitle{Visualizzare ed Elaborare documenti XML}
        \addtocounter{nframe}{1}
        
        \begin{block}{Fogli di Stile CSS}
           
           \begin{itemize}
               \item Il formato CSS (Cascading Style Sheets) è nato per essere applicato alle pagine HTML.
               \item ottenere una separazione tra contenuto e forma
               \item Le specifiche dei fogli di stile CSS sono disponibili sul sito \url{http://www.w3.org/Style/CSS}
           \end{itemize}
        \end{block}
        
    \end{frame}

    \begin{frame}
        \frametitle{Visualizzare ed Elaborare documenti XML}
        \addtocounter{nframe}{1}
        
        \begin{block}{Fogli di Stile CSS}
           
           \begin{itemize}
               \item Al momento è in corso di definizione la successiva versione CSS4
               \item La situazione è in costante miglioramento grazie alla nuova generazione
               di navigatori (Firefox, Opera, Konqueror, Safari, Chrome, ecc.)
               \item Possono essere utilizzate anche per documenti XML
           \end{itemize}
        \end{block}
        
    \end{frame}


    \begin{frame}
        \frametitle{Visualizzare ed Elaborare documenti XML}
        \addtocounter{nframe}{1}
        
        \begin{block}{Fogli di Stile CSS}
           
           \begin{itemize}
               \item Offrono limitatissimi mezzi per modificare il documento al quale vengono applicati (in particolare aggiungere testo)
               \item Sono basati su una sintassi specifica piuttosto semplice
           \end{itemize}
        \end{block}
        
    \end{frame}


    \begin{frame}
        \frametitle{Visualizzare ed Elaborare documenti XML}
        \addtocounter{nframe}{1}
        \begin{center}
            \textbf{Sintassi CSS}
        \end{center}
       
        \begin{center}
            \includegraphics[width=.9\textwidth]{imgs/css-sintassi.png}
        \end{center}
        %\textit{In parte già disponibili nei moduli TEI di base}
    
    \end{frame}

    \begin{frame}
        \frametitle{Visualizzare ed Elaborare documenti XML}
        \addtocounter{nframe}{1}
        \begin{block}{Fogli di Stile CSS: invocazione}
            
            \begin{itemize}
                \item usando \textbf{l’elemento} \texttt{<style>} nel gruppo \texttt{<head>} di un documento
                HTML/XHTML.
                \item con un \textbf{collegamento} all’interno del gruppo \texttt{<head>} di un documento HTML/XHTML
                \item con una \textbf{istruzione} specifica all’inizio di un documento XML
            \end{itemize}
         \end{block}
    
    \end{frame}

    \begin{frame}
        \frametitle{Visualizzare ed Elaborare documenti XML}
        \addtocounter{nframe}{1}
        \begin{block}{Fogli di Stile CSS: invocazione HTML}

            \texttt{<link rel="stylesheet" }
                \\\texttt{ type="text/css" href="default.css" media="screen" > }

         \end{block}

         \begin{block}{Fogli di Stile CSS: invocazione XML}
            
            \texttt{<?xml-stylesheet} 
                \\\texttt{type="text/css" href="style.css"?>}

         \end{block}
    
    \end{frame}

    % \begin{frame}
    %     \frametitle{Modularità della TEI}
    %     \addtocounter{nframe}{1}
        
    %    % \begin{center}
    %     % \includegraphics[width=.2\textwidth]{../imgs/tei-r.pdf}
    %     % \end{center}
    
    %     \begin{itemize}
            
    %         \item<1-> parleremo del sistema Modulare della TEI
    %             \begin{itemize}
    %                 \item<1-> Moduli
    %                 \item<1-> Classi
    %                 \item<1-> Macro
    %                 \item<1-> Datatype
    %             \end{itemize} 
    %         \item<2-> parleremo degli elementi basilari
    %             \begin{itemize}
    %                 \item<2-> Intestazione TEI (TEIHeader)
    %                 \item<2-> Elementi e attributi presenti in tutti i documenti TEI
    %                 \item<2-> Esempi di codifica
    %             \end{itemize} 
    %     \end{itemize}
        
    % \end{frame}
    
    \section{XSL Transformations: Caratteristiche Fondamentali}
    % \begin{frame}
%     \frametitle{Visualizzare ed Elaborare documenti XML}
%     \addtocounter{nframe}{1}
    
%     %\begin{center}
%     %    \includegraphics[width=.2\textwidth]{../imgs/tei-r.pdf}
%     %\end{center}
%     %\textit{In parte già disponibili nei moduli TEI di base}

%      \begin{block}{Perché visualizzare il testo}
%     %     \emph{Per la critica testuale indispensabili i moduli}
%          \begin{itemize}
%             \item  Controllare la codifica e correggere i refusi
%              \item Assicurarsi che tutto sia stato trascritto correttamente
%              \item Mostrare il testo a persone che non conoscono XML-TEI
%              \item Disporre di una versione del lavoro fuibile
%         \end{itemize}
%      \end{block}
    
% \end{frame}



\begin{frame}
    \frametitle{Fondamenti Extensible Stylesheet Language}
    \addtocounter{nframe}{1}
    
    %\begin{center}
    %    \includegraphics[width=.2\textwidth]{../imgs/tei-r.pdf}
    %\end{center}
    %\textit{In parte già disponibili nei moduli TEI di base}

     \begin{block}{eXtensible Stylesheet Language (XSL)}
    %     \emph{Per la critica testuale indispensabili i moduli}
         \begin{itemize}
            \item  Specifica del W3C che descrive un metodo per la visualizzazione e manipolazione dei documenti XML.
             \item Maggiore controllo sulla presentazione dei dati XML
             \item Generazione di layout complessi e composti
        \end{itemize}
     \end{block}
    
\end{frame}


\begin{frame}
    \frametitle{Fondamenti Extensible Stylesheet Language}
    \addtocounter{nframe}{1}
    
    %\begin{center}
    %    \includegraphics[width=.2\textwidth]{../imgs/tei-r.pdf}
    %\end{center}
    %\textit{In parte già disponibili nei moduli TEI di base}

     \begin{block}{XSL incorpora tre linguaggi}
    %     \emph{Per la critica testuale indispensabili i moduli}
         \begin{itemize}
            \item \textbf{XSL Transformations (XSL-T)}: \textit{trasformazione di un documento XML in un altro tipo di documento} (es.: HTML)
            \item \textbf{XSL Formatting Objects (XSL-FO)}: \textit{applicazione degli stili e della resa grafica di un documento XML}
            \item \textbf{XML Path (XPath)}: \textit{usato nei fogli di stile XSLT per selezionare le parti di un documento XML}
        \end{itemize}
     \end{block}
    
\end{frame}

\begin{frame}
    \frametitle{Fondamenti Extensible Stylesheet Language}
    \addtocounter{nframe}{1}
    
    \begin{center}
        \includegraphics[width=.8\textwidth]{imgs/SchemaXSLTprocessing.png}
    \end{center}
    %\textit{In parte già disponibili nei moduli TEI di base}

\end{frame}

\begin{frame}
    \frametitle{Fondamenti Extensible Stylesheet Language}
    \addtocounter{nframe}{1}
    
    %\begin{center}
    %    \includegraphics[width=.2\textwidth]{../imgs/tei-r.pdf}
    %\end{center}
    %\textit{In parte già disponibili nei moduli TEI di base}

     \begin{block}{XSL Transformations}
    %     \emph{Per la critica testuale indispensabili i moduli}
         \begin{itemize}
            \item XSLT è un vero e proprio linguaggio di programmazione che usa la sintassi XML
            \item Usa namespace differenti per distinguere fra istruzioni proprie (precedute da \textbf{xsl:}) e output
            \item Legge e scrive alberi XML (ma è possibile ottenere come output anche del
            codice HTML o del testo semplice)
            \item Versione attuale: XSLT 3.0 \url{(https://www.w3.org/TR/xslt-30/)}
        \end{itemize}
     \end{block}
    
\end{frame}

\begin{frame}
    \frametitle{Fondamenti Extensible Stylesheet Language}
    \addtocounter{nframe}{1}
    
    %\begin{center}
    %    \includegraphics[width=.2\textwidth]{../imgs/tei-r.pdf}
    %\end{center}
    %\textit{In parte già disponibili nei moduli TEI di base}

     \begin{block}{XSL Capacità di trasformazione}
    %     \emph{Per la critica testuale indispensabili i moduli}
         \begin{itemize}
            \item generazione di testo costante;
            \item soppressione del contenuto;
            \item spostamento del testo (es., scambio ordine di nome e cognome);
        \end{itemize}
     \end{block}
    
\end{frame}

\begin{frame}
    \frametitle{Fondamenti Extensible Stylesheet Language}
    \addtocounter{nframe}{1}
    
    %\begin{center}
    %    \includegraphics[width=.2\textwidth]{../imgs/tei-r.pdf}
    %\end{center}
    %\textit{In parte già disponibili nei moduli TEI di base}

     \begin{block}{XSL Capacità di trasformazione}
    %     \emph{Per la critica testuale indispensabili i moduli}
         \begin{itemize}
            \item duplicazione del testo (ad es., tabella di contenuti copiando i titoli);
            \item ordinamento dei contenuti (ad es, termini in ordine alfabetico);
            \item elaborazione di nuove informazioni in base a quelle esistenti (es. statistiche)
        \end{itemize}
     \end{block}
    
\end{frame}

\begin{frame}
    \frametitle{Fondamenti Extensible Stylesheet Language}
    \addtocounter{nframe}{1}
    
    %\begin{center}
    %    \includegraphics[width=.2\textwidth]{../imgs/tei-r.pdf}
    %\end{center}
    %\textit{In parte già disponibili nei moduli TEI di base}

     \begin{block}{Caratteristiche fondamentali di XSLT}
    %     \emph{Per la critica testuale indispensabili i moduli}
         \begin{itemize}
            \item Basato su regole di trasformazione (\textit{modello pattern-matching})
            \item Le regole sono dichiarative \textit{(specificano che cosa deve essere generato quando si incontra un certo modello nel documento)}
            \item Le regole possono essere disposte in qualsiasi ordine
        \end{itemize}
     \end{block}
    
\end{frame}


\begin{frame}
    \frametitle{Fondamenti Extensible Stylesheet Language}
    \addtocounter{nframe}{1}
    
    %\begin{center}
    %    \includegraphics[width=.2\textwidth]{../imgs/tei-r.pdf}
    %\end{center}
    %\textit{In parte già disponibili nei moduli TEI di base}

     \begin{block}{Modalità di Trasformazioni XSLT}
    %     \emph{Per la critica testuale indispensabili i moduli}
         \begin{itemize}
            \item \textbf{Lato server}, utilizzando script Java, ASP, PHP ecc. , per produrre "al volo" pagine HTML sulla base di documenti XML (es. Cocoon);
            \item \textbf{Lato client}, sui Browser che supportano questa tecnologia;
            \item tramite un programma separato (come ad esempio Oxygen), che permette di applicare uno o più scenari di trasformazione.
        \end{itemize}
     \end{block}
    
\end{frame}

\begin{frame}
    \frametitle{Fondamenti Extensible Stylesheet Language}
    \addtocounter{nframe}{1}
    
    %\begin{center}
    %    \includegraphics[width=.2\textwidth]{../imgs/tei-r.pdf}
    %\end{center}
    %\textit{In parte già disponibili nei moduli TEI di base}

     \begin{block}{Componenti di Base di un foglio XSLT}
    %     \emph{Per la critica testuale indispensabili i moduli}
         \begin{itemize}
            \item Intestazione XML
            \item Elemento radice \textit{stylesheet} e namespace
            \item Eventuali istruzioni di elaborazione
            \item Serie di template rules
        \end{itemize}
     \end{block}
    
\end{frame}

\begin{frame}
    \frametitle{Fondamenti Extensible Stylesheet Language}
    \addtocounter{nframe}{1}
    
    %\begin{center}
    %    \includegraphics[width=.2\textwidth]{../imgs/tei-r.pdf}
    %\end{center}
    %\textit{In parte già disponibili nei moduli TEI di base}

     \begin{block}{Intestazione XML}
        \texttt{<?xml version="1.0" encoding="UTF-8" ?>}
     \end{block}

     \begin{block}{Elemento radice}
        \texttt{<xsl:stylesheet version='2.0'}
            \\\texttt{  xmlns:xsl='http://www.w3.org/1999/XSL/Transform'>}
     \end{block}
    
\end{frame}

\begin{frame}
    \frametitle{Fondamenti Extensible Stylesheet Language}
    \addtocounter{nframe}{1}
    
    %\begin{center}
    %    \includegraphics[width=.2\textwidth]{../imgs/tei-r.pdf}
    %\end{center}
    %\textit{In parte già disponibili nei moduli TEI di base}

     \begin{block}{Eventuali istruzioni di elaborazione}
        \texttt{<xsl:output method="xml" version="1.0" indent="yes"/>}
     \end{block}

     \begin{block}{Serie di template rules}
        \texttt{<xsl:template match="/" > ...</xsl:template>} 
        \\\texttt{<xsl:template match="title" > ... </xsl:template>}
     \end{block}
    
\end{frame}

\begin{frame}
    \frametitle{Fondamenti Extensible Stylesheet Language}
    \addtocounter{nframe}{1}
    
    \begin{center}
        \includegraphics[width=.9\textwidth]{imgs/template-modello.png}
    \end{center}
    %\textit{In parte già disponibili nei moduli TEI di base}

\end{frame}

\begin{frame}
    \frametitle{Fondamenti Extensible Stylesheet Language}
    \addtocounter{nframe}{1}
    
    %\begin{center}
    %    \includegraphics[width=.2\textwidth]{../imgs/tei-r.pdf}
    %\end{center}
    %\textit{In parte già disponibili nei moduli TEI di base}

     \begin{block}{Come vengono applicate le regole XSLT}
        \emph{Il processore XSLT}
         \begin{itemize}
            \item Legge il documento XML in input e crea l’albero corrispondente
            \item Inizia a percorrere l’albero leggendo i singoli nodi
            \item Confronta Ogni nodo con le regole presenti nel foglio di stile
            \item Produce l’output secondo le istruzioni della regola
            \item Restituisce un albero di output
        \end{itemize}
     \end{block}
    
\end{frame}

\begin{frame}
    \frametitle{Fondamenti Extensible Stylesheet Language}
    \addtocounter{nframe}{1}
    
    \begin{center}
        \includegraphics[width=.9\textwidth]{imgs/Processo-xslt.png}
    \end{center}
    %\textit{In parte già disponibili nei moduli TEI di base}

\end{frame}


\begin{frame}
    \frametitle{Fondamenti Extensible Stylesheet Language}
    \addtocounter{nframe}{1}
    
    %\begin{center}
    %    \includegraphics[width=.2\textwidth]{../imgs/tei-r.pdf}
    %\end{center}
    %\textit{In parte già disponibili nei moduli TEI di base}

     \begin{block}{Esempio di trasformazione}
         Nella sotto-cartella xsl della cartella src leggere il file d'esercizio e individuare le varie componenti.
     \end{block}
    
\end{frame}

\begin{frame}
    \frametitle{Fondamenti Extensible Stylesheet Language}
    \addtocounter{nframe}{1}
    
    %\begin{center}
    %    \includegraphics[width=.2\textwidth]{../imgs/tei-r.pdf}
    %\end{center}
    %\textit{In parte già disponibili nei moduli TEI di base}

     \begin{block}{Tipi di nodo nell'albero XML}
         \begin{itemize}
            \item \textbf{Radice} del Documento
            \item \textbf{Elementi} con contentuto del sotto albero
            \item \textbf{Attributi}
            \item \textbf{Testo} compresi gli spazi vuoti
            \item \textbf{Commenti} (contenuto tra \texttt{<!-- -->})
            \item \textbf{Namespace} con riferimenti e URI
            \item \textbf{Istruzioni di elaborazione} (contenuto tra \texttt{<? ?>})
        \end{itemize}
     \end{block}
    
\end{frame}

\begin{frame}
    \frametitle{Fondamenti Extensible Stylesheet Language}
    \addtocounter{nframe}{1}
    
    %\begin{center}
    %    \includegraphics[width=.2\textwidth]{../imgs/tei-r.pdf}
    %\end{center}
    %\textit{In parte già disponibili nei moduli TEI di base}

     \begin{block}{Tipi di nodo nell'albero XML}
         \begin{itemize}
            \item Il \textbf{documento} stesso costituisce la radice (\textit{l'elemento radice XML non è la radice dell'albero di rappresentazione!})
            \item L'intero albero è suddivisibile in sotto-alberi
            \item I nodi più importanti sono gli elementi e i loro attributi
            \item Le entità vengono "tradotte" nel testo loro assegnato al momento
            della dichiarazione
            \item Lo spazio "vuoto" può essere considerato o no
        \end{itemize}
     \end{block}
    
\end{frame}

\begin{frame}
    \frametitle{Fondamenti Extensible Stylesheet Language}
    \addtocounter{nframe}{1}
    
    \begin{center}
        \includegraphics[width=.75\textwidth]{imgs/Schema-trasformazione.png}
    \end{center}
    %\textit{In parte già disponibili nei moduli TEI di base}

\end{frame}


    
    \section{Template (rules/named)}
    % \begin{frame}
%     \frametitle{Visualizzare ed Elaborare documenti XML}
%     \addtocounter{nframe}{1}
    
%     %\begin{center}
%     %    \includegraphics[width=.2\textwidth]{../imgs/tei-r.pdf}
%     %\end{center}
%     %\textit{In parte già disponibili nei moduli TEI di base}

%      \begin{block}{Perché visualizzare il testo}
%     %     \emph{Per la critica testuale indispensabili i moduli}
%          \begin{itemize}
%             \item  Controllare la codifica e correggere i refusi
%              \item Assicurarsi che tutto sia stato trascritto correttamente
%              \item Mostrare il testo a persone che non conoscono XML-TEI
%              \item Disporre di una versione del lavoro fuibile
%         \end{itemize}
%      \end{block}
    
% \end{frame}

\begin{frame}
    \frametitle{Visualizzare ed Elaborare documenti XML}
    \addtocounter{nframe}{1}
    
    %\begin{center}
    %    \includegraphics[width=.2\textwidth]{../imgs/tei-r.pdf}
    %\end{center}
    %\textit{In parte già disponibili nei moduli TEI di base}

     \begin{block}{Elemento \texttt{<xsl:template>}}
        Definisce una regola (ovvero un modello) di trasformazione per i nodi di un particolare tipo/contesto.
     \end{block}
    
\end{frame}

\begin{frame}
    \frametitle{Visualizzare ed Elaborare documenti XML}
    \addtocounter{nframe}{1}
    
    \begin{center}
        \includegraphics[width=.95\textwidth]{imgs/Schema-template.png}
    \end{center}
    %\textit{In parte già disponibili nei moduli TEI di base}

\end{frame}

\begin{frame}
    \frametitle{Visualizzare ed Elaborare documenti XML}
    \addtocounter{nframe}{1}
    
    %\begin{center}
    %    \includegraphics[width=.2\textwidth]{../imgs/tei-r.pdf}
    %\end{center}
    %\textit{In parte già disponibili nei moduli TEI di base}

     \begin{block}{Attributi Elemento \texttt{<xsl:template>}}
    %     \emph{Per la critica testuale indispensabili i moduli}
         \begin{itemize}
             \item \textbf{name}: nome del template;
             \item \textbf{match}: pattern che indica l'elemento su cui applicare il modello;
             \item \textbf{priority}: priorità del modello;
             \item \textbf{mode}: modalità di elaborazione, che consente all'elemento di essere elaborato più volte per produrre un risultato diverso ogni volta.
        \end{itemize}
     \end{block}
    
\end{frame}


\begin{frame}
    \frametitle{Visualizzare ed Elaborare documenti XML}
    \addtocounter{nframe}{1}
    
    %\begin{center}
    %    \includegraphics[width=.2\textwidth]{../imgs/tei-r.pdf}
    %\end{center}
    %\textit{In parte già disponibili nei moduli TEI di base}

     \begin{block}{Elemento \texttt{<xsl:template>}}
        \textit{I template XSLT possono avere due forme: }
        \begin{itemize}
            \item \textbf{"tempate rules"} che specificano una regola con pattern-matching ( \texttt{<xsl:apply-templates>})
            \item \textbf{named templates} che specificano regole che possono essere chiamate esplicitamente con \texttt{<xsl:call-template>}
        \end{itemize}

     \end{block}
    
\end{frame}


%% value-of

\begin{frame}
    \frametitle{Visualizzare ed Elaborare documenti XML}
    \addtocounter{nframe}{1}
    
    %\begin{center}
    %    \includegraphics[width=.2\textwidth]{../imgs/tei-r.pdf}
    %\end{center}
    %\textit{In parte già disponibili nei moduli TEI di base}

     \begin{block}{Elemento \texttt{<xsl:value-of>}}
        Restituisce il contenuto del nodo selezionato secondo l'espressione XPath indicata.
        \\ \textit{(Il contenuto di un elemento è costituito da tutti i caratteri che si trovano fra tag di apertura e tag di chiusura)}
     \end{block}
    
\end{frame}

\begin{frame}
    \frametitle{Visualizzare ed Elaborare documenti XML}
    \addtocounter{nframe}{1}
    
    \begin{center}
        \includegraphics[width=.95\textwidth]{imgs/Schema-value-of.png}
    \end{center}
    %\textit{In parte già disponibili nei moduli TEI di base}

\end{frame}

\begin{frame}
    \frametitle{Visualizzare ed Elaborare documenti XML}
    \addtocounter{nframe}{1}
    
    %\begin{center}
    %    \includegraphics[width=.2\textwidth]{../imgs/tei-r.pdf}
    %\end{center}
    %\textit{In parte già disponibili nei moduli TEI di base}

     \begin{block}{Attributi Elemento \texttt{<xsl:value-of>}}
    %     \emph{Per la critica testuale indispensabili i moduli}
         \begin{itemize}
             \item \textbf{select}: espressione XPath da valutare nel contesto corrente
             \item \textbf{disable-output-escaping}: default "no"; se "yes", il testo di
             output non esclude i caratteri XML dal testo
        \end{itemize}
     \end{block}
    
\end{frame}

\begin{frame}
    \frametitle{Visualizzare ed Elaborare documenti XML}
    \addtocounter{nframe}{1}
    
    %\begin{center}
    %    \includegraphics[width=.2\textwidth]{../imgs/tei-r.pdf}
    %\end{center}
    %\textit{In parte già disponibili nei moduli TEI di base}

     \begin{block}{Esempio Elemento \texttt{<xsl:value-of>}}
        
        \texttt{<xsl:template match="fileDesc" >}
        \\\texttt{<h1>File Desc</h1>}
        \\\texttt{<p>}
            \\\texttt{<xsl:value-of select="titleStmt/title" disable-output-escaping="no" />}
        \\\texttt{</p>}
        \\\texttt{</xsl:template>}

     \end{block}
    
\end{frame}

%% Apply-templates

\begin{frame}
    \frametitle{Visualizzare ed Elaborare documenti XML}
    \addtocounter{nframe}{1}
    
    %\begin{center}
    %    \includegraphics[width=.2\textwidth]{../imgs/tei-r.pdf}
    %\end{center}
    %\textit{In parte già disponibili nei moduli TEI di base}

     \begin{block}{Elemento \texttt{<xsl:apply-templates>}}
        Elaborare in modo ricorsivo i nodi di un documento XML a partire da un punto preciso dell'albero XML.
     \end{block}

     \begin{block}{Elemento \texttt{<xsl:apply-templates>}}
        Confronta ogni nodo presente all’interno del nodo selezionato con le template rules del foglio di stile e se viene trovata una regola applicabile questa viene applicata.
     \end{block}

\end{frame}

\begin{frame}
    \frametitle{Visualizzare ed Elaborare documenti XML}
    \addtocounter{nframe}{1}
    
    \begin{center}
        \includegraphics[width=.95\textwidth]{imgs/Schema-apply-templates.png}
    \end{center}
    %\textit{In parte già disponibili nei moduli TEI di base}

\end{frame}

\begin{frame}
    \frametitle{Visualizzare ed Elaborare documenti XML}
    \addtocounter{nframe}{1}
    
    %\begin{center}
    %    \includegraphics[width=.2\textwidth]{../imgs/tei-r.pdf}
    %\end{center}
    %\textit{In parte già disponibili nei moduli TEI di base}

     \begin{block}{Attributi Elemento \texttt{<xsl:apply-templates>}}
    %     \emph{Per la critica testuale indispensabili i moduli}
         \begin{itemize}
             \item \textbf{select}: espressione XPath
             \item \textbf{mode}: modalità di elaborazione
        \end{itemize}
     \end{block}
    
\end{frame}

\begin{frame}
    \frametitle{Visualizzare ed Elaborare documenti XML}
    \addtocounter{nframe}{1}
    
    %\begin{center}
    %    \includegraphics[width=.2\textwidth]{../imgs/tei-r.pdf}
    %\end{center}
    %\textit{In parte già disponibili nei moduli TEI di base}

     \begin{block}{Esempio Elemento \texttt{<xsl:apply-templates>}}
        
        \texttt{<xsl:template match="/" >}
        \\\texttt{<html><head>}
        \\\texttt{<title><xsl:value-of select="TEI/teiHeader/fileDesc/title"/></title>}
        \\\texttt{</head><body><div><span>1.</span>}
        \\\texttt{<xsl:apply-templates select="TEI/teiHeader/fileDesc" />}
        \\\texttt{</div></body></html></xsl:template>}

     \end{block}
    
\end{frame}


%% xsl:for-each

\begin{frame}
    \frametitle{Visualizzare ed Elaborare documenti XML}
    \addtocounter{nframe}{1}
    
    %\begin{center}
    %    \includegraphics[width=.2\textwidth]{../imgs/tei-r.pdf}
    %\end{center}
    %\textit{In parte già disponibili nei moduli TEI di base}

     \begin{block}{Elemento \texttt{<xsl:for-each>}}
        Se sono presenti più nodi con lo stesso nome (e manca un’istruzione ricorsiva precedente) \texttt{<xsl:value-of>} restituisce il valore del primo che incontra.
     \end{block}

     \begin{block}{Elemento \texttt{<xsl:for-each>}}
        è quindi possibile usare l’istruzione \texttt{<xsl:for-each>} e applicare un’istruzione \texttt{<xsl:value-of>} a tutti i nodi identificati dalla regola.
     \end{block}

\end{frame}

\begin{frame}
    \frametitle{Visualizzare ed Elaborare documenti XML}
    \addtocounter{nframe}{1}
    
    \begin{center}
        \includegraphics[width=.95\textwidth]{imgs/Schema-for-each.png}
    \end{center}
    %\textit{In parte già disponibili nei moduli TEI di base}

\end{frame}

\begin{frame}
    \frametitle{Visualizzare ed Elaborare documenti XML}
    \addtocounter{nframe}{1}
    
    %\begin{center}
    %    \includegraphics[width=.2\textwidth]{../imgs/tei-r.pdf}
    %\end{center}
    %\textit{In parte già disponibili nei moduli TEI di base}

     \begin{block}{Attributi Elemento \texttt{<xsl:for-each>}}
    %     \emph{Per la critica testuale indispensabili i moduli}
         \begin{itemize}
             \item \textbf{select}: espressione XPath
        \end{itemize}
     \end{block}
    
\end{frame}

\begin{frame}
    \frametitle{Visualizzare ed Elaborare documenti XML}
    \addtocounter{nframe}{1}
    
    %\begin{center}
    %    \includegraphics[width=.2\textwidth]{../imgs/tei-r.pdf}
    %\end{center}
    %\textit{In parte già disponibili nei moduli TEI di base}

     \begin{block}{Esempio Elemento \texttt{<xsl:for-each>}}
        
        \texttt{<xsl:template match="/" >}
        \\\texttt{<html><head>}
        \\\texttt{<title><xsl:value-of select="TEI/teiHeader/fileDesc/title"/></title>}
        \\\texttt{</head><body><div>}
        \\\texttt{<xsl:for-each select="//div" >}
        \\\texttt{<div><xsl:value-of select="./p" /></div>}
        \\\texttt{</xsl:for-each></div></body></html>}
     \end{block}
    
\end{frame}


%% xsl:text


\begin{frame}
    \frametitle{Visualizzare ed Elaborare documenti XML}
    \addtocounter{nframe}{1}
    
    %\begin{center}
    %    \includegraphics[width=.2\textwidth]{../imgs/tei-r.pdf}
    %\end{center}
    %\textit{In parte già disponibili nei moduli TEI di base}

     \begin{block}{Elemento \texttt{<xsl:text>}}
        Permette di inserire una stringa di testo nell’albero di output.
     \end{block}

     \begin{block}{Elemento \texttt{<xsl:text>}}
        Molto utile se si è deciso di eliminare tutti gli spazi e gli a capo.
     \end{block}
     

\end{frame}

\begin{frame}
    \frametitle{Visualizzare ed Elaborare documenti XML}
    \addtocounter{nframe}{1}
    
    \begin{center}
        \includegraphics[width=.95\textwidth]{imgs/Schema-text.png}
    \end{center}
    %\textit{In parte già disponibili nei moduli TEI di base}

\end{frame}

\begin{frame}
    \frametitle{Visualizzare ed Elaborare documenti XML}
    \addtocounter{nframe}{1}
    
    %\begin{center}
    %    \includegraphics[width=.2\textwidth]{../imgs/tei-r.pdf}
    %\end{center}
    %\textit{In parte già disponibili nei moduli TEI di base}

     \begin{block}{Attributi Elemento \texttt{<xsl:text>}}
    %     \emph{Per la critica testuale indispensabili i moduli}
         \begin{itemize}
             \item \textbf{disable-output-escaping}: se "yes", consente di copiare i caratteri di marcatura non identificati nell'albero di output.
        \end{itemize}
     \end{block}
    
\end{frame}

\begin{frame}
    \frametitle{Visualizzare ed Elaborare documenti XML}
    \addtocounter{nframe}{1}
    
    %\begin{center}
    %    \includegraphics[width=.2\textwidth]{../imgs/tei-r.pdf}
    %\end{center}
    %\textit{In parte già disponibili nei moduli TEI di base}

     \begin{block}{Esempio Elemento \texttt{<xsl:text>}}
        
        \texttt{<xsl:for-each select="\$attr" >}
        \\\texttt{<xsl:value-of select="concat('[',position(),']',current())" />}
        \\\texttt{<xsl:text>\&\#32;</xsl:text>}
        \\\texttt{</xsl:for-each>}

     \end{block}

\end{frame}

%% if
\begin{frame}
    \frametitle{Visualizzare ed Elaborare documenti XML}
    \addtocounter{nframe}{1}
    
    %\begin{center}
    %    \includegraphics[width=.2\textwidth]{../imgs/tei-r.pdf}
    %\end{center}
    %\textit{In parte già disponibili nei moduli TEI di base}

     \begin{block}{Elemento \texttt{<xsl:if>}}
        Identifica una condizione semplice: la regola viene eseguita soltanto se la condizione viene soddisfatta.
     \end{block}

\end{frame}

\begin{frame}
    \frametitle{Visualizzare ed Elaborare documenti XML}
    \addtocounter{nframe}{1}
    
    \begin{center}
        \includegraphics[width=.95\textwidth]{imgs/Schema-if.png}
    \end{center}
    %\textit{In parte già disponibili nei moduli TEI di base}

\end{frame}

\begin{frame}
    \frametitle{Visualizzare ed Elaborare documenti XML}
    \addtocounter{nframe}{1}
    
    %\begin{center}
    %    \includegraphics[width=.2\textwidth]{../imgs/tei-r.pdf}
    %\end{center}
    %\textit{In parte già disponibili nei moduli TEI di base}

     \begin{block}{Attributi Elemento \texttt{<xsl:if>}}
    %     \emph{Per la critica testuale indispensabili i moduli}
         \begin{itemize}
             \item \textbf{test}: l’espressione di test. Se restituisce true, il contenuto di \texttt{<xsl:if>} viene valutato e inserito nell’albero di output; altrimenti viene ignorato
        \end{itemize}
     \end{block}
    
\end{frame}

\begin{frame}
    \frametitle{Visualizzare ed Elaborare documenti XML}
    \addtocounter{nframe}{1}
    
    %\begin{center}
    %    \includegraphics[width=.2\textwidth]{../imgs/tei-r.pdf}
    %\end{center}
    %\textit{In parte già disponibili nei moduli TEI di base}

     \begin{block}{Esempio Elemento \texttt{<xsl:if>}}
        
        \texttt{<xsl:if test="@n='23'" >...</xsl:if>}
        \\\texttt{<xsl:if test="title[@level='m']" >...</xsl:if>}
        \\\texttt{<xsl:if test="count(verse) > 3" >...</xsl:if>}
        \\\texttt{}

     \end{block}

\end{frame}


%% xsl:choose
\begin{frame}
    \frametitle{Visualizzare ed Elaborare documenti XML}
    \addtocounter{nframe}{1}
    
    %\begin{center}
    %    \includegraphics[width=.2\textwidth]{../imgs/tei-r.pdf}
    %\end{center}
    %\textit{In parte già disponibili nei moduli TEI di base}

     \begin{block}{Elemento \texttt{<xsl:choose>}}
        Permette di definire condizioni multiple tra cui scegliere.
     \end{block}

     \begin{block}{Elemento \texttt{<xsl:choose>}}
        Il content model di \textit{choose} prevede uno o più elementi \texttt{<xsl:when>} e opzionalmente l'elemento \texttt{<xsl:otherwise>}.
     \end{block}

\end{frame}

\begin{frame}
    \frametitle{Visualizzare ed Elaborare documenti XML}
    \addtocounter{nframe}{1}
    
    \begin{center}
        \includegraphics[width=.95\textwidth]{imgs/Schema-choose.png}
    \end{center}
    %\textit{In parte già disponibili nei moduli TEI di base}

\end{frame}

% \begin{frame}
%     \frametitle{Visualizzare ed Elaborare documenti XML}
%     \addtocounter{nframe}{1}
    
%     %\begin{center}
%     %    \includegraphics[width=.2\textwidth]{../imgs/tei-r.pdf}
%     %\end{center}
%     %\textit{In parte già disponibili nei moduli TEI di base}

%      \begin{block}{Attributi Elemento \texttt{<xsl:if>}}
%     %     \emph{Per la critica testuale indispensabili i moduli}
%          \begin{itemize}
%              \item \textbf{test}: l’espressione di test. Se restituisce true, il contenuto di \texttt{<xsl:if>} viene valutato e inserito nell’albero di output; altrimenti viene ignorato
%         \end{itemize}
%      \end{block}
    
% \end{frame}

\begin{frame}
    \frametitle{Visualizzare ed Elaborare documenti XML}
    \addtocounter{nframe}{1}
    
    %\begin{center}
    %    \includegraphics[width=.2\textwidth]{../imgs/tei-r.pdf}
    %\end{center}
    %\textit{In parte già disponibili nei moduli TEI di base}

     \begin{block}{Esempio Elemento \texttt{<xsl:choose>}}
        
        \texttt{<xsl:template match="tei:title" >}
        \\\texttt{<xsl:choose>}
        \\\texttt{<xsl:when test="@level = 'm' or @level = 'u'" >}
        \\\texttt{<i><xsl:apply-templates/>. </i> </xsl:when>}
        \\\texttt{<xsl:when test="@level = 'j'" >}
        \\\texttt{<i><xsl:apply-templates/></i>}
        \\\texttt{</xsl:when>}
        \\\texttt{<xsl:otherwise>}
        \\\texttt{<i><xsl:apply-templates/></i></xsl:otherwise>}
        \\\texttt{</xsl:choose></xsl:template>}
     \end{block}

\end{frame}

% xsl:sort

\begin{frame}
    \frametitle{Visualizzare ed Elaborare documenti XML}
    \addtocounter{nframe}{1}
    
    %\begin{center}
    %    \includegraphics[width=.2\textwidth]{../imgs/tei-r.pdf}
    %\end{center}
    %\textit{In parte già disponibili nei moduli TEI di base}

     \begin{block}{Elemento \texttt{<xsl:sort>}}
        Permette di riorganizzare l’ordine in cui vengono scritti i nodi nell’albero di output.
     \end{block}

     \begin{block}{Elemento \texttt{<xsl:sort>}}
        Deve comparire all’interno di un’istruzione \texttt{<xsl:apply-templates>} o \texttt{<xsl:for-each>}.
     \end{block}

\end{frame}

\begin{frame}
    \frametitle{Visualizzare ed Elaborare documenti XML}
    \addtocounter{nframe}{1}
    
    \begin{center}
        \includegraphics[width=.95\textwidth]{imgs/Schema-sort.png}
    \end{center}
    %\textit{In parte già disponibili nei moduli TEI di base}

\end{frame}

\begin{frame}
    \frametitle{Visualizzare ed Elaborare documenti XML}
    \addtocounter{nframe}{1}
    
    %\begin{center}
    %    \includegraphics[width=.2\textwidth]{../imgs/tei-r.pdf}
    %\end{center}
    %\textit{In parte già disponibili nei moduli TEI di base}

     \begin{block}{Attributi Elemento \texttt{<xsl:sort>}}
    %     \emph{Per la critica testuale indispensabili i moduli}
         \begin{itemize}
             \item \textbf{select}: espressione XPath per individuare gli elementi in base ai quali effettuare l’ordinamento.
             \item \textbf{lang}: linguaggio utilizzato per l’ordinamento.
             \item \textbf{data-type}: il tipo degli elementi rispetto ai quali stiamo effettuando l’ordinamento.
             \item \textbf{order}: ordinamento crescente (ascending) o discendente (descending)
             \item \textbf{case-order}: indica se dare precedenza ai caratteri minuscoli (lower-first) o maiuscoli (upper-first)
        \end{itemize}
     \end{block}
    
\end{frame}

\begin{frame}
    \frametitle{Visualizzare ed Elaborare documenti XML}
    \addtocounter{nframe}{1}
    
    %\begin{center}
    %    \includegraphics[width=.2\textwidth]{../imgs/tei-r.pdf}
    %\end{center}
    %\textit{In parte già disponibili nei moduli TEI di base}

     \begin{block}{Esempio Elemento \texttt{<xsl:sort>}}
        
        \texttt{<div><ul><xsl:for-each select="TEI/text/body/div" >}
        \\\texttt{<xsl:sort select="@n" data-type="number" order="descending" />}
        \\\texttt{<li><xsl:value-of select="@n" />}
        \\\texttt{<xsl:text>|</xsl:text>}
        \\\texttt{<xsl:value-of select="current()" /></li>}
        \\\texttt{</xsl:for-each></ul></div>}

     \end{block}
\end{frame}

%% xsl:variable

\begin{frame}
    \frametitle{Visualizzare ed Elaborare documenti XML}
    \addtocounter{nframe}{1}
    
    %\begin{center}
    %    \includegraphics[width=.2\textwidth]{../imgs/tei-r.pdf}
    %\end{center}
    %\textit{In parte già disponibili nei moduli TEI di base}

     \begin{block}{Elemento \texttt{<xsl:variable>}}
        Permette di definire una variabile, ovvero una posizione di memorizzazione denominata in un modo personalizzato, che contiene i risultati di una espressione valutata a runtime.
     \end{block}

     \begin{block}{Elemento \texttt{<xsl:variable>}}
        L’accesso ad una variabile avviene anteponendo il carattere \$ al nome della variabile (es \$unaVariabile).
     \end{block}

\end{frame}

\begin{frame}
    \frametitle{Visualizzare ed Elaborare documenti XML}
    \addtocounter{nframe}{1}
    
    \begin{center}
        \includegraphics[width=.95\textwidth]{imgs/Schema-variable.png}
    \end{center}
    %\textit{In parte già disponibili nei moduli TEI di base}

\end{frame}

\begin{frame}
    \frametitle{Visualizzare ed Elaborare documenti XML}
    \addtocounter{nframe}{1}
    
    %\begin{center}
    %    \includegraphics[width=.2\textwidth]{../imgs/tei-r.pdf}
    %\end{center}
    %\textit{In parte già disponibili nei moduli TEI di base}

     \begin{block}{Attributi Elemento \texttt{<xsl:variable>}}
    %     \emph{Per la critica testuale indispensabili i moduli}
         \begin{itemize}
             \item \textbf{name}: nome della variabile.
             \item \textbf{select}: seleziona il contenuto della variabile, se presente; altrimenti come contenuto viene usato il contenuto dell’istruzione 
        \end{itemize}
     \end{block}
    
\end{frame}

\begin{frame}
    \frametitle{Visualizzare ed Elaborare documenti XML}
    \addtocounter{nframe}{1}
    
    %\begin{center}
    %    \includegraphics[width=.2\textwidth]{../imgs/tei-r.pdf}
    %\end{center}
    %\textit{In parte già disponibili nei moduli TEI di base}
    \textbf{La definizione di variabili può assumere tre distinte forme:}

     \begin{block}{Esempio Elemento \texttt{<xsl:variable>}}
        \begin{itemize}
            \item creazione di una variabile il cui valore è una stringa vuota
            \item creazione di una variabile avente valore definito dall'attributo select
            \item creazione mediante inclusione di contenuto nel corpo dell'elemento
        \end{itemize}
     \end{block}
\end{frame}

\begin{frame}
    \frametitle{Visualizzare ed Elaborare documenti XML}
    \addtocounter{nframe}{1}
    
    %\begin{center}
    %    \includegraphics[width=.2\textwidth]{../imgs/tei-r.pdf}
    %\end{center}
    %\textit{In parte già disponibili nei moduli TEI di base}
    \textbf{La definizione di variabili può assumere tre distinte forme:}

     \begin{block}{Esempio Elemento \texttt{<xsl:variable>}}
        \begin{itemize}
            \item \texttt{<xsl:variable name="myVar" />}
            \item \texttt{<xsl:variable name="myVar" select="150" />}
            \item \texttt{<xsl:variable name="myVar" >}
            \item[] \texttt{<xs:value-of="@n"/>}
            \item[] \texttt{</xsl:variable>}
        \end{itemize}

     \end{block}
\end{frame}

%% xsl:param

\begin{frame}
    \frametitle{Visualizzare ed Elaborare documenti XML}
    \addtocounter{nframe}{1}
    
    %\begin{center}
    %    \includegraphics[width=.2\textwidth]{../imgs/tei-r.pdf}
    %\end{center}
    %\textit{In parte già disponibili nei moduli TEI di base}

     \begin{block}{Elemento \texttt{<xsl:param>}}
        E' simile ad una \texttt{<xsl:variable>}, ma il suo valore può essere modificato in base al modo in cui il template viene chiamato o dal foglio di stile stesso.
     \end{block}

     \begin{block}{Elemento \texttt{<xsl:param>}}
        Può essere inserita come primo figlio di un \texttt{<xsl:template>}
     \end{block}

\end{frame}

\begin{frame}
    \frametitle{Visualizzare ed Elaborare documenti XML}
    \addtocounter{nframe}{1}
    
    \begin{center}
        \includegraphics[width=.95\textwidth]{imgs/Schema-param.png}
    \end{center}
    %\textit{In parte già disponibili nei moduli TEI di base}

\end{frame}

\begin{frame}
    \frametitle{Visualizzare ed Elaborare documenti XML}
    \addtocounter{nframe}{1}
    
    %\begin{center}
    %    \includegraphics[width=.2\textwidth]{../imgs/tei-r.pdf}
    %\end{center}
    %\textit{In parte già disponibili nei moduli TEI di base}

     \begin{block}{Attributi Elemento \texttt{<xsl:param>}}
    %     \emph{Per la critica testuale indispensabili i moduli}
         \begin{itemize}
             \item \textbf{name}: nome della variabile.
             \item \textbf{select}: seleziona il contenuto della variabile, se presente; altrimenti come contenuto viene usato il contenuto dell’istruzione stessa
        \end{itemize}
     \end{block}
    
\end{frame}

\begin{frame}
    \frametitle{Visualizzare ed Elaborare documenti XML}
    \addtocounter{nframe}{1}
    
    %\begin{center}
    %    \includegraphics[width=.2\textwidth]{../imgs/tei-r.pdf}
    %\end{center}
    %\textit{In parte già disponibili nei moduli TEI di base}

     \begin{block}{Esempio Elemento \texttt{<xsl:param>}}
        
        \texttt{<xsl:template name="body" >}
        \\\texttt{<xsl:param name="style" >color:blue</xsl:param>}
        \\\texttt{<div><xsl:attribute name="style" >}
        \\\texttt{<xsl:value-of select="\$style" />}
        \\\texttt{</xsl:attribute>}
        \\\texttt{<xsl:value-of select="." /></div>}
        \\\texttt{</xsl:template>}

     \end{block}
\end{frame}

%% xsl:call-template
\begin{frame}
    \frametitle{Visualizzare ed Elaborare documenti XML}
    \addtocounter{nframe}{1}
    
    %\begin{center}
    %    \includegraphics[width=.2\textwidth]{../imgs/tei-r.pdf}
    %\end{center}
    %\textit{In parte già disponibili nei moduli TEI di base}

     \begin{block}{Elemento \texttt{<xsl:call-template>}}
        Dopo aver assegnato un nome ad un template, è possibile richiamarlo con l’istruzione \texttt{<xsl:call-template>}.
     \end{block}

     \begin{block}{Elemento \texttt{<xsl:call-template>}}
        Per invocare un template passando dei parametri è possibile utilizzare l’elemento \texttt{<xsl:with-param>} nel corpo dell’elemento \texttt{<xsl:call-template>} o \texttt{<xsl:apply-templates>} indicando il nome del parametro ed il valore.
     \end{block}

\end{frame}

\begin{frame}
    \frametitle{Visualizzare ed Elaborare documenti XML}
    \addtocounter{nframe}{1}
    
    \begin{center}
        \includegraphics[width=.95\textwidth]{imgs/Schema-call-template.png}
    \end{center}
    %\textit{In parte già disponibili nei moduli TEI di base}

\end{frame}

\begin{frame}
    \frametitle{Visualizzare ed Elaborare documenti XML}
    \addtocounter{nframe}{1}
    
    %\begin{center}
    %    \includegraphics[width=.2\textwidth]{../imgs/tei-r.pdf}
    %\end{center}
    %\textit{In parte già disponibili nei moduli TEI di base}

     \begin{block}{Attributi Elemento \texttt{<xsl:call-template>}}
    %     \emph{Per la critica testuale indispensabili i moduli}
         \begin{itemize}
             \item \textbf{name}: il nome del template da richiamare. Il foglio di stile deve necessariamente contenere un \texttt{<xsl:template>} con tale nome specificato
        \end{itemize}
     \end{block}
    
\end{frame}

\begin{frame}
    \frametitle{Visualizzare ed Elaborare documenti XML}
    \addtocounter{nframe}{1}
    
    %\begin{center}
    %    \includegraphics[width=.2\textwidth]{../imgs/tei-r.pdf}
    %\end{center}
    %\textit{In parte già disponibili nei moduli TEI di base}

     \begin{block}{Esempio Elemento \texttt{<xsl:call-template>}}
        
        \texttt{<body>}
        \\\texttt{<xsl:call-template name="body" >}
        \\\texttt{<xsl:with-param name="style" >}
        \\\texttt{color:red </xsl:with-param>}
        \\\texttt{</xsl:call-template></body>}

     \end{block}
\end{frame}


%% xsl:element
\begin{frame}
    \frametitle{Visualizzare ed Elaborare documenti XML}
    \addtocounter{nframe}{1}
    
    %\begin{center}
    %    \includegraphics[width=.2\textwidth]{../imgs/tei-r.pdf}
    %\end{center}
    %\textit{In parte già disponibili nei moduli TEI di base}

     \begin{block}{Elemento \texttt{<xsl:element>}}
        Per creare elementi è possibile utilizzare l'istruzione \texttt{<xsl:element>}.
     \end{block}

\end{frame}

\begin{frame}
    \frametitle{Visualizzare ed Elaborare documenti XML}
    \addtocounter{nframe}{1}
    
    \begin{center}
        \includegraphics[width=.95\textwidth]{imgs/Schema-element.png}
    \end{center}
    %\textit{In parte già disponibili nei moduli TEI di base}

\end{frame}

\begin{frame}
    \frametitle{Visualizzare ed Elaborare documenti XML}
    \addtocounter{nframe}{1}
    
    \begin{center}
        \includegraphics[width=.95\textwidth]{imgs/element-example.png}
    \end{center}
    %\textit{In parte già disponibili nei moduli TEI di base}

\end{frame}


%% xsl:attribute

\begin{frame}
    \frametitle{Visualizzare ed Elaborare documenti XML}
    \addtocounter{nframe}{1}
    
    %\begin{center}
    %    \includegraphics[width=.2\textwidth]{../imgs/tei-r.pdf}
    %\end{center}
    %\textit{In parte già disponibili nei moduli TEI di base}

     \begin{block}{Elemento \texttt{<xsl:attribute>}}
        E' possibile creare attributi utilizzando l’elemento \texttt{<xsl:attribute>} ed indicando negli attributi \textit{name} e \textit{namespace} (opzionale) il nome e il namespace di appartenenza dell’attributo.
     \end{block}

\end{frame}

\begin{frame}
    \frametitle{Visualizzare ed Elaborare documenti XML}
    \addtocounter{nframe}{1}
    
    \begin{center}
        \includegraphics[width=.95\textwidth]{imgs/Schema-attributo.png}
    \end{center}
    %\textit{In parte già disponibili nei moduli TEI di base}

\end{frame}


\begin{frame}
    \frametitle{Visualizzare ed Elaborare documenti XML}
    \addtocounter{nframe}{1}
    
    %\begin{center}
    %    \includegraphics[width=.2\textwidth]{../imgs/tei-r.pdf}
    %\end{center}
    %\textit{In parte già disponibili nei moduli TEI di base}

     \begin{block}{Esempio Elemento \texttt{<xsl:attribute>}}
        
        \texttt{<xsl:element name="div" >}
        \\\texttt{<xsl:attribute name="id" >}
        \\\texttt{<xsl:value-of select="@id"/>}
        \\\texttt{</xsl:attribute>}
        \\\texttt{</xsl:element>}

     \end{block}
\end{frame}


%% xsl:comment
\begin{frame}
    \frametitle{Visualizzare ed Elaborare documenti XML}
    \addtocounter{nframe}{1}
    
    %\begin{center}
    %    \includegraphics[width=.2\textwidth]{../imgs/tei-r.pdf}
    %\end{center}
    %\textit{In parte già disponibili nei moduli TEI di base}

     \begin{block}{Altri Elemento di creazione}
        \begin{itemize}
            \item \textbf{commenti}: mediante \texttt{<xsl:comment>} specificando fra i tag di apertura e chiusura il testo del commento
            \item \textbf{processing instruction}: si utilizza \texttt{<xsl:processing-instruction>} specificando mediante l’attributo name il nome ed inserendone il contenuto tra i tag di apertura e chiusura
            \item \textbf{testo}: si usa \texttt{<xsl:text>} specificando nel corpo il contenuto della sezione CDATA.
        \end{itemize}
     \end{block}

\end{frame}

% spazi bianchi
\begin{frame}
    \frametitle{Visualizzare ed Elaborare documenti XML}
    \addtocounter{nframe}{1}
    
    %\begin{center}
    %    \includegraphics[width=.2\textwidth]{../imgs/tei-r.pdf}
    %\end{center}
    %\textit{In parte già disponibili nei moduli TEI di base}

     \begin{block}{Gestione spazi bianchi}
        La gestione degli spazi bianchi nel documento di origine è specificato dalle regole di scarto attraverso le istruzioni \textbf{xsl:preserve-space} e \textbf{xsl:strip-space}.
     \end{block}

\end{frame}


%% xsl:preserve-space
\begin{frame}
    \frametitle{Visualizzare ed Elaborare documenti XML}
    \addtocounter{nframe}{1}
    
    %\begin{center}
    %    \includegraphics[width=.2\textwidth]{../imgs/tei-r.pdf}
    %\end{center}
    %\textit{In parte già disponibili nei moduli TEI di base}

     \begin{block}{Elemento \texttt{<xsl:preserve-space>}}
         Elenca gli elementi dell'albero di origine in cui devono essere conservati gli spazi bianchi originali.
     \end{block}

     \begin{block}{Esempio \texttt{<xsl:preserve-space>}}
        \texttt{<xsl:preserve-space elements="p head"/>}
    \end{block}

\end{frame}

\begin{frame}
    \frametitle{Visualizzare ed Elaborare documenti XML}
    \addtocounter{nframe}{1}
    
    \begin{center}
        \includegraphics[width=.95\textwidth]{imgs/Schema-preserve-space.png}
    \end{center}

\end{frame}

%%  xsl:strip-space

\begin{frame}
    \frametitle{Visualizzare ed Elaborare documenti XML}
    \addtocounter{nframe}{1}
    
    %\begin{center}
    %    \includegraphics[width=.2\textwidth]{../imgs/tei-r.pdf}
    %\end{center}
    %\textit{In parte già disponibili nei moduli TEI di base}

     \begin{block}{Elemento \texttt{<xsl:strip-space>}}
        Elenca gli elementi dell'albero di origine in cui devono essere scartati gli spazi bianchi originali
     \end{block}

     \begin{block}{Esempio \texttt{<xsl:strip-space>}}
       \texttt{ <xsl:strip-space elements = "*"/>}
     \end{block}

\end{frame}

\begin{frame}
    \frametitle{Visualizzare ed Elaborare documenti XML}
    \addtocounter{nframe}{1}
    
    \begin{center}
        \includegraphics[width=.95\textwidth]{imgs/Schema-strip-space.png}
    \end{center}

\end{frame}

\begin{frame}
    \frametitle{Visualizzare ed Elaborare documenti XML}
    \addtocounter{nframe}{1}
    
    \begin{block}{Esercizio}
        Modificare opportunamente il file template.xsl aggiungendo variabili, parametri e call template.
    \end{block}

\end{frame}



    
    \section{XPath: selezione dei nodi ed espression axes}
    % \begin{frame}
%     \frametitle{Visualizzare ed Elaborare documenti XML}
%     \addtocounter{nframe}{1}
    
%     %\begin{center}
%     %    \includegraphics[width=.2\textwidth]{../imgs/tei-r.pdf}
%     %\end{center}
%     %\textit{In parte già disponibili nei moduli TEI di base}

%      \begin{block}{Perché visualizzare il testo}
%     %     \emph{Per la critica testuale indispensabili i moduli}
%          \begin{itemize}
%             \item  Controllare la codifica e correggere i refusi
%              \item Assicurarsi che tutto sia stato trascritto correttamente
%              \item Mostrare il testo a persone che non conoscono XML-TEI
%              \item Disporre di una versione del lavoro fuibile
%         \end{itemize}
%      \end{block}
    
% \end{frame}

% \begin{frame}
%     \frametitle{Visualizzare ed Elaborare documenti XML}
%     \addtocounter{nframe}{1}
    
%     \begin{center}
%         \includegraphics[width=.9\textwidth]{imgs/SchemaXSLTprocessing.png}
%     \end{center}
%     %\textit{In parte già disponibili nei moduli TEI di base}

% \end{frame}

\begin{frame}
    \frametitle{Visualizzare ed Elaborare documenti XML}
    \addtocounter{nframe}{1}
    
    %\begin{center}
    %    \includegraphics[width=.2\textwidth]{../imgs/tei-r.pdf}
    %\end{center}
    %\textit{In parte già disponibili nei moduli TEI di base}

    \begin{block}{XPath}
        
        XPath è un \textit{expression language} fondamentale per realizzare fogli di stile XSLT.
        
    \end{block}
     
    \begin{block}{XPath}
        \begin{itemize}
            \item Selezionare nodi in un documento XML
            \item Fare match nell'albero source per selezionare il corretto template
            \item Manipolare dati attraverso funzioni
        \end{itemize}
    \end{block}
    
\end{frame}

\begin{frame}
    \frametitle{Visualizzare ed Elaborare documenti XML}
    \addtocounter{nframe}{1}
    
    %\begin{center}
    %    \includegraphics[width=.2\textwidth]{../imgs/tei-r.pdf}
    %\end{center}
    %\textit{In parte già disponibili nei moduli TEI di base}

    \begin{block}{XPath}
        XPath offre una sintassi estesa (piuttosto verbosa) e una sintassi abbreviata.
    \end{block}

    \begin{block}{XPath}
        Le espressioni XPath permettono di selezionare con grande precisione elementi, attributi, ecc.
    \end{block}
    
\end{frame}

\begin{frame}
    \frametitle{Visualizzare ed Elaborare documenti XML}
    \addtocounter{nframe}{1}
    
    %\begin{center}
    %    \includegraphics[width=.2\textwidth]{../imgs/tei-r.pdf}
    %\end{center}
    %\textit{In parte già disponibili nei moduli TEI di base}

    \begin{block}{Esempi XPath selezione con match}
        \emph{Selezionare \textit{documento XML} oppure tutti i nodi \texttt{<quote>}}
        \begin{itemize}
            \item \texttt{<xsl:template match="/" >}
            \item \texttt{<xsl:template match="quote" >}
        \end{itemize}
        
    \end{block}
     
    \begin{block}{Esempi XPath selezione con match}
        \emph{Per selezionare i nodi \texttt{<title>} "figli", "nipoti" o comunque discendenti di \texttt{<quote>}}
        \begin{itemize}
            \item \texttt{match="quote/title"} \textit{(discendente diretto - figlio)}
            \item \texttt{match="quote/*/title"} \textit{(discendente di secondo livello - nipote)}
            \item \texttt{match="quote//title"} \textit{(discendente a qualsiasi livello)}
        \end{itemize}     
    \end{block}
    
\end{frame}

\begin{frame}
    \frametitle{Visualizzare ed Elaborare documenti XML}
    \addtocounter{nframe}{1}
    
    %\begin{center}
    %    \includegraphics[width=.2\textwidth]{../imgs/tei-r.pdf}
    %\end{center}
    %\textit{In parte già disponibili nei moduli TEI di base}

    \begin{block}{Esempi XPath selezione con match}
        \emph{Per selezionare i nodi \texttt{<quote>} con attributo \texttt{@type} (e valore \textit{book})}
        \begin{itemize}
            \item \texttt{match="quote[@type]"}
            \item \texttt{match="quote[@type='book']"}
        \end{itemize}
        
    \end{block}
     
    \begin{block}{Esempi XPath selezione con match}
        \emph{Per selezionare i nodi usando il carattere "jolly"}
        \begin{itemize}
            \item \texttt{match="*"}
            \item \texttt{match="*[@type]"}
            \item \texttt{match="*[@type='book']"}
        \end{itemize}
    \end{block}
    
\end{frame}


\begin{frame}
    \frametitle{Visualizzare ed Elaborare documenti XML}
    \addtocounter{nframe}{1}
    
    %\begin{center}
    %    \includegraphics[width=.2\textwidth]{../imgs/tei-r.pdf}
    %\end{center}
    %\textit{In parte già disponibili nei moduli TEI di base}

    \begin{block}{Esempi XPath selezione con match}
        \emph{Per selezionare più di un elemento (operatore OR) oppure per selezionare un nodo con uno specifico ID}
        \begin{itemize}
            \item \texttt{match="title | author"}
            \item \texttt{match="id('stella_2007’)"}
        \end{itemize}
        
    \end{block}
    
\end{frame}

% selezione nodi con select

\begin{frame}
    \frametitle{Visualizzare ed Elaborare documenti XML}
    \addtocounter{nframe}{1}
    
    %\begin{center}
    %    \includegraphics[width=.2\textwidth]{../imgs/tei-r.pdf}
    %\end{center}
    %\textit{In parte già disponibili nei moduli TEI di base}

    \begin{block}{XPath selezione con select}
        Il valore di questo attributo è un'espressione conforme al linguaggio XPath
    \end{block}

    \begin{block}{XPath selezione con select}
        L'attributo select può ricorrere a una sintassi più complessa rispetto a match
    \end{block}
    
\end{frame}

\begin{frame}
    \frametitle{Visualizzare ed Elaborare documenti XML}
    \addtocounter{nframe}{1}
    
    %\begin{center}
    %    \includegraphics[width=.2\textwidth]{../imgs/tei-r.pdf}
    %\end{center}
    %\textit{In parte già disponibili nei moduli TEI di base}

    \begin{block}{L’attributo select è usato con le istruzioni XSLT}
        \begin{itemize}
            \item xsl:apply-templates
            \item xsl:value-of
            \item xsl:copy-of
            \item xsl:for-each
            \item xsl:sort
            \item xsl:variable
            \item xsl:param
        \end{itemize}
    \end{block}

\end{frame}

\begin{frame}
    \frametitle{Visualizzare ed Elaborare documenti XML}
    \addtocounter{nframe}{1}
    
    %\begin{center}
    %    \includegraphics[width=.2\textwidth]{../imgs/tei-r.pdf}
    %\end{center}
    %\textit{In parte già disponibili nei moduli TEI di base}

    \begin{block}{XPath}
        Le espressioni XPath permettono di "navigare" l'albero del documento XML usando assi di navigazione (\textit{expression axes}).
    \end{block}

    \begin{block}{XPath}
        La selezione può essere assoluta o relativa al \textbf{nodo corrente} e si compone di tre parti: (\textbf{Assi}, \textbf{Test}, \textbf{Predicato})
    \end{block}
    
\end{frame}

\begin{frame}
    \frametitle{Visualizzare ed Elaborare documenti XML}
    \addtocounter{nframe}{1}
    
    \begin{center}
        \includegraphics[width=.9\textwidth]{imgs/Schema-Assi-1.png}
    \end{center}

\end{frame}

\begin{frame}
    \frametitle{Visualizzare ed Elaborare documenti XML}
    \addtocounter{nframe}{1}
    
    \begin{center}
        \includegraphics[width=.9\textwidth]{imgs/Schema-Assi-2.png}
    \end{center}

\end{frame}

\begin{frame}
    \frametitle{Visualizzare ed Elaborare documenti XML}
    \addtocounter{nframe}{1}
    
    \begin{center}
        \includegraphics[width=.9\textwidth]{imgs/SchemaAssi-Xpath.png}
    \end{center}

\end{frame}

\begin{frame}
    \frametitle{Visualizzare ed Elaborare documenti XML}
    \addtocounter{nframe}{1}
    
    \begin{center}
        \includegraphics[width=.9\textwidth]{imgs/Sintassi-Abbreviata.png}
    \end{center}

\end{frame}


\begin{frame}
    \frametitle{Visualizzare ed Elaborare documenti XML}
    \addtocounter{nframe}{1}
    
    \begin{center}
        \includegraphics[width=.9\textwidth]{imgs/Sintassi-Abbreviata-Estesa.png}
    \end{center}

\end{frame}


\begin{frame}
    \frametitle{Visualizzare ed Elaborare documenti XML}
    \addtocounter{nframe}{1}
    
    \begin{center}
        \includegraphics[width=.9\textwidth]{imgs/Tab-Operatori-Predicato.png}
    \end{center}

\end{frame}

\begin{frame}
    \frametitle{Visualizzare ed Elaborare documenti XML}
    \addtocounter{nframe}{1}
    
    %\begin{center}
    %    \includegraphics[width=.2\textwidth]{../imgs/tei-r.pdf}
    %\end{center}
    %\textit{In parte già disponibili nei moduli TEI di base}

    \begin{block}{Esempio sintassi estesa - sintassi abbreviata}
        \begin{itemize}
            \item \texttt{<xsl:value-of select="child::author"/>}
            \item \texttt{<xsl:value-of select="author"/>}
        \end{itemize}
    \end{block}

\end{frame}

\begin{frame}
    \frametitle{Visualizzare ed Elaborare documenti XML}
    \addtocounter{nframe}{1}
    
    %\begin{center}
    %    \includegraphics[width=.2\textwidth]{../imgs/tei-r.pdf}
    %\end{center}
    %\textit{In parte già disponibili nei moduli TEI di base}

    \begin{block}{Esempio sintassi estesa - sintassi abbreviata}
        \begin{itemize}
            \item \texttt{<xsl:value-of select="parent::quote"/>}
            \item \texttt{<xsl:value-of select="ancestor::quote"/>}
            \item \texttt{<xsl:value-of select=".."/>}
        \end{itemize}
    \end{block}

\end{frame}

\begin{frame}
    \frametitle{Visualizzare ed Elaborare documenti XML}
    \addtocounter{nframe}{1}
    
    %\begin{center}
    %    \includegraphics[width=.2\textwidth]{../imgs/tei-r.pdf}
    %\end{center}
    %\textit{In parte già disponibili nei moduli TEI di base}

    \begin{block}{Esempio predicati}
        \begin{itemize}
            \item \texttt{//div[@type='chapter']}
            \item \texttt{//div[@type!='chapter']}
            \item \texttt{//div[@n > 2]}
            \item \texttt{//div[1]}
            \item \texttt{//div[last()]}
            \item \texttt{//div[position() = last() - 1]}
            \item \texttt{//div[position() mod 2 = 0]}
        \end{itemize}
    \end{block}

\end{frame}


    
    %\section{Personalizzare TEI}
    %\begin{frame}
	\frametitle{Intro Text Encoding Initiative}
	\framesubtitle{Schemi di codifica TEI – Personalizzare TEI}
	\addtocounter{nframe}{1}

    \begin{block}{Personalizzare TEI}
        Nessun progetto di codifica richiede di utilizzare tutte le specifiche definite dalle linee guida della TEI.
    \end{block}
    \begin{block}{Personalizzare TEI}
        
        La TEI fornisce un insieme specifico di elementi che può essere usato per creare uno schema TEI puntuale e ritagliato sulle specifiche del progetto di codifica in corso.

    \end{block}
    
\end{frame}

% \begin{frame}
% 	\frametitle{Intro Text Encoding Initiative}
% 	\framesubtitle{Schemi di codifica TEI – Personalizzare TEI}
% 	\addtocounter{nframe}{1}

%     \begin{block}{Personalizzare TEI}
%         % The TEI provides a special set of elements which can be used to create such a schema specification.
%         La TEI fornisce un insieme specifico di elementi che può essere usato per creare uno schema TEI puntuale e ritagliato sulle specifiche del progetto di codifica in corso.

%     \end{block}
    
% \end{frame}



\begin{frame}
	\frametitle{Intro Text Encoding Initiative}
	\framesubtitle{Schemi di codifica TEI – Personalizzare TEI}
    \addtocounter{nframe}{1}
    
    \begin{block}{Personalizzare TEI - schemaSpec}
        \texttt{<schemaSpec ident="tei-custom" start="TEI teiCorpus" >}
        \\\texttt{<moduleRef key="analysis" include="interp interpGrp pc s w" />}
        \\\texttt{<moduleRef key="linking" include="anchor seg" />}
        %\texttt{<moduleRef key="tagdocs" include="att code eg gi ident val "/> }
        \\\texttt{<moduleRef key="tei "/> }
        %\texttt{<moduleRef key="textstructure" include="TEI argument back body byline closer dateline div docAuthor docDate docEdition docImprint docTitle epigraph front group imprimatur opener postscript salute signed text titlePage titlePart trailer "/> }
        \\\texttt{</schemaSpec>}
    \end{block}
    \textit{Si selezionano o si escudono gli elementi attraverso gli attributi @\textbf{include} e @\textbf{exclude}}
\end{frame}

\begin{frame}
	\frametitle{Intro Text Encoding Initiative}
	\framesubtitle{Schemi di codifica TEI – Personalizzare TEI}
	\addtocounter{nframe}{1}
    \textit{Modifica agli elementi e attributi dei Moduli}
    \begin{block}{Personalizzare TEI - classSpec}
        \texttt{<classSpec type="atts" ident="att.datable.w3c" module="tei" mode="change" >}
        \\\texttt{<attList> }
        \\\texttt{<attDef ident="notAfter" mode="delete" />}
        \\\texttt{<attDef ident="from" mode="delete" />}
        \\\texttt{<attDef ident="to" mode="delete" /> }
        \\\texttt{</attList>}
        \\\texttt{</classSpec>}
    \end{block}
    
\end{frame}



% ODD document, Selezione dei Moduli per lo schema, nuovi elementi, profilo personalizzato TEI, capitolo 22 delle linee guida (Documentation Elements), capitolo 23 delle linee guida (Using the TEI)


% serious use of the TEI requires careful consideration of exactly which of its elements is appropriate to the of things which the project needs to specify more exactly than the TEI does.


% a document using these elements is provides information for a computer to process along with documentation of that information for a human being to read in a single integrated XML document.

% tei_ all tei_lite epidoc

%% esempio da Exemplars tei_lite.odd
% A quick glance at the XML source code for the TEI Lite ODD shows that it appears to be a typical TEI document

% <schemaSpec ident="tei_lite" start="TEI teiCorpus">
%    <moduleRef key="analysis" include="interp interpGrp pc s w" />
%    <moduleRef key="linking" include="anchor seg" />
%    <moduleRef key=" tagdocs " include="att code eg gi ident val "/> 
%    <moduleRef key="tei "/> 
%    <moduleRef key="textstructure " include="TEI argument back body byline closer dateline div docAuthor docDate docEdition docImprint docTitle epigraph front group imprimatur opener postscript salute signed text titlePage titlePart trailer "/> 
% </schemaSpec>


% TEI currently defines 22 modules
% foto con la tabella

% definizione degli elementi, degli attributi, dei datatype e dei valori di default 




%% esempio epidoc

% <elementSpec   ident="div"   mode="change"   module="textstructure">
%    <attList>
%        <attDef ident="type"     mode="replace"     usage="req">
%            <valList type="closed">
%                <valItem ident="apparatus">
%                    <desc>to contain apparatus criticus or textual notes</desc>
%                    </valItem>
%                    <valItem ident="bibliography">
%                        <desc>to contain bibliographical information, previous publications,            etc.
%                        </desc>
%                        </valItem>
%                        <valItem ident="commentary">
%                            <desc>to contain all editorial commentary, historical/prosopographical            discussion, etc.</desc>
%                            </valItem>
%                            <valItem ident="edition">
%                                <desc>to contain the text of the edition itself; may include multiple            text-parts
%                                </desc>
%                                </valItem>
%                                <valItem ident="textpart">
%                                    <desc>used to divide a div[type=edition] into multiple parts (fragments,            columns,
%                                        faces, etc.)</desc>
%                                    </valItem>
%                                    <valItem ident="translation">
%                                        <desc>to contain a translation of the text into one or more modern            languages
%                                        </desc>
%                                        </valItem>
%                                       </valList>
%        </attDef>
%    </attList>
% </elementSpec>

% esempio aggiunta nuovo elemento: <SpaciesName />

%  Choices can be made explicit in a customized schema, and hence tell us which of the many very different approaches to tagging an individual’s name has been adopted in a given set of documents.


    
    \section{Conclusioni}
    \begin{frame}
    \frametitle{Visualizzare ed Elaborare documenti XML}
    \addtocounter{nframe}{1}
    
    %\begin{center}
    %    \includegraphics[width=.2\textwidth]{../imgs/tei-r.pdf}
    %\end{center}
    %\textit{In parte già disponibili nei moduli TEI di base}

     \begin{block}{Perché visualizzare il testo}
    %     \emph{Per la critica testuale indispensabili i moduli}
         \begin{itemize}
            \item  Controllare la codifica e correggere i refusi
             \item Assicurarsi che tutto sia stato trascritto correttamente
             \item Mostrare il testo a persone che non conoscono XML-TEI
             \item Disporre di una versione del lavoro fuibile
        \end{itemize}
     \end{block}
    
\end{frame}

\begin{frame}
    \frametitle{Visualizzare ed Elaborare documenti XML}
    \addtocounter{nframe}{1}
    
    \begin{center}
        \includegraphics[width=.9\textwidth]{imgs/SchemaXSLTprocessing.png}
    \end{center}
    %\textit{In parte già disponibili nei moduli TEI di base}

\end{frame}

% Controllo e gestione degli spazi bianchi p 54 slide di Chiara
    
    \end{document}