% DOM, Javascript XML, XSL
% capitolo 7 testo e automa di ciotti
% SLIDE Chiara sui fogli di stile
% Capitolo DOM professional Web Dev e XML processing
%

\documentclass{beamer}
    
    %    \usepackage[english]{babel}
        %\usepackage[latin1]{inputenc}
        %\usepackage[T1]{fontenc}
    
    \mode<presentation>{
      \setbeamertemplate{background canvas}[vertical shading]
      \usetheme{Berkeley}
      \useoutertheme{himinfolines}
    }
      
    \usepackage{ucs}
    \usepackage[utf8]{inputenc}
    \usepackage[english,polutonikogreek,italian,UKenglish,british]{babel}
    \usepackage{graphicx}
    \usepackage{colortbl}
    \usepackage{multicol}
    \usepackage{ulem}
    \usepackage{verbatim}
    \usepackage{alltt}
    \usepackage{ccicons}
    \usepackage{MnSymbol,wasysym}
    \usepackage{tikzsymbols}
    \usepackage{textcomp}
    \usepackage{xmpincl}
    
    \usepackage{parskip}
    \setcounter{nframes}{100}
    \setcounter{nframe}{1}
    \setbeamercovered{dynamic}
    \newenvironment{grcenv}{\begin{otherlanguage}{greek}}{\end{otherlanguage}}
    \newcommand{\g}[1]{\textgreek{#1}}
    \definecolor{darkgreen}{rgb}{0,0.5,0}
    \definecolor{darkblue}{rgb}{0,0,0.5}
    \definecolor{grey}{rgb}{0.5,0.5,0.5}
    \setcounter{tocdepth}{5}
    
    \makeatletter
    
    \makeatother
    %\includexmp{LicencesAndLicensing}
    
    %frame00 metadata
        \title{Codifica TEI - Visualizzazione ed Elaborazione XML}
        \author[A.M. Del Grosso]{Angelo Mario Del Grosso \\ \tiny\textit{..}}
        \institute{\texttt{angelo.delgrosso@ilc.cnr.it} \\\textit{CNR-ILC-LicoLab} \\\url{http://licolab.ilc.cnr.it/}}
        \date{Istituto di Linguistica Computazionale ``A. Zampolli'', \today}
        \AtBeginSection[]{
        \begin{frame}<beamer>
        \addtocounter{nframe}{1}
        \footnotesize
        \frametitle{Progress status}
        \tableofcontents[currentsection,hideothersubsections]
        \end{frame}
        }
    
    \begin{document}
    
    \begin{frame}
        \maketitle
    \end{frame}
    
    \begin{frame}
        \frametitle{Sommario della Lezione}
        \tableofcontents
    \end{frame}
    
    \section{Introduzione}
    
    \begin{frame}
        \frametitle{Visualizzare ed Elaborare documenti XML}
        \addtocounter{nframe}{1}
        
        %\begin{center}
        %    \includegraphics[width=.2\textwidth]{../imgs/tei-r.pdf}
        %\end{center}
        %\textit{In parte già disponibili nei moduli TEI di base}

         \begin{block}{Perché visualizzare ed elaborare il testo}
        %     \emph{Per la critica testuale indispensabili i moduli}
             \begin{itemize}
                \item Controllare la codifica e correggere i refusi
                \item Assicurarsi che tutto sia stato trascritto correttamente
                \item Mostrare il testo a persone che non conoscono XML-TEI
                \item Disporre di una versione del lavoro fruibile
                \item Manipolare e analizzare le informazioni codificate
            \end{itemize}
         \end{block}
        
    \end{frame}
    
    \begin{frame}
        \frametitle{Visualizzare ed Elaborare documenti XML}
        \addtocounter{nframe}{1}
        
        \begin{block}{I fogli di stile (style sheet)}
           \begin{itemize}
               \item Descrive il modo in cui un documento elettronico deve essere visualizzato
               \item Il mezzo di rendering può variare: lo schermo di un computer, la stampa, i sintetizzatori vocali, ecc.
           \end{itemize}
        \end{block}
        
    \end{frame}

    \begin{frame}
        \frametitle{Visualizzare ed Elaborare documenti XML}
        \addtocounter{nframe}{1}
        
        \begin{block}{Modello del Documento}
           \begin{itemize}
               \item Descrive il modo in cui un documento elettronico deve essere rappresentato e navigato da agenti software
               \item Indipendenza dal linguaggio e dalla piattaforma
           \end{itemize}
        \end{block}
        
    \end{frame}
    
    \begin{frame}
        \frametitle{Visualizzare ed Elaborare documenti XML}
        \addtocounter{nframe}{1}
        
        \begin{block}{Scopo dei fogli di stile}
           \begin{itemize}
               \item \emph{Separazione forma-contenuto}: \textit{la visualizzazione del documento è un processo indipendente (e successivo)}
               \item \emph{Gestione della resa grafica} per molti documenti contemporaneamente: \textit{massima uniformità dello stile}
               \item \emph{Gestione di mezzi diversi} dal monitor: smartphone, sintetizzatore
               vocale, stampante braille, ecc.
           \end{itemize}
        \end{block}
        
    \end{frame}

    \begin{frame}
        \frametitle{Visualizzare ed Elaborare documenti XML}
        \addtocounter{nframe}{1}
        
        \begin{block}{Scopo di un modello per i documenti}
           \begin{itemize}
               \item \emph{Astrazione}: \textit{la navigazione e la elaborazione sono specificate da uno standard e non cambia nelle sue varie implementazioni}
               \item \emph{Uniformità}: \textit{le interfacce sono comuni a tutti i sistemi che implementano il modello}
               \item \emph{Efficacia ed Efficienza}: il modello identifica oggetti, metodi e proprietà utili alla gestione e al controllo programmatico dei documenti. 
           \end{itemize}
        \end{block}
        
    \end{frame}

    \begin{frame}
        \frametitle{Visualizzare ed Elaborare documenti XML}
        \addtocounter{nframe}{1}
        
        \begin{block}{Metodi e Tecnologie}
            \textbf{Quelli più noti e utilizzati sono standard internazionali definiti dal consorzio W3 \url{(http://www.w3.org/Style/CSS)}.}
           \begin{itemize}
            \item XSL: eXtensible Stylesheet Language
            \item DOM: Document Object Model
           \end{itemize}
        \end{block}
        
    \end{frame}

    % \begin{frame}
    %     \frametitle{Modularità della TEI}
    %     \addtocounter{nframe}{1}
        
    %    % \begin{center}
    %     % \includegraphics[width=.2\textwidth]{../imgs/tei-r.pdf}
    %     % \end{center}
    
    %     \begin{itemize}
            
    %         \item<1-> parleremo del sistema Modulare della TEI
    %             \begin{itemize}
    %                 \item<1-> Moduli
    %                 \item<1-> Classi
    %                 \item<1-> Macro
    %                 \item<1-> Datatype
    %             \end{itemize} 
    %         \item<2-> parleremo degli elementi basilari
    %             \begin{itemize}
    %                 \item<2-> Intestazione TEI (TEIHeader)
    %                 \item<2-> Elementi e attributi presenti in tutti i documenti TEI
    %                 \item<2-> Esempi di codifica
    %             \end{itemize} 
    %     \end{itemize}
        
    % \end{frame}
    
    \section{XSLT: Documenti Multipli e Versione 2.0}
    \begin{frame}
    \frametitle{Visualizzare ed Elaborare documenti XML}
    \addtocounter{nframe}{1}
    
    %\begin{center}
    %    \includegraphics[width=.2\textwidth]{../imgs/tei-r.pdf}
    %\end{center}
    %\textit{In parte già disponibili nei moduli TEI di base}

     \begin{block}{XML Slyle Sheet Transformation: Modularità}
        \begin{itemize}
            \item XSLT consente di importare ed includere fogli XSLT dentro altri documenti XSLT
            \item mediante gli elementi \texttt{<xsl:import>} ed \texttt{<xsl:include>} 
        \end{itemize}
     \end{block}

     \begin{block}{XML Slyle Sheet Transformation}
        \begin{itemize}
            \item \textbf{inclusione}: le regole definite nel documento incluso hanno la stessa priorità delle regole definite nel documento XSLT principale.
            \item \textbf{importazione}: le regole definite nel documento principale hanno una priorità maggiore.
        \end{itemize}

     \end{block}


\end{frame}


\begin{frame}
    \frametitle{Visualizzare ed Elaborare documenti XML}
    \addtocounter{nframe}{1}
    
    \textbf{Dichiarazione XSD dell'elemento xsl:include}

    \begin{center}
        \includegraphics[width=.8\textwidth]{imgs/elementXSL-Include.png}
    \end{center}

\end{frame}

\begin{frame}
    \frametitle{Visualizzare ed Elaborare documenti XML}
    \addtocounter{nframe}{1}
    
    %\begin{center}
    %    \includegraphics[width=.2\textwidth]{../imgs/tei-r.pdf}
    %\end{center}
    %\textit{In parte già disponibili nei moduli TEI di base}

     \begin{block}{XML Slyle Sheet Transformation: xsl:include}
        Dato un foglio esterno \texttt{esterno.xsl} per includerlo nel foglio principale si impiega l'elemento: \texttt{<xsl:include href="esterno.xsl"/>}
     \end{block}

\end{frame}

\begin{frame}
    \frametitle{Visualizzare ed Elaborare documenti XML}
    \addtocounter{nframe}{1}
    
    \textbf{Dichiarazione XSD dell'elemento xsl:import}

    \begin{center}
        \includegraphics[width=.8\textwidth]{imgs/elementXSL-Import.png}
    \end{center}

\end{frame}

\begin{frame}
    \frametitle{Visualizzare ed Elaborare documenti XML}
    \addtocounter{nframe}{1}
    
    %\begin{center}
    %    \includegraphics[width=.2\textwidth]{../imgs/tei-r.pdf}
    %\end{center}
    %\textit{In parte già disponibili nei moduli TEI di base}

     \begin{block}{XML Slyle Sheet Transformation: xsl:import}
        \texttt{<xsl:import href="esterno.xsl"/>}
        \\\texttt{<xsl:template match="/" >}
        \\\texttt{ <xsl:apply-imports/>}
        \\\texttt{</xsl:template>}
     \end{block}

\end{frame}

\begin{frame}
    \frametitle{Visualizzare ed Elaborare documenti XML}
    \addtocounter{nframe}{1}
    
    %\begin{center}
    %    \includegraphics[width=.2\textwidth]{../imgs/tei-r.pdf}
    %\end{center}
    %\textit{In parte già disponibili nei moduli TEI di base}

     \begin{block}{XML Slyle Sheet Transformation: documenti multipli}
        \begin{itemize}
            \item Accedere ai nodi di un documento XML esterno a quello che si sta elaborando
            \item Utilizzando la funzionalità \texttt{document()}
        \end{itemize}

     \end{block}

     \begin{block}{XML Slyle Sheet Transformation: documenti multipli}
        \texttt{<xsl:value-of}
        \\\texttt{ select="document(’altro.xml’)/TEI/text/@type"}
        \\\texttt{/>}

     \end{block}

\end{frame}

\begin{frame}
    \frametitle{Visualizzare ed Elaborare documenti XML}
    \addtocounter{nframe}{1}
    
    %\begin{center}
    %    \includegraphics[width=.2\textwidth]{../imgs/tei-r.pdf}
    %\end{center}
    %\textit{In parte già disponibili nei moduli TEI di base}

     \begin{block}{XML Slyle Sheet Transformation: XSLT 2.0}
        La \textit{recommendation} W3C per XSLT 2.0 è stata pubblicata nel 2007
     \end{block}

     \begin{block}{XML Slyle Sheet Transformation: documenti multipli}
       
        \texttt{<xsl:stylesheet} 
            \\\texttt{xmlns:xsl="http://www.w3.org/1999/XSL/Transform"} 
            \\\texttt{version="2.0" >}


     \end{block}

\end{frame}


\begin{frame}
    \frametitle{Visualizzare ed Elaborare documenti XML}
    \addtocounter{nframe}{1}
    
    %\begin{center}
    %    \includegraphics[width=.2\textwidth]{../imgs/tei-r.pdf}
    %\end{center}
    %\textit{In parte già disponibili nei moduli TEI di base}

     \begin{block}{XML Slyle Sheet Transformation: XSLT 2.0}
        XSLT 2.0 che permette di creare un documento di output (xml, html, xhtml, text).
     \end{block}

     \begin{block}{XML Slyle Sheet Transformation: documenti multipli}
       
        \texttt{<xsl:template match="/" >}
        \\\texttt{ <xsl:result-document method="html" href="output.html" >}
        \\\texttt{ <xsl:apply-templates/>}
        \\\texttt{ </xsl:result-document>}
        \\\texttt{</xsl:template>}
            
     \end{block}

\end{frame}

\begin{frame}
    \frametitle{Visualizzare ed Elaborare documenti XML}
    \addtocounter{nframe}{1}
    
    %\begin{center}
    %    \includegraphics[width=.2\textwidth]{../imgs/tei-r.pdf}
    %\end{center}
    %\textit{In parte già disponibili nei moduli TEI di base}

     \begin{block}{XML Slyle Sheet Transformation: XSLT 2.0}
        XSLT 2.0 che permette di creare anche documenti multipli.
     \end{block}

     \begin{block}{XML Slyle Sheet Transformation: XSLT 2.0 esempio}
        \texttt{<template match="//div/[@type='book']" >}
        \\\texttt{<xsl:for-each select="div/[@type='chapter']" >}
        \\\texttt{<xsl:result-document}
           \\\texttt{method="html"}
           \\\texttt{href="{@n}.html" >}
           \\\texttt{<xsl:apply-templates/>}
        \\\texttt{</xsl:result-document>}
        \\\texttt{</xsl:for-each>}
        \\\texttt{</xsl:template>}
    
    \end{block}

\end{frame}

\begin{frame}
    \frametitle{Visualizzare ed Elaborare documenti XML}
    \addtocounter{nframe}{1}
    
    %\begin{center}
    %    \includegraphics[width=.2\textwidth]{../imgs/tei-r.pdf}
    %\end{center}
    %\textit{In parte già disponibili nei moduli TEI di base}

     \begin{block}{XML Slyle Sheet Transformation: XSLT 2.0}
        \begin{itemize}
            \item elemento \texttt{<xsl:for-each-group>}: permette di selezionare un set di items
            \item processare un gruppo per volta
        \end{itemize}
        
     \end{block}

     \begin{block}{XML Slyle Sheet Transformation: XSLT 2.0}
        \textit{Possibili attributi}
        \begin{itemize}
            \item group-by
            \item group-adjacent
            \item group-starting-with
            \item group-ending-with
        \end{itemize}
    
    \end{block}

\end{frame}

\begin{frame}
    \frametitle{Visualizzare ed Elaborare documenti XML}
    \addtocounter{nframe}{1}
    
        \textit{Esempio elemento for-each-group}

    \begin{center}
        \includegraphics[width=.8\textwidth]{imgs/esempio-groupBy.png}
    \end{center}

\end{frame}

\begin{frame}
    \frametitle{Visualizzare ed Elaborare documenti XML}
    \addtocounter{nframe}{1}
    
        \textit{Espressioni condizionali if()}

    \begin{center}
        \includegraphics[width=.8\textwidth]{imgs/esempio-espressioneCondizionale.png}
    \end{center}

\end{frame}




    
    \section{XML Transformations: Un esempio con XML-TEI}
    \begin{frame}
    \frametitle{Visualizzare ed Elaborare documenti XML}
    \addtocounter{nframe}{1}
    
        \textit{Tutti i nomi di elementi TEI devono essere preceduti dal
        prefisso \textbf{tei:}}

    \begin{center}
        \includegraphics[width=.9\textwidth]{imgs/EsempioCommentato1.png}
    \end{center}

\end{frame}


\begin{frame}
    \frametitle{Visualizzare ed Elaborare documenti XML}
    \addtocounter{nframe}{1}
    
        \textit{Creare lo “scheletro” del documento finale, nel nostro caso un HTML}

    \begin{center}
        \includegraphics[width=.95\textwidth]{imgs/EsempioCommentato2.png}
    \end{center}

\end{frame}

\begin{frame}
    \frametitle{Visualizzare ed Elaborare documenti XML}
    \addtocounter{nframe}{1}
    
        \textit{Definire una regola per l’intestazione TEI (\texttt{<teiHeader>})}

    \begin{center}
        \includegraphics[width=.8\textwidth]{imgs/EsempioCommentato3.png}
    \end{center}

\end{frame}


\begin{frame}
    \frametitle{Visualizzare ed Elaborare documenti XML}
    \addtocounter{nframe}{1}
    
        \textit{Componenti dell'elemento \textbf{tei:text}}

    \begin{center}
        \includegraphics[width=.8\textwidth]{imgs/EsempioCommentato4.png}
    \end{center}

\end{frame}

\begin{frame}
    \frametitle{Visualizzare ed Elaborare documenti XML}
    \addtocounter{nframe}{1}
    
        \textit{Divisioni del testo}

    \begin{center}
        \includegraphics[width=.8\textwidth]{imgs/EsempioCommentato5.png}
    \end{center}

\end{frame}


\begin{frame}
    \frametitle{Visualizzare ed Elaborare documenti XML}
    \addtocounter{nframe}{1}
    
        \textit{Per la poesia}

    \begin{center}
        \includegraphics[width=.9\textwidth]{imgs/EsempioCommentato6.png}
    \end{center}

\end{frame}


\begin{frame}
    \frametitle{Visualizzare ed Elaborare documenti XML}
    \addtocounter{nframe}{1}
    
        \textit{Elementi Phrase-Level}

    \begin{center}
        \includegraphics[width=.95\textwidth]{imgs/EsempioCommentato7.png}
    \end{center}

\end{frame}


\begin{frame}
    \frametitle{Visualizzare ed Elaborare documenti XML}
    \addtocounter{nframe}{1}
    
        \textit{Interventi Editoriali}

    \begin{center}
        \includegraphics[width=.8\textwidth]{imgs/EsempioCommentato8a.png}
    \end{center}

\end{frame}

\begin{frame}
    \frametitle{Visualizzare ed Elaborare documenti XML}
    \addtocounter{nframe}{1}
    
        \textit{Interventi Editoriali}

    \begin{center}
        \includegraphics[width=.8\textwidth]{imgs/EsempioCommentato8b.png}
    \end{center}

\end{frame}

\begin{frame}
    \frametitle{Visualizzare ed Elaborare documenti XML}
    \addtocounter{nframe}{1}
    
        \textit{Altri elementi}

    \begin{center}
        \includegraphics[width=.8\textwidth]{imgs/EsempioCommentato9.png}
    \end{center}

\end{frame}

\begin{frame}
    \frametitle{Visualizzare ed Elaborare documenti XML}
    \addtocounter{nframe}{1}
    
        \textit{Elemento \texttt{<hi>} distinto in base all'attributo \texttt{@rend}}

    \begin{center}
        \includegraphics[width=.85\textwidth]{imgs/EsempioCommentato10.png}
    \end{center}

\end{frame}

\begin{frame}
    \frametitle{Visualizzare ed Elaborare documenti XML}
    \addtocounter{nframe}{1}
    
        \textit{Elementi vuoti e milestone}

    \begin{center}
        \includegraphics[width=.9\textwidth]{imgs/EsempioCommentato11.png}
    \end{center}

\end{frame}





    
    \section{Panoramica sul Document Object Model (DOM)}
    The Document Object Model
The document object model (DOM) is, as previously mentioned, a way of representing the document
independent of browser type. It allows a developer to access the document via a common set of objects,
properties, methods, and events, and to alter the contents of the web page dynamically using scripts.
You should be aware that some small variations are usually added to the DOM by the browser
vendor. So, to guarantee that you don’t fall afoul of a particular implementation, the W3C has
provided a generic set of objects, properties, and methods that should be available in all browsers, in
the form of the DOM standard.
The DOM Standard
We haven’t talked about the DOM standard so far, and for a particular reason: It’s not the easiest
standard to follow. Supporting a generic set of properties and methods has proved to be a very
complex task, and the DOM standard has been broken down into separate levels and sections
to deal with the different areas. The different levels of the standard are all at differing stages of
completion.

Level 0
Level 0 is a bit of a misnomer, because there wasn’t really a level 0 of the standard. This term in fact
refers to the “old way” of doing things—the methods implemented by the browser vendors before the
DOM standard. Someone mentioning level 0 properties is referring to a more linear notation of accessing
properties and methods. For example, typically you’d reference items on a form with the following code:
document.forms[0].elements[1].value = "button1";
We’re not going to cover such properties and methods in this chapter, because they have been
superseded by newer methods.

Level 1
Level 1 is the first version of the standard. It is split into two sections: One is defined as core
(objects, properties, and methods that can apply to both XML and HTML) and the other as HTMLThe Document Object Model
❘ 235
(HTML‐specific objects, properties, and methods). The first section deals with how to go about
navigating and manipulating the structure of the document. The objects, properties, and methods in
this section are very abstract. The second section deals with HTML only and offers a set of objects
corresponding to all the HTML elements. This chapter mainly deals with the second section—
level 1 of the standard.
In 2000, level 1 was revamped and corrected, though it only made it to a working draft and not to a
full W3C recommendation.

Level 2
Level 2 is complete and many of the properties, methods, and events have been implemented
by today’s browsers. It has sections that add specifications for events and style sheets to the
specifications for core and HTML‐specific properties and events. (It also provides sections on
views and traversal ranges, neither of which is covered in this book; you can find more information
at www.w3.org/TR/2000/PR‐DOM‐Level‐2‐Views‐20000927/ and www.w3.org/TR/2000/
PR‐DOM‐Level‐2‐Traversal‐Range‐20000927/ .)

Level 3
Level 3 achieved recommendation status in 2004. It is intended to resolve a lot of the complications
that still exist in the event model in level 2 of the standard, and adds support for XML features,
such as content models and being able to save the DOM as an XML document.
Level 4
In May 2014, DOM level 4 reached candidate recommendation status. It consolidates DOM level 3
with several independent components. At the time of this writing, no modern browser supports
DOM level 4, although that will change in the future.
    
    \section{XML DOM Programming API: Un esempio TEI}
    \begin{frame}
    \frametitle{Visualizzare ed Elaborare documenti XML}
    \addtocounter{nframe}{1}
    
    %\begin{center}
    %    \includegraphics[width=.2\textwidth]{../imgs/tei-r.pdf}
    %\end{center}
    %\textit{In parte già disponibili nei moduli TEI di base}

     \begin{block}{Documenti Object Model (DOM): Esercizio}
       Caricare il file testTeiNS.xml con javascript e manipolarlo con DOM e Xpath per costruire una visualizzazione appropriata
     \end{block}

     \begin{block}{Documenti Object Model (DOM)}
        Scrivere un foglio di stile e applicarlo con javascript al documento DOM risultante dal caricamento del file testTeiNS.xml
     \end{block}


\end{frame}
    
    \end{document}