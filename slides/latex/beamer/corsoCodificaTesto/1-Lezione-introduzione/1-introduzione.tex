% introduzione al corso di codifica del testo anno accademico 2018/2019

\documentclass{beamer}
    
%    \usepackage[english]{babel}
    %\usepackage[latin1]{inputenc}
    %\usepackage[T1]{fontenc}

\mode<presentation>{
  \setbeamertemplate{background canvas}[vertical shading]
  \usetheme{Berkeley}
  \useoutertheme{himinfolines}
}
  
\usepackage{ucs}
\usepackage[utf8]{inputenc}
\usepackage[english,polutonikogreek,italian,UKenglish,british]{babel}
\usepackage{graphicx}
\usepackage{colortbl}
\usepackage{multicol}
\usepackage{ulem}
\usepackage{verbatim}
\usepackage{alltt}
\usepackage{ccicons}
\usepackage{MnSymbol,wasysym}
\usepackage{tikzsymbols}
\usepackage{textcomp}
\usepackage{xmpincl}

\usepackage{parskip}
\setcounter{nframes}{70}
\setcounter{nframe}{1}
\setbeamercovered{dynamic}
\newenvironment{grcenv}{\begin{otherlanguage}{greek}}{\end{otherlanguage}}
\newcommand{\g}[1]{\textgreek{#1}}
\definecolor{darkgreen}{rgb}{0,0.5,0}
\definecolor{darkblue}{rgb}{0,0,0.5}
\definecolor{grey}{rgb}{0.5,0.5,0.5}
\setcounter{tocdepth}{5}

\makeatletter

\makeatother
%\includexmp{LicencesAndLicensing}

%frame00 metadata
    \title{Presentazione del Corso Codifica di Testi \\a.a. 2018-2019}
    \author[A.M. Del Grosso]{Angelo Mario Del Grosso}
    %\institute{\texttt{angelo.delgrosso@ilc.cnr.it} \\\bigskip\textit{CNR-ILC-LicoLab} \\\bigskip\url{http://licolab.ilc.cnr.it/}}
    \institute{\texttt{angelo.delgrosso@ilc.cnr.it} \\\bigskip\textit{CNR-ILC-LicoLab}}
    \date{Istituto di Linguistica Computazionale ``A. Zampolli'', \today}
    \AtBeginSection[]{
    \begin{frame}<beamer>
    \addtocounter{nframe}{1}
    \footnotesize
    \frametitle{Progress status}
    \tableofcontents[currentsection,hideothersubsections]
    \end{frame}
    }

\begin{document}

\begin{frame}
	\maketitle
\end{frame}

\begin{frame}
	\frametitle{Piano della presentazione}
	\tableofcontents
\end{frame}

\section{Presentazione}

\begin{frame}
	\frametitle{Di cosa mi occupo}
	\addtocounter{nframe}{1}

	\begin{block}{Filologia Computazionale}
		%slide di presentazione: chi sono, piccola bio, di cosa mi occupo
		%\\prendere dal curriculum alcuni pezzi mettere mail istituzionale ed eventualmente telefono
		Analisi, progettazione e sviluppo di componenti software per sistemi Web di linguistica e filologia digitale/computazionale volti al trattamento di testi di tradizione medievale, a stampa e di autori moderni e contemporanei.
	\end{block}

	\begin{block}{Modelli Object Oriented per il Textual Scholarship}
		Impiego delle nuove tecnologie nell’ambito delle Digital Humanities (DH) per la progettazione
		object–oriented di strumenti digitali Web–based rispondenti alle esigenze degli utenti accademici, studenti e sviluppatori.
	\end{block}




\end{frame}

\begin{frame}
	\frametitle{Presentazione del Corso}
	\addtocounter{nframe}{1}

	\begin{center}
		\includegraphics[width=.5\textwidth]{../imgs/tei-r.pdf}
	\end{center}

	%\begin{itemize}
	%	\item<1-> Introduzione
	%	\item<2-> Codifica dei Caratteri
	%   \item<3-> Codifica dei Testi
	%   \item<4-> Ecosistema XML (Linee Guida TEI)
	%   \item<5-> Conclusioni
	%\end{itemize}

\end{frame}



\section{Introduzione}

% sezione intro frame 01
\begin{frame}
    \frametitle{Introduzione al Corso di Codifica di Testi}
    \framesubtitle{Obiettivi, competenze e conoscenze}
    \addtocounter{nframe}{1}
    
    \begin{block}{Obiettivo}
        Illustrare i principi di modellazione e le prassi di codifica del testo per una adeguata rappresentazione ed elaborazione digitale di risorse testuali.  
    \end{block}

    \begin{block}{Rationale}
       Fornire gli strumenti e le conoscenze necessarie per progettare e realizzare criticamente una codifica digitale di testi complessi, in particolare testi letterari e di interesse storico-culturale, usando le linee guida della Text Encoding Initiative (TEI).
    \end{block}

\end{frame}

% sezione intro frame 02
\begin{frame}
    \frametitle{Argomenti trattati}
    \framesubtitle{Obiettivi, competenze e conoscenze}
    \addtocounter{nframe}{1}
    
    \begin{block}{Competenze attese}
        Al termine del corso sarete in grado di valutare il metodo di codifica più appropriato allo scenario d'interesse, di creare uno schema di codifica TEI e di usare gli strumenti più idonei per la codifica e la (semplice) elaborazione e visualizzazione di un testo.
    \end{block}

\end{frame}

% sezione intro frame 03
\begin{frame}
    \frametitle{Principali Argomenti}
    \framesubtitle{Obiettivi, competenze e conoscenze}
    \addtocounter{nframe}{1}

    
        \begin{itemize}
            \item Codifica dei caratteri e di testi
            \item I linguaggi di markup e introduzione a XML
            \item Creazione di schemi di codifica
            \item Le norme TEI (Text Encoding Initiative)
            \item Alcuni specifici Moduli TEI
            \item Definizione di schemi di codifica personalizzati
            \item introduzione ai fogli di stile XSLT
            \item elaborazione documenti XML-TEI (XSLT2.0, DOM)
            \item esempi, esercitazioni e seminari 
        \end{itemize}

\end{frame}


% sezione intro frame 04
\begin{frame}
    \frametitle{Perché è importante la codifica dei testi}
    \framesubtitle{Motivazioni teoriche}
    \addtocounter{nframe}{1}
    
    \begin{block}{Perché codificare}
        Il rapporto tra sapere umanistico e informatica non è solo una questione meramente strumentale. 
        \\ L'informatica non è solo un utensile dalle notevoli capacità.
        \\ Salto di paradigma sia dal punto di vista teorico e metodologico sia da quello pratico.

        L'attività di codifica come funzione metodologica nell'ambito delle discipline che si occupano del testo.
        \\ Il linguaggio di codifica adottato può essere considerato come un linguaggio teorico.
        \\ Esplicitare le ipotesi interpretative su un certo oggetto di studio
    \end{block}

\end{frame}

% sezione intro frame 05
\begin{frame}
    \frametitle{Perché è importante la codifica dei testi}
    \framesubtitle{Motivazioni pratiche}
    \addtocounter{nframe}{1}
    
    \begin{block}{Perché codificare}
        Nella nostra cultura la quasi totalità dei testi è \underline{registrata} su materiali fisici di varia natura e forma (manoscritti su pietra, pergamena, papiri, carta, libri a stampa, incunabula, cinquecentine, etc).

        Per rendere disponibile questo patrimonio attraverso i sistemi per la gestione dell'informazione digitali e computazionali è necessario effettuare una trasposizione/trascodifica dei testi (procedimento di conversione dei dati codificati secondo un sistema verso un sistema diverso) dal loro supporto originario verso il nuovo supporto elettronico.
    \end{block}

\end{frame}

% sezione intro frame 06
\begin{frame}
    \frametitle{Perché è importante la codifica dei testi}
    \framesubtitle{In sintesi}
    \addtocounter{nframe}{1}
    
    \begin{block}{Perché codificare}
    Il focus del corso sarà diretto alla rappresentazione digitale del testo.
    \\ Per ottenere tale rappresentazione ci sono diversi formati e formalismi:
    \\ la nostra scelta ricade sulle norme suggerite dal consorzio TEI.
    \\ molte questioni ancora non risolte e controverse, sia teorico-metodologico, sia pratico-tecnologico.
    \\ I formalismi e le tecnologie adottate possono essere viste come isomorfe.
    \end{block}

\end{frame}

% sezione intro frame 07
\begin{frame}
    \frametitle{Perché è importante la codifica dei testi}
    \framesubtitle{Ma in definitiva}
    \addtocounter{nframe}{1}
    
    \begin{block}{Perché codificare}

        \begin{center}
            \textit{Le differenze di formato sono più che altro estetiche e non sostanziali}
        \end{center}

    \end{block}
     

    \begin{block}{Perché codificare}

        \begin{center}
            \textbf{Ma anche l'occhio \underline{umano} vuole la sua parte}
        \end{center}
       

    \end{block}

\end{frame}



\section{Codifica dei Caratteri}
% frame 00
\begin{frame}
	\frametitle{Elementi di Codifica dei Caratteri}
	\framesubtitle{Problemi di rappresentazione}
	\addtocounter{nframe}{1}

	\begin{center}
		\includegraphics[width=.9\textwidth]{imgs/ascii-67.pdf}
	\end{center}

	[DA completare]

\end{frame}

% frame 00
\begin{frame}
	\frametitle{Elementi di Codifica dei Caratteri}
	\framesubtitle{Definizioni}
	\addtocounter{nframe}{1}

	\begin{block}{Rappresentare il testo in formato digitale}
		L’adozione di metodologie informatiche per il trattamento dei testi richiede in primo luogo la disponibilità di un'adeguata rappresentazione dei dati testuali in formato digitale.
	\end{block}

\end{frame}

% frame 01
\begin{frame}
	\frametitle{Elementi di Codifica dei Caratteri}
	\framesubtitle{Definizioni}
	\addtocounter{nframe}{1}

	\begin{block}{Perché è importante la codifica dei caratteri}
		La codifica dei caratteri costituisce il grado zero (basso livello) della rappresentazione di testi su supporto digitale.
		\begin{center}
			\textit{Le codifiche dei caratteri sono la base di qualsiasi schema di codifica testuale}.
		\end{center}
	\end{block}



	\begin{block}{Rappresentazione digitale dei caratteri}
		I caratteri vengono rappresentati all’interno di un elaboratore mediante una sequenza di codici binari formati da opportune disposizioni di cifre composte da 0 e 1: 01100001 \textit{lettera a}
	\end{block}

\end{frame}



% frame 03
\begin{frame}
	\frametitle{Elementi di Codifica dei Caratteri}
	\framesubtitle{Definizioni}
	\addtocounter{nframe}{1}

	\begin{block}{Tabella Code Page ASCII 7 bit}
		%immagine di esempio Code Page ASCII (cp1252)
		\begin{center}
			\includegraphics[width=.9\textwidth]{imgs/ascii-67.pdf}
		\end{center}

	\end{block}
	%\hline
	\begin{tiny}
		\begin{center}
			7 bit = 128 possibili caratteri; 32 caratteri di controllo; 96 caratteri effettivi
		\end{center}

	\end{tiny}

\end{frame}

% frame 0
\begin{frame}
	\frametitle{Elementi di Codifica dei Caratteri}
	\framesubtitle{Esempio codifica binaria}
	\addtocounter{nframe}{1}

	\begin{block}{codifica \textit{ciao mondo!} 7 bit ASCII}
		\begin{center}
			\textsc{6369 616f 206d 6f6e 646f 210a}
		\end{center}
	\end{block}

	\begin{block}{codifica \textit{ciao è mondo!} 8 bit ASCII}
		\begin{center}
			\textmd{6369 616f 20\textbf{e8} 206d 6f6e 646f 210a       }
		\end{center}
	\end{block}

	\begin{block}{codifica \textbf{ciao è mondo!} UNICODE UTF-8}
		\begin{center}
			6369 616f 20\textbf{c3 a8}20 6d6f 6e64 6f21 0a
		\end{center}
	\end{block}

\end{frame}

% frame 02
\begin{frame}
	\frametitle{Elementi di Codifica dei Caratteri}
	\framesubtitle{Definizioni}
	\addtocounter{nframe}{1}

	% \begin{block}{Character set, Code Set}
	%  - Character set
	%  - Code Set
	%  - Character encoding
	%  - Tabella del Code page
	% \end{block}

	\begin{description}
		\item [Character set] Per le discipline che studiano i sistemi di scrittura e l'analisi del linguaggio naturale, un insieme di caratteri astratti è detto Character set (unità alfabetiche). Astratto perché non riguarda la rappresentazione materiale della forma sul supporto, ma è relativo alla forma mentale, fatta di simboli di codifica (referenti).
		\item [Coded Char Set] Per poter trattare un insieme di unità alfabetiche in formato digitale bisogna assegnare a ciascun carattere un numero intero non negativo detto code point.
		
	\end{description}

\end{frame}

% frame 02b
\begin{frame}
	\frametitle{Elementi di Codifica dei Caratteri}
	\framesubtitle{Definizioni}
	\addtocounter{nframe}{1}

	% \begin{block}{Character set, Code Set}
	%  - Character set
	%  - Code Set
	%  - Character encoding
	%  - Tabella del Code page
	% \end{block}

	\begin{description}
		\item [Character encoding]  Il fine ultimo della codifica è quello di rappresentare una sequenza di caratteri in una sequenza di byte. La codifica di un carattere utilizza uno ``encoding schema'' che a sua volta mappa o trasforma ciascun code point in una sequenza di byte e quindi in ultima istanza in una sequenza di bit. 
		\item [Tabella del code page] Generalmente i code points sono espressi attraverso un sistema numerico esadecimale e disposti in una tabella di associazione.
	\end{description}

\end{frame}

% frame 02c
\begin{frame}
	\frametitle{Elementi di Codifica dei Caratteri}
	\framesubtitle{In sintesi}
	\addtocounter{nframe}{1}


	\begin{block}{Codifica dei caratteri}
		Quindi trasformare una sequenza di caratteri appartenenti ad un char set in una sequenza di byte (bit) significa prima di tutto trasformare/mappare ciascun carattere nel proprio corrispettivo code point e successivamente codificare/serializzare questo code point nella relativa sequenza di byte (bit).
	\end{block}

\end{frame}


% frame 0
\begin{frame}
	\frametitle{Elementi di Codifica dei Caratteri}
	\framesubtitle{Complessità e rappresentazione}
	\addtocounter{nframe}{1}

	\begin{block}{Complessità di rappresentazione universale dei caratteri}
		Se si considerano tutti i possibili alfabeti del mondo e le molteplici esigenze poste dalla scrittura delle fonti manoscritte antiche e medievali, ci si accorge che la realizzazione di un sistema universale per la codifica dei caratteri è un progetto molto complesso con svariate sfide da affrontare.
	\end{block}

\end{frame}

% frame 0
\begin{frame}
	\frametitle{Complessità e rappresentazione di Codifica dei Caratteri}
	\framesubtitle{Un Esempio}
	\addtocounter{nframe}{1}

	\begin{center}
		\includegraphics[width=.9\textwidth]{imgs/ascii-67.pdf}
	\end{center}

	[DA COMPLETARE]

\end{frame}


\begin{frame}
	\frametitle{Elementi di Codifica dei Caratteri}
	\framesubtitle{Unicode}
	\addtocounter{nframe}{1}

	\begin{block}{Complessità di rappresentazione universale}
		Ad oggi, lo standard de facto per la codifica dei caratteri è lo UNICODE. Esso è in grado di codificare più di un milione di differenti unità alfabetiche, segni di interpunzione e diacritici, appartenenti a centinaia di diverse lingue.
	\end{block}

	\begin{block}{Complessità di rappresentazione universale}
		%(1.114.111)
		Unicode assegna i propri code point in un range che va da $0x0$ a $0x10FFFF$. In Unicode il code point viene  indicato con una ``U'' seguita da un segno ``+'' seguito a sua volta dall'esadecimale con padding del codice (es: U+0041 lettera a).
	\end{block}

\end{frame}

\begin{frame}
	\frametitle{Elementi di Codifica dei Caratteri}
	\framesubtitle{Unicode}
	\addtocounter{nframe}{1}

	\begin{block}{Unicode Transformation Format}
		Lo Unicode è un Coded Char Set e per essere concretamente serializzato su un supporto elettronico deve essere trasformato attraverso qualche tipo di schema di codifica.
		L'UTF (Unicode Transformation Format) mappa i code point Unicode in sequenze di byte (bit).
	\end{block}

	\begin{block}{UTF standards}
		Esistono tre tipi di schemi di codifica che vanno sotto il nome di UTF, ciascuno è identificato dal minimo numero di bit necessario a codificare ciascun code point: UTF-8; UTF-16; UTF-32. 
	\end{block}

\end{frame}



\section{Codifica dei Testi}
% riprendere qualcosa dalle slide Fiormonte Ciotti (Introduzione alla codifica XML per i testi umanistici)
% one of the purposes of the TEI Guidelines is to guide encoding practice.

\begin{frame}
    \frametitle{Elementi di Codifica del testo}
    \framesubtitle{Tabella Formalismi}
    \addtocounter{nframe}{1}
    
    \begin{block}{Formalismi}
	    \includegraphics[width=.5\textwidth]{imgs/TabellaFormalismiCodificaTesto.png}
    \end{block}
    courtesy of \textit{Fabio Vitali}

\end{frame}


\begin{frame}
    \frametitle{Elementi di Codifica del testo}
    \framesubtitle{lista di formati}
    \addtocounter{nframe}{1}
    
    \begin{block}{Formati dato}
Data structures – CSV and tabular data
– JSON
– RDF
Plain text formats – Plain text
– TeX, LaTeX, etc.
– Markdown, CommonMark and wiki syntaxes
Markup formats
– HTML, HTML5
– XML
– HTML5+ Embedded annotations (e.g., HTML5 + RDFa)
– Markup spinoffs for overlapping (e.g. LMNL, TexMECS, etc.) 

    \end{block}

    \begin{block}{Riferimenti TEI}
        Capitolo sul character encoding e modulo Ganji 
    \end{block}

\end{frame}


% Representing text digitally 
Records
– Structures describing entities by enumerating their
properties
• Tables
– Collections of data as lists of homogeneous
records
• Trees
– Hierarchies of data and collections
• Graphs
– Networks of information structures more or less
densely intertwined 


% mia slide sulle possibili rappresentazioni del testo


% Slide Vitali
Text is hard
• Text contains relevant information, but its
structure predates digitally representable
information collections.
• It is not data. It is not a structure. It is not a
collection.
• It is not organized in records, tables, trees,
graphs. 

% altra slide Vitali
• Text has characters, including punctuation
– We all (sort of) agree on this
• Texts is ordered
– In "To be or not to be", it is important that "To be"
comes before "not to be"
• Text has structure
• Text has presentation
• Text has grammar
• Texts has semantics
• Text has variants
• Text has a lot of things that can be said about it 



% dalla slide vitali
The ecosystem of text
• Editing
• Validating
• Printing & displaying
• Transforming and converting
• Annotating
– Annotating annotations
• Searching
• ... 

% altra slide vitali
• A data format is not only what you use to
represent the data (or text) that you have to
deal with.
• An ugly format is, basically, a format for which
you have to work a lot to obtain the things you
need.
– Like putting make up on a pig to make it pretty
• It very much depends on what is the use you
plan for your data (or text). 

% slide codifica e Markup
In a sense, of course, every piece of markup added to a text represents the result of an analysis, whether human or automatic, and so it is natural to think of representing such annotations incrementally by means of markup to a digital text.

\section{Ecosistema XML}
% Da slide Vitali:

XML allows to create markup languages that are
readable, generic, structured, hierarchical.
– Data: no problem
– Hierarchical data: no problem
– Text: no problem
– Presentation of data: after transformation via XSLT
– Hierarchical text: no problem
– Validation: no problem
– References: no problem
– Annotations: as attributes or ad hoc sections of the
document 

\section{Conclusioni}
% strumenti per l'editing XML
%% editor di testi plain text
%% Oxygen
%% VSCode
%% editiXML


% bibliografia di riferimento
%% Ciotti
%% Burnard
%% TEI guide lines
%% Pierazzo (due libri)
%% Slide (del corso)
%% XML specification e technical report W3C (https://www.w3.org/TR/xml/)
%% XML visual
%% XSL XPATH
%% XSD (art of XSD - SQL validation)
%% DTD (libro visual XML)
%% RELAXNG (libro relaxng, tutorial)

% esercitazioni e approfondimenti

% modalità di esame
% progetto
% materiali e slide del corso
%% github


\section*{Bibliografia}
% bibliografia di riferimento
%% Ciotti
%% Burnard
%% TEI guide lines
%% Pierazzo (due libri)
%% Slide (del corso)
%% XML specification e technical report W3C (https://www.w3.org/TR/xml/)
%% XML visual
%% XSL XPATH
%% XSD (art of XSD - SQL validation)
%% DTD (libro visual XML)
%% RELAXNG (libro relaxng, tutorial)

%bibliografia
\begin{frame}
    \frametitle{References}
    \addtocounter{nframe}{1}
    \begin{thebibliography}{10}
        \setbeamertemplate{bibliography item}[paper]
        \tiny\bibitem{Lenci2016} Lenci, A., Montemagni S., and Pirrelli V. (2016). Testo e Computer. Elementi Di Linguistica Computazionale. Aulamagna. Carocci.
        \tiny\bibitem{Pierazzo2015} Pierazzo, E. (2015). Digital Scholarly Editing : Theories, Models and Methods. Farnham Surrey: Ashgate.
        \tiny\bibitem{orlandi2010} Orlandi, T. (2010). Informatica testuale: teoria e prassi. Laterza.
        \tiny\bibitem{Pierazzo2016} Driscoll, M. J., and Pierazzo, E. (Eds.). (2016). Digital Scholarly Editing: Theories and Practices (Vol. 4). Open Book Publishers.
        \tiny\bibitem{ciotti2012} Ciotti F., e Crupi G, a c. di. (2012). Dall’Informatica umanistica alle culture digitali. ROMA : Casa Editrice Università La Sapienza. \href{http://www.editricesapienza.it/node/7688}{open access: http://www.editricesapienza.it/node/7688}
        \tiny\bibitem{Williams2009} Williams, I. (2009). Beginning XSLT and XPath: Transforming XML Documents and Data. Wiley.
        \tiny\bibitem{Kay2011} Kay, M. (2011). XSLT 2.0 and XPath 2.0 Programmer’s Reference. Wiley.
    \end{thebibliography}

\end{frame}

\begin{frame}
    \frametitle{References}
    \addtocounter{nframe}{1}
    \begin{thebibliography}{10}
        
        \setbeamertemplate{bibliography item}[online]
        \tiny\bibitem{MSDN} \textit{XML Standards Reference}, MSDN. \url{https://msdn.microsoft.com/en-us/library/ms256177(v=vs.110).aspx}
        \tiny\bibitem{IBMXML1} IBM XML \textit{Tutorial}, \url{https://www.ibm.com/developerworks/xml/tutorials/xmlintro/xmlintro.html}
        \tiny\bibitem{w3school} W3School Tutorial \url{https://www.w3schools.com/xml/default.asp}

    \end{thebibliography}

\end{frame}


\end{document}