% frame 01
\begin{frame}
    \frametitle{Elementi di Codifica dei Caratteri}
    \framesubtitle{Definizioni}
    \addtocounter{nframe}{1}
    
    \begin{block}{Perché è importante}
    La codifica dei caratteri rappresenta il grado zero della rappresen- tazione di testi su supporto digitale.
    \\ Essa costituisce la base di tutti gli ulteriori sistemi di codifica testuale
    \end{block}



    \begin{block}{Codifica dei caratteri}
        %Capitolo sul character encoding e modulo Ganji 
        Come qualsiasi altro tipo di dati, anche i caratteri vengono rappresentati all’interno di un elaboratore mediante una codifica numerica binaria
        \\Le codifiche dei caratteri sono la base di qualsiasi schema di codi- fica testuale,
    \end{block}

\end{frame}

% frame 02
\begin{frame}
    \frametitle{Elementi di Codifica dei Caratteri}
    \framesubtitle{Definizioni}
    \addtocounter{nframe}{1}
    
    \begin{block}{Character set, Code Set}
     - Character set
     - Code Set
     - Character encoding
     - Tabella del Code page
    \end{block}

    \begin{block}{Tabella Code Page}
    immagine di esempio Code Page ASCII (cp1252)
    \end{block}

\end{frame}

% Se poi si considerano gli altri alfabeti mondiali, anche limitandosi a forme normalizzate e moderne, o le esigenze poste dalla trascrizione di fonti manoscritte antiche e medievali, ci si accorge che la progettazione di un sistema universale per la codifica di caratteri è assai complessa: