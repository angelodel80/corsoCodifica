\documentclass{beamer}
    
%    \usepackage[english]{babel}
    %\usepackage[latin1]{inputenc}
    %\usepackage[T1]{fontenc}

\mode<presentation>{
  \setbeamertemplate{background canvas}[vertical shading]
  \usetheme{Berkeley}
  \useoutertheme{himinfolines}
}
  
\usepackage{ucs}
\usepackage[utf8]{inputenc}
\usepackage[english,polutonikogreek,italian,UKenglish,british]{babel}
\usepackage{graphicx}
\usepackage{colortbl}
\usepackage{multicol}
\usepackage{ulem}
\usepackage{verbatim}
\usepackage{alltt}
\usepackage{ccicons}
\usepackage{MnSymbol,wasysym}
\usepackage{tikzsymbols}
\usepackage{textcomp}
\usepackage{xmpincl}

\usepackage{parskip}
\setcounter{nframes}{100}
\setcounter{nframe}{1}
\setbeamercovered{dynamic}
\newenvironment{grcenv}{\begin{otherlanguage}{greek}}{\end{otherlanguage}}
\newcommand{\g}[1]{\textgreek{#1}}
\definecolor{darkgreen}{rgb}{0,0.5,0}
\definecolor{darkblue}{rgb}{0,0,0.5}
\definecolor{grey}{rgb}{0.5,0.5,0.5}
\setcounter{tocdepth}{5}

\makeatletter

\makeatother
\includexmp{LicencesAndLicensing}

%frame00 metadata
    \title{Presentazione Corso Codifica del Testo}
    \author[A.M. Del Grosso]{Angelo Mario Del Grosso}
    \institute{\texttt{angelo.delgrosso@ilc.cnr.it} \\\bigskip\textit{CNR-ILC-LicoLab} \\\bigskip\url{http://licolab.ilc.cnr.it/}}
    \date{Istituto di Linguistica Computazionale ``A. Zampolli'', \today}
    \AtBeginSection[]{
    \begin{frame}<beamer>
    \addtocounter{nframe}{1}
    \footnotesize
    \frametitle{Progress status}
    \tableofcontents[currentsection,hideothersubsections]
    \end{frame}
    }

\begin{document}

\begin{frame}
	\maketitle
\end{frame}

\begin{frame}
	\frametitle{Piano della presentazione}
	\tableofcontents
\end{frame}

\section{Presentazione}
\begin{frame}
	\frametitle{Presentazione del Corso}
	\addtocounter{nframe}{1}
    
    \begin{center}
	    \includegraphics[width=.4\textwidth]{./imgs/tei-r.pdf}
	\end{center}

	\begin{itemize}
		\item<1-> parleremo di TEI
		\item<2-> parleremo anche di TEI
		\item<3-> \textbf{non} parleremo di altro se non di TEI
	\end{itemize}
\end{frame}

\section{Introduzione}
\begin{frame}
    \frametitle{Includi frame esterno}
    \addtocounter{nframe}{1}

    \begin{enumerate}[<+->]
        \item dal master indicare il file di input
        \item dal modulo inserire il frame
        \item con il titolo del frame e il contatore
    \end{enumerate}
\end{frame}

\section{Chi siamo}
\begin{frame}
    \frametitle{chi siamo}
    \framesubtitle{nel vero senso della parola}
    \addtocounter{nframe}{1}

    \begin{block}{Sappiamo}
        noi chi siamo?!
    \end{block}
    
\end{frame}

\section{Interessanti}
\begin{frame}
    \frametitle{Cose interessanti}
    \framesubtitle{il sottotitolo del frame!}
    \addtocounter{nframe}{1}
    
    \begin{block}{un blocco nel frame}
        Il contenuto del blocco
    \end{block}
    
\end{frame}

\section{Conclusioni}
%conclusioni f01
\begin{frame}
    \frametitle{References}
    \addtocounter{nframe}{1}
    \begin{thebibliography}{10}

        \setbeamertemplate{bibliography item}[paper]
        \tiny\bibitem{ciotti2011} Ciotti 2011

        \setbeamertemplate{bibliography item}[online]
        \tiny\bibitem{CCwikiBY2014} CC Wiki \textit{Best practices for attribution}, CC Wiki 2014, \url{https://wiki.creativecommons.org/wiki/Best\_practices\_for\_attribution}

    \end{thebibliography}

\end{frame}


\end{document}