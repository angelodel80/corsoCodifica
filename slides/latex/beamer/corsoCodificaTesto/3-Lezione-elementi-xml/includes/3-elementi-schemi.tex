%slide di introduzione ai fondamenti degli schemi xml
% [+]leggere capitolo due "XML Schema Languages" del libro Advanced XML Applications from the Experts at The XML Guild
% Only TYPED xml can be validated with an XML schema
% When XML documents need to follow a certain structure and the values of attributes and elements need to follow a set of rules
% A method to describe the XML structure 
% A method to validate the XML data
%  get a clear idea about the required XML structure from the schema.
% Since a schema explains the required XML structure much better than a word document or a document of any other kind, there are no chances of misunderstanding or misinterpreting what is required.
% Validation of the XML data submitted by the client applications became easier.
%  define the validation rules.
%  list of validations that had to be included in the Schema

% XML Schema provides a kind of document grammar, naming the possible components and constraining the organization of an document. 
% XML Schema makes it possible to express constraints
% The Rules can be easly checked by an automatic processor: a validator

% An XML schema describes and validates XML documents.

% Should this piece of data go in an element or in an attribute? On many occasions, your piece of data could be stored equally well in an attribute as in an element with no loss of information. It is being largely a matter of personal taste



% For different users to share a vocabulary effectively, rules must be developed that
% specifically control what code and content a document from that vocabulary might
% contain. This is done by attaching either a Document Type Definition (DTD) or a
% schema to the XML document containing the data. Both DTDs and schemas contain
% rules for how data in a document vocabulary should be structured. A DTD defines the
% structure of the data and, very broadly, the types of data allowable. A schema more
% precisely defines the structure of the data and specific data restrictions.

% If an XML document is part of a vocabulary with a defined DTD or schema, it also
% must be tested to ensure that it satisfies the rules of that vocabulary. A well-formed XML
% document that satisfies the rules of a DTD or schema is said to be a valid document.

% frame 02
\begin{frame}
    \frametitle{Elementi per la definizione degli schemi xml}
    \framesubtitle{principi}
    \addtocounter{nframe}{1}

    we need to make sure that the data that we receive follows a certain XML structure and should contain values which are coherent. 

    Your function needs to make sure that the caller passes correct XML data. You could make use of an XML Schema to perform this validation.

    Performing such validations without the help of a SCHEMA will be extremely difficult most of the time.


\end{frame}

% frame 03
\begin{frame}
    \frametitle{Elementi per la definizione degli schemi xml}
    \framesubtitle{principi}
    \addtocounter{nframe}{1}

    Make sure that the XML document is structured exactly the way your function expects it to be.

   We need an XML schema when we need to make sure that the XML document that we need to work with is in the expected format.
   Make sure that the values of elements and attributes are within the accepted range.

   When data is managed and exchanged in XML format, there needs to be clear agreement about the structure of the XML document.

   There needs to be a contract between the caller and the callee about the XML document being exchanged.

   validate the XML document to make sure that it adheres to the format defined in the contract.


\end{frame}



% frame 04
\begin{frame}
    \frametitle{Elementi per la definizione degli schemi xml}
    \framesubtitle{principi}
    \addtocounter{nframe}{1}

    A Schema provides such the contract. 
    \\ It defines the structure of the XML document. 
    \\ It defines rules to validate the value of elements and attributes as well as their formats. 
    \\ Once a schema is defined, a Schema Validator can validate an XML document against the rules defined in the Schema.


\end{frame}


% frame 05
\begin{frame}
    \frametitle{Elementi per la definizione degli schemi xml}
    \framesubtitle{Tipi di formalismi per definire schemi XML}
    \addtocounter{nframe}{1}

   DTD, XDR, SOX, Schematron, DSD, DCD, DDML, RELAX NG

\end{frame}

\subsection{DTD}
\input{dtd.tex}

\subsection{XSD}
\input{xsd.tex}

\subsection{RELAXGN}
\input{relaxng.tex}