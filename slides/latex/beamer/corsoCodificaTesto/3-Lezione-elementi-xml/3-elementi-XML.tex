% slide di panoramica sulla tecnologia XML
% % Panoramica XML: definizione, sintassi, documento ben formato, documento valido
% cos'è XML
% XML is used when you want to mark up a textual resource and to store data in a suitable and recognized standard way.

%% slide premesse


\documentclass{beamer}
    
    %    \usepackage[english]{babel}
        %\usepackage[latin1]{inputenc}
        %\usepackage[T1]{fontenc}
    
    \mode<presentation>{
      \setbeamertemplate{background canvas}[vertical shading]
      \usetheme{Berkeley}
      \useoutertheme{himinfolines}
    }
      
    \usepackage{ucs}
    \usepackage[utf8]{inputenc}
    \usepackage[english,polutonikogreek,italian,UKenglish,british]{babel}
    \usepackage{graphicx}
    \usepackage{colortbl}
    \usepackage{multicol}
    \usepackage{ulem}
    \usepackage{verbatim}
    \usepackage{alltt}
    \usepackage{ccicons}
    \usepackage{MnSymbol,wasysym}
    \usepackage{tikzsymbols}
    \usepackage{textcomp}
    \usepackage{xmpincl}
    
    \usepackage{parskip}
    \setcounter{nframes}{100}
    \setcounter{nframe}{1}
    \setbeamercovered{dynamic}
    \newenvironment{grcenv}{\begin{otherlanguage}{greek}}{\end{otherlanguage}}
    \newcommand{\g}[1]{\textgreek{#1}}
    \definecolor{darkgreen}{rgb}{0,0.5,0}
    \definecolor{darkblue}{rgb}{0,0,0.5}
    \definecolor{grey}{rgb}{0.5,0.5,0.5}
    \setcounter{tocdepth}{5}
    
    \makeatletter
    
    \makeatother
    %\includexmp{LicencesAndLicensing}
    
    %frame00 metadata
        \title{Codifica di Testi - Introduzione XML Markup \\a.a. 2018-2019}
        \author[A.M. Del Grosso]{Angelo Mario Del Grosso}
        \institute{\texttt{angelo.delgrosso@ilc.cnr.it} \\\bigskip\textit{CNR-ILC-LicoLab}}
        \date{Istituto di Linguistica Computazionale ``A. Zampolli'', \today}
        \AtBeginSection[]{
        \begin{frame}<beamer>
        \addtocounter{nframe}{1}
        \footnotesize
        \frametitle{Progress status}
        \tableofcontents[currentsection,hideothersubsections]
        \end{frame}
        }
    
\begin{document}

\begin{frame}
	\maketitle
\end{frame}

\begin{frame}
	\frametitle{Contenuto della lezione}
	\tableofcontents
\end{frame}

\section{I linguaggi di codifica}
\begin{frame}
	\frametitle{I linguaggi di codifica}
	\framesubtitle{introduzione}
	\addtocounter{nframe}{1}

	\begin{block}{Definizione di codifica digitale del testo}
		Per \textbf{codifica} digitale dei testi intendiamo la \textit{rappresentazione formale} di un \textbf{testo} ad un qualche livello descrittivo, su di un supporto digitale, in un formato utilizzabile da un elaboratore (\textit{Machine Readable Form}) mediante un opportuno \textbf{linguaggio informatico} (F. Ciotti).
	\end{block}

\end{frame}

\begin{frame}
	\frametitle{I linguaggi di codifica}
	\framesubtitle{Riassumendo}
	\addtocounter{nframe}{1}

	\begin{block}{Impostazione teorico-pratica}

		\begin{itemize}
			\item  un testo è molto di più della sequenza di caratteri che lo
			      compongono
			\item per mezzo della codifica vogliamo rendere esplicite le
			      caratteristiche che vogliamo analizzare
			\item  solo quello che è esplicito può essere interpretato ed
			      elaborato dal computer
			\item vogliamo codificare il testo per quello che è, non per quello che
			      sembra
			\item codifica da effettuare mediante linguaggio di markup
		\end{itemize}

	\end{block}

\end{frame}


\begin{frame}
	\frametitle{I linguaggi di codifica}
	\framesubtitle{Linguaggi di marcatura}
	\addtocounter{nframe}{1}

	\begin{block}{Il markup}
		Il termine markup è stato utilizzato in passato per denotare i segni grafici che accompagnavano un testo apposti sul documento per indicare correzioni o modalità grafiche di stampa.
	\end{block}

\end{frame}

\begin{frame}
	\frametitle{I linguaggi di codifica}
	\framesubtitle{Linguaggi di marcatura}
	\addtocounter{nframe}{1}

	\begin{center}
		\includegraphics[width=.7\textwidth]{imgs/markup001.jpg}
	\end{center}

\end{frame}

\begin{frame}
	\frametitle{I linguaggi di codifica}
	\framesubtitle{Linguaggi di marcatura}
	\addtocounter{nframe}{1}

	\begin{center}
		\includegraphics[width=.8\textwidth]{imgs/MarkupConvention.png}
	\end{center}

\end{frame}

\begin{frame}
	\frametitle{I linguaggi di codifica}
	\framesubtitle{Linguaggi di marcatura}
	\addtocounter{nframe}{1}

	\begin{block}{Il markup}
		La codifica con linguaggi di marcatura (markup) è in definitiva un insieme di convenzioni, rese attraverso specifiche sequenze di caratteri, etichette, codici, (detti tags) intercalati nel testo per permettere agli elaboratori elettronici di distinguere le varie parti di un documento.
	\end{block}

	\begin{block}{Il markup formale}
		Un linguaggio di markup è un sistema formale per scambiare e pubblicare informazioni in formato testo in modo strutturato.
	\end{block}


\end{frame}

\begin{frame}
	\frametitle{I linguaggi di codifica}
	\framesubtitle{Linguaggi di marcatura}
	\addtocounter{nframe}{1}

	\begin{block}{Il markup formale}
		Markup formale: costituito da un sistema ben preciso di istruzioni, ognuna delle quali è dotata di una specifica semantica e sintassi.
	\end{block}


\end{frame}

\begin{frame}
	\frametitle{I linguaggi di codifica}
	\framesubtitle{Linguaggi di marcatura}
	\addtocounter{nframe}{1}

	\begin{block}{Diversi tipi di markup}
		Esistono diversi linguaggi di markup, per rappresentare diversi tipi di documenti.
		\begin{itemize}
			\item Linguaggi procedurali (specific markup languages)
			\item Linguaggi dichiarativi (generic markup languages)
		\end{itemize}
	\end{block}
\end{frame}

\begin{frame}
	\frametitle{I linguaggi di codifica}
	\framesubtitle{Linguaggi di marcatura procedurale}
	\addtocounter{nframe}{1}

	\begin{block}{Linguaggi procedurali}
		\begin{itemize}
			\item Orientati al documento, indicano come deve essere elaborato e
			      disposto il testo
			\item Istruzioni da inserire nel testo per specificarne specifiche
			      caratteristiche
			\item Font, dimensione, spaziatura del carattere, posizionamento
			      nella pagina, colore, etc.
		\end{itemize}
	\end{block}

	\textit{Esempi: TeX e LaTeX, RTF}

\end{frame}

\begin{frame}[fragile]
	\frametitle{I linguaggi di codifica}
	\framesubtitle{Linguaggi di marcatura procedurale}
	\addtocounter{nframe}{1}

	\defverbatim{\rtf}{%
		\begin{tiny}
			\begin{verbatim}

{\rtf1\ansi\deff0\adeflang1025
{\fonttbl{\f0\froman\fprq2\fcharset0 Times New Roman;}
{\f1\froman\fprq2\fcharset0 Times New Roman;}
{\f2\fnil\fprq2\fcharset0 Lucida Sans Unicode;}
{\colortbl;\red0\green0\blue0;\red128\green128\blue128;}
{\stylesheet{\s1\cf0{\*\hyphen2\hyphlead2\hyphtrail2\hyphmax0}
\rtlch\af5\afs24\lang255\ltrch\dbch\af2\afs24\langfe255\loch\f0\fs24\lang1040\snext1 Standard;}

        \end{verbatim}
		\end{tiny}
	}

	\begin{block}{Esempio RTF}
		{\rtf}
	\end{block}

\end{frame}

\begin{frame}
	\frametitle{I linguaggi di codifica}
	\framesubtitle{Linguaggi di marcatura procedurale}
	\addtocounter{nframe}{1}

	\begin{block}{Esempio LaTex}
        \begin{tiny}
            Ciao
			% \backslash documentclass[a4paper,10pt]$\{article\}$\\
			% \backslash usepackage[utf8]$\{inputenc\}$\\
			% \backslash usepackage[T1]$\{fontenc\}$\\
			% \backslash usepackage[italian]$\{babel\}$\\
			% \backslash title$\{Il mio primo documento\}$\\
			% \backslash author$\{Angelo Mario Del Grosso\}$\\
			% \backslash begin$\{document\}$\\
			% \backslash maketitle\\
			% \backslash begin$\{abstract\}$\\
			% Primo tentativo di scrivere in \backslash LaTeX .\\
			% \backslash end$\{abstract\}$\\
			% \backslash section$\{titolo della sezione\}$\\
			% Questo documento è vuoto.\\
			% \backslash footnote$\{nota a piè di pagina.\}$\\
			% \backslash end$\{document\}$\\
		\end{tiny}
	\end{block}

\end{frame}

\begin{frame}
	\frametitle{I linguaggi di codifica}
	\framesubtitle{Linguaggi di marcatura}
	\addtocounter{nframe}{1}

	\begin{block}{Il markup procedurale}
		L’unico utilizzo di un testo codificato tramite un linguaggio procedurale è la creazione di un output orientato alla visualizzazione.
	\end{block}
\end{frame}

\begin{frame}
	\frametitle{I linguaggi di codifica}
	\framesubtitle{Linguaggi di marcatura procedurale}
	\addtocounter{nframe}{1}

	\begin{center}
		\includegraphics[width=.8\textwidth]{imgs/LatexDoc.jpg}
	\end{center}

\end{frame}

\begin{frame}
	\frametitle{I linguaggi di codifica}
	\framesubtitle{Linguaggi di marcatura dichiarativi}
	\addtocounter{nframe}{1}

	\begin{block}{Linguaggi dichiarativi}
		Orientati al testo, annotano la struttura, la funzione ed il significato degli elementi costitutivi
		del testo, tralasciandone l’aspetto.
		\begin{itemize}
			\item la posizione che il brano in questione occupa all’interno del documento (markup strutturale)
			\item peculiarità del testo stesso (markup semantico)
			\item I fogli di stile definiscono la formattazione dell’output
			\item Molteplici usi del medesimo testo
		\end{itemize}
	\end{block}
	\textit{Esempio: famiglia SGML, XML}
\end{frame}

\begin{frame}
	\frametitle{I linguaggi di codifica}
	\framesubtitle{Linguaggi di marcatura dichiarativi}
	\addtocounter{nframe}{1}

	\begin{block}{Markup dichiarativi: contenuto e presentazione}
		La separazione tra contenuto e presentazione non solo è intenzionale, ma è la caratteristica principale di questi sistemi di marcatura: essa permette di concentrarsi sull'annotazione logica-semantica per funzioni di ricerca e di analisi, lasciando ad altro (ai fogli di stile) la resa grafica.
	\end{block}

	\begin{block}{Unico testo più usi}
		In questo modo si ha inoltre la possibilità di utilizzare uno stesso testo codificato con
		finalità o formattazioni differenti, a seconda delle varie esigenze.
	\end{block}

\end{frame}

\begin{frame}
	\frametitle{I linguaggi di codifica}
	\framesubtitle{Markup dichiarativi: esempio SGML}
	\addtocounter{nframe}{1}

	\begin{center}
		\includegraphics[width=.8\textwidth]{imgs/testo-sgml.png}
	\end{center}

\end{frame}

\begin{frame}
	\frametitle{I linguaggi di codifica}
	\framesubtitle {Markup dichiarativi vs Markup procedurali}
	\addtocounter{nframe}{1}

	\begin{center}
		\includegraphics[width=.8\textwidth]{imgs/Procedurale-Dichiarativo.png}
	\end{center}

\end{frame}

\begin{frame}
	\frametitle{I linguaggi di codifica}
	\framesubtitle{Linguaggi di marcatura}
	\addtocounter{nframe}{1}

	\begin{block}{linguaggi semi-dichiarativi e/o semi-procedurali}
		Esistono anche linguaggi che possono essere definiti
		semi-procedurali, o semi-dichiarativi, che come si intuisce utilizzano le istruzioni sia
		per una codifica di tipo procedurale, sia per una codifica di tipo descrittivo o
		dichiarativo. Uno degli esempi più famosi è HTML che tra le sue etichette mescola
		istruzioni di tipo procedurale per indicare come devono essere rese determinate
		porzioni di testo, a istruzioni di tipo dichiarativo che hanno una base semantica e
		vengono rese in modo differente a seconda del browser utilizzato.
	\end{block}

\end{frame}


\section{Fondamenti del linguaggio XML}
%%  la codifica testuale è la rappresentazione formale di un testo e delle sue caratteristiche mediante un linguaggio informa- tico. (ciotti)

% le descrizioni per la rappresentazione del testo devono essere opportunamente formalizzati per poter essere "comprensibili" dal colcolatore/sistema software.

% La codifica informatica del testo è un linguaggio teorico che per- mette allo studioso di costruire modelli formali del testo.


% La rappresentazione che vogliamo eseguire deve essere eseguira mediante le istruzioni, le convenzioni e i costrutti  messi a disposizione da un opportuno linguaggio che sarà definito formalmente da una specifica sintassi e da una precisa semantica.
% linguaggio in cui tutti i termini sono definiti esplicitamente e usati in modo conforme a tali definizioni.

% Si deve notare che «ogni dato su cui l’elaboratore deve operare viene rappresentato a livello elementare mediante una sequenza (o stringa) di simboli

% Tra questi hanno una notevole importanza ai fini della modellizzazione di testi, quei sistemi basati sui cosiddetti markup language.

% Il termine inglese markup designava nella stampa tipografica tutte le indicazioni e annotazioni simboliche aggiunte dall’autore o dall’editore su un manoscritto o su un dattiloscritto per istruire il tipografo

% Similmente un markup language è costituito da un set di istruzioni di un vero e proprio linguaggio orientato alla descrizioni dei fenomeni di composizione e struttura del testo.

% Un linguaggio di Markup, quindi, è un formalismo artificiale con il quale poter esprimente la rappresentazione o il modello del testo considerato.
% Un linguaggio (formale) sull'alfabeto A non è altro che un sottoinsieme di A*. Una grammatica formale serve proprio a definire un certo sottoinsieme di stringhe tra tutte quelle possibili su un dato alfabeto.

% Crediamo che una adeguata soluzione pragmatica di questi problemi sia da individuare nel- l’associazione di un modello implementato da un markup language, fin dove è possibile, e di un modello grafico del documento, implementato in uno dei formati grafici standard, tra quelli sviluppati nell’ambito delle tecnologie di computer grafica.

% Una tale forma di modellizzazione informatica di un do- cumento testuale risulterebbe, peraltro, la più adeguata nel caso speci- fico dei manoscritti, per i quali nessuna descrizione di tipo linguistico sarebbe in grado di rappresentare tutte le informazioni visuali che una immagine digitale è in grado di veicolare.

% l’applicazione di metodologie computazionali nell’ambito della ricerca umanistica comporta due tipi, o meglio due fasi di formalizzazione:
• definizione e implementazione di strutture dati adeguate alla cattura dei fenomeni di interesse dell’umanista, e in particolare alla rappresentazione formale dei testi;
• specificazione di algoritmi che, applicati alle strutture dati, sia- no in grado di simulare i processi di manipolazione dei testi ti- pici della ricerca umanistica o in generale delle pratiche sociali che hanno a che fare in vario modo con i testi.
Il problema della codifica testuale rientra in generale nel primo tipo di formalizzazione,

% definizione e implementazione di un linguaggio formale che deve essere a un tempo processabile da un elaboratore e sufficientemente espressivo per rappresentare la complessità dell’oggetto testo.

%Più precisamente uno schema di codifica associa un insieme di caratteristiche o elementi costituenti di un oggetto testuale a un insieme di simboli, e le relazioni tra gli elementi testuali a relazioni sintattiche tra i simboli.
%% Un esempio..

% I linguaggi per la codifica testuale vengono denominati nella lette- ratura anglosassone markup language, linguaggi di marcatura.

% i linguaggi di markup infatti, consistono di un insieme di simboli che vengono inse- riti all’interno o accanto al testo verbale.

% Codificare un testo significa esplicitare i processi inferenziali effet- tuati da un interprete nella comprensione del testo stesso.

% Il principale requisito di uno schema di codifica, pertanto, è la capacità rappresen- tazionale che esso offre allo studioso

% rappresentare adeguatamente i differenti fenomeni testuali che vengono studiati da varie discipline;

% In modo parallelo ai linguaggi di programmazione, anche i linguaggi di markup possono essere divisi in due tipologie: linguaggi procedurali, che nella letteratura vengono indicati anche come specific markup language; e linguaggi dichiarativi o descrittivi, detti anche generic markup language10.

% I sistemi di codifica procedurale sono per definizione orientati a una singola applicazione. la portabilità di un testo co- dificato con sistemi procedurali è molto limitata.

% invece di specificare quali operazioni di formattazione vanno effettuate in un particolare punto del testo, si di- chiara che un dato segmento testuale è istanza di un tipo di struttura editoriale del testo; insomma, si dichiara: “questo è un titolo”

% Un sistema di codifica dichiarativo dunque è orientato alla rappresentazione delle caratteristiche o elementi che costituiscono un testo, indipenden- temente dalle finalità specifiche per le quali il testo è stato memorizzato e codificato.
%Come i linguaggi procedurali, anche quelli dichiarativi vengono u- tilizzati inserendo all’interno del file di testo sequenze di caratteri. generalmente dette tag (etichette o marche)

% In ultima analisi, la codifica informatica di un testo può essere vista come il prodotto di un insieme di inferenze che vengono espresse mediante un linguaggio formalizzato. (ciotti)

% La codifica informatica di un testo provvede un sistema linguistico formalizzato che permette a uno studioso di «rendere esplicita una in- terpretazione di un testo» [Burnard, 1995: 43], e le varie operazioni inferenziali implicite che la hanno prodotta. I sistemi dichiarativi forniscono un potente dispositivo metalinguistico.

%La visione pluralista del testo portata alle sue estreme conseguenze, eccede i limiti sintattici di un formalismo di codifica come XML. Lo standard, infatti, non è dotato di costrutti sintattici adeguati alla rappresentazione di molteplici sottoprospettive gerarchi- che concorrenti che si sovrappongono ma che possono anche collegar- si e interrelarsi.

% non possiamo dire apriori che uno schema di codifica testuale coglie l’essenza del testo più e meglio di un altro in base a un qualche assun- to metafisico. Ma neppure si può affermare che ogni rappresentazione è vera in quanto costituisce il suo oggetto testo secondo esigenze spe- cifiche e locali.

Ogni model- lo descrive le caratteristiche del testo a un determinato livello, in base al punto di vista dell’osservatore, ma non coincide con esse.

% 

\section{Validare XML }
%% 	    %\includegraphics[width=.5\textwidth]{../imgs/comunicazioneUmana.png}
% definizione di segre
% definizione di tito orlandi

\begin{frame}
	\frametitle{Modellare il testo}
	\framesubtitle{Il testo come oggetto del dominio di studio}
	\addtocounter{nframe}{1}

	\begin{block}{Informatica nelle scienze umane}
		L'informatica umanistica ruota attorno alla rappresentazione e all'elaborazione degli oggetti che costituiscono il dominio delle discipline umanistiche.
	\end{block}

	\begin{block}{Testo come oggetto di studio}
		Il testo è tra questi, l'oggetto più ricorrente
	\end{block}

\end{frame}

\begin{frame}
	\frametitle{Modellare il testo}
	\framesubtitle{qual è la natura del testo?}
	\addtocounter{nframe}{1}

	\begin{block}{Modellare il testo}
		Per trattare e rappresentare il testo in ambiente digitale, bisogna formulare un suo modello
	\end{block}

	\begin{block}{Modello del testo}
		Un efficace modello del testo non può prescindere da una determinazione di cosa si intenda per testo e la sua natura. Il fatto è che la rappresentazione (e a maggior ragione l’elaborazione) informatica è ontologicamente formale in senso stretto (Ciotti).
	\end{block}

\end{frame}

\begin{frame}
	\frametitle{Modellare il testo}
	\framesubtitle{Testo: oggetto complesso}
	\addtocounter{nframe}{1}

	\begin{block}{Cos'è il testo}
		Il testo è un oggetto complesso in quanto è in grado di veicolare significato su più livelli strutturali (logico, ontologico, linguistico, autoriale, editoriale, fisico, etc), anche attraverso l’instaurazione di molteplici relazioni tra più livelli.
	\end{block}

	\begin{block}{Ma..}
		Non possediamo nessuna teoria sufficientemente completa del testo
	\end{block}

\end{frame}

\begin{frame}
	\frametitle{Modellare il testo}
	\framesubtitle{Testo: oggetto complesso}
	\addtocounter{nframe}{1}

	\begin{block}{Cos'è il testo}
		Si tratta di una entità informativa complessa in quanto è il prodotto di più agenti e più fattori
	\end{block}

\end{frame}

\begin{frame}
	\frametitle{Modellare il testo}
	\framesubtitle{Tentativo..}
	\addtocounter{nframe}{1}

	\begin{block}{Cos'è il testo}
		Quale concezione o modello ontologico del testo è implicata nella rappresentazione informatica di esso?
	\end{block}

	\begin{block}{A quali teorie possiamo rivolgerci?}
		Molteplici teorie del testo, quasi tutte sbilanciate sul livello verbale-semantico (linguistica testuale).
	\end{block}

\end{frame}

\begin{frame}
	\frametitle{Modellare il testo}
	\framesubtitle{Tentativo..}
	\addtocounter{nframe}{1}

	\begin{block}{Cos'è il testo}
		Il termine ``testo''  si riferisce a un oggetto plurale, in cui esiste
		\begin{itemize}
			\item Un livello astratto: la sequenza verbale, la quale a sua volta genera una serie di livelli di contenuti semantici.
			\item Un livello materiale: il supporto e le tracce d’inchiostro.
			\item Un livello dinamico: il testo viene creato da un autore e ricreato dal lettore.
		\end{itemize}

	\end{block}

\end{frame}


\begin{frame}
	\frametitle{Modellare il testo}
	\framesubtitle{Il testo: Cercando una definizione}
	\addtocounter{nframe}{1}

	\begin{block}{Cos'è il testo}
		\begin{itemize}
			\item Un documento materiale composto da fogli che contengono inchiostro variamente disposte?
			\item Un discorso linguistico fissato tramite la scrittura su un documento materiale?
			\item Un’opera dell’ingegno che viene costituita e fissata tramite un insieme di simboli/segni?
			\item Il contenuto di un messaggio?
		\end{itemize}
	\end{block}
\end{frame}

\begin{frame}
	\frametitle{Modellare il testo}
	\framesubtitle{Il testo: Cercando una definizione}
	\addtocounter{nframe}{1}

	\begin{block}{Cos'è il testo - uno sguardo più specialistico}
		\begin{itemize}
			\item Lo stato linguistico di un singolo testimone materiale di un’opera?
			\item Lo stato linguistico di un medesimo testimone di un’opera che presenta diverse lezioni identificabili?
			\item Una versione edita di un’opera?
			\item Una sequenza coerente di enunciati in una lingua naturale?
			\item Uno scritto che può essere trasmesso anche oralmente?
		\end{itemize}
	\end{block}

\end{frame}

\begin{frame}
	\frametitle{Modellare il testo}
	\framesubtitle{Definizioni}
	\addtocounter{nframe}{1}

	\begin{block}{Il testo}
		Dal latino \textit{textum}, participio passato di texere ``tessere'' quindi un testo è un ``tessuto'', un ordito di unità di significato (\textbf{monemi}) veicolato da simboli (\textbf{grafemi})
	\end{block}

\end{frame}

\begin{frame}
	\frametitle{Modellare il testo}
	\framesubtitle{Definizioni}
	\addtocounter{nframe}{1}

	\begin{block}{Il testo secondo Segre 1985}
		Il testo è dunque una successione fissa di significati grafici. Questi significati sono poi portatori di significati semantici.
	\end{block}

\end{frame}

\begin{frame}
	\frametitle{Modellare il testo}
	\framesubtitle{Definizioni}
	\addtocounter{nframe}{1}

	\begin{block}{Il testo secondo Tito Orlandi 2010}
		La produzione di un'attività linguistica intesa in senso stretto, cioè in una delle lingue storicamente date, secondo il concetto saussuriano di \textit{langue} che produce la \textit{parole}. Il fatto che essa sia finalizzata a trasmettere un messaggio, pur importante in sé, non interessa per il momento se non marginalmente.
	\end{block}

\end{frame}


\begin{frame}
	\frametitle{Modellare il testo}
	\framesubtitle{Definizioni}
	\addtocounter{nframe}{1}

	\begin{block}{Ancora Tito Orlandi}
		Il Testo si presenta al lettore sotto la forma di rappresentazione di un'espressione linguistica, e poiché l'attività linguistica è multiforme, potrà essere una rappresentazione puramente mentale, astratta, ovvero fonica (verbale), ovvero grafica (scrittura).
	\end{block}


\end{frame}


\begin{frame}
	\frametitle{Modellare il testo}
	\framesubtitle{Definizioni}
	\addtocounter{nframe}{1}

	\begin{center}
		\includegraphics[width=.9\textwidth]{imgs/TestoDante.png}
	\end{center}

	\begin{block}{Possiamo concludere che}
		Il testo è l'invariante, la successione di valori, rispetto alle variabili dei caratteri, della scrittura.
	\end{block}

\end{frame}


% Teso è l'invariante rispetto ai segni, è la succesione di valori invariabili. Il testo, dunque è ciò che permane, l’invariante, in ogni operazione di riproduzione materiale della sequenza di simboli grafici.

% Questa definizione di testo come un oggetto astratto allografico sembra fornire un criterio di individuazione di un testo in base al principio di identità per sostituzione

\begin{frame}
	\frametitle{Modellare il testo}
	\framesubtitle{Complessità del testo}
	\addtocounter{nframe}{1}

	\begin{block}{Il testo oggetto complesso multidimensionale/pluralista}
		\begin{itemize}
			\item analisi grafematica (caratteri, ideogrammi)
			\item analisi strutturale (struttura editoriale, interna)
			\item analisi metrica (piedi, versi, stanze, strofe)
			\item analisi stilistica e retorica
			\item analisi tematica
			\item analisi semantica
			\item ecc..
		\end{itemize}
	\end{block}

\end{frame}

\begin{frame}
	\frametitle{Modellare il testo}
	\framesubtitle{Il testo letterario}
	\addtocounter{nframe}{1}

	\begin{block}{Particolarità del testo letterario}
		\begin{itemize}
			\item L'emittente prepara il messaggio, il destinatario potrebbe riceverlo anche dopo molto tempo
			\item Il feedback molto spesso non è possibile
			\item Il codice dell'emittente e quello del destinatario possono risultare diversi (o solo parzialmente compatibili)
		\end{itemize}
	\end{block}

\end{frame}


% Testo vs Documento


\begin{frame}
	\frametitle{Modellare il testo}
	\framesubtitle{Il documento}
	\addtocounter{nframe}{1}

	\begin{block}{Il Testo non è il Documento}
		La parola documento deriva dalla parola latina ``documentum'' che ha la stessa radice di ``doceo'' che significa ``insegnare''/``dimostrare'' con l'aggiunta del suffisso ``-umentum'' che intende lo strumento per fare.
	\end{block}

	\begin{block}{Il Documento}
		La parola ``documento'' designa quindi uno strumento per insegnare o uno strumento atto a dimostrare qualcosa.
	\end{block}


\end{frame}

\begin{frame}
	\frametitle{Modellare il testo}
	\framesubtitle{Il documento: Cercando una definizione}
	\addtocounter{nframe}{1}

	\begin{block}{Cos'è un documento}
		\begin{itemize}
			\item Una prova o un evidenza?
			\item Un atto cartaceo originale e/o ufficiale attraverso il quale dimostrare o provare qualcosa?
			\item Qualsiasi cosa materiale che serva come evidenza o prova?
			\item Un qualcosa di scritto contenente informazioni?
			\item Un supporto materiale sul quale sono stati rappresentati fatti o pensieri attraverso un qualche tipo di convenzione fatta di simboli e segni?
		\end{itemize}

	\end{block}

\end{frame}


%The first definition is more general. 
%The second one catches the most intuitive association of a document to a paper support, 
%while the third one extends the definition to all other kinds of support that may have the same function. 
%While all these definitions mainly focus on the bureaucratic, administrative or juridic aspects
%of documents, 
%the fourth and fifth ones are more interested in its role of information bearer that can be exploited for study, research information. 
%Again, the former covers the classical meaning of documents as written papers, while the latter extends it to any kind of support and representation. 
%Summing up, three aspects can be considered as relevant in identifying a document: its original meaning is captured
%by definitions 4 and 5, while definition 1 extends it to underline its importance as a proof of something, and definitions 2 and 3 in some way formally recognize this role in the social establishment.



\begin{frame}
	\frametitle{Modellare il testo}
	\framesubtitle{Definizioni}
	\addtocounter{nframe}{1}

	\begin{block}{Testo}
		Il testo è una entità astratta invariante, che in ogni operazione di realizzazione materiale della sequenza di simboli grafici determina la struttura fisica di un oggetto sensibilmente concreto (ovvero capace di attivare uno dei canali recettivi dell’uomo verso stimoli esterni).
	\end{block}

	\begin{block}{Documento}
		L'oggetto materiale, sensibile, concreto che costituisce il supporto, stabile e riproducibile dell’informazione testuale che determina una modalità d’interazione (lettura, decifrazione).
	\end{block}

\end{frame}

\begin{frame}
	\frametitle{Modellare il testo}
	\framesubtitle{Struttura del testo}
	\addtocounter{nframe}{1}

	\begin{block}{Elementi di un testo}
		Oltre alla sequenza di grafemi, un testo si presenta strutturato in segmentazioni logiche e partizioni interne di interi blocchi
	\end{block}

	\begin{block}{Struttura di un testo}
		Il testo ha una certa struttura, i cui elementi sono determinati dalla struttura logico-semantica del discorso con finalità e funzioni ben distinte (titolo di un capitolo, corpo di un paragrafo)
	\end{block}

\end{frame}

% esiste anche un modello documento
\begin{frame}
	\frametitle{Modellare il testo}
	\framesubtitle{Struttura del documento}
	\addtocounter{nframe}{1}

	\begin{block}{Struttura di un documento}
		Un documento, ugualmente, è costituito da elementi significativi, non solo verbali, da esplicitare formalmente (layout, segmentazione, foliazione, aspetti paratestuali, aspetti tipografici, aspetti topografici).
		% tipografica: Della tipografia, che ha relazione con la tipografia, cioè con la stampa
		%topografica: dislocazione o posizione reciproca dei varî elementi o delle varie parti che concorrono a formarlo
	\end{block}

\end{frame}

\begin{frame}
	\frametitle{Modellare il testo}
	\framesubtitle{Codifica del documento}
	\addtocounter{nframe}{1}

	\begin{block}{Il testo non è il documento}
		La codifica di un documento è diversa dalla codifica di un
		testo \textbf{(testo != documento)}. Si può parlare di documenti cartacei e di documenti
		digitali, ma non di testi cartacei e testi digitali!
	\end{block}

	\begin{block}{Documento digitale}
		Corrispondenza perfetta con l'originale: documento digitale
	\end{block}


\end{frame}


\begin{frame}
	\frametitle{Modellare il testo}
	\framesubtitle{Modelli concettuali}
	\addtocounter{nframe}{1}

	\begin{block}{Il testo è strutturato}
		Il testo è un oggetto reale dotato di una sua struttura che corrisponde alla struttura del linguaggio di rappresentazione;
	\end{block}

	\begin{block}{La natura del testo come OHCO}
		In un famoso articolo (DeRose et al., 1990) gli autori tentarono di dimostrare che il testo è una ``ordered hierarchy of content objects (OHCO)'': una gerarchia ordinata di oggetti di contenuto.
	\end{block}
\end{frame}


\begin{frame}
	\frametitle{Modellare il testo}
	\framesubtitle{ordered hierarchy of content objects (OHCO)}
	\addtocounter{nframe}{1}

	\begin{block}{La natura del testo come OHCO}
		Gli oggetti di contenuto testuale sono perlopiù le strutture editoriali astratte di cui si compone un testo (divisioni logiche, segmentazioni e partizioni di un libro).
	\end{block}

	\begin{block}{La natura del testo come OHCO}
		Questi oggetti sono gerarchici e di contenuto poiché tali elementi ne possono contenere altri, ma sempre che veicolano contenuto.  Sono ordinati in quanto esiste una relazione lineare tra due oggetti posti sul medesimo livello gerarchico (pensiamo alle parole di una frase all'interno di un paragrafo).
	\end{block}
\end{frame}

\begin{frame}
	\frametitle{Modellare il testo}
	\framesubtitle{OHCO come primo modello concettuale TEI}
	\addtocounter{nframe}{1}

	\begin{block}{OHCO e TEI}
		Il primo modello implementato dalla Text Encoding Initiative si è basato su questo impianto teorico.
	\end{block}

	\begin{block}{OHCO e TEI}
		Sono però stati riscontrati una serie di problemi di rappresentazione che determinano diversi insiemi di elementi di contenuto i quali non possono essere ricondotti a una struttura gerarchica unitaria (pensiamo alla rappresentazione di parole e di righe).
	\end{block}

\end{frame}

\begin{frame}
	\frametitle{Modellare il testo}
	\framesubtitle{Prospettive analitiche}
	\addtocounter{nframe}{1}

	\begin{block}{Strutture sovrapposte}
		Gli elementi di contenuto che si sovrappongono si comportano come se appartenessero a diverse gerarchie di oggetti testuali.
	\end{block}

	\begin{block}{Prospettiva analitica}
		Ogni prospettiva analitica su un testo determina una struttura gerarchica di oggetti di contenuto. (F. Ciotti)
		%Un corollario operativo di questa tesi è che se due elementi a e b si sovrappongono, allora essi appartengono a due diverse prospettive analitiche.
	\end{block}
\end{frame}

\begin{frame}
	\frametitle{Modellare il testo}
	\framesubtitle{Sottoprospettive}
	\addtocounter{nframe}{1}

	\begin{block}{Sottoprospettive analitiche}
		Se due oggetti testuali evidenziati da una prospettiva teorica si sovrappongono, allora essi appartengono rispettivamente a due sottoprospettive diverse della prospettiva teorica principale.(F. Ciotti)
	\end{block}

	\begin{block}{Sottoprospettive analitiche}
		In questa ottica il testo diventa un sistema a più livelli, che corrispondono a diversi punti di vista dell’osservatore (sistema multidimensionale). (F. Ciotti)
	\end{block}

\end{frame}

\begin{frame}
	\frametitle{Modellare il testo}
	\framesubtitle{La struttura gerarchica del testo}
	\addtocounter{nframe}{1}

	\begin{block}{Le gerarchie del testo}
		Le strutture del testo sono strutture gerarchiche: le sottoprospettive, comunque esse siano definite, danno luogo a gerarchie.
	\end{block}
\end{frame}

%\begin{frame}
%	\frametitle{Modellare il testo}
%	\framesubtitle{}
%	\addtocounter{nframe}{1}
%
%	\begin{block}{Blocco}
%		Alcune teorie avanzano la necessità di abbandonare l’assunto ontologico che il testo sia un oggetto reale del mondo, dotato di una struttura intrinseca: Il testo è dunque una entità che viene costruita e non scoperta e analizzata dalla attività scientifica. soluzione interessante e intellettualmente stimolante alle aporie determinate dalla teoria gerarchica del testo, specialmente nella tematizzazione del ruolo dell’osservatore nei processi di rappresentazione.
%	\end{block}
%\end{frame}

\begin{frame}
	\frametitle{Modellare il testo}
	\framesubtitle{OHCO}
	\addtocounter{nframe}{1}

	\begin{block}{OHCO: che cosa è un testo?}
		 Un testo è un oggetto linguistico astratto organizzato secondo una struttura gerarchica ordinata di oggetti di contenuto (e varie successive riformulazioni). 
	\end{block}

	\begin{block}{OHCO: ricapitolando}
		L’idea di una preminenza della struttura gerarchica nella testualità ha mantenuto un ruolo descrittivo ed esplicativo essenziale.
	\end{block}
	
\end{frame}

%\begin{frame}
%	\frametitle{Modellare il testo}
%	\framesubtitle{}
%	\addtocounter{nframe}{1}
%
%	\begin{block}{Blocco}
%		il genere determina gli elementi che costituiscono il testo, quindi il tipo di documento. Reciprocamente un genere testuale è individuato dalla classe di oggetti di contenuto che contiene.
%	\end{block}
%\end{frame}

% Con il termine grafema si indica il segno elementare e non ulteriormente suddivisibile che costituisce l'unità minima dei sistemi di scrittura: 

\section{Document Typed Definition - DTD}
%% Vedere Slide CHiara. Roberto
% Portabilità e riutilizzabilità
% schema di codifica
% TEI XML focuses on the meaning of text, rather than its appearance.

\begin{frame}
	\frametitle{Importanza della codifica digitale}
	\framesubtitle{Perché effettuare la codifica}
	\addtocounter{nframe}{1}

	\begin{block}{Digitalizzare un testo}
		Digitalizzare per poter favorire l'elaborazione e il trattamento automatico dei testi
	\end{block}

	\begin{block}{Trattamento dei testi}
        \begin{itemize}
            \item  analisi di tipo linguistico (linguistica computazionale,
            database testuali, corpora linguistics)
            \item analisi di altro tipo (metrica, stilistica, etc.)
            \item ricerca testuale avanzata
            \item pubblicazione in vari formati (sul web, come ebook, a
            stampa)
            \item didattica
        \end{itemize}
 
    \end{block}
\end{frame}

\begin{frame}
	\frametitle{Importanza della codifica digitale}
	\framesubtitle{Perché effettuare la codifica}
	\addtocounter{nframe}{1}

	\begin{block}{Digitalizzare un testo}
		Per facilitare e garantire una universalità di accesso al loro contenuto 
	\end{block}

	\begin{block}{Vantaggi della digitalizzazione}
        \begin{itemize}
            \item Edizioni elettroniche garantiscono diffusione capillare
            (via web) e nuove funzionalità (ipertesti, ricerca, etc.)
            \item Permettono anche di preservare i documenti più antichi
            (e fragili) riducendone la consultazione diretta
        \end{itemize}
     \end{block}
\end{frame}

\begin{frame}
	\frametitle{Importanza della codifica digitale}
	\framesubtitle{Perché effettuare la codifica}
	\addtocounter{nframe}{1}

	\begin{block}{Superare i problemi dei documenti digitali}
		\begin{itemize}
            \item disponibilità di hardware e software
            \item sistemi proprietari chiusi
            \item elevata obsolescenza e limitata manutenibilità
            \item difficile portabilità su piattaforme diverse
        \end{itemize}
	\end{block}
	
\end{frame}

\begin{frame}
	\frametitle{Importanza della codifica digitale}
	\framesubtitle{Perché effettuare la codifica}
	\addtocounter{nframe}{1}

	\begin{block}{Sistemi non adatti}
		\begin{itemize}
            \item word processing: WYSIWYG (What You See Is What You Get)
            \item sistemi proprietari chiusi (Word, Adobe, etc)
            \item elevata obsolescenza e limitata manutenibilità
            \item difficile portabilità su piattaforme diverse (windows, linux)
        \end{itemize}
	\end{block}
	
\end{frame}

\begin{frame}
	\frametitle{Importanza della codifica digitale}
	\framesubtitle{Perché effettuare la codifica}
	\addtocounter{nframe}{1}

	\begin{block}{Massimizzare le seguenti proprietà: Portabilità}
		\begin{itemize}
            \item indipendenza dall’hardware: processore, supporto, output
            \item indipendenza dal software: sistemi operativi, applicazioni di authoring, applicazioni di visualizzazione
            \item indipendenza dai sistemi di codifica dei caratteri
            \item indipendenza logica: da un particolare processo applicativo
        \end{itemize}
	\end{block}
	
\end{frame}




%La codifica dell’informazione gode delle seguenti proprietà:
% indipendenza dall’hardware, ovvero da una particolare architettura elaborativa (processore), da un particolare supporto digitale (disco magnetico, disco ottico, etc.), o da un particolare dispositivo o sistema di output (video, stampa);
% indipendenza dal software, sia sistemi operativi, sia applicazioni deputate alla creazione, analisi, manipolazione e visualizzazione di testi elettronici; (formati di dati proprietari mutamente incompatibili)
% indipendenza logica dalle applicazioni ovvero indipendenza semantica dello schema di codifica da un particolare processo applicativo.

% L’archiviazione su supporto digitale del patrimonio letterario e culturale delle culture mondiali deve misurarsi con questi problemi, e adottare degli schemi di codifica capaci di garantire la massima portabilità. 



% Una risorsa informativa digitale è portabile se è intercambiabile tra sistemi diversi, riutilizzabile in molteplici processi computazionali anche a distanza di tempo, e integrabile da ulteriori risorse informative omogenee


%  Esso deve divenire uno standard
% I vantaggi di uno standard formale o informale, oltre alla portabilità sta anche nella sua apertura, ovvero nella disponibilità pubblica delle sue specifiche.



% La disposizione alla rappresentazione di strutture astratte non pone limiti alla natura e tipologia delle caratteristiche testuali che si possono codificare in un testo elettronico. Queste possono essere utilizzate indifferentemente 

% Se il linguaggio è dotato di una sintassi che permette di specificare le relazioni tra gli elementi, essa può essere usata per rappresentare la struttura e l’organizzazione del testo a un determinato livello di descrizione, o i rapporti tra elementi appartenenti a diversi livelli.


% un sistema di codifica dichiarativa assiste un autore nel processo di scrittura, poiché focalizza l’attenzione sul contenuto di un testo (o sulla struttura del contenuto) piuttosto che sulla sua forma grafica.

% I sistemi di markup dichiarativo introducono consistenti vantaggi anche nei processi produttivi editoriali e nella gestione dei flussi informativi aziendali. Poiché un medesimo schema di codifica dichiarativo può essere utilizzato in molteplici forme di trattamento informatico, i costi di produzione e gestione di una base dati testuale vengono fortemente ridotti.

% I sistemi di codifica dichiarativa peraltro si prestano ottimamente per rappresentare strutture complesse come riferimenti incrociati e collegamenti tra elementi all’interno di un testo, ma anche tra più testi

% Infatti un database offre dei notevoli vantaggi dal punto di vista delle prestazioni computazionali e della velocità di ricerca, anche se richiede in generale una ingente quantità di memoria per l’archiviazione.

% La codifica permette allo studioso di esplicitare le sue ipotesi interpretative.

% OHCO: efficienza computazionale che la struttura gerarchica mostra; 

% I linguaggi di markup dichiarativi permettono di predicare l’appartenenza di un dato segmento testuale a una classe di strutture testuali definita dall’utente;
%  Così è possibile descrivere formalmente le caratteristiche di un testo in modo indipendente da particolari finalità di trattamento
% da contingenti forme di presentazione grafica su un qualsivoglia supporto fisico

% I linguaggi di markup dichiarativi, e in particolare SGML e XML, si sono rivelati dei veri e propri strumenti di supporto all’analisi computazionale dei testi
% la sintassi del linguaggio di codifica può essere usata per rappresentare le relazioni tra gli elementi strutturali di un testo, a un determinato livello di descrizione.



%\section{Linguaggi di marcatura}
%% VEdere Slide Rosselli-Chiara lezione sui markup languages (le prime slide)

% intro linguaggi di Markup

% La rappresentazione che vogliamo eseguire deve essere eseguira mediante le istruzioni, le convenzioni e i costrutti  messi a disposizione da un opportuno linguaggio che sarà definito formalmente da una specifica sintassi e da una precisa semantica.
% linguaggio in cui tutti i termini sono definiti esplicitamente e usati in modo conforme a tali definizioni.

% Si deve notare che «ogni dato su cui l’elaboratore deve operare viene rappresentato a livello elementare mediante una sequenza (o stringa) di simboli


% In modo parallelo ai linguaggi di programmazione, anche i linguaggi di markup possono essere divisi in due tipologie: linguaggi procedurali, che nella letteratura vengono indicati anche come specific markup language; e linguaggi dichiarativi o descrittivi, detti anche generic markup language.

% I sistemi di codifica procedurale sono per definizione orientati a una singola applicazione. la portabilità di un testo codificato con sistemi procedurali è molto limitata.


% nei linguaggi di markup dichiarativi/descrittivi invece di specificare quali operazioni di formattazione vanno effettuate in un particolare punto del testo, si dichiara che un dato segmento testuale è istanza di un tipo di struttura editoriale del testo; insomma, si dichiara: “questo è un titolo”


% Un sistema di codifica dichiarativo dunque è orientato alla rappresentazione delle caratteristiche o elementi che costituiscono un testo, indipendentemente dalle finalità specifiche per le quali il testo è stato memorizzato e codificato.

% Tra questi hanno una notevole importanza ai fini della modellizzazione di testi, quei sistemi basati sui cosiddetti markup language.

% Il termine inglese markup designava nella stampa tipografica tutte le indicazioni e annotazioni simboliche aggiunte dall’autore o dall’editore su un manoscritto o su un dattiloscritto per istruire il tipografo

% Similmente un markup language è costituito da un set di istruzioni di un vero e proprio linguaggio orientato alla descrizioni dei fenomeni di composizione e struttura del testo.

% i linguaggi di markup infatti, consistono di un insieme di simboli che vengono inseriti all’interno o accanto al testo verbale.

% Un linguaggio di Markup, quindi, è un formalismo artificiale con il quale poter esprimente la rappresentazione o il modello del testo considerato.
% Un linguaggio (formale) sull'alfabeto A non è altro che un sottoinsieme di A*. Una grammatica formale serve proprio a definire un certo sottoinsieme di stringhe tra tutte quelle possibili su un dato alfabeto.



%Come i linguaggi procedurali, anche quelli dichiarativi vengono utilizzati inserendo all’interno del file di testo sequenze di caratteri. generalmente dette tag (etichette o marche)

%Più precisamente uno schema di codifica associa un insieme di caratteristiche o elementi costituenti di un oggetto testuale a un insieme di simboli, e le relazioni tra gli elementi testuali a relazioni sintattiche tra i simboli.
%% Un esempio (per esempio capitolo-titolo-paragrafo)..

\section{XML Schema Definition - XSD}
%% l’applicazione di metodologie computazionali nell’ambito della ricerca umanistica comporta due tipi, o meglio due fasi di formalizzazione:
% definizione e implementazione di strutture dati adeguate alla cattura dei fenomeni di interesse dell’umanista, e in particolare alla rappresentazione formale dei testi;
% specificazione di algoritmi che, applicati alle strutture dati, siano in grado di simulare i processi di manipolazione dei testi tipici della ricerca umanistica o in generale delle pratiche sociali che hanno a che fare in vario modo con i testi.

%% lo schema di codifica TEI impone al responsabile della codifica di effettuare delle scelte teoriche e interpretative che non sono pertinenti alla sua opera di semplice trascrittore.

%  uno di carattere epistemologico, riguarda la natura della codifica come processo di rappresentazione.
% carattere ontologico, e concerne il concetto generale di testo che «emerge» dalle teorie dei sistemi di codifica.

% domanda: codifica è un processo interpretativo oppure un processo riproduttivo?
%% lo schema di codifica TEI impone al responsabile della codifica di effettuare delle scelte teoriche e interpretative che non sono pertinenti alla sua opera di semplice trascrittore.


% la rappresentazione informatica è un processo semiotico: Ogni atto rappresentazionale o semiotico implica dei processi interpretativi 

% indagare più a fondo la natura della codifica e dell’idea di testo che la codifica presuppone.


% Naturalmente questo è possibile se tale descrizione del supporto fisico di un testo è riducibile a un struttura gerarchica.

% I problemi e le difficoltà determinati dagli schemi SGML per una codifica presentazionale in effetti, sono determinati proprio da questa metastruttura

% problema: utilizzazione dei simboli del linguaggio informatico in funzione di representare dei caratteri alfanumerici del testo

\begin{frame}
	\frametitle{Approfondimenti e Conclusioni}
	\framesubtitle{per comprendere la codifica}
	\addtocounter{nframe}{1}

	\begin{block}{Codifica del testo}
		L’applicazione di metodologie computazionali nell’ambito della ricerca umanistica comporta due aspetti di formalizzazione
	\end{block}

	\begin{block}{Due elementi}
		Formalizzazione dei dati e formalizzazione dell'elaborazione
	\end{block}

\end{frame}

\begin{frame}
	\frametitle{Approfondimenti e Conclusioni}
	\framesubtitle{per comprendere la codifica}
	\addtocounter{nframe}{1}

	\begin{block}{Formalizzazione dei dati}
		Definizione e implementazione di strutture dati adeguate alla cattura dei fenomeni di interesse dell’umanista, e in particolare alla rappresentazione formale dei testi;
	\end{block}

	\begin{block}{Formalizzazione dell'elaborazione}
		specificazione di algoritmi che, applicati alle strutture dati, siano in grado di simulare i processi di manipolazione dei testi tipici della ricerca umanistica o in generale delle pratiche sociali che hanno a che fare in vario modo con i testi.
	\end{block}

\end{frame}


\begin{frame}
	\frametitle{Approfondimenti e Conclusioni}
	\framesubtitle{per comprendere la codifica}
	\addtocounter{nframe}{1}

	\begin{block}{Codifica del testo}
		Il problema della codifica testuale rientra in generale nel primo tipo di formalizzazione (Dati).
	\end{block}

\end{frame}

\begin{frame}
	\frametitle{Approfondimenti e Conclusioni}
	\framesubtitle{per comprendere la codifica}
	\addtocounter{nframe}{1}

	\begin{block}{problemi teorici}
		La codifica è un processo assai più complesso delle semplice e meccanica correlazione biunivoca di strutture rappresentazionali.
	\end{block}

\end{frame}

\begin{frame}
	\frametitle{Approfondimenti e Conclusioni}
	\framesubtitle{per comprendere la codifica}
	\addtocounter{nframe}{1}

	\begin{block}{problema del testo}
		la specificazione di cosa sia un testo e di quale legame sussista tra questa specificazione, i processi dell’interpretazione e i linguaggi formali con i quali essa viene descritta.
	\end{block}

\end{frame}

\begin{frame}
	\frametitle{Approfondimenti e Conclusioni}
	\framesubtitle{per comprendere la codifica}
	\addtocounter{nframe}{1}

	\begin{block}{Norme TEI}
		Le linee guida di codifica TEI impongono a chi codifica di effettuare delle scelte teoriche e interpretative che non sono imputabili alla semplice trascrizione.
    \end{block}
    
    \begin{block}{Processo di codifica}
        La codifica è un processo interpretativo non solo un processo riproduttivo.
        \\Non è quindi un semplice processo di trascrizione!
    \end{block}
    

\end{frame}


\begin{frame}
	\frametitle{Approfondimenti e Conclusioni}
	\framesubtitle{per comprendere la codifica}
	\addtocounter{nframe}{1}

	\begin{block}{Codifica come interpretazione}
		Conseguentemente ogni processo di codifica (inclusi quelli di codifica informatica del testo) è il risultato di una interpretazione.
    \end{block}
    
    \begin{block}{Rappresentazione e interpretazione}
        In ogni caso non esiste nessun genere di rappresentazione di un testo che si possa definire libera da processi interpretativi.
    \end{block}

\end{frame}

\begin{frame}
	\frametitle{Approfondimenti e Conclusioni}
	\framesubtitle{per comprendere la codifica}
	\addtocounter{nframe}{1}

	\begin{block}{Esempio}
		L’assunzione che una data traccia grafica ``A'' sia una istanza di un data classe astratta di tracce che identifichiamo come il carattere ``a''. Richiesti molti sforzi interpretativi.
    \end{block}
    
    \begin{block}{Rappresentazione e interpretazione}
        In linea di principio non è sempre possibile dire in modo non ambiguo che una traccia su un supporto fisico appartiene a una certa classe di iscrizioni che chiamiamo carattere.
    \end{block}

\end{frame}


\begin{frame}
	\frametitle{Approfondimenti e Conclusioni}
	\framesubtitle{per comprendere la codifica}
	\addtocounter{nframe}{1}

	\begin{block}{Certezza e soggettività}
		\textbf{Ogni interpretazione può godere di diversi gradi di certezza e di soggettività. In ogni caso non esiste nessun genere di rappresentazione di un testo che si possa definire libera da processi interpretativi.}
    \end{block}
   
\end{frame}

\begin{frame}
	\frametitle{Approfondimenti e Conclusioni}
	\framesubtitle{per comprendere la codifica}
	\addtocounter{nframe}{1}

	\begin{block}{Linguaggio teorico}
		Lo schema di codifica è un linguaggio teorico usato per costruire teorie o modelli di fenomeni testuali
    \end{block}

    \begin{block}{Linguaggio teorico}
        La stessa costruzione di un linguaggio teorico riflette un determinato modello del mondo (soggettivo o condiviso).
        \\ \textit{si presuppone una teoria ontologica del testo}.
    \end{block}
   
\end{frame}

\begin{frame}
	\frametitle{Approfondimenti e Conclusioni}
	\framesubtitle{per comprendere la codifica}
	\addtocounter{nframe}{1}

	\begin{block}{Obiettivo}
		\begin{itemize}
			\item Sviluppare teorie e modelli formali del testo (o di alcuni suoi livelli descrittivi)
			\item Individuare formalismi atti a esprimerli in modo computazionalmente accettabile
		\end{itemize}
	
    \end{block}
   
\end{frame}



% OHCO: La ragione di tanto attaccamento all’idea di struttura gerarchica ovviamente non è immotivata. Il fatto è che XML (e SGML) può essere considerato sia un formalismo sia un modello di dati espresso da quel formalismo, e tale (meta)modello è appunto un albero ordinato etichettato.

% Il prezzo costituito dall’adozione di un modello di dati così vincolante, d’altra parte, paga il vantaggio di potere validare in modo automatico ogni istanza di dati rispetto al modello mediante algoritmi generali ben conosciuti e computazionalmente trattabili, ciò che a sua volta consente di costruire sistemi di elaborazione degli stessi dati consistenti ed efficaci (al netto dei costrutti ID/IDREF).

% Le manifestazioni di queste difficoltà sono state comunemente rubricate come il problema delle gerarchie sovrapposte (overlapping hierarchies)
% In termini semplici il problema OH dal punto di vista sintattico consiste nel fatto che, dati due oggetti logici presenti in un testo, le coppie di tag bilanciati che li rappresentano non si annidano propriamente ma si sovrappongono.
%Tale situazione è sintatticamente e semanticamente vietata in XML

% ipotetiche soluzioni
% Non è un caso che negli ultimi dieci anni, proprio in parallelo con l’inarrestabile diffusione di XML nel mondo dell’elaborazione testuale (e non solo) e della TEI nella comunità umanistica si sono moltiplicati i tentativi di trovare delle soluzioni definitive al problema.
% soluzioni interne e soluzioni esterne al paradigma XML. Per questo alla completezza e congruenza va affiancata una serie di ulteriori criteri valutativi di natura teorica, tecnica e pragmatica

% In generale tutte le soluzioni proposte finora hanno grossi limiti per quel che concerne la facilità di gestione e manutenzione

% La visione pluralista del testo portata alle sue estreme conseguenze, eccede i limiti sintattici di un formalismo di codifica come XML. Lo standard, infatti, non è dotato di costrutti sintattici adeguati alla rappresentazione di molteplici sottoprospettive gerarchiche concorrenti che si sovrappongono ma che possono anche collegarsi e interrelarsi.

% Ogni modello descrive le caratteristiche del testo a un determinato livello, in base al punto di vista dell’osservatore, ma non coincide con esse.

% esempio con codice XML e grafico/immagine








% problema della complessità testuale
%% Se esistono proprietà dei testi irriducibili a qualsiasi formalizzazione anche minimale, allora queste non possono per definizione essere rappresentate e trattate con metodi computazionali.
% nella categoria delle gerarchie sovrapposte si possono distinguere diversi sottoproblemi di complessità crescente
% distinguere una “gerarchia delle gerarchie sovrapposte”.
% 1. Il caso più semplice e comune è quello della compresenza di due o più strutture (livelli) gerarchiche i cui elementi si sovrappongono.
% 2. concettualmente più complicato è quello di un elemento appartenente a una gerarchia che si estende oltre i confini dell’elemento in cui inizia o persino di uno dei suoi predecessori
% 3. Tecnicamente simile ma concettualmente distinto il caso di elementi composti da segmenti discontinui e non contigui 
% 4. Infine il caso più complesso che si verifica quando un dato elemento può auto-sovrapporsi illimitatamente.


% Leggibilità e facilità di comprensione da parte di un utente umano (sono esclusi dunque tutti i formati binari)
% Facilità di manutenzione e modifica
% Disponibilità di implementazioni software
% Compatibilità sintattica con XML
% Facilità di validazione (eventualmente sopravanzando le capacità di validazione di un parser XML standard)
% Possibilità di validazione incrociata tra gerarchie diverse Possibilità di formattazione ed elaborazione grafica e presentazionale
% Possibilità di estrapolare molteplici viste basate su uno o più tra le gerarchie presenti
% Possibilità di estrapolare sottoinsiemi gerarchici delle caratteri- stiche testuali
% Continuità del contenuto testuale serializzato

% Soluzioni interne al paradigma XML rientrano artifici sintattici che mantengono la conformità a XML
% XML come puro formalismo di serializzazione per modelli di dati non gerarchici
% Segmentazione: Un elemento logico che si sovrappone ai confini di un’altro (o di più altri) viene diviso in due (o n) elementi XML dello stesso tipo correlati mediante apposti attributi.
% Questa soluzione permetterebbe di risolvere sovrapposizioni di tipo (1) , (2) e (3), e può essere parzialmente validata mediante un oculato uso di attributi ID/IDREF.
% Elementi di congiunzione: Consiste nell’introduzione di un elemento XML (elemento di congiunzione) con la funzione metatestuale di esprimere l’unità logica di un fenomeno testuale rappresentato da più elementi XML distinti. Questa tecnica consente di trattare in linea teorica ogni caso di sovrapposizione e di non contiguità, di ordinamento inverso e di relazione n-aria tra oggetti testuali. Questo ultimo problema può essere risolto parzialmente adottando gli schemi di puntamento basati su range di caratteri previsto nello standard XPointer ma in questo caso l’elaborazione dei riferimenti richiederebbe l’uso di software ad hoc.
% Markup esterno (stand-off markup): Le tecniche basate su markup esterno sono di fatto identiche a quelle basate su elementi join. La differenza consiste nel fatto che in questo caso gli elementi che esprimono il collegamento di segmenti testuali nel documento XML base sono in un documento XML esterno. Il fatto di poter collocare i collegamenti in un documento autonomo consente di adottare un linguaggio XML per descrivere una struttura principale del documento base e uno diverso per la rappresentazione dei livelli di descrizione ulteriori.

% Ad esempio si può usare la TEI per la codifica del documento e XTM (XML Topic Maps) o RDF per esprimere le relazioni tra gli oggetti testuali. (interessante). Questa tecnica ha il vantaggio di poter disporre di modelli di dati e di sistemi di elaborazione complessi.

% Elementi Milestone: Un elemento milestone è un elemento XML vuoto che segnala un punto monodimensionale in un documento XML. si possono collocare liberamente sintatticamente in un documento XML. Questa strategia è ampiamente utilizzata nella TEI per veicolare le indicazioni sulla messa in pagina di un testo nel documento fonte da cui è stato memorizzato. Gli elementi milestone possono essere usati in coppie virtuali per segnalare i confini di segmenti arbitrari di testo che si sovrappongono agli elementi standard. Insieme a elementi di congiunzione o stand-off markup possono rappresentare virtualmente ogni genere di sovrapposizione, auto-sovrapposizione e segmentazione non contigua. 

%Gli elementi milestone offrono la massima flessibilità sintattica senza costringere a separare markup e contenuto. un parser XML può validare la corretta collocazione di un elemento vuoto rispetto a un modello di contenuto, o verificare che due elementi vuoti siano stati correlati mediante coppie di ID/IDREF. ma non può in alcun modo attribuire funzione strutturale alla sequenza di caratteri contenuta tra due elementi vuoti. non è accessibile come tale a una applicazione XML standard.



% Per questo negli ultimi anni la ricerca teorica sullo sviluppo di si- stemi di markup non gerarchici ha avuto un notevole stimolo

% L’ostacolo maggiore consiste nella individuazione di un modello di dati e di un formalismo a esso associato che possa essere validato ed elaborato mediante algoritmi generali e computazionalmente trattabili come avviene per il modello ad albero di XML.
%% rappresentazione di alberi concorrenti in un medesimo documento XML (Xconcur, JIITs)
%% In LMNL la soluzione del problema OH viene trovata uscendo definitivamente fuori dal paradigma gerarchico di XML
%% TexMECS e GODDAD: Markup Languages for Complex Documents. una notazione non molto dissimile da XML (provvede infatti anche strutture come gli attributi) la quale tuttavia permette di esprimere facilmente strutture sovrapposte, auto- sovrapposte e non contigue. In linea teorica questo grafo può esprimere tutte le possibili relazioni tra oggetti testuali linearizzati sottoforma di stringhe di caratteri etichettate mediante markup, inclusi i più complessi casi di auto- sovrapposizione o di frammentazione non contigua e non linearmente ordinata. 



% non possiamo dire apriori che uno schema di codifica testuale coglie l’essenza del testo più e meglio di un altro in base a un qualche assunto metafisico. Ma neppure si può affermare che ogni rappresentazione è vera in quanto costituisce il suo oggetto testo secondo esigenze specifiche e locali.

%% l’applicazione di procedure informatiche al trattamento dei testi richiede anche la simulazione dei processi che su di essi vengono effettuati.
% scrittura (il momento in cui il testo ha origine), edizione, lettura, analisi, interpretazione, archiviazione, catalogazione.

% Alcune caratteristiche sono comuni a tutti o a molti di questi tipi di testi, mentre altre sono assolutamente specifiche.

% In conclusione secondo la teoria OHCO un testo è una struttura gerarchica di oggetti logici, e la codifica non fa altro che esplicitare questa sua struttura essenziale.

% Ci accorgiamo dunque che la pratica comune nelle discipline che stu- diano i testi è quella di definire il loro oggetto a partire da un punto di vista interno alla disciplina stessa. Ciò dà luogo alla individuazione di diversi unità o elementi di conte- nuto testuale.

% TEI ha una struttura modulare, in cui ogni modulo corrisponde alla rappresentazione di un determinato punto di vi- sta metodologico sul testo.

% L’impianto finale dello schema di codifica della Text Encoding Initiative ha accolto in parte la nozione di sottoprospettiva ed ha svi- luppato una serie di costrutti sintattici e semantici in grado di rappre- sentare adeguatamente fenomeni di sovrapposizione e di parallelismo tra elementi testuali in XML.

% Una rappresentazione codificata di un testo dunque è «vera» se è internamente coerente, accettabile razionalmente nell’ambito di una teoria, in grado di rappresentare i fenomeni testuali rilevanti nel conte- sto di quella teoria o prospettiva metodologica, ed eventualmente di rendere conto dei rapporti tra strutture e fenomeni emergenti da rap- presentazioni (punti di vista) diverse.

% Il trasferimento del testo su supporto informatico propone allo studioso una serie di quesiti teorici (oltre a numerosi problemi pratici) a partire dal momento della decisione su quale particolare oggetto del mondo sia da considerare come “fonte” della memorizzazione.

% E come rileva lo stesso Dennet, possiamo benissimo sbagliarci nell’interpretare.

% La codifica elettronica di un testo, in quanto rappresentazione di un testo e delle sue caratteristiche median- te un linguaggio formale, si colloca interamente all’interno del proces- so analitico-interpretativo,

% si tratta quindi di individuare o sviluppare un sistema di codifica abbastanza potente da permettere a ogni studio- so, da qualsiasi punto di vista disciplinare, di rappresentare le caratte- ristiche testuali che lo interessano e di poter esplicitare le sue interpre- tazioni sul ruolo di tali caratteristiche.

% Occorre dunque tenere presente nella rappresentazione del testo anche i possibili processi ap- plicativi a cui esso può essere sottoposto.

% Queste assunzioni a loro volta non sono individuali: esse sono condivise da una comunità di studiosi che hanno in comune metodologie e pratiche disciplinari, ontologie pratiche, criteri di accettabilità razionale, anche se possono divergere sulla interpretazione di particolari fenomeni.

% attività pratica è stata la continua riflessione circa i migliori metodi e strumenti formali per condurre il delicato compito di rappresentare quegli oggetti complessi, plurali e multiformi che sono i testi, soprattutto quelli che rientrano nella difficilmente definibile categoria dei testi letterari.

% L’utilizzazione delle tecnologie e delle metodologie informatiche ci impone di esplicitare il complesso di nozioni implicite nel dominio degli studi linguistici e letterari, e di sottoporle a verifica sperimentale.
%  quell’oltre si apre lo spazio dell’interpretazione. È questo spazio tra la sequenza discorsiva dei significanti e l’universo semantico e sintattico della narrazione a essere difficilmen- te valicabile.
% l’ambito in cui l’applicazione dell’informatica allo studio dei testi letterari ha avuto il maggiore sviluppo è l'analisi quantitativa della rappresentazione lineare del testo.

% Se, invece, ci si propone di studiare i fenomeni testuali dell’intreccio dei campi semantici sui quali si basa la narrazione, solo un preventivo intervento interpretativo dello studioso può fornire i dati a un sistema informatico che sia in grado di analizzarli in modo quantitativo. (il calcolatore è bravo a fare conti).
% assumere la prospettiva di un’analisi del testo (letterario) as- sistita dal computer.
% riassumento:
%%elaborazione di un quadro teorico di riferimento entro cui col- locare i procedimenti analitici;
%%definizione di un modello di rappresentazione informatica o codifica del testo e delle strutture rilevanti in relazione al con- testo di riferimento;
%%individuazione di metodi e processi di analisi testuale applica- bili al modello del testo e loro definizione sottoforma di proce- dure formali o algoritmi;
%%implementazione del modello di rappresentazione e dei proces- si di analisi mediante adeguati linguaggi informatici;
%%applicazione delle procedure informatiche al testo digitalizza- to;
%%analisi e interpretazione critica dei risultati.
% Si tratta di costruire un modello informatico del testo nel quadro di un determinato contesto teorico, per poi interrogare oppor- tunamente tale modello e avanzare ipotesi interpretative sul testo.

% Il fatto è che quel testo elettronico rende concreto semplicemente uno dei modelli possibili 
% L’operazione di codifica resta dunque un’opera d’interpretazione
%  Sarà possibile, in tal modo, instaurare una sorta di rapporto dialettico tra testo, dati scaturiti dall’elaborazione informatica e ipotesi dello studioso, che re- almente potrebbe aiutarci a disegnare un profilo del tutto nuovo dell’operazione di critica testuale. Per arrivare, infine, alla realizzazione di un mo- dello attendibile e utile del documento da studiare, stabilendone e di- chiarandone i livelli di rappresentazione

\section{Conclusioni}
%\input{includes/2-conclusion.tex}

\section*{Bibliografia}
%\input{includes/3-bibliografia.tex}

\end{document}




