% capitoli 3 e 4 delle linee guida 
% estratti dal libro what is TEI (The structural organization, Varieties of textual structure, parte della TEI cornucopia part 1)
% fare esempio facsimile e codifica dei fenomeni (vedere anche slide del turco e fiormonte-ciotti)
% Elements available in All TEI Documents

% esplicitare le informazioni implicite all'interno del testo.

% i documenti TEI sono organizzati in modo molto simile tra loro: esistono elementi presenti in quasi tutti i testi codificati in TEI.

% Un testo codificato in TEI inizia sempre con l'elemento <text>, suddiviso a sua volta in frontespizio (<front>), corpo del testo (<body>) e appendice (<back>).


\begin{frame}
	\frametitle{Intro Text Encoding Initiative}
	\framesubtitle{TEI}
	\addtocounter{nframe}{1}

	\begin{block}{Un documento TEI P5 ‘minimo’}
        \begin{itemize}
            \item prologo XML
            \item intestazione TEI
            \item elementi strutturali
            \item elementi semantici dei moduli base
        \end{itemize}
    \end{block}
    
\end{frame}

\begin{frame}
	\frametitle{Intro Text Encoding Initiative}
	\framesubtitle{TEI}
	\addtocounter{nframe}{1}

   \textbf{ Moduli di base: \textit{tei}, \textit{header}, \textit{textstructure}, \textit{core}}

	\begin{block}{un documento TEI P5 ‘minimo’}
        Anche usando soltanto i moduli essenziali si ha a disposizione
        uno schema adatto alla marcatura di numerosi tipi di testi.
    \end{block}
 
       \textit{Schemi ``leggeri'' consigliati: la TEI Lite, o se necessario una
        versione più ridotta della P5 (TEI
        Absolutely Bare)}
\end{frame}



\begin{frame}
	\frametitle{Intro Text Encoding Initiative}
	\framesubtitle{Schemi di codifica TEI – Moduli base}
	\addtocounter{nframe}{1}

	\begin{block}{Caratteristiche degli elementi illustrati}
        \begin{itemize}
            \item gli elementi TEI rientrano nelle categorie generali di
            elementi XML che abbiamo visto
            \item elementi che possono contenere solo altri elementi (=
            elementi strutturali)
            \item elementi che possono contenere altri elementi e testo
            \item elementi che possono contenere solo testo
            \item  elementi vuoti (es. \texttt{<pb/>})
            \item  gli elementi vuoti marcano una gerarchia differente
        \end{itemize}
    \end{block}
\end{frame}


\begin{frame}
	\frametitle{Intro Text Encoding Initiative}
	\framesubtitle{Schemi di codifica TEI – Moduli base}
	\addtocounter{nframe}{1}

	\begin{block}{Gerarchie multiple}
        
        \texttt{<?xml version="1.0" encoding="utf-8"?>
        <text>
        <titolo>Gli assassinii della Rue Morgue</titolo>
        <intestazione> I </intestazione>
        \emph{<pagina n=``5''>}
        <p>Le facoltà mentali che si sogliono chiamare analitiche sono, di
        per se stesse, poco suscettibili di analisi [...]</p>
        \emph{<p>}La facoltà di risolvere è probabilmente molto rinfor-
        \emph{</pagina>}
        <pagina n=``6''>
        zata dallo studio delle matematiche e in modo particolare
        dell’altissimo ramo di questa scienza che[...] \emph{</p>}
        </pagina>
        </text>}
    \end{block}

\end{frame}



\begin{frame}
	\frametitle{Intro Text Encoding Initiative}
	\framesubtitle{Schemi di codifica TEI – Moduli base}
	\addtocounter{nframe}{1}

	\begin{block}{Gerarchie multiple- elementi vuoti}
        \texttt{<?xml version="1.0" encoding="utf-8"?>
        <text>
        <titolo>Gli assassinii della Rue Morgue</titolo>
        <intestazione> I </intestazione>
        \emph{<pagina n="5"/>}
        <p>Le facoltà mentali che si sogliono chiamare analitiche sono, di
        per se stesse [...]</p>
        <p>La facoltà di risolvere è probabilmente molto rinfor-
        \emph{<pagina n="6"/>}
        zata dallo studio delle matematiche e in modo particolare
        dell’altissimo ramo di questa scienza che [...] </p> </text>}
       
    \end{block}
    
   

\end{frame}

\begin{frame}
	\frametitle{Intro Text Encoding Initiative}
	\framesubtitle{Schemi di codifica TEI – Moduli base}
	\addtocounter{nframe}{1}

	\begin{block}{Struttura di un documento TEI}
        \begin{itemize}
            \item \textit{struttura fondamentale all’interno della radice (\texttt{<TEI>})}
            \item una intestazione TEI (\texttt{<teiHeader>})
            \item un testo: \texttt{<text>} (o più testi, cfr. infra)
        \end{itemize}
    \end{block}
    
\end{frame}




\begin{frame}
	\frametitle{Intro Text Encoding Initiative}
	\framesubtitle{Schemi di codifica TEI:  Moduli base}
	\addtocounter{nframe}{1}

	\begin{block}{Esempio}
       \texttt{<?xml version="1.0" encoding="utf-8"?>}
         \\\texttt{<!DOCTYPE TEI SYSTEM ``tei\_lite.dtd''>}
         \\\texttt{<TEI xmlns=``http://www.tei\-c.org/ns/1.0''>}
         \\\texttt{<teiHeader> </teiHeader>}
         \\\texttt{<text>}
             \\\texttt{<div><p></p></div>}
        \\\texttt{</text>}
        \\\texttt{</TEI>}
    \end{block}
\end{frame}

\begin{frame}
	\frametitle{Intro Text Encoding Initiative}
	\framesubtitle{Schemi di codifica TEI – Moduli base}
	\addtocounter{nframe}{1}

	\begin{block}{Esempio}
        \texttt{<?xml version="1.0" encoding="utf-8"?>}
        \\\texttt{<?xml-model href="tei-lite.rng"?>}
        \\\texttt{<TEI xmlns=``http://www.tei-c.org/ns/1.0''>}
        \\\texttt{<teiHeader>...</teiHeader>}
        \\\texttt{<text>}
        \\\texttt{<div><p></p></div>}
        \\\texttt{</text>}
        \\\texttt{</TEI>}
    \end{block}
\end{frame}



\begin{frame}
	\frametitle{Intro Text Encoding Initiative}
	\framesubtitle{Schemi di codifica TEI – Moduli base}
	\addtocounter{nframe}{1}

	\begin{block}{Elementi strutturali}
        
       \begin{itemize}
           \item \texttt{<text>} un singolo testo di qualsiasi tipo (punto di partenza della gerarchia).
           \item \texttt{<facsimile>} riproduzione della fonte primaria, può affiancare o sostituire \texttt{<text>}
           \item \texttt{<front>} figlio di \texttt{<text>} materiale che precede il testo
           \item \texttt{<body>} figlio di \texttt{<text>} rappresenta il testo stesso
           \item \texttt{<back>} figlio di \texttt{<text>} materiale che segue il testo
           \item \texttt{<group>} figlio di \texttt{<text>} alternativo a \texttt{<body>}, raggruppa testi diversi
       \end{itemize}
        
    \end{block}
    
   

\end{frame}



\begin{frame}
	\frametitle{Intro Text Encoding Initiative}
	\framesubtitle{Schemi di codifica TEI – Moduli base}
	\addtocounter{nframe}{1}

	\begin{block}{Esempio schematico di documento TEI}
        \texttt{<TEI>}
        \\\texttt{<teiHeader> [informazioni del TEI Header]</teiHeader>}
        \\\texttt{<text>}
        \\\texttt{<front> [premessa, dedica ...] </front>}
        \\\texttt{<body> [corpo del testo ...] </body>}
        \\\texttt{<back> [postfazione, appendice ...]</back>}
        \\\texttt{</text>}
        \\\texttt{</TEI>}
    \end{block}
    
   

\end{frame}



\begin{frame}
	\frametitle{Intro Text Encoding Initiative}
	\framesubtitle{Schemi di codifica TEI – Moduli base}
	\addtocounter{nframe}{1}
    \begin{block}{Costruire documenti compositi}
        \begin{itemize}
            \item rimpiazzando il \texttt{<body>} con un gruppo (\texttt{<group>}) di testi si ottiene un documento composito
            \item ciascuno di questi testi è rappresentato secondo una struttura
            standard
            \item un’altra possibilità è creare un corpus con \texttt{<teiCorpus>}
            \item intestazioni (\texttt{<teiHeader>}) separate per il corpus e per
            ciascun gruppo di testi
            \item struttura più complessa, su più livelli
        \end{itemize}
    \end{block}
\end{frame}


\begin{frame}
	\frametitle{Intro Text Encoding Initiative}
	\framesubtitle{Schemi di codifica TEI – Moduli base}
	\addtocounter{nframe}{1}

        \texttt{<TEI>
        <teiHeader> [ intestazione del testo composito ] </teiHeader>
        <text>
        <front> [ front matter del composito ] </front>
        <group>
        <text>
        <front> [ front matter del primo testo ] </front>
        <body> [ body del primo testo ]
        </body>
        <back> [ back matter del primo testo ] </back>
        </text>
        <text>
        <front> [ front matter del secondo testo] </front>
        <body> [ body del secondo testo ]
        </body>
        <back> [ back matter del secondo testo ] </back>
        </text>
        ...
        [ altri testi o gruppi di testi ]
        ...
        </group>
        <back>
        [ back matter del composito ]
        </back>
        </text>
        </TEI>}

\end{frame}

\begin{frame}
	\frametitle{Intro Text Encoding Initiative}
	\framesubtitle{Schemi di codifica TEI – Moduli base}
	\addtocounter{nframe}{1}

    \begin{block}{Costruzione di corpora TEI}
        \texttt{<teiCorpus>
<teiHeader> [metadati per il corpus] </teiHeader>
<TEI>
<teiHeader> [metadati relativi al I testo]</teiHeader>
<text> [primo testo del corpus] </text>
</TEI>
<TEI>
<teiHeader>[metadati relativi al II testo]</teiHeader>
<text> [secondo testo del corpus] </text>
</TEI>
</teiCorpus>}
    \end{block}
\end{frame}


% \begin{frame}
% 	\frametitle{Intro Text Encoding Initiative}
% 	\framesubtitle{TEI}
% 	\addtocounter{nframe}{1}

% 	\begin{block}{TEI}
% 	 Schemi di codifica TEI – Moduli base
%   IMMAGINE
%     \end{block}
% \end{frame}


\begin{frame}
	\frametitle{Intro Text Encoding Initiative}
	\framesubtitle{Schemi di codifica TEI – Moduli base}
	\addtocounter{nframe}{1}

	\begin{block}{Altri elementi strutturali fondamentali}
        \begin{itemize}
            \item suddivisioni del testo, non numerate: \texttt{<div> }(nessun limite di nidificazione)
            \item suddivisioni del testo, numerate: \texttt{<div1> ... <div7> }(massimo 7 livelli)
            \item paragrafi: \texttt{<p>}
            \item testo riferito: \texttt{<q>} (discorso diretto, citazioni, ecc.)
        \end{itemize}
        
    \end{block}
\end{frame}


\begin{frame}
	\frametitle{Intro Text Encoding Initiative}
	\framesubtitle{Schemi di codifica TEI – Moduli base}
	\addtocounter{nframe}{1}

	\begin{block}{Altri elementi strutturali fondamentali}
        \begin{itemize}
            \item versi: strofe \texttt{<lg>} e singoli versi \texttt{<l>}
            \item testi teatrali: discorsi \texttt{<sp>} che possono contenere paragrafi
            \texttt{<p>} o versi \texttt{<l>}, oltre a direzioni di scena \texttt{<stage>}
            \item milestone tags: \texttt{<pb/>, <lb/>, <cb/>, <milestone/>}
            \item notare che un \texttt{<div>} può contenere un \texttt{<floatingText>} (possibilità di introdurre gerarchie complesse).
        \end{itemize}
        
    \end{block}

\end{frame}

\begin{frame}
	\frametitle{Intro Text Encoding Initiative}
	\framesubtitle{Schemi di codifica TEI – Moduli base}
	\addtocounter{nframe}{1}

	\begin{block}{Apertura e chiusura di un \texttt{<div>}}
        \begin{itemize}
            \item \texttt{<head>}: qualunque tipo di intestazione: il titolo di un'opera, l’intestazione di un paragrafo, di una sezione, ecc.
            \item l'attributo \textit{type} permette di classificare in base a una tipologia
            \item \texttt{<epigraph>} citazione all’inizio del testo, o nella pagina del titolo, eventualmente con riferimento bibliografico
            \item \texttt{<opener>} raggruppa un serie di elementi (data, luogo, saluti, ecc.) all’inizio del \texttt{<div>}, specie di una lettera
        \end{itemize}
    \end{block}

\end{frame}

\begin{frame}
	\frametitle{Intro Text Encoding Initiative}
	\framesubtitle{Schemi di codifica TEI – Moduli base}
	\addtocounter{nframe}{1}

	\begin{block}{Apertura e chiusura di un \texttt{<div>}}
        \begin{itemize}
            \item \texttt{<argument>}: lista degli argomenti trattati nel \texttt{<div>}
            \item \texttt{<trailer>} frase che compare alla fine del \texttt{<div>} (ad esempio ``Fine del capitolo 1'')
            \item \texttt{<closer>} raggruppa un serie di elementi (data, luogo, saluti, ecc.) alla fine del \texttt{<div>}, specie di una lettera
        \end{itemize}
    \end{block}

\end{frame}


\begin{frame}
	\frametitle{Intro Text Encoding Initiative}
	\framesubtitle{Schemi di codifica TEI – Moduli base}
	\addtocounter{nframe}{1}        
        \texttt{<div type="lettera”>
        <opener>
        <dateline>
        <name type=``place''>Pisa</name>
        <date>20 marzo 2015</date>
        </dateline>
        <salute>Gentilissima Prof.ssa Scannagatti,</salute>
        </opener>
        <p>sono spiacente di doverle comunicare che un’invasione di cavallette si
        è abbattuta sui miei quaderni incautamente lasciati in giardino, e li ha
        divorati interamente.</p>
        <p>Questo purtroppo significa che non posso mostrare i compiti svolti,
        come sempre, con solerzia e assiduo impegno.</p>
        <closer>
        <salute>Certo di poter contare sulla sua comprensione le porgo i miei
        migliori saluti,</salute>
        <signed>Pierino Rossi</signed>
        </closer>
        </div>}

\end{frame}



\begin{frame}
	\frametitle{Intro Text Encoding Initiative}
	\framesubtitle{Schemi di codifica TEI – Moduli base}
	\addtocounter{nframe}{1}

	\begin{block}{Errori frequenti}
        I titoli sono codificati con \texttt{<title>} soltanto nel caso di riferimenti bibliografici. \\I titoli del testo, dei capitoli ecc. si marcano con \texttt{<head>}
    \end{block}
    

\end{frame}


\begin{frame}
	\frametitle{Intro Text Encoding Initiative}
	\framesubtitle{Schemi di codifica TEI – Moduli base}
	\addtocounter{nframe}{1}

	\begin{block}{Nota sugli errori possibili}
        \textbf{Tre categorie:}
        \begin{itemize}
            \item \textbf{errori sintattici}: un elemento inserito in un punto sbagliato
            della gerarchia, o che non può contenere testo ecc.
            \item \textbf{errori di marcatura semantica}: usare un elemento inadatto
            allo scopo, ad esempio marcare un titolo con \texttt{<emph>}
            \item \textbf{errori di interpretazione} del testo (che portano al II tipo o
            all’assenza del markup che andrebbe inserito)
        \end{itemize}
       
    \end{block}

\end{frame}


% \begin{frame}
% 	\frametitle{Intro Text Encoding Initiative}
% 	\framesubtitle{Schemi di codifica TEI – Moduli base}
% 	\addtocounter{nframe}{1}

% 	\begin{block}{ A proposito di <div>}
%         domanda che ricorre periodicamente: perché non usare nomi
%         più significativi invece di un contenitore generico come <div>
%         (= ‘suddivisione, parte di un testo’)?
%         perché non usare <book>, <chapter>, <section>, ecc. come
%         fa, ad esempio, lo schema DOCBOOK?
%         troppa variabilità nell’uso comune, meglio usare un termine
%         generico che possa poi essere specificato usando l’attributo
%         type (es. <div type=”chapter”>)
%         i <div>, numerati o meno, possono essere ‘nidificati’, nessun
%         problema nel riproporre la struttura editoriale visibile
%     \end{block}

% \end{frame}


\begin{frame}
	\frametitle{Intro Text Encoding Initiative}
	\framesubtitle{Schemi di codifica TEI – Moduli base}
	\addtocounter{nframe}{1}

    \textbf{Alcuni attributi possono essere usati con qualsiasi elemento (v. la classe att.global)}

    \begin{block}{ Attributi globali}
        \begin{itemize}
            \item \textbf{n} un numero o un nome non univoco, possibilmente breve, per identificare un elemento
            \item \textbf{rend} informazioni relative all’aspetto (\textit{originale}!) del testo
            \item \textbf{rendition} simile a \textit{@rend}, ma fa riferimento a elementi
            \texttt{<rendition>} inseriti nell’\texttt{<encodingDesc>} (dentro \texttt{<tagsDecl>})
        \end{itemize}
    \end{block}
\end{frame}

\begin{frame}
	\frametitle{Intro Text Encoding Initiative}
	\framesubtitle{Schemi di codifica TEI – Moduli base}
	\addtocounter{nframe}{1}

    \begin{block}{ Attributi globali}
        \begin{itemize}
            \item \textbf{xml:lang} la lingua del testo contenuto da un elemento
            \item \textbf{xml:id} un identificatore univoco per l’elemento
        \end{itemize}
       \textit{NOTA: in base ai moduli usati nello schema sono disponibili ulteriori attributi globali}
    \end{block}
\end{frame}

\begin{frame}
	\frametitle{Intro Text Encoding Initiative}
	\framesubtitle{Schemi di codifica TEI – Moduli base}
	\addtocounter{nframe}{1}

	\begin{block}{Esempio}
        \texttt{<text>
        <body>
        \emph{<div n="ch1" type=``chapter''>}
        <pb n="1"/>[...]
        <p>[...] risulta chiaro se avete letto \emph{<title
        rend="underline" xml:lang=``fra''>}Les fleurs du
        mal</title> [...]</p>
        <p>[...] un grande esempio di <foreign
        xml:lang=``fra''>savoir faire</foreign> [...]</p>
        [...]
        </div>
        [ altri div ... ]
        </body>
        </text>}
    \end{block}

\end{frame}


\begin{frame}
	\frametitle{Intro Text Encoding Initiative}
	\framesubtitle{Schemi di codifica TEI: Moduli base}
	\addtocounter{nframe}{1}

	\begin{block}{Esempio}
       \texttt{
           <text>
        <body>
        <div n="ch1" type=``chapter''> <pb n="1"/> [...]
        <p n=``1''>[...] descritto altrove (si veda ad
        esempio \emph{<ref target=``\#Rossi94''>Rossi 1994</ref>})
        [...] </p> [...]
        </div>
        [ altri div ... ]
        <div n="bib" type=``bibliography''>
        [...]
        \emph{<bibl xml:id=``Rossi94''>
        <author>Rossi, M.</author>[...]</bibl>}
        [...]
        </div>
        </body>
        </text>}
    \end{block}
\end{frame}

\begin{frame}
	\frametitle{Intro Text Encoding Initiative}
	\framesubtitle{Schemi di codifica TEI – Moduli base}
	\addtocounter{nframe}{1}

    \begin{block}{Errori frequenti}
        \texttt{<div>} non può essere usato allo stesso livello gerarchico
        di \texttt{<p>}, in altre parole non si può alternare \texttt{<div>} con
        \texttt{<p>}
    \end{block}
    
    \begin{block}{Errore!}
        \texttt{<div> [...] </div>
        <p> [...] </p>
        <div> [...] </div>}
    \end{block}
\end{frame}


\begin{frame}
	\frametitle{Intro Text Encoding Initiative}
	\framesubtitle{Schemi di codifica TEI – Moduli base}
	\addtocounter{nframe}{1}

    \begin{block}{Errori frequenti}
        \texttt{<div>} e tutti gli altri elementi strutturali \textit{puri} non
        possono contenere testo.
       
    \end{block}
    
    \begin{block}{Errore!}
        \texttt{<div>Pippo</div>
        <person>Pippo</person>}
    \end{block}
    

\end{frame}


\begin{frame}
	\frametitle{Intro Text Encoding Initiative}
	\framesubtitle{Schemi di codifica TEI – Moduli base}
	\addtocounter{nframe}{1}

	\begin{block}{Enfasi e termini particolari}
       
        \begin{itemize}
            \item \texttt{<emph>} parole o frasi enfatizzate nel testo. (\texttt{Questo è il <emph>mio</emph> computer!})
            \item \texttt{<foreign>} parola o frase in una lingua diversa. (\texttt{In quel punto entrò il bidello a dare il <foreign xml:lang=``lat''>finis</foreign>}).
        \end{itemize}
    \end{block}

\end{frame}

\begin{frame}
	\frametitle{Intro Text Encoding Initiative}
	\framesubtitle{Schemi di codifica TEI – Moduli base}
	\addtocounter{nframe}{1}

	\begin{block}{Enfasi e termini particolari}
       
        \begin{itemize}
            \item \texttt{<distinct>} “diverso” dal testo perché arcaico, gergale, ecc. (\texttt{Saltò in groppa al <distinct>fido destriero</distinct>})
            \item \texttt{<hi>} elemento generico. (\texttt{<hi rend=``double''>N</hi>el mezzo del cammin di nostra vita.
            Il suo nome è <hi rend=``italic''>Mario Rossi</hi>})
        \end{itemize}
    \end{block}

\end{frame}

\begin{frame}
	\frametitle{Intro Text Encoding Initiative}
	\framesubtitle{Schemi di codifica TEI: Moduli base}
	\addtocounter{nframe}{1}

    \begin{block}{Enfasi e termini particolari}
        \begin{itemize}
            \item  \texttt{<mentioned>} parola o frase menzionata ma non usata.
            (\texttt{Il termine corretto è <mentioned>epigrafe</mentioned>})
            \item \texttt{<soCalled>} parola o espressione da cui ci si distanzia
            (\texttt{il cosiddetto <soCalled>darwinismo sociale</soCalled>})
        \end{itemize}
    \end{block}
    
\end{frame}

\begin{frame}
	\frametitle{Intro Text Encoding Initiative}
	\framesubtitle{Schemi di codifica TEI: Moduli base}
	\addtocounter{nframe}{1}

    \begin{block}{Enfasi e termini particolari}
        \begin{itemize}
            \item \texttt{<term>} una o più parole considerate termine tecnico.
            (\texttt{Possiamo definire il <term xml:id="NPL" rend=``italic''>neopositivismo logico</term>})
            \item \texttt{<gloss>} una spiegazione o glossa riguardo il testo.
            (\texttt{<gloss target=``\#NPL''>una corrente filosofica basata
            sul principio che la filosofia debba aspirare al rigore
            proprio della scienza </gloss>})
        \end{itemize}
    \end{block}
    
\end{frame}

\begin{frame}
	\frametitle{Intro Text Encoding Initiative}
	\framesubtitle{Schemi di codifica TEI – Moduli base}
	\addtocounter{nframe}{1}

	\begin{block}{Esercizio}
       \textbf{Marcare un testo plain text di circa 3000 caratteri a piacere.}
        \begin{itemize}
            \item inserire prologo XML
            \item marcare la struttura usando gli elementi fin qui descritti
            in particolare marcare tutti i paragrafi usando \texttt{<p>} e la struttura editoriale usando \texttt{<div>}
            \item verificare che sia ben formato con xmllint
            \item salvare il file XML su github
        \end{itemize}
    \end{block}
\end{frame}


%%%%%%%%%%%%%%%%%%%%%%%%%%%%%%%%%%%%%%%%%%%%%%%%%


\begin{frame}
	\frametitle{Intro Text Encoding Initiative}
	\framesubtitle{Schemi di codifica TEI – Moduli base}
	\addtocounter{nframe}{1}

	\begin{block}{Citazioni}
        \begin{itemize}
            \item \texttt{<q>} testo citato da altre fonti: discorso diretto, esempi
            (nei dizionari), ecc.
            
        \end{itemize}
    \end{block}

    \begin{block}{Citazioni - Esempio}
        (\texttt{La mia maestra della prima superiore mi salutò di
        sulla porta della classe e mi disse: <q rend=``PRE
        mdash''>Enrico, tu vai al piano di sopra, quest'anno;
        non ti vedrò nemmen più passare!</q>})         
    \end{block}
    

\end{frame}


\begin{frame}
	\frametitle{Intro Text Encoding Initiative}
	\framesubtitle{Schemi di codifica TEI – Moduli base}
	\addtocounter{nframe}{1}

	\begin{block}{Citazioni}
        \begin{itemize}
            
            \item \texttt{<quote>} frase o brano attribuito a fonte esterna
           
            
        \end{itemize}
        
    \end{block}

    \begin{block}{Citazioni - Esempio}
            
        (\texttt{<p>E allora disse: <q rend=``PRE lsquo POST
            rsquo''>Ecco come comincia la Divina Commedia:
            <quote>Nel mezzo del cammin di nostra vita / Mi
            ritrovai per una selva oscura</quote>.</q></p>})
        
        \end{block}
    
\end{frame}



\begin{frame}
	\frametitle{Intro Text Encoding Initiative}
	\framesubtitle{Schemi di codifica TEI – Moduli base}
	\addtocounter{nframe}{1}

	\begin{block}{Citazioni}
        \begin{itemize}
            \item \texttt{<said>} testo pronunciato ad alta voce o pensato
            \item \texttt{<cit>} citazione con riferimento bibliografico
        \end{itemize}
        
        
        
    \end{block}


\end{frame}

\begin{frame}
	\frametitle{Intro Text Encoding Initiative}
	\framesubtitle{Schemi di codifica TEI – Moduli base}
	\addtocounter{nframe}{1}

	\begin{block}{Citazioni - Esempio}

        
       \texttt{Lexicography has shown little sign of being affected by the work
        of followers of J.R. Firth, probably best summarized in his
        slogan, \underline{<cit>}
        \underline{<quote>}You shall know a word by the company it keeps.\underline{</quote>}
        \underline{<ref>}(Firth, 1957)\underline{</ref>}
        \underline{</cit>}}
        
    \end{block}

    \textit{Semplice riferimento bibliografico nell’esempio, possibile
        aggiungere un collegamento (a capitolo/sezione o a una
        specifica entrata bibliografica) usando l’attributo \emph{@target}}

\end{frame}


\begin{frame}
	\frametitle{Intro Text Encoding Initiative}
	\framesubtitle{Schemi di codifica TEI: Moduli base}
	\addtocounter{nframe}{1}

	\begin{block}{TEI: elementi di citazione}
        A proposito di \texttt{<q> e <quote>}
        possono contenere non solo altri elementi simili (\texttt{<q> e
        <quote>}) ma anche elementi come \texttt{<p>, <l>, ecc.}:
    \end{block}

    \textit{Possibili problemi relativi alla gerarchia (tag overlap, gerarchie
    sovrapposte):
    \texttt{<p>Allora disse: <q>Sarò breve.</p><p>Ho finito.</q></p>}}
\end{frame}


\begin{frame}
	\frametitle{Intro Text Encoding Initiative}
	\framesubtitle{Schemi di codifica TEI: Moduli base}
	\addtocounter{nframe}{1}

	\begin{block}{TEI: esempio citazione con elementi annidati} 
       \texttt{<p><q>The Lord! The Lord! It is Sakya Muni himself,</q> the lama half
         sobbed; and under his breath began the wonderful Buddhist invocation:<q>}
        \\\texttt{<quote>}
         \\\texttt{<l>To Him the Way \- the Law \- Apart \- </l>}
         \\\texttt{<l>Whom Maya held beneath her heart</l>}
         \\\texttt{<l>Ananda's Lord \- the Bodhisat</l>}
        \\\texttt{</quote>}
        \\\texttt{And He is here! The Most Excellent Law is here also. My
        pilgrimage is well begun. And what work! What work!</q></p>}
    \end{block}
\end{frame}

\begin{frame}
	\frametitle{Intro Text Encoding Initiative}
	\framesubtitle{ Schemi di codifica TEI: Moduli base}
	\addtocounter{nframe}{1}

	\begin{block}{Nomi, numeri e date}
       \begin{itemize}
           \item \texttt{<rs>} letteralmente ``referring string'', nome o etichetta generica
           
           \texttt{(<q>Mio caro <rs type=``person''>Filippo</rs></q>,
           gli disse <rs type=``person''>sua moglie</rs>.)}

           \item  \texttt{<name>} nome proprio (di persona, luogo, ecc.)
           
           \texttt{<q>Mio caro <name type=``person''>Filippo</name></q>}
       \end{itemize}
        
    \end{block}   

\end{frame}


\begin{frame}
	\frametitle{Intro Text Encoding Initiative}
	\framesubtitle{Schemi di codifica TEI: Moduli base}
	\addtocounter{nframe}{1}

    \begin{block}{Nomi, numeri e date}
        \begin{itemize}
            \item \texttt{<num>} un numero in qualsiasi formato
            \texttt{<num value=``23''>XXIII</num>}
            \item \texttt{<date>} una data in qualsiasi formato
            \texttt{nato il <date when=``18680210''>10 febb. 1868</date>}
            \item \texttt{<time>} l'ora in qualsiasi formato
            \texttt{alle <time when=``8.00''>otto del mattino</time>}
        \end{itemize}
    \end{block}
\end{frame}


\begin{frame}
	\frametitle{Intro Text Encoding Initiative}
	\framesubtitle{Schemi di codifica TEI – Moduli base}
	\addtocounter{nframe}{1}

	\begin{block}{Nomi, numeri e date}
        
        \textbf{possibile usare un modulo specifico:}
        \\\textit{13 Names, Dates, People, and Places} (
        \url{http://www.tei-c.org/release/doc/tei-p5-doc/en/html/ND.html} )
       
        
        \texttt{(<persName>, <forename>, <surname>, <roleName>,
        <addName>, ecc.)}

    \end{block}

    \textit{Questo modulo permette una granularità molto maggiore
    grazie agli elementi specifici che mette a disposizione.
    \\Possibile creare sistema prosopografico.
    \\Ricchezza degli attributi (anche nella versione base)}
    
\end{frame}


\begin{frame}
	\frametitle{Intro Text Encoding Initiative}
	\framesubtitle{Schemi di codifica TEI – Moduli base}
	\addtocounter{nframe}{1}

	\begin{block}{La pagina del titolo}
        
        elemento \texttt{<titlePage> }può contenere:
        \begin{itemize}
            \item \texttt{<docTitle>} e (o direttamente) \texttt{<titlePart>}: titolo anche in più parti
            \item \texttt{<docEdition>}: informazioni riguardo l’edizione
            \item \texttt{<byline>} e (o direttamente) \texttt{<docAuthor>}: autore
            \item \texttt{<docImprint>}: informazioni di stampa, a sua volta include
        \end{itemize}
    \end{block}
\end{frame}

\begin{frame}
	\frametitle{Intro Text Encoding Initiative}
	\framesubtitle{Schemi di codifica TEI – Moduli base}
	\addtocounter{nframe}{1}

	\begin{block}{La pagina del titolo}
        
        elemento \texttt{<titlePage> }può contenere:
        \begin{itemize}
            \item \texttt{<publisher>}: editore
            \item \texttt{<pubPlace>}: luogo di stampa
            \item \texttt{<date>}: data
            \item \texttt{<docDate>}: data
        \end{itemize}
    \end{block}
\end{frame}


\begin{frame}
	\frametitle{Intro Text Encoding Initiative}
	\framesubtitle{Schemi di codifica TEI – Moduli base}
	\addtocounter{nframe}{1}

	\begin{block}{Esempio di pagina del titolo}
        \texttt{<titlePage>}
        \\\texttt{<titlePart>Cuore</titlePart>}
        \\\texttt{<byline>di <docAuthor>E. de Amicis</docAuthor></byline>}
        \\\texttt{<docEdition>Edizione integrale</docEdition>}
        \\\texttt{<docImprint>}
        \\\texttt{<publisher>Newton Compton editori</publisher>}
        \\\texttt{<pubPlace>Roma</pubPlace>}
        \\\texttt{<date>1994</date>}
        \\\texttt{</docImprint>}
        \\\texttt{</titlePage>}
    \end{block}
\end{frame}



\begin{frame}
	\frametitle{Intro Text Encoding Initiative}
	\framesubtitle{Schemi di codifica TEI – Moduli base}
	\addtocounter{nframe}{1}

	\begin{block}{Esempio di pagina del titolo}   
        \texttt{<titlePage>}
        \\\texttt{<docAuthor>E. DE AMICIS</docAuthor>}
        \\\texttt{<docTitle>}
        \\\texttt{<titlePart type= ``main'' >CUORE</titlePart>}
        \\\texttt{<titlePart>Libro per i ragazzi</titlePart>}
        \\\texttt{</docTitle>}
        \\\texttt{<docEdition>98.a edizione</docEdition>}
        \\\texttt{<graphic url= ``publisher.png'' >}
        \\\texttt{<docImprint>}
        \\\texttt{<pubPlace>MILANO</pubPlace>}
        \\\texttt{<publisher>FRATELLI TREVES, EDITORI</publisher>}
        \\\texttt{<date>1889</date>}
        \\\texttt{</docImprint>}
        \\\texttt{</titlePage>}
    \end{block}
    
\end{frame}



% \begin{frame}
% 	\frametitle{Intro Text Encoding Initiative}
% 	\framesubtitle{Schemi di codifica TEI – Moduli base}
% 	\addtocounter{nframe}{1}

% 	\begin{block}{TEI}
        
%         Associare uno schema di codifica
%         per qualsiasi progetto anche mediamente impegnativo è
%         preferibile creare un proprio schema TEI
%         cominciamo a validare sulla base di uno schema a partire
%         dall’esercizio ese02.txt
%         associazione di uno schema al documento TEI XML:
%         XML Copy Editor: XML → Associa → DTD di sistema
%         Oxygen: Document → Schema → Associate Schema...
%         altri editor: inserire manualmente il <!DOCTYPE> nel
%         caso si usi una DTD, o la processing instruction <?xml-
%         model> descritta nel capitolo A Gentle Introduction to XML
%     \end{block}
% \end{frame}



% \begin{frame}
% 	\frametitle{Intro Text Encoding Initiative}
% 	\framesubtitle{TEI}
% 	\addtocounter{nframe}{1}

% 	\begin{block}{TEI}
%         % Schemi di codifica TEI: Moduli base
%         % Nota su XML Copy Editor
%         % XCE non supporta ancora l’uso di schemi di codifica nel
%         % formato RelaxNG, necessario ricorrere alla vecchia DTD
%         % XCE controlla anche la coerenza interna dello schema
%         % un messaggio come quello che segue può essere
%         % ignorato, l’informazione essenziale è se il documento è
%         % valido oppure no (riportato in cima):
%         % ese02-*******.xml is valid
%         % Attenzione at line 2, column 37: element
%         % '_DUMMY_model.resourceLike' is referenced in a content
%         % model but was never declared
%         % Attenzione at line 2, column 37: element
%         % '_DUMMY_model.gLike' is referenced in a content model but
%         % was never declared
%     \end{block}
    
   

% \end{frame}


% \begin{frame}
% 	\frametitle{Intro Text Encoding Initiative}
% 	\framesubtitle{TEI}
% 	\addtocounter{nframe}{1}

% 	\begin{block}{TEI}
%         Schemi di codifica TEI – Moduli base
%         Esercizio 2
%         marcare il testo (file ese02.txt) come segue:
%         editor da usare: XML Copy Editor
%         aprire il sito delle norme TEI P5
%         marcare la struttura per prima cosa
%         marcare tutte le espressioni enfatizzate (underscore =
%         corsivo), i discorsi diretti, le citazioni, i nomi, ecc.
%         verificare che sia ben formato premendo F2 e validare
%         salvare il file XML come ese02-Cognome.xml
%         testi tratti da http://www.gutenberg.org/
%     \end{block}
    
   

% \end{frame}




\begin{frame}
	\frametitle{Intro Text Encoding Initiative}
	\framesubtitle{Schemi di codifica TEI: Moduli base}
	\addtocounter{nframe}{1}

	\begin{block}{Collegamenti interni ed esterni}
        
        \begin{itemize}
            \item \texttt{<ptr/>} specifica un puntatore a un altro \textit{luogo} (un altro
            punto dello stesso testo o di un altro testo)
        \end{itemize}

        \texttt{<p>Il sistema di puntatori è basato sul meccanismo W3C
        Xpointer. Per maggiori informazioni si veda il
        capitolo <ptr target="\#SAid"/>; per le specifiche si
        veda <ptr target=``http://www.w3.org/TR/xptrxpointer/''/>; 
        anche <ptr type="image" target="\#fig22"/>.</p>}
    \end{block}
\end{frame}

\begin{frame}
	\frametitle{Intro Text Encoding Initiative}
	\framesubtitle{Schemi di codifica TEI: Moduli base}
	\addtocounter{nframe}{1}

	\begin{block}{Collegamenti interni ed esterni}
        
        \begin{itemize}
            \item \texttt{<ref>} specifica un puntatore a un altro \textit{luogo}, può
            includere del testo.            
        \end{itemize}
       
        \texttt{Si veda il <ref>terzo capitolo, p. 24</ref>.
            Si veda il <ref target=``\#cap3.24''>terzo capitolo,
            par. 24</ref>}
    \end{block}
\end{frame}





\begin{frame}
	\frametitle{Intro Text Encoding Initiative}
	\framesubtitle{Schemi di codifica TEI – Moduli base}
	\addtocounter{nframe}{1}

    \begin{block}{Liste e tabelle}
        \begin{itemize}
            \item \texttt{<list>} qualsiasi tipo di lista (type per specificare)
            \item \texttt{<head>} intestazione (titolo) della lista
            \item \texttt{<item>} un elemento della lista
            \item \texttt{<label>} numero o esponente associato all’\texttt{<item>}
            \item \texttt{<headLabel>} intestazione per gli esponenti della lista
            \item \texttt{<headItem>} intestazione per gli elementi della lista
        \end{itemize}   
    \end{block}
\end{frame}




\begin{frame}
	\frametitle{Intro Text Encoding Initiative}
	\framesubtitle{Schemi di codifica TEI – Moduli base}
	\addtocounter{nframe}{1}

    \textit{Agli item può essere associata o meno una etichetta (label), ma in caso
    affermativo deve essere presente per tutti.
   }

	\begin{block}{Liste e tabelle}
        
        \texttt{<list><head>Ingredienti:</head> <item>un cucchiaio di
        zucchero;</item> <item>mezzo chilo di farina;</item>
        <item>due uova.</item></list>}
    \end{block}
    \textit{Usare l'attributo type per specificare il tipo di lista.
        \\Il testo non deve essere necessariamente ordinato come lista
        per essere marcato come tale}
\end{frame}



\begin{frame}
	\frametitle{Intro Text Encoding Initiative}
	\framesubtitle{Schemi di codifica TEI – Moduli base}
	\addtocounter{nframe}{1}

	\begin{block}{Lista semplice}
        \texttt{<list type=``simple''>}
        \\\texttt{<head>Lista della spesa:</head>}
        \\\texttt{<item>pane;</item>}
        \\\texttt{<item>frutta;</item>}
        \\\texttt{<item>verdura;</item>}
        \\\texttt{<item>latte;</item>}
        \\\texttt{<item>farina;</item>}
        \\\texttt{<item>uova;</item>}
        \\\texttt{<item>tovaglioli;</item>}
        \\\texttt{<item>bicchieri;</item>}
        \\\texttt{<item>piatti.</item>}
        \\\texttt{</list>}
    \end{block}
    
\end{frame}


\begin{frame}
	\frametitle{Intro Text Encoding Initiative}
	\framesubtitle{ Schemi di codifica TEI: Moduli base}
	\addtocounter{nframe}{1}

	\begin{block}{Glossario}
        \texttt{<list type=``gloss''>}
        \\\texttt{<head>Tecnologie XML:</head>}
        \\\texttt{<label>XSL</label>}
        \\\texttt{<item>eXtensible Stylesheet Language</item>}
        \\\texttt{<label>XSLT</label>}
        \\\texttt{<item>XSL Transformations</item>}
        \\\texttt{<label>XSLFO</label>}
        \\\texttt{<item>XSL \- Formatting Objects</item>}
        \\\texttt{<label>XQuery</label>}
        \\\texttt{<item>XML Query Language</item>}
        \\\texttt{<label>XPAth</label>}
        \\\texttt{<item>XML Path Language</item>}
        \\\texttt{</list>}
    \end{block}
    
\end{frame}


\begin{frame}
	\frametitle{Intro Text Encoding Initiative}
	\framesubtitle{Schemi di codifica TEI: Moduli base}
    \addtocounter{nframe}{1}
    
    \begin{block}{Tabelle}
        \texttt{gli elementi per tabelle sono <table>, <row>, <cell> disponibili
        con il modulo figures, v. il cap. 14 Tables, Formulae and
        Graphics (\url{http://www.tei-c.org/release/doc/tei-p5-doc/en/html/FT.html})}
    \end{block}

\end{frame}

\begin{frame}
	\frametitle{Intro Text Encoding Initiative}
	\framesubtitle{Schemi di codifica TEI: Moduli base}
	\addtocounter{nframe}{1}


	\begin{block}{Tabella}
        
        \texttt{<table rows="3" cols=``2''>
        <row>
        <cell>HTML</cell>
        <cell>Derivato da SGML (come applicazione), il
        linguaggio del World Wide Web.</cell>
        </row>
        <row>
        <cell>XML</cell>
        <cell>Derivato da SGML (per semplificazione).</cell>
        </row>
        <row>
        <cell>XSLT</cell>
        <cell>Fogli di stile per XML.</cell>
        </row>
        </table>}
    \end{block}
    
\end{frame}



\begin{frame}
	\frametitle{Intro Text Encoding Initiative}
	\framesubtitle{Schemi di codifica TEI – Moduli base}
	\addtocounter{nframe}{1}

	\begin{block}{Note}
        
        
        \begin{itemize}
            \item l’elemento \texttt{<note>} permette di inserire una nota di
            qualsiasi tipo (attributi \textit{type} e \textit{place})
            \item la nota può appartenere al testo codificato, o può essere
            opera di chi lo codifica (attributo \textit{resp})
            \item se non nel flusso del testo usare gli attributi \textit{target} e
            \textit{targetEnd} per stabilire collegamento preciso
        \end{itemize}
    \end{block}
\end{frame}

\begin{frame}
	\frametitle{Intro Text Encoding Initiative}
	\framesubtitle{Schemi di codifica TEI – Moduli base}
	\addtocounter{nframe}{1}

	\begin{block}{Note - Esempio}
        \texttt{Le <title>Guidelines</title><note n="2" place="foot" resp=``amdg''>C.M. SperbergMcQueen and Lou Burnard, <title>Guidelines for Electronic Text Encoding and Interchange</title> (Chicago, Oxford: Text Encoding initiative, 2002).</note> sono nate con lo scopo di ...}
    \end{block}
\end{frame}

\begin{frame}
	\frametitle{Intro Text Encoding Initiative}
	\framesubtitle{Schemi di codifica TEI – Moduli base}
	\addtocounter{nframe}{1}

    \begin{block}{Esercizio}
        Utilizzare puntatori, note, lista, glossario marcare nomi, rs, term.  
        % marcare il testo del file ese03.jpg come segue:
        % editor da usare: XML Copy Editor o Editix
        % marcare tutto quello che si vede nella pagina
        % marcare come glossario usando <list> oppure
        % in alternativa: marcare come glossario <term> + <gloss>
        % usare identificatori e puntatori/riferimenti
        % inserire un paio di note a piè di pagina
        % verificare costantemente che il documento sia valido
        % salvare il file XML come ese03-Cognome.xml
    \end{block}
    

\end{frame}


\begin{frame}
	\frametitle{Intro Text Encoding Initiative}
	\framesubtitle{Schemi di codifica TEI – Moduli base}
	\addtocounter{nframe}{1}

    %\textit{I riferimenti bibliografici, compresa la bibliografia vera e
     %   propria, sono codificati usando i seguenti elementi:}
	\begin{block}{Bibliografia}
        \begin{itemize}
            \item \texttt{<bibl>} citazione bibliografica di tipo \textit{flessibile}: possono essere presenti solo alcuni elementi; usata anche nel corpo del testo
            \item \texttt{<biblStruct>} citazione bibliografica di tipo strutturato: gli
            elementi devono seguire uno schema fisso
            \item \texttt{<biblFull>} citazione bibliografica di tipo strutturato completo:
            gli elementi devono seguire uno schema fisso e devono essere tutti presenti
            \item \texttt{<listBibl>} lista di citazioni bibliografiche in uno qualsiasi
            dei formati citati sopra
        \end{itemize}
       
    \end{block}
    
\end{frame}



\begin{frame}
	\frametitle{Intro Text Encoding Initiative}
	\framesubtitle{Schemi di codifica TEI – Moduli base}
	\addtocounter{nframe}{1}

    \textit{Una voce bibliografica può essere codificata impiegando tre possibili tipi:}
	\begin{block}{Tipi di pubblicazione}
        
        \begin{itemize}
            \item \texttt{<analytic>} “analitico”: informazioni su un titolo che
            non costituisce una pubblicazione autonoma
            (articolo di rivista o in una miscellanea)
            \item \texttt{<monogr>} “monografico”: informazioni su una
            pubblicazione, anche di tipo periodico (rivista)
            \item \texttt{<series>} “serie”: informazioni su di una serie
            editoriale
        \end{itemize}
        
    \end{block}
    
    \textit{tipicamente impiegati con gli elementi \texttt{<biblStruct>} e \texttt{<biblFull>}, più flessibilità con \texttt{<bibl>}}
    
\end{frame}


\begin{frame}
	\frametitle{Intro Text Encoding Initiative}
	\framesubtitle{Schemi di codifica TEI – Moduli base}
	\addtocounter{nframe}{1}

	\begin{block}{Titolo, autore, curatore}
        \begin{itemize}
            \item \texttt{<title>} il titolo della pubblicazione; più livelli (attr. level)
            \item \textit{level=”a”} articolo (article)
            \item \textit{level=”j”} rivista (journal)
            \item \textit{level=”m”} monografia (monograph)
            \item \textit{level=”s”} serie (series)
            \item \textit{level=”u”} non pubblicato (unpublished)
        \end{itemize}
    \end{block}
    
\end{frame}


\begin{frame}
	\frametitle{Intro Text Encoding Initiative}
	\framesubtitle{Schemi di codifica TEI – Moduli base}
	\addtocounter{nframe}{1}

    \begin{block}{Titolo, autore, curatore}
        \begin{itemize}
            \item \texttt{<author>} autore della pubblicazione
            \item \texttt{<editor>} curatore della pubblicazione 
        \end{itemize} 
    \end{block}
    \textit{Questi elementi possono contenere direttamente il testo con il nome oppure elementi più complessi (\texttt{<persName> e sub-elementi})}
\end{frame}


\begin{frame}
	\frametitle{Intro Text Encoding Initiative}
	\framesubtitle{Schemi di codifica TEI – Moduli base}
	\addtocounter{nframe}{1}

    \begin{block}{Elementi descrittivi}
        \begin{itemize}
            \item \texttt{<imprint>} raggruppa informazioni relative alla stampa
            o diffusione di una pubblicazione
            \item \texttt{<publisher>} l’editore responsabile della produzione a
            stampa o diffusione
            \item \texttt{<pubPlace>} il luogo di pubblicazione
            \item \texttt{<date>} la data di pubblicazione
        \end{itemize}
    \end{block}
\end{frame}

\begin{frame}
	\frametitle{Intro Text Encoding Initiative}
	\framesubtitle{Schemi di codifica TEI – Moduli base}
	\addtocounter{nframe}{1}

	\begin{block}{Elementi descrittivi}
    \begin{itemize}
        \item \texttt{<biblScope>} l’estensione (in pagine, volumi) della
        pubblicazione;
        \item con attributo \textit{unit} permette di specificare il tipo di estensione
        ad es. se si tratta di pagine
        \item \texttt{<citedRange>} porzione di lavoro citato.
    \end{itemize}

    \end{block}

    \texttt{<biblScope> può essere usato anche all’interno di <series>}
\end{frame}


\begin{frame}
	\frametitle{Intro Text Encoding Initiative}
	\framesubtitle{Schemi di codifica TEI – Moduli base}
	\addtocounter{nframe}{1}

	\begin{block}{Esempio: monografia con autore unico:}
        \texttt{<biblStruct xml:id=``Robinson1993d''>}
        \\\texttt{ <monogr>}
        \\\texttt{  <author>Robinson, P.</author>}
        \\\texttt{  <title level=``m''>The Digitization of Primary Textual Sources</title>}
        \\\texttt{  <imprint>}
        \\\texttt{   <pubPlace>Oxford</pubPlace>}
        \\\texttt{   <publisher>Office for Humanities Communication</publisher>}
        \\\texttt{   <date>1993</date>}
        \\\texttt{  </imprint>}
        \\\texttt{ </monogr>}
        \\\texttt{</biblStruct>}
    \end{block}
\end{frame}


\begin{frame}
	\frametitle{Intro Text Encoding Initiative}
	\framesubtitle{Schemi di codifica TEI: Moduli base}
	\addtocounter{nframe}{1}

	\begin{block}{Articolo di miscellanea, due autori}
        \texttt{<biblStruct xml:id=``Mohler2001''>
                <analytic>
                <author>Mohler, Peter Ph.</author>
                <author>Zuell, Cornelia</author>
                <title level=``a''>Applied Text Theory: Qualitative Analysis of
                Answers to Open Ended Questions</title>
                </analytic>
                <monogr>
                <editor>Mark D. West</editor>
                <title level=``m''>Application of Computer Content Analysis</title>
                <imprint>
                <pubPlace>Westport CN</pubPlace>
                <publisher>Ablex</publisher>
                <date>2001</date>
                <biblScope unit=``pages''>116</biblScope>
                </imprint>
                </monogr>
                </biblStruct>}
    \end{block}
\end{frame}


\begin{frame}
	\frametitle{Intro Text Encoding Initiative}
	\framesubtitle{Schemi di codifica TEI – Moduli base}
	\addtocounter{nframe}{1}

	\begin{block}{articolo su rivista, un solo autore}
       \texttt{ <biblStruct xml:id=``Ester1994''>
        <analytic>
        <author>Ester, Michael</author>
        <title level=``a''>Digital Images in the Context of Visual
        Collections and Scholarship</title>
        </analytic>
        <monogr>
        <title level=``j''>Visual Resources</title>
        <imprint>
        <date>1994</date>
        </imprint>
        <biblScope unit=``vol''>X, 1</biblScope>
        <biblScope unit=``pages''>23</biblScope>
        </monogr>
        <note type=``key''>iproc</note>
        </biblStruct>}
    \end{block}
    
\end{frame}





\begin{frame}
	\frametitle{Intro Text Encoding Initiative}
	\framesubtitle{Schemi di codifica TEI: Moduli base}
	\addtocounter{nframe}{1}

	\begin{block}{Componenti non testuali: immagini} 
       \begin{itemize}
           \item in formato XML \textit{(SVG, il formato TEI)}
           \item in formato binario \textit{(TIFF, JPEG, PNG)}
       \end{itemize}
    \end{block}
    
\end{frame}


\begin{frame}
	\frametitle{Intro Text Encoding Initiative}
	\framesubtitle{Schemi di codifica TEI: Moduli base}
	\addtocounter{nframe}{1}

	\begin{block}{Componenti non testuali: immagini} 
       \begin{itemize}
           \item \texttt{<graphic/>} contiene un puntatore a una immagine
           \item attributo \textbf{url} può essere locale o esterno
           \item \texttt{<binaryObject>} dati binari che costituiscono una immagine
           \item \texttt{<media>} riferimento a file audio, video ecc.
       \end{itemize}

    \end{block}
\end{frame}

\begin{frame}
	\frametitle{Intro Text Encoding Initiative}
	\framesubtitle{Schemi di codifica TEI: Moduli base}
	\addtocounter{nframe}{1}

    \begin{block}{Componenti non testuali: immagini}
        \begin{itemize}
            \item \texttt{<figure>} raggruppa informazioni relative a un’immagine
            \item \texttt{<head>} titolo o didascalia relativa all’immagine
            \item \texttt{<figDesc>} descrizione dell'immagine
            \item \texttt{<graphic>} puntatore a un'immagine
        \end{itemize}
    \end{block}
    \texttt{<figure> nella P5 fa parte del modulo figures 
    (\url{http://www.tei-c.org/release/doc/tei-p5-doc/en/html/FT.html\#FTGRA} )}
    
\end{frame}



\begin{frame}
	\frametitle{Intro Text Encoding Initiative}
	\framesubtitle{Schemi di codifica TEI – Moduli base}
	\addtocounter{nframe}{1}
    
    \textit{Nella TEI P4 bisogna inserire dichiarazioni
    per le immagini nell’intestazione}
    
	\begin{block}{Esempio TEI P4}
        \texttt{<?xml version='1.0' encoding='utf-8'?>
        ...
        <!NOTATION JPEG PUBLIC
        `ISO DIS 10918//NOTATION JPEG Graphics Format//EN'>
        <!ENTITY PDatabase SYSTEM `support/PDatabase.jpeg' NDATA JPEG>
        ]>
        ...
        <p>The system is build around a relational database, called the
        <soCalled>palaeographical database</soCalled> ... </p>
        \underline{<figure entity=``PDatabase''>}
        <figDesc>Schematic of the SPI system.</figDesc>
        </figure>}
    \end{block}
\end{frame}


\begin{frame}
	\frametitle{Intro Text Encoding Initiative}
	\framesubtitle{Schemi di codifica TEI – Moduli base}
	\addtocounter{nframe}{1}

	\begin{block}{Immagine in un documento P5}
        
        \texttt{<p>Questo fenomeno è chiaramente visibile nella figura che
        segue:
        \underline{<graphic url="diagramma.png"/>}
        Altre variazioni delle crescite sono
        disponibili...</p>}


       \texttt{<head>
        <graphic
        url="http://www.name.org/hpral02.gif"/>
        </head>}

    \end{block}
    \textit{immagine impiegata come intestazione in un documento P5}
\end{frame}

\begin{frame}
	\frametitle{Intro Text Encoding Initiative}
	\framesubtitle{Schemi di codifica TEI – Moduli base}
	\addtocounter{nframe}{1}

	\begin{block}{Esercizi documenti di tipo e genere diversi}
        \begin{itemize}
            \item codificare usando gli opportuni elementi TEI un articolo di rivista
            \item codificare usando gli opportuni elementi TEI una cartolina 
        \end{itemize}

    \end{block}
\end{frame}



% Documento TEI Minimale.

%<TEI xmlns="http://www.tei-c.org/ns/1.0">
%    <teiHeader>
%        <fileDesc>
%            <titleStmt>
%                <title>The life and opinions of Tristram Shandy, Gentleman: TEI edition</title>
%            </titleStmt>
%            <publicationStmt>
%                <publisher>Web Head Press</publisher>
%                <date>2013</date> edition, 1708</p>
%                <sourceDesc />
%        </fileDesc>
%    </teiHeader>
%   <text>
%       <front>frontespizio qui</front>
%
%        <body>
%            <div type="volume" xml:id="TS01">
%                <div type="chapter" xml:id="TS0101">
%
%                    <head>Chap. I</head>
%                    <p>I wish either my father or my mother, or indeed both of them, as they were in duty both equally bound
%                        to it, had minded what they were about when they begot me; ...</p>
%                    <!-- remainder of chapter 1 here -->
%                </div>
%                <div type="chapter" xml:id="TS0102">
%
%                    <head>Chap. II</head>
%                    <p> — Then, positively, there is nothing in the question, that I can see, either good or bad. — Then let
%                        me tell you, Sir, it was a very unseasonable question at least ...</p>
%                    <!-- remainder of chapter 2 here -->
%                </div>
%                <!-- remaining chapters of volume 1 here -->
%            </div>
%            <!-- remaining volumes of work here -->
%        </body>

%        <back>appendice qui</back>
%    </text>
% </TEI>

%% gruppi principali:
% Paragrafi
% Punteggiatura
% Evidenziazione e Citazione (quotation)
% Principali elementi editoriali
% Nomi, Numeri, Date, Abbreviazioni, Indirizzi
% Collegamenti e Riferimenti
% Liste
% Note, Annotazioni e Indici
% Grafici e altre componenti non testuali
% Sistema bibliografico
% Poesia e teatro

% tra gli elementi di uso generico nella codifica dei testi più comuni, l'elemento paragrafo (<p>), insime all'elemento divisione (<div>), è di sicuro il più impiegato. L'elemento paragrafo è un elemento di classe chunk, quindi non può annidarsi.

%% Punteggiatura

%% Evidenziazione
% due classi: model.hiLike e model.emphLike.
% sono elementi phrase-level che possono comparire in paragraph level
% elementi in model.emphLike: <foreign> <emph> <distinct>
% esempio <foreign>:
% John eats a <foreign xml:lang="fr">croissant</foreign> every morning.
% esempio <emph>.

%<q>
% <emph rend="italic">What does Christopher Robin do in the morning
%   nowadays?</emph>
%</q>

%<q>
% <emph style="font-style: italic">What does Christopher Robin do in
%   the morning nowadays?</emph>
%</q>

%<l>Doth sometimes Counsel take —
% and sometimes <emph rendition="#italic">Tea</emph>.</l>
%<!-- in the header ... -->
%<rendition xml:id="italic" scheme="css">font-style: italic</rendition>

% esempio <destinct>

%Next morning a boy in that dormitory confided to his
% bosom friend, a
% <distinct time="1900" space="GB"
% social="publicschool">fag</distinct>
% of Macrea's, that there was trouble in their midst which
% King <distinct time="archaic">would fain</distinct> keep
% secret

% esempio <hi>
%<hi rend="gothic">And this Indenture further witnesseth</hi>
% that the said <hi rend="italic">Walter Shandy</hi>, merchant,
% in consideration of the said intended marriage ...

%% Quotation
% <q> <said> <quote> <cit> <mentioned> <soCalled>
% esempio <q>
% <q rendition="#quoteBefore #quoteAfter">Four score and seven years ago...</q>


% esempio <said>
%<said rend="pre(‘) post(’)">Who-e debel
% you?</said> — he at last said —
%<said rend="pre(‘) post(’)">you no speak-e,
% damme, I kill-e.</said> And so saying,
% the lighted tomahawk began flourishing
% about me in the dark.

% Attributi @who e @toWhom
% esempio:
%Adolphe se tourna vers lui :
%<said who="#Adolphe">— Alors, Albert, quoi de neuf?</said>
%<said who="#Albert">— Pas grand-chose.</said>
%<said who="#Robert">— Il fait beau,</said> dit Robert.
%
%<!-- ... elsewhere in the document -->
%<list type="speakers">
% <item xml:id="Adolphe"/>
% <item xml:id="Albert"/>
% <item xml:id="Robert"/>
%</list>

%% esempio <quote> e <cit>

% <div xml:id="mm01" type="chapter">
% <head>Chapter 1</head>
% <epigraph>
%  <cit>
%   <quote>
%    <l>Since I can do no good because a woman</l>
%    <l>Reach constantly at something that is near it.</l>
%   </quote>
%   <bibl>
%    <title>The Maid's Tragedy</title>
%    <author>Beaumont and Fletcher</author>
%   </bibl>
%  </cit>
% </epigraph>
% <p>Miss Brooke had that kind of beauty which seems to be thrown into
%   relief by poor dress...</p>
% </div>
%%%%
% <bibl xml:id="tlk_36">Tolkien (1936)</bibl> tells us that
% <quote source="#tlk_36">
% <title>Beowulf</title> is in fact so interesting as
% poetry, in places poetry so powerful, that this quite
% overshadows the historical content
% </quote>.

% esempio <soCalled>
%He hated <soCalled>good</soCalled> books.

%% Terms, Glosses, Equivalents, and Descriptions
% <term> <gloss> membri della classe model.emphLike
% esempio:
% <gloss rend="unmarked" target="#PRSR">A computational device that infers
% structure from grammatical strings of words</gloss> is known as a
% <term xml:id="PRSR">parser</term>, and much of the history of NLP over the
% last 20 years has been occupied with the design of parsers.

% l'elemento <term> ha anche l'attributo @ref usato per collegare un termine ad un entrata di vocabolario/glossario.

% esempio enfasi e funzione dell'enfasi:

%%% controllare italico nel testo.. %%%
%A pretty common case, I believe; in all vehement debatings. She says I am too witty; Anglicé, too pert; I, that she is %too wise; that is to say, being likewise put into English, not so young as she has been: in short, she is grown so much %into a mother, that she had forgotten she ever was a daughter. ...

%A pretty common case, I believe; in all <emph>vehement</emph>
%debatings. She says I am <q rend="italic">too witty</q>;
%<foreign xml:lang="la" rend="roman">Anglicé</foreign>,
%<gloss rend="italic">too pert</gloss>; I, that she is
%<q rend="italic"> too wise</q>; that is to say, being likewise
% put into English, <gloss rend="italic">not so young as she has
% been</gloss>: in short, she is grown so much into a
%<hi rend="italic">mother</hi>, that she had forgotten she ever
% was a <hi rend="italic">daughter</hi>.
 
% Editorial changes
% attributi di att.global.responsibility, attributi di classe att.global.source, att.editLike e attributi di classe att.dimensions.

% elemento <choice>
% gli elementi di classe model.choicePart possono essere utilizzati per codificare diverse forme del testo.
% correzioni, normalizzazioni, aggiunte, espunzioni e omissioni

% Errori
% esempio:
% ... marginal comments which indicate that the
%<choice>
% <corr resp="#msm" cert="high">dates</corr>
% <sic>date's</sic>
%</choice> mentioned in the main body of the text are
% incorrect.
%
%<!-- within the header for this document ... -->
%<respStmt xml:id="msm">
% <resp>editor</resp>
% <name>C.M. Sperberg-McQueen</name>
%</respStmt>

% quando l'errore è stato corretto direttamente sulla fonte si utilizzano gli elementi <add> e <del>, se l'elemento viene corretto perché illeggibile si utilizza l'elemento <supplied>, mentre se se scioglie una abbreviazione si utilizza l'elemento <ex>

%% Normalizzazione
% esempio:

%<p>...how godly a <choice>
%  <orig>dede</orig>
%  <reg>deed</reg>
% </choice> it is to
%<choice>
%  <orig>overthrowe</orig>
%  <reg>overthrow</reg>
% </choice> so wicked a race the
% world may judge: for my part I <choice>
%  <orig>thinke</orig>
%  <reg>think</reg>
% </choice>
% there <choice>
%  <orig>canot</orig>
%  <reg>cannot</reg>
%</choice> be a greater
%<choice>
%  <orig>sacryfice</orig>
%  <reg resp="#AMDG" cert="high">sacrifice</reg>
% </choice> to God.</p>

%% Aggiunte, cancellazioni, omissioni
% gli elementi definiti dalla TEI per codificare tali fenomeni sono l'elemento <gap>, l'elemento <unclear>, l'elemento <add> e l'elemento <del>
% esempi <gap>
% <gap reason="illegible" unit="word" quantity="2"/>
% <gap reason="overwriting illegible" extent="several characters"/>
% <gap reason="sampling" quantity="120" unit="lines">
 %<desc>irrelevant commentary</desc>
%</gap>

% esempio <unclear>
% <l>And where the sandy mountain Fenwick scald</l>
%<l>
%<unclear reason="ink blot">The</unclear> sea between
%yet hence his pray'r prevail'd
%</l>

% esempio add / deletion
% <l>
%  <del rend="overstrike">Inviolable</del>
%  <add place="below">Inexplicable</add>
%  splendour of Corinthian white and gold
% </l>
