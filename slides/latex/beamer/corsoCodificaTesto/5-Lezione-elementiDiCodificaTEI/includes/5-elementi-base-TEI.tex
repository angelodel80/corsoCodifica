% capitoli 3 e 4 delle linee guida 
% estratti dal libro what is TEI (The structural organization, Varieties of textual structure, parte della TEI cornucopia part 1)
% fare esempio facsimile e codifica dei fenomeni (vedere anche slide del turco e fiormonte-ciotti)
% Elements available in All TEI Documents

% esplicitare le informazioni implicite all'interno del testo.

% i documenti TEI sono organizzati in modo molto simile tra loro: esistono elementi presenti in quasi tutti i testi codificati in TEI.

% Un testo codificato in TEI inizia sempre con l'elemento <text>, suddiviso a sua volta in frontespizio (<front>), corpo del testo (<body>) e appendice (<back>).

% Documento TEI Minimale.

%<TEI xmlns="http://www.tei-c.org/ns/1.0">
%    <teiHeader>
%        <fileDesc>
%            <titleStmt>
%                <title>The life and opinions of Tristram Shandy, Gentleman: TEI edition</title>
%            </titleStmt>
%            <publicationStmt>
%                <publisher>Web Head Press</publisher>
%                <date>2013</date> edition, 1708</p>
%                </sourceDesc>
%        </fileDesc>
%    </teiHeader>
%   <text>
%       <front>frontespizio qui</front>
%
%        <body>
%            <div type="volume" xml:id="TS01">
%                <div type="chapter" xml:id="TS0101">
%
%                    <head>Chap. I</head>
%                    <p>I wish either my father or my mother, or indeed both of them, as they were in duty both equally bound
%                        to it, had minded what they were about when they begot me; ...</p>
%                    <!-- remainder of chapter 1 here -->
%                </div>
%                <div type="chapter" xml:id="TS0102">
%
%                    <head>Chap. II</head>
%                    <p> — Then, positively, there is nothing in the question, that I can see, either good or bad. — Then let
%                        me tell you, Sir, it was a very unseasonable question at least ...</p>
%                    <!-- remainder of chapter 2 here -->
%                </div>
%                <!-- remaining chapters of volume 1 here -->
%            </div>
%            <!-- remaining volumes of work here -->
%        </body>

%        <back>appendice qui</back>
%    </text>
% </TEI>

%% gruppi principali:
% Paragrafi
% Punteggiatura
% Evidenziazione e Citazione (quotation)
% Principali elementi editoriali
% Nomi, Numeri, Date, Abbreviazioni, Indirizzi
% Collegamenti e Riferimenti
% Liste
% Note, Annotazioni e Indici
% Grafici e altre componenti non testuali
% Sistema bibliografico
% Poesia e teatro

% tra gli elementi di uso generico nella codifica dei testi più comuni, l'elemento paragrafo (<p>), insime all'elemento divisione (<div>), è di sicuro il più impiegato. L'elemento paragrafo è un elemento di classe chunk, quindi non può annidarsi.

%% Punteggiatura

%% Evidenziazione
% due classi: model.hiLike e model.emphLike.
% sono elementi phrase-level che possono comparire in paragraph level
% elementi in model.emphLike: <foreign> <emph> <distinct>
% esempio <foreign>:
% John eats a <foreign xml:lang="fr">croissant</foreign> every morning.
% esempio <emph>.

%<q>
% <emph rend="italic">What does Christopher Robin do in the morning
%   nowadays?</emph>
%</q>

%<q>
% <emph style="font-style: italic">What does Christopher Robin do in
%   the morning nowadays?</emph>
%</q>

%<l>Doth sometimes Counsel take —
% and sometimes <emph rendition="#italic">Tea</emph>.</l>
%<!-- in the header ... -->
%<rendition xml:id="italic" scheme="css">font-style: italic</rendition>

% esempio <destinct>

%Next morning a boy in that dormitory confided to his
% bosom friend, a
% <distinct time="1900" space="GB"
% social="publicschool">fag</distinct>
% of Macrea's, that there was trouble in their midst which
% King <distinct time="archaic">would fain</distinct> keep
% secret

% esempio <hi>
%<hi rend="gothic">And this Indenture further witnesseth</hi>
% that the said <hi rend="italic">Walter Shandy</hi>, merchant,
% in consideration of the said intended marriage ...

%% Quotation
% <q> <said> <quote> <cit> <mentioned> <soCalled>
% esempio <q>
% <q rendition="#quoteBefore #quoteAfter">Four score and seven years ago...</q>


% esempio <said>
%<said rend="pre(‘) post(’)">Who-e debel
% you?</said> — he at last said —
%<said rend="pre(‘) post(’)">you no speak-e,
% damme, I kill-e.</said> And so saying,
% the lighted tomahawk began flourishing
% about me in the dark.

% Attributi @who e @toWhom
% esempio:
%Adolphe se tourna vers lui :
%<said who="#Adolphe">— Alors, Albert, quoi de neuf?</said>
%<said who="#Albert">— Pas grand-chose.</said>
%<said who="#Robert">— Il fait beau,</said> dit Robert.
%
%<!-- ... elsewhere in the document -->
%<list type="speakers">
% <item xml:id="Adolphe"/>
% <item xml:id="Albert"/>
% <item xml:id="Robert"/>
%</list>

%% esempio <quote> e <cit>

% <div xml:id="mm01" type="chapter">
% <head>Chapter 1</head>
% <epigraph>
%  <cit>
%   <quote>
%    <l>Since I can do no good because a woman</l>
%    <l>Reach constantly at something that is near it.</l>
%   </quote>
%   <bibl>
%    <title>The Maid's Tragedy</title>
%    <author>Beaumont and Fletcher</author>
%   </bibl>
%  </cit>
% </epigraph>
% <p>Miss Brooke had that kind of beauty which seems to be thrown into
%   relief by poor dress...</p>
% </div>
%%%%
% <bibl xml:id="tlk_36">Tolkien (1936)</bibl> tells us that
% <quote source="#tlk_36">
% <title>Beowulf</title> is in fact so interesting as
% poetry, in places poetry so powerful, that this quite
% overshadows the historical content
% </quote>.

% esempio <soCalled>
%He hated <soCalled>good</soCalled> books.

%% Terms, Glosses, Equivalents, and Descriptions
% <term> <gloss> membri della classe model.emphLike
% esempio:
% <gloss rend="unmarked" target="#PRSR">A computational device that infers
% structure from grammatical strings of words</gloss> is known as a
% <term xml:id="PRSR">parser</term>, and much of the history of NLP over the
% last 20 years has been occupied with the design of parsers.

% l'elemento <term> ha anche l'attributo @ref usato per collegare un termine ad un entrata di vocabolario/glossario.

% esempio enfasi e funzione dell'enfasi:

%%% controllare italico nel testo.. %%%
%A pretty common case, I believe; in all vehement debatings. She says I am too witty; Anglicé, too pert; I, that she is %too wise; that is to say, being likewise put into English, not so young as she has been: in short, she is grown so much %into a mother, that she had forgotten she ever was a daughter. ...

%A pretty common case, I believe; in all <emph>vehement</emph>
%debatings. She says I am <q rend="italic">too witty</q>;
%<foreign xml:lang="la" rend="roman">Anglicé</foreign>,
%<gloss rend="italic">too pert</gloss>; I, that she is
%<q rend="italic"> too wise</q>; that is to say, being likewise
% put into English, <gloss rend="italic">not so young as she has
% been</gloss>: in short, she is grown so much into a
%<hi rend="italic">mother</hi>, that she had forgotten she ever
% was a <hi rend="italic">daughter</hi>.
 
% Editorial changes
% attributi di att.global.responsibility, attributi di classe att.global.source, att.editLike e attributi di classe att.dimensions.

% elemento <choice>
% gli elementi di classe model.choicePart possono essere utilizzati per codificare diverse forme del testo.
% correzioni, normalizzazioni, aggiunte, espunzioni e omissioni

% Errori
% esempio:
% ... marginal comments which indicate that the
%<choice>
% <corr resp="#msm" cert="high">dates</corr>
% <sic>date's</sic>
%</choice> mentioned in the main body of the text are
% incorrect.
%
%<!-- within the header for this document ... -->
%<respStmt xml:id="msm">
% <resp>editor</resp>
% <name>C.M. Sperberg-McQueen</name>
%</respStmt>

% quando l'errore è stato corretto direttamente sulla fonte si utilizzano gli elementi <add> e <del>, se l'elemento viene corretto perché illeggibile si utilizza l'elemento <supplied>, mentre se se scioglie una abbreviazione si utilizza l'elemento <ex>

%% Normalizzazione
% esempio:

%<p>...how godly a <choice>
%  <orig>dede</orig>
%  <reg>deed</reg>
% </choice> it is to
%<choice>
%  <orig>overthrowe</orig>
%  <reg>overthrow</reg>
% </choice> so wicked a race the
% world may judge: for my part I <choice>
%  <orig>thinke</orig>
%  <reg>think</reg>
% </choice>
% there <choice>
%  <orig>canot</orig>
%  <reg>cannot</reg>
%</choice> be a greater
%<choice>
%  <orig>sacryfice</orig>
%  <reg resp="#AMDG" cert="high">sacrifice</reg>
% </choice> to God.</p>

%% Aggiunte, cancellazioni, omissioni
% gli elementi definiti dalla TEI per codificare tali fenomeni sono l'elemento <gap>, l'elemento <unclear>, l'elemento <add> e l'elemento <del>
% esempi <gap>
% <gap reason="illegible" unit="word" quantity="2"/>
% <gap reason="overwriting illegible" extent="several characters"/>
% <gap reason="sampling" quantity="120" unit="lines">
 %<desc>irrelevant commentary</desc>
%</gap>

% esempio add / deletion
% <l>
%  <del rend="overstrike">Inviolable</del>
%  <add place="below">Inexplicable</add>
%  splendour of Corinthian white and gold
% </l>