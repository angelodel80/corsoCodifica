% elementi raccolti dai primi capitoli delle linee guida e dal libro What is the text encoding initiative (Lou Burnard)
% what is the Text Encoding Initiative

\documentclass{beamer}
    
%    \usepackage[english]{babel}
    %\usepackage[latin1]{inputenc}
    %\usepackage[T1]{fontenc}

\mode<presentation>{
  \setbeamertemplate{background canvas}[vertical shading]
  \usetheme{Berkeley}
  \useoutertheme{himinfolines}
}
  
\usepackage{ucs}
\usepackage[utf8]{inputenc}
\usepackage[english,polutonikogreek,italian,UKenglish,british]{babel}
\usepackage{graphicx}
\usepackage{colortbl}
\usepackage{multicol}
\usepackage{ulem}
\usepackage{verbatim}
\usepackage{alltt}
\usepackage{ccicons}
\usepackage{MnSymbol,wasysym}
\usepackage{tikzsymbols}
\usepackage{textcomp}
\usepackage{xmpincl}

\usepackage{parskip}
\setcounter{nframes}{100}
\setcounter{nframe}{1}
\setbeamercovered{dynamic}
\newenvironment{grcenv}{\begin{otherlanguage}{greek}}{\end{otherlanguage}}
\newcommand{\g}[1]{\textgreek{#1}}
\definecolor{darkgreen}{rgb}{0,0.5,0}
\definecolor{darkblue}{rgb}{0,0,0.5}
\definecolor{grey}{rgb}{0.5,0.5,0.5}
\setcounter{tocdepth}{5}

\makeatletter

\makeatother
%\includexmp{LicencesAndLicensing}

%frame00 metadata
    \title{Elementi Codifica TEI}
    \author[A.M. Del Grosso]{Angelo Mario Del Grosso}
    \institute{\texttt{angelo.delgrosso@ilc.cnr.it} \\\bigskip\textit{CNR-ILC-LicoLab} \\\bigskip\url{http://licolab.ilc.cnr.it/}}
    \date{Istituto di Linguistica Computazionale ``A. Zampolli'', \today}
    \AtBeginSection[]{
    \begin{frame}<beamer>
    \addtocounter{nframe}{1}
    \footnotesize
    \frametitle{Progress status}
    \tableofcontents[currentsection,hideothersubsections]
    \end{frame}
    }

\begin{document}

\begin{frame}
	\maketitle
\end{frame}

\begin{frame}
	\frametitle{Sommario della Lezione}
	\tableofcontents
\end{frame}

\section{Intro}

\begin{frame}
	\frametitle{Introduzione}
	\addtocounter{nframe}{1}
    
    %\begin{center}
	%    \includegraphics[width=.2\textwidth]{../imgs/tei-r.pdf}
	%\end{center}

    \begin{block}{Cos'è la TEI}
        la TEI - \textit{acronimo di Text Encoding Initiative} - rappresenta un punto di riferimento per tutte le iniziative in cui scopo principale è quello di digitalizzare risorse testuali in ambito umanistico per fini di ricerca e di conservazione.
    \end{block}

    \begin{block}{Qual è l'obiettivo della TEI}
        L'obiettivo della TEI è quello di fornire linee guida per la creazione e la gestione in forma digitale di qualsiasi tipo di dato creato e usato in ambito umanistico.
        \\ E per questo motivo investe molto nella accessibilità e nella divulgazione del sistema tecnologico che da anni sviluppa.
    \end{block}
    
\end{frame}

\begin{frame}
	\frametitle{Modularità della TEI}
	\addtocounter{nframe}{1}
    
   % \begin{center}
    % \includegraphics[width=.2\textwidth]{../imgs/tei-r.pdf}
    % \end{center}

    \begin{itemize}
        
        \item<1-> parleremo del sistema Modulare della TEI
            \begin{itemize}
                \item<1-> Moduli
                \item<1-> Classi
                \item<1-> Macro
                \item<1-> Datatype     
            \end{itemize} 
        \item<2-> parleremo degli elementi basilari
            \begin{itemize}
                \item<2-> Intestazione TEI (TEIHeader)
                \item<2-> Elementi e attributi presenti in tutti i documenti TEI
                \item<2-> Esempi di codifica
            \end{itemize} 
    \end{itemize}
    
\end{frame}

\begin{frame}
	\frametitle{I principi fondamentali della TEI}
	\addtocounter{nframe}{1}
    
    \begin{center}
	    \includegraphics[width=.2\textwidth]{../imgs/tei-r.pdf}
	\end{center}

    \begin{itemize}
        
        \item<1-> Le linee guida della TEI privilegiano il "significato" (meaning) del testo piuttosto che l'"aspetto" (layout); privileggia il modello del testo, piuttosto che il formato.
          
        \item<2-> La TEI è stata progettata per essere indipendente dagli strumenti software che la usano per la creazione oppure per l'elaborazione dei documenti elettronici.

        \item<3-> La TEI cresce, matura, si evolve sulla base delle indicazioni e delle ricerche dalla propria comunità di riferimento (community-driven).
           
    \end{itemize}
    
\end{frame}

\section{Sezione I}
\input{includes/infrastruttura-TEI.tex}

\section{Sezione II}
%% frame 01
% Every TEI document must have a TEI Header, represented by a <teiHeader> element.
% The TEI header has four main components,
% file description, encoding description, profile description, revision description

% esempio TEI header Minimale
% <teiHeader>
%    <fileDesc>
%        <titleStmt>
%            <title>Title of the work</title>
%        </titleStmt>
%        <publicationStmt>
%            <p>Information about the publication of the work</p>
%        </publicationStmt>
%        <sourceDesc>
%            <p>Information about the source from which the work was derived</p>
%        </sourceDesc>
%    </fileDesc>
% </teiHeader>

% Rifarsi al capitolo due delle linee guida per una descrizione esaustiva e sistematica

% esempio TEI descrizione file
% tre componenti obbligatorie: title statement, publication statement, descrizione della fonte.

% <titleStmt>
%    <title xml:lang="sk">Yogadarśanam (arthāt yogasūtrapūphah).</title>
%    <title>The Yoga sūtras of Patañjali: a digital edition.</title>
%    <author>Patañjali</author>
%    <funder>Wellcome Institute for the History of Medicine</funder>
%    <principal>Dominik Wujastyk</principal>
%    <respStmt>
%        <name>Wieslaw Mical</name>
%        <resp>data entry and proof</resp>
%    </respStmt>
%    <respStmt> 
%        <name>Jan Hajic</name> 
%        <resp>conversion to TEI-conformant markup</resp> 
%    </respStmt>
% </titleStmt>

% esempio publication statement

%% Descrizione della fonte
% document formally the object or objects from which the TEI document has been derived, using traditional bibliographic
% Born Digital
% Printed Source
% sbobbinatura
% manoscritto (msDesc)

%% Esempio di sourceDesc
% <sourceDesc>
%    <bibl xml:id="Sue1846">
%        <author>
%            <surname>Sue</surname>,
%            <forename>Eugène</forename>
%        </author>
%        <title level="m">Martin, l’enfant trouvé : Mémoires d’un valet de chambre</title>
%        <imprint>
%            <publisher>C. Muquardt</publisher>
%            <pubPlace>Bruxelles</pubPlace>
%            <pubPlace>Leipzig</pubPlace>
%            <date when="1846">MDCCCXLVI</date>
%        </imprint>
%    </bibl>
% </sourceDesc>


% file description 
\subsection{File Description}

 esempio TEI descrizione file
% tre componenti obbligatorie: title statement, publication statement, descrizione della fonte.

% <titleStmt>
%    <title xml:lang="sk">Yogadarśanam (arthāt yogasūtrapūphah).</title>
%    <title>The Yoga sūtras of Patañjali: a digital edition.</title>
%    <author>Patañjali</author>
%    <funder>Wellcome Institute for the History of Medicine</funder>
%    <principal>Dominik Wujastyk</principal>
%    <respStmt>
%        <name>Wieslaw Mical</name>
%        <resp>data entry and proof</resp>
%    </respStmt>
%    <respStmt> 
%        <name>Jan Hajic</name> 
%        <resp>conversion to TEI-conformant markup</resp> 
%    </respStmt>
% </titleStmt>

% esempio publication statement

%% Descrizione della fonte
% document formally the object or objects from which the TEI document has been derived, using traditional bibliographic
% Born Digital
% Printed Source
% sbobbinatura
% manoscritto (msDesc)

%% Esempio di sourceDesc
% <sourceDesc>
%    <bibl xml:id="Sue1846">
%        <author>
%            <surname>Sue</surname>,
%            <forename>Eugène</forename>
%        </author>
%        <title level="m">Martin, l’enfant trouvé : Mémoires d’un valet de chambre</title>
%        <imprint>
%            <publisher>C. Muquardt</publisher>
%            <pubPlace>Bruxelles</pubPlace>
%            <pubPlace>Leipzig</pubPlace>
%            <date when="1846">MDCCCXLVI</date>
%        </imprint>
%    </bibl>
% </sourceDesc>

% encoding description
\subsection{Encoding Description}
% Encoding descsription (<encodingDesc />) is used to supply information about almost any aspect of the encoding process itself

%% esempio descrizione della codifica
% <encodingDesc>
%    <projectDesc>
%        <p>Texts collected for use in the Claremont Shakespeare Clinic, June 1990.</p>
%    </projectDesc>
%    <samplingDecl>
%        <p>Each text contains a sample of up to 2000 words, running from the start of the document to the end of the sentence
%            after the 2000 word mark. For the purposes of word counting, hyphens and apostrophes were treated as spaces.
%            </p>
%    </samplingDecl>
%    <editorialDecl>
%        <normalization>
%            <p>Word forms broken by end of line hyphenation have been reconstructed without comment. The hyphen has been removed
%                except for hyphenated forms attested elsewhere in the text. </p>
%        </normalization>
%        <quotation marks="all" form="std">
%            <p>All quotation marks have been removed. Direct speech is represented by the use of the
%                <gi>said</gi> tag; other quoted material is represented by means of the
%                <gi>q</gi> tag. </p>
%        </quotation>
%    </editorialDecl>
% </encodingDesc>

%% for human reader and for automated process

% esempio charDecl
% <charDecl>
%    <glyph xml:id="z103">
%        <glyphName>LATIN LETTER Z WITH TWO STROKES</glyphName>
%        <mapping type="standardized">z</mapping>
%        <mapping type="PUA">U+E304</mapping>
%    </glyph>
% </charDecl>

% segue nel testo 
% <p> ... mulct<g ref="#z103"/> ... </p>

% tassonomia

% <classDecl>
%    <taxonomy xml:id="size">
%        <category xml:id="large">
%            <catDesc>story occupies more than half a page</catDesc>
%        </category>
%        <category xml:id="medium">
%            <catDesc>story occupies between quarter and a half page</catDesc>
%        </category>
%        <category xml:id="small">
%            <catDesc>story occupies less than a quarter page</catDesc>
%        </category>
%        <!-- etc -->
%    </taxonomy>
%    <taxonomy xml:id="topic">
%        <category xml:id="politics-domestic">
%            <catDesc>Refers to domestic political events</catDesc>
%        </category>
%        <category xml:id="politics-foreign">
%            <catDesc>Refers to foreign political events</catDesc>
%        </category>
%        <category xml:id="social-women">
%            <catDesc>refers to role of women in society</catDesc>
%        </category>
%        <category xml:id="social-servants">
%            <catDesc>refers to role of servants in society</catDesc>
%        </category>
%        <!-- etc -->
%    </taxonomy>
% </classDecl>

% uso nel testo 
% <catRef target="#small #social-women"/>

%% esempio dichiarazione dei tag
% <tagsDecl>
%    <rendition xml:id="IT" scheme="css">font-style:italic</rendition>
%    <rendition xml:id="FontRoman" scheme="css">font-family: serif</rendition>
%    <namespace name="http://www.tei-c.org/ns/1.0">
%        <tagUsage gi="emph" render="#IT" />
%        <tagUsage gi="hi" render="#IT" />
%        <tagUsage gi="text" render="#FontRoman" />
%    </namespace>
% </tagsDecl>

% uso nel testo
% @rendition attribute

% profile description
\subsection{Profile Description}
% elemento che descrive il profilo del documento TEI
% elementi di metadatazione non bibliografici
% elementi per edentificare i vari passaggi genetici elemento <listChange />
 %% esempio proust di Pierazzo
 %% esempio carteggio

 %% esempio profile
% <profileDesc>
%    <creation>
%        <date when="1962" /> </creation>
%    <textClass>
%        <catRef target="#WRI #ALLTIM1 #ALLAVA2 #ALLTYP3 #WRIDOM5 #WRILEV2 #WRIMED1 #WRIPP5 #WRISAM3 #WRISTA2 #WRITAS0" />
%        <classCode scheme="DLEE">W nonAc: humanities arts</classCode>
%        <keywords scheme="COPAC">
%            <term>History, Modern - 19th century</term>
%            <term>Capitalism - History - 19th century</term>
%            <term>World, 1848-1875</term>
%        </keywords>
%    </textClass>
% </profileDesc>

% revision description
\subsection{Revision Description}
 represented by a <revisionDesc> element which contains a list of <change>
% given first. The <listChange> element mentioned above may also be used here to refer to identified stages in the evolution of the electronic file

% esempio
% <revisionDesc>
%    <listChange>
%        <change when="2013-05-11">First complete draft</change>
%        <change when="2013-04-07">Created header and document structure</change>
%    </listChange>
% </revisionDesc>

\section{Sezione III}
\input{includes/elementi-base-TEI.tex}

\section{Sezione IV}
% ODD document, Selezione dei Moduli per lo schema, nuovi elementi, profilo personalizzato TEI, capitolo 22 delle linee guida (Documentation Elements), capitolo 23 delle linee guida (Using the TEI)


\section{Conclusioni}
%conclusioni f01
\begin{frame}
    \frametitle{References}
    \addtocounter{nframe}{1}
    \begin{thebibliography}{10}

        \setbeamertemplate{bibliography item}[paper]
        \tiny\bibitem{ciotti2011} Ciotti 2011

        \setbeamertemplate{bibliography item}[online]
        \tiny\bibitem{CCwikiBY2014} CC Wiki \textit{Best practices for attribution}, CC Wiki 2014, \url{https://wiki.creativecommons.org/wiki/Best\_practices\_for\_attribution}

    \end{thebibliography}

\end{frame}


\end{document}