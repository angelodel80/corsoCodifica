%  la codifica testuale è la rappresentazione formale di un testo e delle sue caratteristiche mediante un linguaggio informa- tico. (ciotti)

% le descrizioni per la rappresentazione del testo devono essere opportunamente formalizzati per poter essere "comprensibili" dal colcolatore/sistema software.

% La codifica informatica del testo è un linguaggio teorico che per- mette allo studioso di costruire modelli formali del testo.


% La rappresentazione che vogliamo eseguire deve essere eseguira mediante le istruzioni, le convenzioni e i costrutti  messi a disposizione da un opportuno linguaggio che sarà definito formalmente da una specifica sintassi e da una precisa semantica.
% linguaggio in cui tutti i termini sono definiti esplicitamente e usati in modo conforme a tali definizioni.

% Si deve notare che «ogni dato su cui l’elaboratore deve operare viene rappresentato a livello elementare mediante una sequenza (o stringa) di simboli

% Tra questi hanno una notevole importanza ai fini della modellizzazione di testi, quei sistemi basati sui cosiddetti markup language.

% Il termine inglese markup designava nella stampa tipografica tutte le indicazioni e annotazioni simboliche aggiunte dall’autore o dall’editore su un manoscritto o su un dattiloscritto per istruire il tipografo

% Similmente un markup language è costituito da un set di istruzioni di un vero e proprio linguaggio orientato alla descrizioni dei fenomeni di composizione e struttura del testo.

% Un linguaggio di Markup, quindi, è un formalismo artificiale con il quale poter esprimente la rappresentazione o il modello del testo considerato.
% Un linguaggio (formale) sull'alfabeto A non è altro che un sottoinsieme di A*. Una grammatica formale serve proprio a definire un certo sottoinsieme di stringhe tra tutte quelle possibili su un dato alfabeto.

% Crediamo che una adeguata soluzione pragmatica di questi problemi sia da individuare nel- l’associazione di un modello implementato da un markup language, fin dove è possibile, e di un modello grafico del documento, implementato in uno dei formati grafici standard, tra quelli sviluppati nell’ambito delle tecnologie di computer grafica.

% Una tale forma di modellizzazione informatica di un do- cumento testuale risulterebbe, peraltro, la più adeguata nel caso speci- fico dei manoscritti, per i quali nessuna descrizione di tipo linguistico sarebbe in grado di rappresentare tutte le informazioni visuali che una immagine digitale è in grado di veicolare.

% l’applicazione di metodologie computazionali nell’ambito della ricerca umanistica comporta due tipi, o meglio due fasi di formalizzazione:
• definizione e implementazione di strutture dati adeguate alla cattura dei fenomeni di interesse dell’umanista, e in particolare alla rappresentazione formale dei testi;
• specificazione di algoritmi che, applicati alle strutture dati, sia- no in grado di simulare i processi di manipolazione dei testi ti- pici della ricerca umanistica o in generale delle pratiche sociali che hanno a che fare in vario modo con i testi.
Il problema della codifica testuale rientra in generale nel primo tipo di formalizzazione,

% definizione e implementazione di un linguaggio formale che deve essere a un tempo processabile da un elaboratore e sufficientemente espressivo per rappresentare la complessità dell’oggetto testo.

%Più precisamente uno schema di codifica associa un insieme di caratteristiche o elementi costituenti di un oggetto testuale a un insieme di simboli, e le relazioni tra gli elementi testuali a relazioni sintattiche tra i simboli.
%% Un esempio..

% I linguaggi per la codifica testuale vengono denominati nella lette- ratura anglosassone markup language, linguaggi di marcatura.

% i linguaggi di markup infatti, consistono di un insieme di simboli che vengono inse- riti all’interno o accanto al testo verbale.

% Codificare un testo significa esplicitare i processi inferenziali effet- tuati da un interprete nella comprensione del testo stesso.

% Il principale requisito di uno schema di codifica, pertanto, è la capacità rappresen- tazionale che esso offre allo studioso

% rappresentare adeguatamente i differenti fenomeni testuali che vengono studiati da varie discipline;

% In modo parallelo ai linguaggi di programmazione, anche i linguaggi di markup possono essere divisi in due tipologie: linguaggi procedurali, che nella letteratura vengono indicati anche come specific markup language; e linguaggi dichiarativi o descrittivi, detti anche generic markup language10.

% I sistemi di codifica procedurale sono per definizione orientati a una singola applicazione. la portabilità di un testo co- dificato con sistemi procedurali è molto limitata.

% invece di specificare quali operazioni di formattazione vanno effettuate in un particolare punto del testo, si di- chiara che un dato segmento testuale è istanza di un tipo di struttura editoriale del testo; insomma, si dichiara: “questo è un titolo”

% Un sistema di codifica dichiarativo dunque è orientato alla rappresentazione delle caratteristiche o elementi che costituiscono un testo, indipenden- temente dalle finalità specifiche per le quali il testo è stato memorizzato e codificato.
%Come i linguaggi procedurali, anche quelli dichiarativi vengono u- tilizzati inserendo all’interno del file di testo sequenze di caratteri. generalmente dette tag (etichette o marche)

% In ultima analisi, la codifica informatica di un testo può essere vista come il prodotto di un insieme di inferenze che vengono espresse mediante un linguaggio formalizzato. (ciotti)

% La codifica informatica di un testo provvede un sistema linguistico formalizzato che permette a uno studioso di «rendere esplicita una in- terpretazione di un testo» [Burnard, 1995: 43], e le varie operazioni inferenziali implicite che la hanno prodotta. I sistemi dichiarativi forniscono un potente dispositivo metalinguistico.

%La visione pluralista del testo portata alle sue estreme conseguenze, eccede i limiti sintattici di un formalismo di codifica come XML. Lo standard, infatti, non è dotato di costrutti sintattici adeguati alla rappresentazione di molteplici sottoprospettive gerarchi- che concorrenti che si sovrappongono ma che possono anche collegar- si e interrelarsi.

% non possiamo dire apriori che uno schema di codifica testuale coglie l’essenza del testo più e meglio di un altro in base a un qualche assun- to metafisico. Ma neppure si può affermare che ogni rappresentazione è vera in quanto costituisce il suo oggetto testo secondo esigenze spe- cifiche e locali.

Ogni model- lo descrive le caratteristiche del testo a un determinato livello, in base al punto di vista dell’osservatore, ma non coincide con esse.

% 