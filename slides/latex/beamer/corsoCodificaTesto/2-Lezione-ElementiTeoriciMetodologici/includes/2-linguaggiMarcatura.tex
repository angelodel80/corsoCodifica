% VEdere Slide Rosselli-Chiara lezione sui markup languages (le prime slide)

% intro linguaggi di Markup

% La rappresentazione che vogliamo eseguire deve essere eseguira mediante le istruzioni, le convenzioni e i costrutti  messi a disposizione da un opportuno linguaggio che sarà definito formalmente da una specifica sintassi e da una precisa semantica.
% linguaggio in cui tutti i termini sono definiti esplicitamente e usati in modo conforme a tali definizioni.

% Si deve notare che «ogni dato su cui l’elaboratore deve operare viene rappresentato a livello elementare mediante una sequenza (o stringa) di simboli


% In modo parallelo ai linguaggi di programmazione, anche i linguaggi di markup possono essere divisi in due tipologie: linguaggi procedurali, che nella letteratura vengono indicati anche come specific markup language; e linguaggi dichiarativi o descrittivi, detti anche generic markup language.

% I sistemi di codifica procedurale sono per definizione orientati a una singola applicazione. la portabilità di un testo codificato con sistemi procedurali è molto limitata.


% nei linguaggi di markup dichiarativi/descrittivi invece di specificare quali operazioni di formattazione vanno effettuate in un particolare punto del testo, si dichiara che un dato segmento testuale è istanza di un tipo di struttura editoriale del testo; insomma, si dichiara: “questo è un titolo”


% Un sistema di codifica dichiarativo dunque è orientato alla rappresentazione delle caratteristiche o elementi che costituiscono un testo, indipendentemente dalle finalità specifiche per le quali il testo è stato memorizzato e codificato.

% Tra questi hanno una notevole importanza ai fini della modellizzazione di testi, quei sistemi basati sui cosiddetti markup language.

% Il termine inglese markup designava nella stampa tipografica tutte le indicazioni e annotazioni simboliche aggiunte dall’autore o dall’editore su un manoscritto o su un dattiloscritto per istruire il tipografo

% Similmente un markup language è costituito da un set di istruzioni di un vero e proprio linguaggio orientato alla descrizioni dei fenomeni di composizione e struttura del testo.

% i linguaggi di markup infatti, consistono di un insieme di simboli che vengono inseriti all’interno o accanto al testo verbale.

% Un linguaggio di Markup, quindi, è un formalismo artificiale con il quale poter esprimente la rappresentazione o il modello del testo considerato.
% Un linguaggio (formale) sull'alfabeto A non è altro che un sottoinsieme di A*. Una grammatica formale serve proprio a definire un certo sottoinsieme di stringhe tra tutte quelle possibili su un dato alfabeto.



%Come i linguaggi procedurali, anche quelli dichiarativi vengono utilizzati inserendo all’interno del file di testo sequenze di caratteri. generalmente dette tag (etichette o marche)

%Più precisamente uno schema di codifica associa un insieme di caratteristiche o elementi costituenti di un oggetto testuale a un insieme di simboli, e le relazioni tra gli elementi testuali a relazioni sintattiche tra i simboli.
%% Un esempio (per esempio capitolo-titolo-paragrafo)..