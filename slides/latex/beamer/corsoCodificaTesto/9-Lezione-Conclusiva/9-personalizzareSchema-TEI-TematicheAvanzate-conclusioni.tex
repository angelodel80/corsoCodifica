% strumenti per la pubblicazione
% strumenti per l'analisi (TXM, altri)
% slides James Cummings 
% customizzazione TEI vedere capitoli 22 e 23 delle guidelines

% DOM, Javascript XML, XSL
% capitolo 7 testo e automa di ciotti
% SLIDE Chiara sui fogli di stile
% Capitolo DOM professional Web Dev e XML processing
%

\documentclass{beamer}
    
    %    \usepackage[english]{babel}
        %\usepackage[latin1]{inputenc}
        %\usepackage[T1]{fontenc}
    
    \mode<presentation>{
      \setbeamertemplate{background canvas}[vertical shading]
      \usetheme{Berkeley}
      \useoutertheme{himinfolines}
    }
      
    \usepackage{ucs}
    \usepackage[utf8]{inputenc}
    \usepackage[english,polutonikogreek,italian,UKenglish,british]{babel}
    \usepackage{graphicx}
    \usepackage{colortbl}
    \usepackage{multicol}
    \usepackage{ulem}
    \usepackage{verbatim}
    \usepackage{alltt}
    \usepackage{ccicons}
    \usepackage{MnSymbol,wasysym}
    \usepackage{tikzsymbols}
    \usepackage{textcomp}
    \usepackage{xmpincl}
    
    \usepackage{parskip}
    \setcounter{nframes}{100}
    \setcounter{nframe}{1}
    \setbeamercovered{dynamic}
    \newenvironment{grcenv}{\begin{otherlanguage}{greek}}{\end{otherlanguage}}
    \newcommand{\g}[1]{\textgreek{#1}}
    \definecolor{darkgreen}{rgb}{0,0.5,0}
    \definecolor{darkblue}{rgb}{0,0,0.5}
    \definecolor{grey}{rgb}{0.5,0.5,0.5}
    \setcounter{tocdepth}{5}
    
    \makeatletter
    
    \makeatother
    %\includexmp{LicencesAndLicensing}
    
    %frame00 metadata
        \title{Codifica TEI - Personalizzazione e Tematiche avanzate}
        \author[A.M. Del Grosso]{Angelo Mario Del Grosso \\ \tiny\textit{(materiale parzialmente derivato dalle lezioni della Prof. C. Di Pietro))}}
        \institute{\texttt{angelo.delgrosso@ilc.cnr.it} \\\textit{CNR-ILC-LicoLab} \\\url{http://licolab.ilc.cnr.it/}}
        \date{Istituto di Linguistica Computazionale ``A. Zampolli'', \today}
        \AtBeginSection[]{
        \begin{frame}<beamer>
        \addtocounter{nframe}{1}
        \footnotesize
        \frametitle{Progress status}
        \tableofcontents[currentsection,hideothersubsections]
        \end{frame}
        }
    
    \begin{document}
    
    \begin{frame}
        \maketitle
    \end{frame}
    
    \begin{frame}
        \frametitle{Sommario della Lezione}
        \tableofcontents
    \end{frame}
    
    \section{Introduzione}
    
    \begin{frame}
        \frametitle{Visualizzare ed Elaborare documenti XML}
        \addtocounter{nframe}{1}
        
        %\begin{center}
        %    \includegraphics[width=.2\textwidth]{../imgs/tei-r.pdf}
        %\end{center}
        %\textit{In parte già disponibili nei moduli TEI di base}

        %  \begin{block}{Perché visualizzare ed elaborare il testo}
        % %     \emph{Per la critica testuale indispensabili i moduli}
        %      \begin{itemize}
        %         \item Controllare la codifica e correggere i refusi
        %         \item Assicurarsi che tutto sia stato trascritto correttamente
        %         \item Mostrare il testo a persone che non conoscono XML-TEI
        %         \item Disporre di una versione del lavoro fruibile
        %         \item Manipolare e analizzare le informazioni codificate
        %     \end{itemize}
        %  \end{block}
        
    \end{frame}
    
    % \begin{frame}
    %     \frametitle{Modularità della TEI}
    %     \addtocounter{nframe}{1}
        
    %    % \begin{center}
    %     % \includegraphics[width=.2\textwidth]{../imgs/tei-r.pdf}
    %     % \end{center}
    
    %     \begin{itemize}
            
    %         \item<1-> parleremo del sistema Modulare della TEI
    %             \begin{itemize}
    %                 \item<1-> Moduli
    %                 \item<1-> Classi
    %                 \item<1-> Macro
    %                 \item<1-> Datatype
    %             \end{itemize} 
    %         \item<2-> parleremo degli elementi basilari
    %             \begin{itemize}
    %                 \item<2-> Intestazione TEI (TEIHeader)
    %                 \item<2-> Elementi e attributi presenti in tutti i documenti TEI
    %                 \item<2-> Esempi di codifica
    %             \end{itemize} 
    %     \end{itemize}
        
    % \end{frame}
    
    \section{Personalizzare Schemi TEI}
    
    \begin{frame}
        \frametitle{Modularità e personalizzazione della TEI}
        \addtocounter{nframe}{1}
        
       % \begin{center}
        % \includegraphics[width=.2\textwidth]{../imgs/tei-r.pdf}
        % \end{center}
    
        \begin{block}{Vocabolario TEI P5}
            \textbf{Oltre 550 elementi presenti nel vocabolario completo della TEI}
                \begin{itemize}
                    \item Difficilmente serviranno tutti per un singolo progetto
                    \item Probabilmente per un particolare progetto servirà un elemento
                    non previsto
                \end{itemize} 
            \end{block}
    \end{frame}

    \begin{frame}
        \frametitle{Modularità e personalizzazione della TEI}
        \addtocounter{nframe}{1}
        
       % \begin{center}
        % \includegraphics[width=.2\textwidth]{../imgs/tei-r.pdf}
        % \end{center}
    
        \begin{block}{Design dello Schema TEI P5}
            
            \textbf{la TEI P5 è progettata per ottimizzare:}
                \begin{itemize}
                    \item aspetti di \textbf{modularità} che consentano agli utenti di selezionare solo le componenti necessarie (moduli).
                    \item aspetti di \textbf{estensiblità} che consentano agli utenti di aggiungere nuovi elementi e attributi
                    \item aspetti di \textbf{riusabilità} che consentano agli utenti di modificare elementi e attributi già esistenti
                \end{itemize} 
        \end{block}
        
    \end{frame}

    \begin{frame}
        \frametitle{Modularità e personalizzazione della TEI}
        \addtocounter{nframe}{1}
        
       % \begin{center}
        % \includegraphics[width=.2\textwidth]{../imgs/tei-r.pdf}
        % \end{center}
    
        \begin{block}{Personalizzare lo schema TEI P5}
            
            \textbf{la TEI P5 mette a disposizione}
                \begin{itemize}
                    \item vocabolario XML per la codifica di fenomeni testuali (concetti del testo)
                    \item sistema di moduli e di personalizzazione del vocabolario
                    \item strumenti per la personalizzazione: TEI Roma e ODD
                    \item [] \url{http://www.tei-c.org/Roma/}
                \end{itemize} 
        \end{block}
        
    \end{frame}

    \begin{frame}
        \frametitle{Modularità e personalizzazione della TEI}
        \addtocounter{nframe}{1}

        \begin{block}{alcuni elementi strutturali importanti distinguibili in un libro}
                \begin{itemize}
                    \item Il documento, la pagina del titolo, i capitoli
                    \item Le intestazioni, le sezioni e sotto-sezioni, i paragrafi
                    \item Le citazioni, i discorsi diretti
                    \item Le interruzioni di pagina e di linea
                    \item Le figure, i nomi di persone, i nomi di luoghi
                \end{itemize} 
        \end{block}
        
    \end{frame}


    \begin{frame}
        \frametitle{Modularità e personalizzazione della TEI}
        \addtocounter{nframe}{1}

        \begin{block}{Due possibili strade di personalizzazione}
                \begin{itemize}
                    \item Scegliere una delle personalizzazioni pronte all’uso disponibili sul sito della TEI
                    \item Creare da zero il proprio schema TEI
                \end{itemize} 
        \end{block}
        
    \end{frame}


    \begin{frame}
        \frametitle{Modularità e personalizzazione della TEI}
        \addtocounter{nframe}{1}

        \begin{block}{Scegliere personalizzazioni già esistenti}
                \begin{itemize}
                    \item \url{http://www.tei-c.org/Guidelines/Customization/}
                    \item Incentrati su particolari aspetti e caratteristiche del testo
                    \item Da utilizzare così come sono, oppure da sfruttare come punto di
                    partenza per una propria personalizzazione
                \end{itemize} 
        \end{block}
        
    \end{frame}

    \begin{frame}
        \frametitle{Modularità e personalizzazione della TEI}
        \addtocounter{nframe}{1}
        
        \textbf{Utilizzo di customizzazioni già esistenti}

         \begin{center}
            \includegraphics[width=.9\textwidth]{imgs/customization.png}
         \end{center}
       
        
    \end{frame}

    \begin{frame}
        \frametitle{Modularità e personalizzazione della TEI}
        \addtocounter{nframe}{1}

        \begin{block}{Creare da zero il proprio schema TEI}
                \begin{itemize}
                    \item TEI Roma: \url{http://www.tei-c.org/Roma/}
                    \item Selezione e restrizione del modello TEI
                    \item Estensione del modello TEI
                \end{itemize} 
        \end{block}
        
    \end{frame}

    \begin{frame}
        \frametitle{Modularità e personalizzazione della TEI}
        \addtocounter{nframe}{1}
        
        \textbf{Creazione di uno schema con l'applicazione Web Roma}

         \begin{center}
            \includegraphics[width=.95\textwidth]{imgs/Roma1.png}
         \end{center}
       
        
    \end{frame}

    \begin{frame}
        \frametitle{Modularità e personalizzazione della TEI}
        \addtocounter{nframe}{1}
        
        \textbf{Passo 1: Inizializzare la personalizzazione}

         \begin{center}
            \includegraphics[width=.95\textwidth]{imgs/Roma2.png}
         \end{center}
       
        
    \end{frame}

%     ▪ New → scelta del punto di partenza
% ▪ Customize → personalizzazione metadati
% ▪ Language → lingua schema e documentazione
% ▪ Modules → scelta degli elementi TEI
% ▪ Add Elements → aggiunta elementi
% ▪ Change classes → gestione attributi
% ▪ Schema → generazione schema
% ▪ Documentation → creazione documentazione
% ▪ Save Customization → salvataggio file ODD


\begin{frame}
    \frametitle{Modularità e personalizzazione della TEI}
    \addtocounter{nframe}{1}
    
    \textbf{Passo 2: Selezionare la lingua dello schema}

     \begin{center}
        \includegraphics[width=.95\textwidth]{imgs/Roma3.png}
     \end{center}
   
    
\end{frame}

\begin{frame}
    \frametitle{Modularità e personalizzazione della TEI}
    \addtocounter{nframe}{1}
    
    \textbf{Passo 3: Selezionare i moduli da includere}

     \begin{center}
        \includegraphics[width=.9\textwidth]{imgs/Roma4.png}
     \end{center}
   
    
\end{frame}

\begin{frame}
    \frametitle{Modularità e personalizzazione della TEI}
    \addtocounter{nframe}{1}
    
    \textbf{Passo 4:  Rimuovere i moduli già inclusi}


     \begin{center}
        \includegraphics[width=.97\textwidth]{imgs/Roma5.png}
     \end{center}
   
    
\end{frame}

\begin{frame}
    \frametitle{Modularità e personalizzazione della TEI}
    \addtocounter{nframe}{1}
    
    
    \textbf{Passo 5: Selezionare gli elementi da includere/escludere}

     \begin{center}
        \includegraphics[width=.95\textwidth]{imgs/Roma6.png}
     \end{center}
   
    
\end{frame}

\begin{frame}
    \frametitle{Modularità e personalizzazione della TEI}
    \addtocounter{nframe}{1}
    
    \textbf{Passo 6: Selezionare gli attributi da includere/escudere}

     \begin{center}
        \includegraphics[width=.9\textwidth]{imgs/Roma7.png}
     \end{center}
   
    
\end{frame}

    
\begin{frame}
    \frametitle{Modularità e personalizzazione della TEI}
    \addtocounter{nframe}{1}
    
    \textbf{Passo 7: Aggiungere uno o più nuovi elementi}

     \begin{center}
        \includegraphics[width=.9\textwidth]{imgs/Roma9.png}
     \end{center}
    
\end{frame}
   
\begin{frame}
    \frametitle{Modularità e personalizzazione della TEI}
    \addtocounter{nframe}{1}
    
    \textbf{Passo 8: Aggiungere uno o più nuovi attributi}

     \begin{center}
        \includegraphics[width=.9\textwidth]{imgs/Roma14.png}
     \end{center}
     
    
\end{frame}
    
\begin{frame}
    \frametitle{Modularità e personalizzazione della TEI}
    \addtocounter{nframe}{1}

    \textbf{Passo 9: Validare gli elementi e gli attributi inclusi}

     \begin{center}
        \includegraphics[width=.9\textwidth]{imgs/Roma8.png}
     \end{center}
   
    
\end{frame}


\begin{frame}
    \frametitle{Modularità e personalizzazione della TEI}
    \addtocounter{nframe}{1}
    
    \textbf{Passo 10: personalizzare le classi predefinite TEI }

     \begin{center}
        \includegraphics[width=.9\textwidth]{imgs/Roma10.png}
     \end{center}
   
    
\end{frame}

\begin{frame}
    \frametitle{Modularità e personalizzazione della TEI}
    \addtocounter{nframe}{1}
    
    \textbf{Passo 11: Modificare le classi predefinite TEI}

     \begin{center}
        \includegraphics[width=.9\textwidth]{imgs/Roma11.png}
     \end{center}
   
    
\end{frame}

\begin{frame}
    \frametitle{Modularità e personalizzazione della TEI}
    \addtocounter{nframe}{1}
   
   
    \textbf{Passo 12: Generazione dello schema}

     \begin{center}
        \includegraphics[width=.9\textwidth]{imgs/Roma12.png}
     \end{center}
   
    
\end{frame}

\begin{frame}
    \frametitle{Modularità e personalizzazione della TEI}
    \addtocounter{nframe}{1}
    
    \textbf{Passo 13: Generazione della documentazione}

     \begin{center}
        \includegraphics[width=.95\textwidth]{imgs/Roma13.png}
     \end{center}
    
\end{frame}

\begin{frame}
    \frametitle{Modularità e personalizzazione della TEI}
    \addtocounter{nframe}{1}
    
    \textbf{Passo 14: Generazione del documento ODD}

     \begin{center}
        \includegraphics[width=.65\textwidth]{imgs/CustomizationODD.png}
     \end{center}
    
\end{frame}


\begin{frame}
    \frametitle{Modularità e personalizzazione della TEI}
    \addtocounter{nframe}{1}
    
    \textbf{Documento ODD}

     \begin{center}
        \includegraphics[width=.97\textwidth]{imgs/CustomizationODD-1.png}
     \end{center}
    
\end{frame}

\begin{frame}
    \frametitle{Modularità e personalizzazione della TEI}
    \addtocounter{nframe}{1}
    
    \textbf{Documento ODD}

     \begin{center}
        \includegraphics[width=.97\textwidth]{imgs/CustomizationODD-2.png}
     \end{center}
    
\end{frame}

\begin{frame}
    \frametitle{Modularità e personalizzazione della TEI}
    \addtocounter{nframe}{1}
    
    \textbf{Schema dtd con le personalizzazioni}

     \begin{center}
        \includegraphics[width=.97\textwidth]{imgs/TEI-Custom-DTD.png}
     \end{center}
   
    
\end{frame}

\begin{frame}
    \frametitle{Modularità e personalizzazione della TEI}
    \addtocounter{nframe}{1}
    
    \textbf{Schema dtd con le personalizzazioni}

     \begin{center}
        \includegraphics[width=.9\textwidth]{imgs/TEI-Custom-DTD-2.png}
     \end{center}
   
    
\end{frame}



\begin{frame}
    \frametitle{Modularità e personalizzazione della TEI}
    \addtocounter{nframe}{1}

    \begin{block}{Approfondimenti}
            \begin{itemize}
                \item Customizing the TEI with Roma
                \item [] \url{http://www.tei-c.org/Guidelines/Customization/use_roma.xml}
                \item Getting Started with P5 ODDs
                \item [] \url{http://www.tei-c.org/Guidelines/Customization/odds.xml}
                \item TEI By Example: Customising TEI, ODD, Roma
                \item [] \url{http://teibyexample.org/modules/TBED08v00.htm}
            \end{itemize} 
    \end{block}

\end{frame}

    
    \section{Tematiche Avanzate per la codifica del testo in TEI}
    
    \begin{frame}
        \frametitle{Modularità della TEI}
        \addtocounter{nframe}{1}
        
       % \begin{center}
        % \includegraphics[width=.2\textwidth]{../imgs/tei-r.pdf}
        % \end{center}
    
        \begin{itemize}
            
            \item<1-> parleremo del sistema Modulare della TEI
                \begin{itemize}
                    \item<1-> Moduli
                    \item<1-> Classi
                    \item<1-> Macro
                    \item<1-> Datatype
                \end{itemize} 
            \item<2-> parleremo degli elementi basilari
                \begin{itemize}
                    \item<2-> Intestazione TEI (TEIHeader)
                    \item<2-> Elementi e attributi presenti in tutti i documenti TEI
                    \item<2-> Esempi di codifica
                \end{itemize} 
        \end{itemize}
        
    \end{frame}
    
    \section{Conclusioni}
    % strumenti per la pubblicazione
% strumenti per l'analisi (TXM, altri)
% slides James Cummings 
% customizzazione TEI vedere capitoli 22 e 23 delle guidelines
    
    \end{document}