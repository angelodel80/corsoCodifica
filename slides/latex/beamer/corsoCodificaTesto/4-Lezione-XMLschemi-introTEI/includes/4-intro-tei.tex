%%Codifica di testi

% Le norme TEI
% Roberto Rosselli Del Turco
% Dipartimento di Studi Umanistici
% Università di Torino
% roberto.rossellidelturco@fileli.unipi.it
% roberto.rossellidelturco@unito.itLa codifica di testi – Le norme TEI
% La Text Encoding Initiative
% Sito WWW: http://www.tei-c.org/

\begin{frame}
	\frametitle{Intro Text Encoding Initiative}
	\framesubtitle{TEI}
	\addtocounter{nframe}{1}

	\begin{block}{Motto}
		TEI: Yesterday's information tomorrow
	\end{block}

	\begin{block}{Dal sito TEI}
		“an international and interdisciplinary standard that
		enables libraries, museums, publishers, and individual
		scholars to represent a variety of literary and linguistic
		texts for online research, teaching, and preservation”
	\end{block}
\end{frame}


\begin{frame}
	\frametitle{Intro Text Encoding Initiative}
	\framesubtitle{TEI}
	\addtocounter{nframe}{1}

	\begin{block}{testo di riferimento}
		Guidelines for Electronic Text Encoding and Interchange ( http://www.tei-c.org/Guidelines/ )
	\end{block}

	\begin{block}{testo di ausilio}
		BURNARD, Lou. What is the Text Encoding Initiative? How to add intelligent markup to digital resources. Nouva edizione [online]. Marseille: OpenEdition Press, 2014 (creato il 13 octobre 2018). Disponibile su Internet: <http://books.openedition.org/oep/426>. ISBN: 9782821834606. DOI: 10.4000/books.oep.426.

	\end{block}
\end{frame}


\begin{frame}
	\frametitle{Intro Text Encoding Initiative}
	\framesubtitle{TEI}
	\addtocounter{nframe}{1}

	\begin{block}{un po' di storia}
		\begin{itemize}
			\item 1987: necessità di standard che permetta la creazione e l’interscambio di documenti per mezzo di archivi informatici(convegno NY)
			\item 1990: prima versione delle Guidelines (TEI P1)
			\item 1990-94: fondi garantiti da enti quali NEH, Mellon Foundation, la Comunità Europea; supporto di ACH, ACL, ALLC
		\end{itemize}
	\end{block}

\end{frame}

\begin{frame}
	\frametitle{Intro Text Encoding Initiative}
	\framesubtitle{TEI}
	\addtocounter{nframe}{1}

	\begin{block}{un po' di storia}
		\begin{itemize}
			\item 2000: nascita del TEI Consortium, associazione non profit per lo sviluppo dello standard TEI
			\item 2002: passaggio da SGML a XML con la v. P4
			\item 2007: nuova versione TEI P5, continuamente aggiornata
		\end{itemize}
	\end{block}

\end{frame}


\begin{frame}
	\frametitle{Intro Text Encoding Initiative}
	\framesubtitle{TEI}
	\addtocounter{nframe}{1}

	\begin{block}{TEI Guidelines}
		\textit{versioni P1 – P3 basate su SGML}
		\\\textbf{versione P4}
		\begin{itemize}
			\item standard precedente, ancora impiegata
			\item basata su XML, DTD tradizionale
			\item pubblicata in forma definitiva nel 2002
			\item http://www.tei-c.org/Guidelines/P4/
		\end{itemize}
	\end{block}

\end{frame}

\begin{frame}
	\frametitle{Intro Text Encoding Initiative}
	\framesubtitle{TEI}
	\addtocounter{nframe}{1}

	\begin{block}{TEI Guidelines: versione P5}
		\begin{itemize}
			\item basata su XML, schema RelaxNG (e DTD tradizionale)
			\item pubblicata alla fine del 2007, aggiornata due volte l’anno
			\item molte novità interessanti (in particolare: maggior modularità)
			\item  http://www.tei-c.org/Guidelines/P5/
		\end{itemize}
	\end{block}

\end{frame}

\begin{frame}
	\frametitle{Intro Text Encoding Initiative}
	\framesubtitle{TEI}
	\addtocounter{nframe}{1}

	\begin{block}{TEI Guidelines: Obiettivi}
		\begin{itemize}
			\item better interchange and integration of scholarly data
			\item support for all texts, in all languages, from all periods
			\item guidance for the perplexed: what to encode - hence, a user-driven codification of existing best practice
			\item assistance for the specialist: how to encode --- hence, a loose framework into which unpredictable extensions can be fitted
		\end{itemize}
	\end{block}

\end{frame}



\begin{frame}
	\frametitle{Intro Text Encoding Initiative}
	\framesubtitle{TEI}
	\addtocounter{nframe}{1}

	\begin{block}{TEI Guidelines: Obiettivi}
		These apparently incompatible goals result in a highly flexible,
		modular, environment for DTD customization.
		Lou Burnard, TEI and XML: a marriage made in heaven?
		(http://www.tei-c.org/Talks/marriage.xml?style=printable)
	\end{block}

\end{frame}


\begin{frame}
	\frametitle{Intro Text Encoding Initiative}
	\framesubtitle{TEI}
	\addtocounter{nframe}{1}

	\begin{block}{Che cosa offre la TEI}
		% un ricco (e complesso) manuale di codifica, le Guidelines
		% for Electronic Text Encoding and Interchange, che
		% comprende un numero elevato di elementi, sia di tipo
		% strutturale sia semantico, definiti per mezzo di
		% un certo numero di schemi di codifica (DTD o schema
		% language) che sono
		% organizzati in una struttura altamente modulare e
		% personalizzabile.

	\end{block}

\end{frame}

\begin{frame}
	\frametitle{Intro Text Encoding Initiative}
	\framesubtitle{TEI}
	\addtocounter{nframe}{1}

	\begin{block}{Che cosa offre la TEI}
		\begin{itemize}
			\item \textbf{è possibile scegliere soltanto i moduli necessari}
			\item \textbf{è possibile modificare le definizioni degli elementi}
		\end{itemize}

	\end{block}

\end{frame}



\begin{frame}
	\frametitle{Intro Text Encoding Initiative}
	\framesubtitle{TEI}
	\addtocounter{nframe}{1}

	\begin{block}{Supporto per gli utenti}
		\begin{itemize}
			\item il sito del consorzio (http://www.tei-c.org/)
			\item pagine relative alle varie versioni delle Guidelines
			\item software, tutorial, etc.
			\item il wiki: http://www.tei-c.org/wiki/index.php/Main_Page
			\item la mailing list TEI-L
			\item Github: https://github.com/TEIC
			\item TEI by example: http://teibyexample.org/
		\end{itemize}

	\end{block}

\end{frame}


\begin{frame}
	\frametitle{Intro Text Encoding Initiative}
	\framesubtitle{TEI}
	\addtocounter{nframe}{1}

	\begin{block}{Novità della versione P5}
		\begin{itemize}
			\item modulo di descrizione dei manoscritti
			\item grafica e multimedia
			\item standoff markup
			\item supporto per i namespace XML
			\item miglioramenti nel modulo feature structure
			\item miglioramenti (limitati ...) nella gestione di varianti testuali
			\item nuovi meccanismi di linking
			\item personalizzazione semplificata
			\item varie migliorie tecniche
		\end{itemize}

	\end{block}

\end{frame}


\begin{frame}
	\frametitle{Intro Text Encoding Initiative}
	\framesubtitle{TEI}
	\addtocounter{nframe}{1}

	\begin{block}{TEI: Struttura modulare}
		\begin{itemize}
			\item si scelgono soltanto i moduli che corrispondono alle proprie esigenze, in modo da realizzare rapidamente uno schema di codifica appropriato
			\item ogni modulo contiene un certo numero di elementi (tagset)
			\item gli elementi sono organizzati in classi (strutturali,semantiche)
			\item gli attributi sono organizzati in classi (globali e specifici)
		\end{itemize}

	\end{block}

\end{frame}


% classi strutturali: elementi che hanno un ruolo a livello strutturale
% (paragrafi, strofe, etc.)
% classi semantiche: elementi descrittivi del testo (aggiunte,
% correzioni, nomi di persona, etc.)
% anche gli attributi sono organizzati in classi
% attributi globali: disponibili per tutti gli elementi
% attributi specifici: solo per alcuni elementi


\begin{frame}
	\frametitle{Intro Text Encoding Initiative}
	\framesubtitle{TEI}
	\addtocounter{nframe}{1}

	\begin{block}{TEI: Struttura modulare - Moduli essenziali}
		\begin{itemize}
			\item \textbf{tei}: definisce le classi di elementi, le macro e i datatype che verranno usati per tutti i moduli
			\item \textbf{header}: l’intestazione contenente i metadati relativi al documento TEI XML
			\item \textbf{textstructure}: elementi strutturali per qualsiasi tipo di testo
			\item \textbf{core}: elementi utili in qualsiasi tipo di documento
		\end{itemize}

	\end{block}

\end{frame}

\begin{frame}
	\frametitle{Intro Text Encoding Initiative}
	\framesubtitle{TEI}
	\addtocounter{nframe}{1}

	\begin{block}{TEI: Struttura modulare - Moduli facoltativi}
		\begin{itemize}
			\item \textbf{analysis}: strumenti per analisi (linguistica etc.) del testo
			\item \textbf{corpus}: gestione di corpora linguistici
			\item \textbf{drama}: elementi per testi teatrali e drammatici
			\item \textbf{gaiji}: rappresentazione di caratteri e glifi non standard
			\item \textbf{msdescription}: metadati relativi a manoscritti
			\item \textbf{spoken}: trascrizione del parlato
			\item \textbf{textcrit}: apparato critico
			\item \textbf{transcr}:  trascrizione di fonti primarie (manoscritti)
			\item \textbf{verse}: elementi supplementari per testi poetici
		\end{itemize}

	\end{block}

\end{frame}

\begin{frame}
	\frametitle{Intro Text Encoding Initiative}
	\framesubtitle{TEI}
	\addtocounter{nframe}{1}

	\begin{block}{TEI: Struttura modulare - Moduli}

		\textbf{SITO TEI PER ELENCP MODULI}

	\end{block}

\end{frame}


\begin{frame}
	\frametitle{Intro Text Encoding Initiative}
	\framesubtitle{TEI}
	\addtocounter{nframe}{1}

	\begin{block}{TEI Lite}

		una specifica \textbf{personalizzazione} della TEI versione P4/5

	\end{block}

	\textit{a simple demonstration of how the TEI encoding scheme
		might be adopted to meet 90\% of the needs of 90\% of the
		TEI user community (dalla Prefatory Note: http://www.tei-
		c.org/Lite/)}

\end{frame}


\begin{frame}
	\frametitle{Intro Text Encoding Initiative}
	\framesubtitle{TEI}
	\addtocounter{nframe}{1}

	\begin{block}{TEI Lite}

		la versione P4 è stata tradotta anche in italiano: TEI Lite:
		introduzione alla codifica dei testi, a cura di F. Ciotti (
		http://www.tei-c.org/Lite/teiu5_it.html)
		molto usata, ma presenta varie limitazioni
		con la versione P5 è più semplice produrre una versione
		semplificata o personalizzata

	\end{block}


\end{frame}

\begin{frame}
	\frametitle{Intro Text Encoding Initiative}
	\framesubtitle{TEI}
	\addtocounter{nframe}{1}

	\begin{block}{TEI Pizza chef}

		la TEI P4 poteva essere modificata usando un programma su
		web chiamato Pizza chef
		http://www.tei-c.org.uk/pizza.html
		metafora della base e dei condimenti (toppings),
		corrispondenti ai moduli indispensabili e a quelli facoltativi
		meccanismo efficace, soprattutto considerando l’alternativa
		(modifica manuale delle DTD TEI), ma obsoleto
		necessaria comunque modifica manuale per nuovi elementi

	\end{block}


\end{frame}

\begin{frame}
	\frametitle{Intro Text Encoding Initiative}
	\framesubtitle{TEI}
	\addtocounter{nframe}{1}

	\begin{block}{TEI Roma}

		per la P5 si è deciso di proporre una modularizzazione più
		efficace, e uno strumento di personalizzazione più potente.
		Strumento basato sul nuovo formato ODD.


	\end{block}

	\begin{block}{TEI Roma}
		il metodo da seguire per la versione P5 (quella che
		useremo) fino all’arrivo del successore (Byzantium)

	\end{block}

\end{frame}

\begin{frame}
	\frametitle{Intro Text Encoding Initiative}
	\framesubtitle{TEI}
	\addtocounter{nframe}{1}

	\begin{block}{TEI Roma}

		\begin{itemize}
			\item  possibilità di scegliere, escludere, modificare sia gli elementi (e le classi di elementi), sia gli attributi (e le classi di attributi)
			\item possibilità di aggiungere elementi (eventualmente inserendoli nelle classi preesistenti)
			\item possibilità di salvare lo schema in tre formati diversi: DTD  tradizionale, W3C e RelaxNG (anche in forma compatta)
		\end{itemize}

	\end{block}


\end{frame}


\begin{frame}
	\frametitle{Intro Text Encoding Initiative}
	\framesubtitle{TEI}
	\addtocounter{nframe}{1}

	\begin{block}{TEI Roma: Il formato ODD}
		le versioni P1 – P4 delle DTD TEI erano nel formato DTD language e dipendevano dalla sintassi SGML per molti aspetti
	\end{block}

	\begin{block}{TEI Roma: Il formato ODD}
		One Document Does it all (ODD): set di specifiche in base alle quali un semplice documento TEI XML “defines a schema in terms of the modules it requires, together with any possible modifications, such as the desired
		root element” risultato finale: lo schema nel linguaggio desiderato e la relativa documentazione
	\end{block}

\end{frame}


\begin{frame}
	\frametitle{Intro Text Encoding Initiative}
	\framesubtitle{TEI}
	\addtocounter{nframe}{1}

	\begin{block}{TEI Roma: Il formato ODD}

		tutorial: Getting Started with P5 ODDs\\
		http://www.tei-c.org/Guidelines/Customization/odds.xml

	\end{block}


\end{frame}


% 15La codifica di testi – Le norme TEI
% TEI Roma 2
% per motivi vari, la manutenzione di TEI Roma lascia a
% desiderare (per non parlare dello sviluppo di un successore)
% nel passato recente la componente di “controllo” dello
% schema (Sanity checker) non funzionava
% in tal caso è possibile andare sul sito TEI di Oxford dove è in
% esecuzione una versione più vecchia:
% http://tei.it.ox.ac.uk/Roma/
% Roma, in ogni caso, è solo un front-end per il linguaggio ODD
% (cfr. infra)




\begin{frame}
	\frametitle{Intro Text Encoding Initiative}
	\framesubtitle{TEI}
	\addtocounter{nframe}{1}

	\begin{block}{Il futuro della TEI}
		\textbf{negli ultimi anni gli schemi TEI sono stati oggetto di alcune critiche}
	\end{block}

	\begin{block}{Il futuro della TEI}
		\begin{itemize}
			\item la TEI è troppo grande / complicata / piccola
			\item la TEI è basata su XML e questo formato è in declino
			\item la TEI/XML non supporta le gerarchie multiple
			\item la TEI non supporta il markup di tipo stand-off
		\end{itemize}
	\end{block}

\end{frame}


\begin{frame}
	\frametitle{Intro Text Encoding Initiative}
	\framesubtitle{TEI}
	\addtocounter{nframe}{1}

	\begin{block}{Il futuro della TEI}
		\textbf{negli ultimi anni gli schemi TEI sono stati oggetto di alcune critiche}
	\end{block}

	\begin{block}{Il futuro della TEI}
		la cosa importante da ricordare è che il formato XML non è
		la TEI (in passato SGML, in futuro chissà → abstraction level)
		al contrario, se c’è una discrepanza fra Guidelines e schemi,
		la precedenza va alle Guidelines
	\end{block}

\end{frame}


\begin{frame}
	\frametitle{Intro Text Encoding Initiative}
	\framesubtitle{TEI}
	\addtocounter{nframe}{1}

	\begin{block}{text encoding con la TEI}
		è caldamente raccomandato usare direttamente la
		versione più recente della P5.\\
		La flessibilità della P5 permette di definire uno schema di
		codifica che corrisponda precisamente al modello
    \end{block}
    
    \textit{La comunità di utenti e sviluppatori TEI offre un buon supporto.}

\end{frame}




%%%%%%%%%%%%%%%%%%%%%%
% Codifica di testi
% I moduli base della TEI
% Roberto Rosselli Del Turco
% Dipartimento di Studi Umanistici
% Università di Torino
% roberto.rossellidelturco@fileli.unipi.it
% roberto.rossellidelturco@unito.itSchemi di codifica TEI – Moduli base
% Elementi disponibili per tutti i documenti TEI

\begin{frame}
	\frametitle{Intro Text Encoding Initiative}
	\framesubtitle{TEI}
	\addtocounter{nframe}{1}

	\begin{block}{TEI}
        un documento TEI P5 ‘minimo’ è composto da :
        intestazione XML
        intestazione TEI
        elementi strutturali
        elementi semantici dei moduli base (non indispensabili)
        anche usando soltanto i moduli essenziali si ha a disposizione
        uno schema adatto alla marcatura di numerosi tipi di testi
        moduli di base: tei, header, textstructure, core
        schemi ‘leggeri’ consigliati: la TEI Lite, o se necessario una
        versione personalizzata e ancora più ridotta della P5 (TEI
        Absolutely Bare)
    \end{block}
    
   

\end{frame}




begin{frame}
	\frametitle{Intro Text Encoding Initiative}
	\framesubtitle{TEI}
	\addtocounter{nframe}{1}

	\begin{block}{TEI}
        2Schemi di codifica TEI – Moduli base
        Nota sugli elementi spiegati
        punto di riferimento per il corso: la TEI P5 “completa”
        iniziamo con la TEI Lite (= versione “leggera”)
        proposti elementi di uso generale e di uso specifico
        per cominciare, la TEI Lite è più semplice da gestire
        elementi non presenti di default evidenziati in rosso scuro
        in caso di dubbio consultare le liste degli elementi:
        TEI Lite: http://www.tei-c.org/release/doc/tei-p5-exemplars/html/teilite.doc.html
        TEI P5: http://www.tei-c.org/release/doc/tei-p5-doc/en/html/REF-ELEMENTS.html
        NB: quando si crea lo schema controllare che gli elementi
        siano effettivamente inclusi nei moduli prescelti
    \end{block}
    
   

\end{frame}



begin{frame}
	\frametitle{Intro Text Encoding Initiative}
	\framesubtitle{TEI}
	\addtocounter{nframe}{1}

	\begin{block}{TEI}
        3Schemi di codifica TEI – Moduli base
        Caratteristiche degli elementi spiegati
        gli elementi TEI rientrano nelle categorie generali di
        elementi XML che abbiamo visto in precedenza:
        elementi che possono contenere solo altri elementi (=
        elementi strutturali)
        elementi che possono contenere altri elementi e testo
        elementi che possono contenere solo testo
        elementi vuoti (es. <pb/>)
        gli elementi vuoti marcano una gerarchia differente
        NB: inserire il contenuto sbagliato (ad es. testo in un
        elemento strutturale) significa rendere non valido il
        documento TEI
    \end{block}
    
   

\end{frame}






begin{frame}
	\frametitle{Intro Text Encoding Initiative}
	\framesubtitle{TEI}
	\addtocounter{nframe}{1}

	\begin{block}{TEI}
        4Schemi di codifica TEI – Moduli base
        Gerarchie multiple 1
        <?xml version="1.0" encoding="utf-8"?>
        <text>
        <titolo>Gli assassinii della Rue Morgue</titolo>
        <intestazione> I </intestazione>
        <pagina n="5">
        <p>Le facoltà mentali che si sogliono chiamare analitiche sono, di
        per se stesse, poco suscettibili di analisi. Le conosciamo soltanto negli
        effetti. [...]</p>
        <p>La facoltà di risolvere è probabilmente molto rinfor-
        </pagina>
        <pagina n="6">
        zata dallo studio delle matematiche e in modo particolare
        dell’altissimo ramo di questa scienza che – impropriamente e solo in
        ragione delle sue operazioni in senso retrogrado – è stata chiamata
        analisi [...] </p>
        </pagina>
        </text>
    \end{block}
    
   

\end{frame}






begin{frame}
	\frametitle{Intro Text Encoding Initiative}
	\framesubtitle{TEI}
	\addtocounter{nframe}{1}

	\begin{block}{TEI}
        5Schemi di codifica TEI – Moduli base
        Gerarchie multiple 2: elementi vuoti
        <?xml version="1.0" encoding="utf-8"?>
        <text>
        <titolo>Gli assassinii della Rue Morgue</titolo>
        <intestazione> I </intestazione>
        <pagina n="5"/>
        <p>Le facoltà mentali che si sogliono chiamare analitiche sono, di
        per se stesse, poco suscettibili di analisi. Le conosciamo soltanto negli
        effetti. [...]</p>
        <p>La facoltà di risolvere è probabilmente molto rinfor-
        <pagina n="6"/>
        zata dallo studio delle matematiche e in modo particolare
        dell’altissimo ramo di questa scienza che – impropriamente e solo in
        ragione delle sue operazioni in senso retrogrado – è stata chiamata
        analisi [...] </p>
        </text>
    \end{block}
    
   

\end{frame}






begin{frame}
	\frametitle{Intro Text Encoding Initiative}
	\framesubtitle{TEI}
	\addtocounter{nframe}{1}

	\begin{block}{TEI}
        6Schemi di codifica TEI – Moduli base
        Struttura di un documento TEI
        struttura fondamentale all’interno della radice (<TEI>):
        una intestazione TEI (<teiHeader>)
        un testo: <text> (o più testi, cfr. infra)
        contenuto del TEI header:
        metadati relativi al documento (utili per collezioni di testi
        codificati)
        descrizione del file usando <fileDesc> (obbligatoria)
        descrizioni relative al tipo di codifica, al contenuto del
        documento, alle sue revisioni (facoltative)
        è possibile includere testi introduttivi e spiegazioni relative alla
        codifica effettuata (preziosi per l’interscambio!)
    \end{block}
    
   

\end{frame}






begin{frame}
	\frametitle{Intro Text Encoding Initiative}
	\framesubtitle{TEI}
	\addtocounter{nframe}{1}

	\begin{block}{TEI}
        7Schemi di codifica TEI – Moduli base
        Esempio 1
        <?xml version="1.0" encoding="utf-8"?>
        <!DOCTYPE TEI SYSTEM "tei_lite.dtd">
        <TEI xmlns="http://www.tei-c.org/ns/1.0">
        <teiHeader>...</teiHeader>
        <text>
        <div>
        <p></p>
        </div>
        </text>
        </TEI>
    \end{block}
    
   

\end{frame}






begin{frame}
	\frametitle{Intro Text Encoding Initiative}
	\framesubtitle{TEI}
	\addtocounter{nframe}{1}

	\begin{block}{TEI}
        8Schemi di codifica TEI – Moduli base
        Esempio 2
        <?xml version="1.0" encoding="utf-8"?>
        <?xml-model href="tei-lite.rng"?>
        <TEI xmlns="http://www.tei-c.org/ns/1.0">
        <teiHeader>...</teiHeader>
        <text>
        <div>
        <p></p>
        </div>
        </text>
        </TEI>
    \end{block}
    
   

\end{frame}






begin{frame}
	\frametitle{Intro Text Encoding Initiative}
	\framesubtitle{TEI}
	\addtocounter{nframe}{1}

	\begin{block}{TEI}
        9Schemi di codifica TEI – Moduli base
        Struttura di un documento TEI
        schema di intestazione TEI minima:
        <teiHeader>
        <fileDesc>
        <titleStmt>...</titleStmt>
        <publicationStmt>...</publicationStmt>
        <sourceDesc>...</sourceDesc>
        </fileDesc>
        </teiHeader>
        metadati essenziali riguardo il titolo, la modalità di diffusione e
        la fonte originaria di un testo codificato
        permettono classificazione, archiviazione ed elaborazione
        bibliografica
    \end{block}
    
   

\end{frame}





begin{frame}
	\frametitle{Intro Text Encoding Initiative}
	\framesubtitle{TEI}
	\addtocounter{nframe}{1}

	\begin{block}{TEI}
        10Schemi di codifica TEI – Moduli base
        Esempio di intestazione TEI
        <teiHeader>
        <fileDesc>
        <titleStmt>
        <title>La Divina Commedia: versione elettronica</title>
        <respStmt>
        <resp>Conversione TEI P5 a cura di</resp><name>M. Rossi</name>
        </respStmt>
        </titleStmt>
        <publicationStmt>
        <publisher>Università di Pisa</publisher>
        <date>2002-11-07</date>
        <availability status="restricted"><p>Contattare il responsabile del
        progetto, vietata la riproduzione.</p></availability>
        </publicationStmt>
        <sourceDesc>
        <bibl><title>La Divina Commedia</title><author>Dante Alighieri
        </author><publisher>Mondadori</publisher><date>1988</date></bibl>
        </sourceDesc>
        </fileDesc>
        </teiHeader>
    \end{block}
    
   

\end{frame}






begin{frame}
	\frametitle{Intro Text Encoding Initiative}
	\framesubtitle{TEI}
	\addtocounter{nframe}{1}

	\begin{block}{TEI}
        11Schemi di codifica TEI – Moduli base
        Struttura di un documento TEI
        le altre componenti dell’intestazione TEI:
        <encodingDesc> informazioni riguardo lo schema (e il
        modello di codifica) utilizzato
        <profileDesc>
        descrizione del testo: quando è stato
        creato, da chi, usando quali lingue etc.
        <revisionDesc> informazioni sulle versioni del file
        stesso livello di <fileDesc>, vedremo con più dettagli
        i metadati sono una componente essenziale di qualsiasi
        progetto di digitalizzazione
    \end{block}
    
   

\end{frame}






begin{frame}
	\frametitle{Intro Text Encoding Initiative}
	\framesubtitle{TEI}
	\addtocounter{nframe}{1}

	\begin{block}{TEI}
        12Schemi di codifica TEI – Moduli base
        Elementi strutturali
        <text>
        un singolo testo di qualsiasi tipo
        punto di partenza della gerarchia riguardo il contenuto
        può essere affiancato o sostituito da <facsimile>
        all’interno di <text> possiamo trovare quattro elementi:
        <front> materiale che precede il testo (se presente)
        <body> il testo stesso
        <back> materiale che segue il testo (se presente)
        <group> alternativo a <body>, raggruppa testi diversi
    \end{block}
    
   

\end{frame}







begin{frame}
	\frametitle{Intro Text Encoding Initiative}
	\framesubtitle{TEI}
	\addtocounter{nframe}{1}

	\begin{block}{TEI}
        13Schemi di codifica TEI – Moduli base
        Struttura di un documento TEI
        esempio schematico di documento TEI unitario:
        <TEI>
        <teiHeader> [informazioni del TEI Header]
        </teiHeader>
        <text>
        <front> [premessa, dedica ...] </front>
        <body> [corpo del testo ...] </body>
        <back> [postfazione, appendice ...]</back>
        </text>
        </TEI>
    \end{block}
    
   

\end{frame}






begin{frame}
	\frametitle{Intro Text Encoding Initiative}
	\framesubtitle{TEI}
	\addtocounter{nframe}{1}

	\begin{block}{TEI}
        14Schemi di codifica TEI – Moduli base
        Struttura di un documento TEI
        è possibile costruire documenti compositi:
        rimpiazzando il <body> con un gruppo (<group>) di testi si
        ottiene un documento composito
        ciascuno di questi testi è rappresentato secondo la struttura
        vista prima
        un’altra possibilità è creare un corpus con <teiCorpus>:
        intestazioni (<teiHeader>) separate per il corpus e per
        ciascun gruppo di testi
        struttura più complessa, su più livelli
    \end{block}
    
   

\end{frame}







begin{frame}
	\frametitle{Intro Text Encoding Initiative}
	\framesubtitle{TEI}
	\addtocounter{nframe}{1}

	\begin{block}{TEI}
        15Schemi di codifica TEI – Moduli base
        Schema di testo composito
        <TEI>
        <teiHeader> [ intestazione del testo composito ] </teiHeader>
        <text>
        <front> [ front matter del composito ] </front>
        <group>
        <text>
        <front> [ front matter del primo testo ] </front>
        <body> [ body del primo testo ]
        </body>
        <back> [ back matter del primo testo ] </back>
        </text>
        <text>
        <front> [ front matter del secondo testo] </front>
        <body> [ body del secondo testo ]
        </body>
        <back> [ back matter del secondo testo ] </back>
        </text>
        ...
        [ altri testi o gruppi di testi ]
        ...
        </group>
        <back>
        [ back matter del composito ]
        </back>
        </text>
        </TEI>
    \end{block}
    
   

\end{frame}







begin{frame}
	\frametitle{Intro Text Encoding Initiative}
	\framesubtitle{TEI}
	\addtocounter{nframe}{1}

	\begin{block}{TEI}
	% 16Schemi di codifica TEI – Moduli base
% Costruzione di corpora TEI
% esempio schematico di TEI Corpus:
% <teiCorpus>
% <teiHeader> [metadati per il corpus] </teiHeader>
% <TEI>
% <teiHeader> [metadati relativi al I testo]</teiHeader>
% <text> [primo testo del corpus] </text>
% </TEI>
% <TEI>
% <teiHeader>[metadati relativi al II testo]</teiHeader>
% <text> [secondo testo del corpus] </text>
% </TEI>
% </teiCorpus>
    \end{block}
    
   

\end{frame}





begin{frame}
	\frametitle{Intro Text Encoding Initiative}
	\framesubtitle{TEI}
	\addtocounter{nframe}{1}

	\begin{block}{TEI}
	 17Schemi di codifica TEI – Moduli base
 CONTROLLARE IMMAGINE
    \end{block}
    
   

\end{frame}





begin{frame}
	\frametitle{Intro Text Encoding Initiative}
	\framesubtitle{TEI}
	\addtocounter{nframe}{1}

	\begin{block}{TEI}
        18Schemi di codifica TEI – Moduli base
        Altri elementi strutturali fondamentali
        suddivisioni del testo:
        non numerati: <div> (nessun limite di nidificazione)
        numerati: <div1> ... <div7> (massimo 7 livelli)
        paragrafi: <p>
        testo riferito: <q> (discorso diretto, citazioni, etc.)
        versi: strofe <lg> e singoli versi <l>
        testi teatrali: discorsi <sp> che possono contenere paragrafi
        <p> o versi <l>, oltre a direzioni di scena <stage>
        milestone tags: <pb/>, <lb/>, <cb/>, <milestone/>
        notare che un <div> può contenere un <floatingText>:
        possibilità di introdurre gerarchie complesse
    \end{block}
    
   

\end{frame}



begin{frame}
	\frametitle{Intro Text Encoding Initiative}
	\framesubtitle{TEI}
	\addtocounter{nframe}{1}

	\begin{block}{TEI}
        19Schemi di codifica TEI – Moduli base
        Apertura e chiusura di un <div>
        <head>
        qualunque tipo di intestazione: il titolo di un opera,
        l’intestazione di un paragrafo, di una sezione, etc.
        type
        permette di classificare in base a una tipologia
        <epigraph> citazione all’inizio del testo, o nella pagina del titolo,
        eventualmente con riferimento bibliografico
        <opener> raggruppa un serie di elementi (data, luogo, saluti,
        etc.) all’inizio del <div>, specie di una lettera
        <argument> lista degli argomenti trattati nel <div>
        <trailer>
        frase che compare alla fine del <div> (ad esempio
        “Fine del capitolo 1”)
        <closer>
        raggruppa un serie di elementi (data, luogo, saluti,
        etc.) alla fine del <div>, specie di una lettera
    \end{block}
    
   

\end{frame}




begin{frame}
	\frametitle{Intro Text Encoding Initiative}
	\framesubtitle{TEI}
	\addtocounter{nframe}{1}

	\begin{block}{TEI}
        20Schemi di codifica TEI – Moduli base
        Esempio di <div> con apertura/chiusura
        <div type="lettera”>
        <opener>
        <dateline>
        <name type="place">Pisa</name>
        <date>20 marzo 2015</date>
        </dateline>
        <salute>Gentilissima Prof.ssa Scannagatti,</salute>
        </opener>
        <p>sono spiacente di doverle comunicare che un’invasione di cavallette si
        è abbattuta sui miei quaderni incautamente lasciati in giardino, e li ha
        divorati interamente.</p>
        <p>Questo purtroppo significa che non posso mostrare i compiti svolti,
        come sempre, con solerzia e assiduo impegno.</p>
        <closer>
        <salute>Certo di poter contare sulla sua comprensione le porgo i miei
        migliori saluti,</salute>
        <signed>Pierino Rossi</signed>
        </closer>
        </div>
    \end{block}
    
   

\end{frame}





begin{frame}
	\frametitle{Intro Text Encoding Initiative}
	\framesubtitle{TEI}
	\addtocounter{nframe}{1}

	\begin{block}{TEI}
        21Schemi di codifica TEI – Moduli base
        Errori frequenti 1
        molto spesso nell’intestazione si fraintende il significato
        dell’elemento <fileDesc>
        questo elemento serve in primo luogo a dare informazioni
        sul file stesso, non sul testo originale
        il riferimento alla fonte dalla quale è tratto il testo codificato
        deve essere inserito nel <sourceDesc>
        i titoli sono codificati con <title> soltanto nel caso di
        riferimenti bibliografici
        i titoli del testo, dei capitoli etc. si marcano con <head>
    \end{block}
    
   

\end{frame}





begin{frame}
	\frametitle{Intro Text Encoding Initiative}
	\framesubtitle{TEI}
	\addtocounter{nframe}{1}

	\begin{block}{TEI}
        22Schemi di codifica TEI – Moduli base
        Nota sugli errori possibili
        gli errori che si possono commettere durante la codifica
        di un testo ricadono più o meno in tre categorie:
        errori sintattici: un elemento inserito in un punto sbagliato
        della gerarchia, o che non può contenere testo etc.
        errori di marcatura semantica: usare un elemento inadatto
        allo scopo, ad esempio marcare un titolo con <emph>
        errori di interpretazione del testo (che portano al II tipo o
        all’assenza del markup che andrebbe inserito)
        gli errori del primo tipo sono i più facili da individuare e
        correggere, quelli del terzo i più difficili
    \end{block}
    
   

\end{frame}





begin{frame}
	\frametitle{Intro Text Encoding Initiative}
	\framesubtitle{TEI}
	\addtocounter{nframe}{1}

	\begin{block}{TEI}
        23Schemi di codifica TEI – Moduli base
        A proposito di <div>
        domanda che ricorre periodicamente: perché non usare nomi
        più significativi invece di un contenitore generico come <div>
        (= ‘suddivisione, parte di un testo’)?
        perché non usare <book>, <chapter>, <section>, etc. come
        fa, ad esempio, lo schema DOCBOOK?
        troppa variabilità nell’uso comune, meglio usare un termine
        generico che possa poi essere specificato usando l’attributo
        type (es. <div type=”chapter”>)
        i <div>, numerati o meno, possono essere ‘nidificati’, nessun
        problema nel riproporre la struttura editoriale visibile
    \end{block}
    
   

\end{frame}





begin{frame}
	\frametitle{Intro Text Encoding Initiative}
	\framesubtitle{TEI}
	\addtocounter{nframe}{1}

	\begin{block}{TEI}
        24Schemi di codifica TEI – Moduli base
        Attributi globali
        alcuni attributi possono essere usati con qualsiasi
        elemento (v. la classe att.global), in particolare:
        n
        un numero o un nome non univoco, possibilmente breve,
        per identificare un elemento
        rend informazioni relative all’aspetto (originale!) del testo
        rendition simile a @rend, ma fa riferimento a elementi
        <rendition> inseriti nell’<encodingDesc> (dentro <tagsDecl>)
        xml:lang la lingua del testo contenuto da un elemento
        xml:id un identificatore univoco per l’elemento
        NB: in base ai moduli usati nello schema sono disponibili ulteriori attributi globali
    \end{block}
    
   

\end{frame}





begin{frame}
	\frametitle{Intro Text Encoding Initiative}
	\framesubtitle{TEI}
	\addtocounter{nframe}{1}

	\begin{block}{TEI}
        25Schemi di codifica TEI – Moduli base
        Esempio 1
        <text>
        <body>
        <div n="ch1" type="chapter">
        <pb n="1"/>[...]
        <p>[...] risulta chiaro se avete letto <title
        rend="underline" xml:lang="fra">Les fleurs du
        mal</title> [...]</p>
        <p>[...] un grande esempio di <foreign
        xml:lang="fra">savoir faire</foreign> [...]</p>
        [...]
        </div>
        [ altri div ... ]
        </body>
        </text>
    \end{block}
    
   

\end{frame}





begin{frame}
	\frametitle{Intro Text Encoding Initiative}
	\framesubtitle{TEI}
	\addtocounter{nframe}{1}

	\begin{block}{TEI}
        26Schemi di codifica TEI – Moduli base
        Esempio 2
        <text>
        <body>
        <div n="ch1" type="chapter"> <pb n="1"/> [...]
        <p n="1">[...] descritto altrove (si veda ad
        esempio <ref target="#Rossi94">Rossi 1994</ref>)
        [...] </p> [...]
        </div>
        [ altri div ... ]
        <div n="bib" type="bibliography">
        [...]
        <bibl xml:id="Rossi94">
        <author>Rossi, M.</author>[...]</bibl>
        [...]
        </div>
        </body>
        </text>
    \end{block}
    
   

\end{frame}





begin{frame}
	\frametitle{Intro Text Encoding Initiative}
	\framesubtitle{TEI}
	\addtocounter{nframe}{1}

	\begin{block}{TEI}
        27Schemi di codifica TEI – Moduli base
        Esercizio 1
        marcare il testo (file ese01.txt) come segue:
        editor da usare: XML Copy Editor
        inserire intestazione XML
        marcare la struttura usando gli elementi fin qui descritti
        in particolare marcare tutti i paragrafi usando <p>
        verificare che sia ben formato premendo F2
        salvare il file XML come ese01-Cognome.xml
        file nominati diversamente non saranno accettati
        testi tratti da http://www.liberliber.it/
    \end{block}
    
   

\end{frame}





begin{frame}
	\frametitle{Intro Text Encoding Initiative}
	\framesubtitle{TEI}
	\addtocounter{nframe}{1}

	\begin{block}{TEI}
        28Schemi di codifica TEI – Moduli base
        Suggerimenti riguardo gli esercizi
        non fate copia e incolla da Wordpad o Notepad, aprite
        direttamente i testi usando XML Copy editor
        i documenti XML hanno come estensione .xml, quindi
        salvate subito una copia con tale estensione
        aprite le Guidelines TEI nel navigatore
        per prima cosa marcare la struttura: cominciare la
        marcatura semantica come secondo passo
        ogni esercizio ha degli obiettivi da raggiungere:
        controllate prima di spedire
        non lasciate accumulare gli errori
    \end{block}
    
   

\end{frame}



begin{frame}
	\frametitle{Intro Text Encoding Initiative}
	\framesubtitle{TEI}
	\addtocounter{nframe}{1}

	\begin{block}{TEI}
        29Schemi di codifica TEI – Moduli base
        Errori frequenti 2
        <div> non può essere usato allo stesso livello gerarchico
        di <p>, in altre parole non si può alternare <div> con
        <p>:
        <div> [...] </div>
        <p> [...] </p>
        <div> [...] </div>
        ← INVALID!!!
        <div> e tutti gli altri elementi strutturali “puri” non
        possono contenere testo:
        <div>Pippo</div>
        ← INVALID!!!
        <person>Pippo</person> ← INVALID!!!
    \end{block}
    
   

\end{frame}






begin{frame}
	\frametitle{Intro Text Encoding Initiative}
	\framesubtitle{TEI}
	\addtocounter{nframe}{1}

	\begin{block}{TEI}
        30Schemi di codifica TEI – Moduli base
        Enfasi e termini particolari 1
        <emph>
        parole o frasi enfatizzate nel testo
        questo è il <emph>mio</emph> computer!
        <foreign>
        parola o frase in una lingua diversa
        In quel punto entrò il bidello a dare il <foreign
        xml:lang="lat">finis</foreign>.
        <distinct>
        “diverso” dal testo (arcaico, gergale, etc.)
        saltò in groppa al <distinct>fido destriero</distinct>
        <hi>
        elemento generico
        <hi rend="double">N</hi>el mezzo del cammin di nostra vita
        il suo nome è <hi rend="italic">Mario Rossi</hi>
    \end{block}
    
   

\end{frame}




begin{frame}
	\frametitle{Intro Text Encoding Initiative}
	\framesubtitle{TEI}
	\addtocounter{nframe}{1}

	\begin{block}{TEI}
        31Schemi di codifica TEI – Moduli base
        Enfasi e termini particolari 2
        <mentioned> parola o frase menzionata ma non usata
        il termine corretto è <mentioned>epigrafe</mentioned>
        <soCalled>
        parola o espressione da cui ci si distanzia
        il cosiddetto <soCalled>darwinismo sociale</soCalled>
        <term>
        una o più parole considerate termine tecnico
        possiamo definire il <term xml:id="NPL"
        rend="italic">neopositivismo logico</term>
        <gloss>
        una spiegazione o glossa riguardo il testo
        come <gloss target="#NPL">una corrente filosofica basata
        sul principio che la filosofia debba aspirare al rigore
        proprio della scienza</gloss>
    \end{block}
    
   

\end{frame}







begin{frame}
	\frametitle{Intro Text Encoding Initiative}
	\framesubtitle{TEI}
	\addtocounter{nframe}{1}

	\begin{block}{TEI}
        32Schemi di codifica TEI – Moduli base
        Citazioni 1
        <q>
        testo citato da altre fonti: discorso diretto, esempi
        (nei dizionari), etc.
        La mia maestra della prima superiore mi salutò di
        sulla porta della classe e mi disse: <q rend="PRE
        mdash">Enrico, tu vai al piano di sopra, quest'anno;
        non ti vedrò nemmen più passare!</q>
        <quote>
        frase o brano attribuito a fonte esterna
        <p>E allora disse: <q rend="PRE lsquo POST
        rsquo">Ecco come comincia la Divina Commedia:
        <quote>Nel mezzo del cammin di nostra vita / Mi
        ritrovai per una selva oscura</quote>.</q></p>
    \end{block}
    
   

\end{frame}





begin{frame}
	\frametitle{Intro Text Encoding Initiative}
	\framesubtitle{TEI}
	\addtocounter{nframe}{1}

	\begin{block}{TEI}
        33Schemi di codifica TEI – Moduli base
        Citazioni 2
        <said> testo pronunciato ad alta voce o pensato
        <cit> citazione con riferimento bibliografico
        Lexicography has shown little sign of being affected by the work
        of followers of J.R. Firth, probably best summarized in his
        slogan, <cit>
        <quote>You shall know a word by the company it keeps.</quote>
        <ref>(Firth, 1957)</ref>
        </cit>
        semplice riferimento bibliografico nell’esempio, possibile
        aggiungere un collegamento (a capitolo/sezione o a una
        specifica entrata bibliografica) usando l’attributo @target
    \end{block}
    
   

\end{frame}






begin{frame}
	\frametitle{Intro Text Encoding Initiative}
	\framesubtitle{TEI}
	\addtocounter{nframe}{1}

	\begin{block}{TEI}
        34Schemi di codifica TEI – Moduli base
        A proposito di <q> e <quote>
        possono contenere non solo altri elementi simili (<q> e
        <quote>) ma anche elementi come <p>, <l>, etc.:
        <p>
        <q>The Lord! The Lord! It is Sakya Muni himself,</q> the lama half
        sobbed; and under his breath began the wonderful Buddhist invocation:­<q>
        <quote>
        <l>To Him the Way — the Law — Apart —</l>
        <l>Whom Maya held beneath her heart</l>
        <l>Ananda's Lord — the Bodhisat</l>
        </quote>
        And He is here! The Most Excellent Law is here also. My
        pilgrimage is well begun. And what work! What work!</q>
        </p>
        possibili problemi relativi alla gerarchia (tag overlap, gerarchie
        sovrapposte):
        <p>Allora disse: <q>Sarò breve.</p><p>Ho finito.</q></p>
    \end{block}
    
   

\end{frame}






begin{frame}
	\frametitle{Intro Text Encoding Initiative}
	\framesubtitle{TEI}
	\addtocounter{nframe}{1}

	\begin{block}{TEI}
        35Schemi di codifica TEI – Moduli base
        Nomi, numeri e date 1
        <rs>
        lett. referring string, nome o etichetta generica
        <q>Mio caro <rs type="person">Filippo</rs></q>,
        gli disse <rs type="person">sua moglie</rs>...
        <name>
        nome proprio (di persona, luogo, etc.)
        <q>Mio caro <name type="person">Filippo</name></q>,
        <num>
        un numero in qualsiasi formato
        <num value="23">XXIII</num>
        <date>
        una data in qualsiasi formato
        nato il <date when="1868­02­10">10 febb. 1868</date>
        <time>
        l’ora in qualsiasi formato
        alle <time when="8.00">otto del mattino</time>
    \end{block}
    
   

\end{frame}






begin{frame}
	\frametitle{Intro Text Encoding Initiative}
	\framesubtitle{TEI}
	\addtocounter{nframe}{1}

	\begin{block}{TEI}
        36Schemi di codifica TEI – Moduli base
        Nomi, numeri e date 2
        se gli elementi dei moduli di base risultassero insufficienti per
        la codifica è possibile usare un modulo specifico:
        13 Names, Dates, People, and Places (
        http://www.tei-c.org/release/doc/tei-p5-doc/en/html/ND.html )
        questo modulo permette una granularità molto maggiore
        grazie agli elementi specifici che mette a disposizione
        (<persName>, <forename>, <surname>, <roleName>,
        <addName>, etc.)
        possibile creare sistema prosopografico
        ricchezza degli attributi (anche nella versione base)
    \end{block}
    
   

\end{frame}






begin{frame}
	\frametitle{Intro Text Encoding Initiative}
	\framesubtitle{TEI}
	\addtocounter{nframe}{1}

	\begin{block}{TEI}
        37Schemi di codifica TEI – Moduli base
        La pagina del titolo
        elemento <titlePage> può contenere:
        <docTitle> e (o direttamente) <titlePart>: titolo anche in più
        parti
        <docEdition>: informazioni riguardo l’edizione
        <byline> e (o direttamente) <docAuthor>: autore
        <docImprint>: informazioni di stampa, a sua volta include
        <publisher>: editore
        <pubPlace>: luogo di stampa
        <date>: data
        <docDate>: data
    \end{block}
    
   

\end{frame}



begin{frame}
	\frametitle{Intro Text Encoding Initiative}
	\framesubtitle{TEI}
	\addtocounter{nframe}{1}

	\begin{block}{TEI}
        38Schemi di codifica TEI – Moduli base
        Esempio di pagina del titolo
        <titlePage>
        <titlePart>Cuore</titlePart>
        <byline>di <docAuthor>E. de
        Amicis</docAuthor></byline>
        <docEdition>Edizione integrale</docEdition>
        <docImprint>
        <publisher>Newton Compton editori</publisher>
        <pubPlace>Roma</pubPlace>
        <date>1994</date>
        </docImprint>
        </titlePage>
    \end{block}
    
   

\end{frame}








begin{frame}
	\frametitle{Intro Text Encoding Initiative}
	\framesubtitle{TEI}
	\addtocounter{nframe}{1}

	\begin{block}{TEI}
        39Schemi di codifica TEI – Moduli base
        Esempio di pagina del titolo
        <titlePage>
        <docAuthor>E. DE AMICIS</docAuthor>
        <docTitle>
        <titlePart type= " main " >CUORE</titlePart>
        <titlePart>Libro per i ragazzi</titlePart>
        </docTitle>
        <docEdition>98.a edizione</docEdition>
        <graphic url= "publisher.png" >
        <docImprint>
        <pubPlace>MILANO</pubPlace>
        <publisher>FRATELLI TREVES, EDITORI</publisher>
        <date>1889</date>
        </docImprint>
        </titlePage>
    \end{block}
    
   

\end{frame}






begin{frame}
	\frametitle{Intro Text Encoding Initiative}
	\framesubtitle{TEI}
	\addtocounter{nframe}{1}

	\begin{block}{TEI}
        40Schemi di codifica TEI – Moduli base
        Associare uno schema di codifica
        per qualsiasi progetto anche mediamente impegnativo è
        preferibile creare un proprio schema TEI
        cominciamo a validare sulla base di uno schema a partire
        dall’esercizio ese02.txt
        associazione di uno schema al documento TEI XML:
        XML Copy Editor: XML → Associa → DTD di sistema
        Oxygen: Document → Schema → Associate Schema...
        altri editor: inserire manualmente il <!DOCTYPE> nel
        caso si usi una DTD, o la processing instruction <?xml-
        model> descritta nel capitolo A Gentle Introduction to XML
    \end{block}
    
   

\end{frame}






begin{frame}
	\frametitle{Intro Text Encoding Initiative}
	\framesubtitle{TEI}
	\addtocounter{nframe}{1}

	\begin{block}{TEI}
        41Schemi di codifica TEI – Moduli base
        Nota su XML Copy Editor
        XCE non supporta ancora l’uso di schemi di codifica nel
        formato RelaxNG, necessario ricorrere alla vecchia DTD
        XCE controlla anche la coerenza interna dello schema
        un messaggio come quello che segue può essere
        ignorato, l’informazione essenziale è se il documento è
        valido oppure no (riportato in cima):
        ese02-*******.xml is valid
        Attenzione at line 2, column 37: element
        '_DUMMY_model.resourceLike' is referenced in a content
        model but was never declared
        Attenzione at line 2, column 37: element
        '_DUMMY_model.gLike' is referenced in a content model but
        was never declared
    \end{block}
    
   

\end{frame}






begin{frame}
	\frametitle{Intro Text Encoding Initiative}
	\framesubtitle{TEI}
	\addtocounter{nframe}{1}

	\begin{block}{TEI}
        42Schemi di codifica TEI – Moduli base
        Esercizio 2
        marcare il testo (file ese02.txt) come segue:
        editor da usare: XML Copy Editor
        aprire il sito delle norme TEI P5
        marcare la struttura per prima cosa
        marcare tutte le espressioni enfatizzate (underscore =
        corsivo), i discorsi diretti, le citazioni, i nomi, etc.
        verificare che sia ben formato premendo F2 e validare
        salvare il file XML come ese02-Cognome.xml
        testi tratti da http://www.gutenberg.org/
    \end{block}
    
   

\end{frame}






begin{frame}
	\frametitle{Intro Text Encoding Initiative}
	\framesubtitle{TEI}
	\addtocounter{nframe}{1}

	\begin{block}{TEI}
        43Schemi di codifica TEI – Moduli base
        Collegamenti interni ed esterni
        <ptr/>
        specifica un puntatore a un altro “luogo” (un altro
        punto dello stesso testo o di un altro testo)
        <p>Il sistema di puntatori è basato sul meccanismo W3C
        Xpointer. Per maggiori informazioni si veda il
        capitolo <ptr target="#SA­id"/>; per le specifiche si
        veda <ptr target="http://www.w3.org/TR/xptr­xpointer/
        "/>; v. anche <ptr type="image" target="#fig22"/>.</p>
        <ref>
        specifica un puntatore a un altro “luogo”, può
        includere del testo
        Si veda il <ref>terzo capitolo, p. 24</ref>.
        Si veda il <ref target="#cap3.24">terzo capitolo,
        par. 24</ref>
    \end{block}
    
   

\end{frame}






begin{frame}
	\frametitle{Intro Text Encoding Initiative}
	\framesubtitle{TEI}
	\addtocounter{nframe}{1}

	\begin{block}{TEI}
        44Schemi di codifica TEI – Moduli base
        Liste e tabelle
        <list> qualsiasi tipo di lista (type per specificare)
        <head> intestazione (titolo) della lista
        <item> un elemento della lista
        <label> numero o esponente associato all’<item>
        <headLabel> intestazione per gli esponenti della lista
        <headItem> intestazione per gli elementi della lista
    \end{block}
    
   

\end{frame}




begin{frame}
	\frametitle{Intro Text Encoding Initiative}
	\framesubtitle{TEI}
	\addtocounter{nframe}{1}

	\begin{block}{TEI}
        45Schemi di codifica TEI – Moduli base
        Liste e tabelle
        agli item può essere associata o no una label, ma in caso
        affermativo deve essere presente per tutti
        il testo non deve essere necessariamente ordinato come lista
        per essere marcato in quanto tale:
        <list><head>Ingredienti:</head> <item>un cucchiaio di
        zucchero;</item> <item>mezzo chilo di farina;</item>
        <item>due uova.</item></list>
        usare l’attributo type per specificare il tipo di lista
        gli elementi per tabelle sono <table>, <row>, <cell> disponibili
        con il modulo figures, v. il cap. 14 Tables, Formulae and
        Graphics ( http://www.tei-c.org/release/doc/tei-p5-doc/en/html/FT.html )
    \end{block}
    
   

\end{frame}



begin{frame}
	\frametitle{Intro Text Encoding Initiative}
	\framesubtitle{TEI}
	\addtocounter{nframe}{1}

	\begin{block}{TEI}
        46Schemi di codifica TEI – Moduli base
        Lista semplice
        <list type="simple">
        <head>Lista della spesa:</head>
        <item>pane;</item>
        <item>frutta;</item>
        <item>verdura;</item>
        <item>latte;</item>
        <item>farina;</item>
        <item>uova;</item>
        <item>tovaglioli;</item>
        <item>bicchieri;</item>
        <item>piatti.</item>
        </list>
    \end{block}
    
   

\end{frame}




begin{frame}
	\frametitle{Intro Text Encoding Initiative}
	\framesubtitle{TEI}
	\addtocounter{nframe}{1}

	\begin{block}{TEI}
        47Schemi di codifica TEI – Moduli base
        Glossario
        <list type="gloss">
        <head>Tecnologie XML:</head>
        <label>XSL</label>
        <item>eXtensible Stylesheet Language</item>
        <label>XSL­T</label>
        <item>XSL ­ Transformations</item>
        <label>XSL­FO</label>
        <item>XSL – Formatting Objects</item>
        <label>XQuery</label>
        <item>XML Query Language</item>
        <label>XPAth</label>
        <item>XML Path Language</item>
        </list>
    \end{block}
    
   

\end{frame}




begin{frame}
	\frametitle{Intro Text Encoding Initiative}
	\framesubtitle{TEI}
	\addtocounter{nframe}{1}

	\begin{block}{TEI}
        48Schemi di codifica TEI – Moduli base
        Tabella
        <table rows="3" cols="2">
        <row>
        <cell>HTML</cell>
        <cell>Derivato da SGML (come applicazione), il
        linguaggio del World Wide Web.</cell>
        </row>
        <row>
        <cell>XML</cell>
        <cell>Derivato da SGML (per semplificazione).</cell>
        </row>
        <row>
        <cell>XSL­T</cell>
        <cell>Fogli di stile per XML.</cell>
        </row>
        </table>
    \end{block}
    
   

\end{frame}



begin{frame}
	\frametitle{Intro Text Encoding Initiative}
	\framesubtitle{TEI}
	\addtocounter{nframe}{1}

	\begin{block}{TEI}
        49Schemi di codifica TEI – Moduli base
        Note
        l’elemento <note> permette di inserire una nota di
        qualsiasi tipo (attributi type e place)
        la nota può appartenere al testo codificato, o può essere
        opera di chi lo codifica (attributo resp)
        se non nel flusso del testo usare gli attributi target e
        targetEnd per stabilire collegamento preciso
        Le <title>Guidelines</title><note n="2" place="foot"
        resp="rrdt">C.M. Sperberg­McQueen and Lou Burnard,
        <title>Guidelines for Electronic Text Encoding and
        Interchange</title> (Chicago, Oxford: Text Encoding
        initiative, 2002).</note> sono nate con lo scopo di [...]
    \end{block}
    
   

\end{frame}



begin{frame}
	\frametitle{Intro Text Encoding Initiative}
	\framesubtitle{TEI}
	\addtocounter{nframe}{1}

	\begin{block}{TEI}
        50Schemi di codifica TEI – Moduli base
        Esercizio 3
        marcare il testo del file ese03.jpg come segue:
        editor da usare: XML Copy Editor o Editix
        marcare tutto quello che si vede nella pagina
        marcare come glossario usando <list> oppure
        in alternativa: marcare come glossario <term> + <gloss>
        usare identificatori e puntatori/riferimenti
        inserire un paio di note a piè di pagina
        verificare costantemente che il documento sia valido
        salvare il file XML come ese03-Cognome.xml
    \end{block}
    
   

\end{frame}




begin{frame}
	\frametitle{Intro Text Encoding Initiative}
	\framesubtitle{TEI}
	\addtocounter{nframe}{1}

	\begin{block}{TEI}
        51Schemi di codifica TEI – Moduli base
        Bibliografia
        i riferimenti bibliografici, compresa la bibliografia vera e
        propria, sono codificati usando i seguenti elementi:
        <bibl>
        citazione bibliografica di tipo “flessibile”: possono
        essere presenti solo alcuni elementi; usata anche
        nel corpo del testo (NB: stessi elementi del succ.)
        <biblStruct>
        citazione bibliografica di tipo strutturato: gli
        elementi devono seguire uno schema fisso
        <biblFull > citazione bibliografica di tipo strutturato completo:
        gli elementi devono seguire uno schema fisso e
        devono essere tutti presenti
        <listBibl> lista di citazioni bibliografiche in uno qualsiasi
        dei formati citati sopra
    \end{block}
    
   

\end{frame}





begin{frame}
	\frametitle{Intro Text Encoding Initiative}
	\framesubtitle{TEI}
	\addtocounter{nframe}{1}

	\begin{block}{TEI}
        52Schemi di codifica TEI – Moduli base
        Tipi di pubblicazione
        ogni voce bibliografica può essere codificata sulla base di
        tre possibili tipi di pubblicazione:
        <analytic> “analitico”: informazioni su un titolo che
        non costituisce una pubblicazione autonoma
        (articolo di rivista o in una miscellanea)
        <monogr> “monografico”: informazioni su una
        pubblicazione, anche di tipo periodico (rivista)
        <series> “serie”: informazioni su di una serie
        editoriale
        tipicamente impiegati con gli elementi <biblStruct> e
        (meno di frequente) <biblFull>, più flessibilità con <bibl>
    \end{block}
    
   

\end{frame}





begin{frame}
	\frametitle{Intro Text Encoding Initiative}
	\framesubtitle{TEI}
	\addtocounter{nframe}{1}

	\begin{block}{TEI}
        53Schemi di codifica TEI – Moduli base
        Titolo, autore, curatore
        <title> il titolo della pubblicazione; più livelli (attr. level):
        level=”a”
        articolo (article)
        level=”j”
        rivista (journal)
        level=”m”
        monografia (monograph)
        level=”s”
        serie (series)
        level=”u”
        non pubblicato (unpublished)
        <author> autore della pubblicazione; può contenere
        direttamente il nome dell’autore o elementi più
        complessi (<persName> e sub-elementi)
        <editor> curatore della pubblicazione; vedi sopra
    \end{block}
    
   

\end{frame}





begin{frame}
	\frametitle{Intro Text Encoding Initiative}
	\framesubtitle{TEI}
	\addtocounter{nframe}{1}

	\begin{block}{TEI}
	% 54Schemi di codifica TEI – Moduli base
% Elementi descrittivi
% <imprint> raggruppa informazioni relative alla stampa
% o diffusione di una pubblicazione
% <publisher> l’editore responsabile della produzione a
% stampa o diffusione
% <pubPlace> il luogo di pubblicazione
% <date> la data di pubblicazione
% <biblScope> l’estensione (in pagine, volumi) della
% pubblicazione;
% permette di specificare il tipo di estensione
% ad es. se si tratta di pagine
% type
% <biblScope> può essere usato anche all’interno di <series>
    \end{block}
    
   

\end{frame}






begin{frame}
	\frametitle{Intro Text Encoding Initiative}
	\framesubtitle{TEI}
	\addtocounter{nframe}{1}

	\begin{block}{TEI}
        55Schemi di codifica TEI – Moduli base
        Esempio semplice
        il caso di una monografia con autore unico:
        <biblStruct xml:id="Robinson1993d">
        <monogr>
        <author>Robinson, P.</author>
        <title level="m">The Digitization of Primary Textual
        Sources</title>
        <imprint>
        <pubPlace>Oxford</pubPlace>
        <publisher>Office for Humanities
        Communication</publisher>
        <date>1993</date>
        </imprint>
        </monogr>
        </biblStruct>
    \end{block}
    
   

\end{frame}





begin{frame}
	\frametitle{Intro Text Encoding Initiative}
	\framesubtitle{TEI}
	\addtocounter{nframe}{1}

	\begin{block}{TEI}
        56Schemi di codifica TEI – Moduli base
        un articolo di miscellanea con due autori:
        <biblStruct xml:id="Mohler2001">
        <analytic>
        <author>Mohler, Peter Ph.</author>
        <author>Zuell, Cornelia</author>
        <title level="a">Applied Text Theory: Qualitative Analysis of
        Answers to Open Ended Questions</title>
        </analytic>
        <monogr>
        <editor>Mark D. West</editor>
        <title level="m">Application of Computer Content Analysis</title>
        <imprint>
        <pubPlace>Westport CN</pubPlace>
        <publisher>Ablex</publisher>
        <date>2001</date>
        <biblScope type="pages">1­16</biblScope>
        </imprint>
        </monogr>
        </biblStruct>
    \end{block}
    
   

\end{frame}






begin{frame}
	\frametitle{Intro Text Encoding Initiative}
	\framesubtitle{TEI}
	\addtocounter{nframe}{1}

	\begin{block}{TEI}
	% 57Schemi di codifica TEI – Moduli base
% un articolo su rivista di un solo autore:
% <biblStruct xml:id="Ester1994">
% <analytic>
% <author>Ester, Michael</author>
% <title level="a">Digital Images in the Context of Visual
% Collections and Scholarship</title>
% </analytic>
% <monogr>
% <title level="j">Visual Resources</title>
% <imprint>
% <date>1994</date>
% </imprint>
% <biblScope type="vol">X, 1</biblScope>
% <biblScope type="pages">23</biblScope>
% </monogr>
% <note type="key">iproc</note>
% </biblStruct>
    \end{block}
    
   

\end{frame}





begin{frame}
	\frametitle{Intro Text Encoding Initiative}
	\framesubtitle{TEI}
	\addtocounter{nframe}{1}

	\begin{block}{TEI}
        58Schemi di codifica TEI – Moduli base
        Componenti non testuali: immagini
        possibile includere nella marcatura immagini in formato XML
        (SVG, il formato TEI) e in formato binario (TIFF, JPEG, PNG)
        <graphic/>
        contiene un puntatore a una immagine
        url
        può essere locale o esterna
        <binaryObject> dati binari che costituiscono una immagine
        <media>
        riferimento a file audio, video etc.
        <figure>
        raggruppa informazioni relative a un’immagine
        <head>
        titolo o didascalia relativa all’immagine
        <figDesc>
        descrizione dell’immagine
        <graphic>
        puntatore a un’immagine
        NB <figure> nella P5 fa parte del modulo figures ( http://www.tei-
        c.org/release/doc/tei-p5-doc/en/html/FT.html#FTGRA )
    \end{block}
    
   

\end{frame}





begin{frame}
	\frametitle{Intro Text Encoding Initiative}
	\framesubtitle{TEI}
	\addtocounter{nframe}{1}

	\begin{block}{TEI}
	% 59Schemi di codifica TEI – Moduli base
% Esempio TEI P4
% NB: nella TEI P4 è indispensabile inserire dichiarazioni
% relative alle immagini nell’intestazione del documento
% <?xml version='1.0' encoding='utf­8'?>
% ...
% <!NOTATION JPEG PUBLIC
% 'ISO DIS 10918//NOTATION JPEG Graphics Format//EN' >
% <!ENTITY PDatabase SYSTEM "support/PDatabase.png" NDATA PNG>
% ]>
% ...
% <p>The system is build around a relational database, called the
% <soCalled>palaeographical database</soCalled>, that contains all
% the information the application produces and processes:</p>
% <figure entity="PDatabase">
% <figDesc>Schematic of the SPI system.</figDesc>
% </figure>
    \end{block}
    
   

\end{frame}





begin{frame}
	\frametitle{Intro Text Encoding Initiative}
	\framesubtitle{TEI}
	\addtocounter{nframe}{1}

	\begin{block}{TEI}
	% 60Schemi di codifica TEI – Moduli base
% Esempi TEI P5
% inserimento di un’immagine in un documento P5:
% <p>Questo fenomeno è chiaramente visibile nella figura che
% segue:
% <graphic url="diagramma.png"/>
% Altre variazioni delle crescite trimestrali sono
% disponibili nel rapporto allegato alla relazione.</p>
% immagine impiegata come intestazione in un documento P5:
% <head>
% <graphic
% url="http://www.iath.virginia.edu/gants/Ornaments/Heads/hp­
% ral02.gif"/>
% </head>
    \end{block}
    
   

\end{frame}


