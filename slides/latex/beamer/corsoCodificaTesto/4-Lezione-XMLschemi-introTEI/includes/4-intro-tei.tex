%%Codifica di testi

% Le norme TEI
% Roberto Rosselli Del Turco
% Dipartimento di Studi Umanistici
% Università di Torino
% roberto.rossellidelturco@fileli.unipi.it
% roberto.rossellidelturco@unito.itLa codifica di testi – Le norme TEI
% La Text Encoding Initiative
% Sito WWW: http://www.tei-c.org/

\begin{frame}
	\frametitle{Intro Text Encoding Initiative}
	\framesubtitle{TEI}
	\addtocounter{nframe}{1}

	\begin{block}{Motto}
		TEI: Yesterday's information tomorrow
	\end{block}

	\begin{block}{Dal sito TEI}
		“an international and interdisciplinary standard that
		enables libraries, museums, publishers, and individual
		scholars to represent a variety of literary and linguistic
		texts for online research, teaching, and preservation”
	\end{block}
\end{frame}


\begin{frame}
	\frametitle{Intro Text Encoding Initiative}
	\framesubtitle{TEI}
	\addtocounter{nframe}{1}

	\begin{block}{Testo di riferimento}
        Guidelines for Electronic Text Encoding and Interchange 
        \\( \url{http://www.tei-c.org/Guidelines/} )
	\end{block}

	\begin{block}{testo di ausilio}
		BURNARD, Lou. What is the Text Encoding Initiative? How to add intelligent markup to digital resources. Nouva edizione [online]. Marseille: OpenEdition Press, 2014 (creato il 13 octobre 2018). Disponibile su Internet: \url{http://books.openedition.org/oep/426}. ISBN: 9782821834606. DOI: 10.4000/books.oep.426.

	\end{block}
\end{frame}


\begin{frame}
	\frametitle{Intro Text Encoding Initiative}
	\framesubtitle{TEI}
	\addtocounter{nframe}{1}

	\begin{block}{un po' di storia}
		\begin{itemize}
			\item 1987: necessità di standard che permetta la creazione e l’interscambio di documenti per mezzo di archivi informatici(convegno NY)
			\item 1990: prima versione delle Guidelines (TEI P1)
			\item 1990-94: fondi garantiti da enti quali NEH, Mellon Foundation, la Comunità Europea; supporto di ACH, ACL, ALLC
		\end{itemize}
	\end{block}

\end{frame}

\begin{frame}
	\frametitle{Intro Text Encoding Initiative}
	\framesubtitle{TEI}
	\addtocounter{nframe}{1}

	\begin{block}{un po' di storia}
		\begin{itemize}
			\item 2000: nascita del TEI Consortium, associazione non profit per lo sviluppo dello standard TEI
			\item 2002: passaggio da SGML a XML con la v. P4
			\item 2007: nuova versione TEI P5, continuamente aggiornata
		\end{itemize}
	\end{block}

\end{frame}


\begin{frame}
    \frametitle{Intro Text Encoding Initiative}
    \framesubtitle{TEI}
    \addtocounter{nframe}{1}

	\begin{block}{TEI Guidelines}
		\textit{versioni P1 e P3 basate su SGML}
		\\\textbf{versione P4}
		\begin{itemize}
			\item standard precedente, ancora impiegata
			\item basata su XML, DTD tradizionale
			\item pubblicata in forma definitiva nel 2002
			\item \url{http://www.tei-c.org/Guidelines/P4/}
		\end{itemize}
    \end{block}
    
\end{frame}

\begin{frame}
	\frametitle{Intro Text Encoding Initiative}
	\framesubtitle{TEI}
	\addtocounter{nframe}{1}

	\begin{block}{TEI Guidelines: versione P5}
		\begin{itemize}
			\item basata su XML, schema RelaxNG (e DTD tradizionale)
			\item pubblicata alla fine del 2007, aggiornata due volte l’anno
			\item molte novità interessanti (in particolare: maggior modularità)
			\item  http://www.tei-c.org/Guidelines/P5/
		\end{itemize}
	\end{block}

\end{frame}

\begin{frame}
	\frametitle{Intro Text Encoding Initiative}
	\framesubtitle{TEI}
	\addtocounter{nframe}{1}

	\begin{block}{TEI Guidelines: Obiettivi}
		\begin{itemize}
			\item better interchange and integration of scholarly data
			\item support for all texts, in all languages, from all periods
			\item guidance for the perplexed: what to encode - hence, a user-driven codification of existing best practice
			\item assistance for the specialist: how to encode --- hence, a loose framework into which unpredictable extensions can be fitted
		\end{itemize}
	\end{block}

\end{frame}



\begin{frame}
	\frametitle{Intro Text Encoding Initiative}
	\framesubtitle{TEI}
	\addtocounter{nframe}{1}

	\begin{block}{TEI Guidelines: Obiettivi}
		These apparently incompatible goals result in a highly flexible,
		modular, environment for DTD customization.
		Lou Burnard, TEI and XML: a marriage made in heaven?
		(http://www.tei-c.org/Talks/marriage.xml?style=printable)
	\end{block}

\end{frame}


\begin{frame}
	\frametitle{Intro Text Encoding Initiative}
	\framesubtitle{TEI}
	\addtocounter{nframe}{1}

	\begin{block}{Che cosa offre la TEI}
        \begin{itemize}
            \item un ricco (e complesso) manuale di codifica, 
            \\(le Guidelines for Electronic Text Encoding and Interchange)
            \item un numero elevato di elementi (sia strutturale sia semantico)
            \item schemi di codifica
            \item infrastruttua modulare e personalizzabile
        \end{itemize}

	\end{block}

\end{frame}

\begin{frame}
	\frametitle{Intro Text Encoding Initiative}
	\framesubtitle{TEI}
	\addtocounter{nframe}{1}

	\begin{block}{Che cosa offre la TEI}
		\begin{itemize}
			\item \textbf{è possibile scegliere soltanto i moduli necessari}
			\item \textbf{è possibile modificare le definizioni degli elementi}
		\end{itemize}

	\end{block}

\end{frame}

\begin{frame}
	\frametitle{Intro Text Encoding Initiative}
	\framesubtitle{TEI}
	\addtocounter{nframe}{1}

	\begin{block}{Supporto per gli utenti}
		\begin{itemize}
			\item il sito del consorzio (\url{http://www.tei-c.org/})
			\item pagine relative alle varie versioni delle Guidelines
			\item software, tutorial, ecc.
			\item il wiki: \url{http://www.tei-c.org/wiki/index.php/Main_Page}
			\item la mailing list TEI\-L
			\item Github: \url{https://github.com/TEIC}
			\item TEI by example:\url{ http://teibyexample.org/}
		\end{itemize}
	\end{block}
\end{frame}


\begin{frame}
	\frametitle{Intro Text Encoding Initiative}
	\framesubtitle{TEI}
	\addtocounter{nframe}{1}

	\begin{block}{Novità della versione P5}
		\begin{itemize}
			\item modulo di descrizione dei manoscritti
			\item grafica e multimedia
			\item standoff markup
			\item supporto per i namespace XML
			\item miglioramenti nel modulo feature structure
		\end{itemize}

	\end{block}

\end{frame}

\begin{frame}
	\frametitle{Intro Text Encoding Initiative}
	\framesubtitle{TEI}
	\addtocounter{nframe}{1}

	\begin{block}{Novità della versione P5 (cont.)}
		\begin{itemize}
			\item modulo di descrizione dei manoscritti
			\item grafica e multimedia
			\item standoff markup
			\item supporto per i namespace XML
			\item miglioramenti nel modulo feature structure
		\end{itemize}

	\end{block}

\end{frame}

\begin{frame}
	\frametitle{Intro Text Encoding Initiative}
	\framesubtitle{TEI}
	\addtocounter{nframe}{1}

	\begin{block}{TEI: Struttura modulare}
		\begin{itemize}
			\item si scelgono soltanto i moduli che corrispondono alle proprie esigenze, in modo da realizzare rapidamente uno schema di codifica appropriato
			\item ogni modulo contiene un certo numero di elementi (tagset)
			\item gli elementi sono organizzati in classi (strutturali, semantiche)
			\item gli attributi sono organizzati in classi (globali e specifici)
		\end{itemize}

	\end{block}

\end{frame}


% classi strutturali: elementi che hanno un ruolo a livello strutturale
% (paragrafi, strofe, etc.)
% classi semantiche: elementi descrittivi del testo (aggiunte,
% correzioni, nomi di persona, etc.)
% anche gli attributi sono organizzati in classi
% attributi globali: disponibili per tutti gli elementi
% attributi specifici: solo per alcuni elementi


\begin{frame}
	\frametitle{Intro Text Encoding Initiative}
	\framesubtitle{TEI}
	\addtocounter{nframe}{1}

	\begin{block}{TEI: Struttura modulare - Moduli essenziali}
		\begin{itemize}
			\item \textbf{tei}: definisce le classi di elementi, le macro e i datatype che verranno usati per tutti i moduli
			\item \textbf{header}: l’intestazione contenente i metadati relativi al documento TEI XML
			\item \textbf{textstructure}: elementi strutturali per qualsiasi tipo di testo
			\item \textbf{core}: elementi utili in qualsiasi tipo di documento
		\end{itemize}

	\end{block}

\end{frame}

\begin{frame}
	\frametitle{Intro Text Encoding Initiative}
	\framesubtitle{TEI}
	\addtocounter{nframe}{1}

	\begin{block}{TEI: Struttura modulare - Moduli facoltativi}
		\begin{itemize}
			\item \textbf{analysis}: strumenti per analisi (linguistica etc.) del testo
			\item \textbf{corpus}: gestione di corpora linguistici
			\item \textbf{drama}: elementi per testi teatrali e drammatici
			\item \textbf{gaiji}: rappresentazione di caratteri e glifi non standard
			\item \textbf{msdescription}: metadati relativi a manoscritti
		\end{itemize}

	\end{block}

\end{frame}


\begin{frame}
	\frametitle{Intro Text Encoding Initiative}
	\framesubtitle{TEI}
	\addtocounter{nframe}{1}

	\begin{block}{TEI: Struttura modulare - Moduli facoltativi}
		\begin{itemize}
			\item \textbf{spoken}: trascrizione del parlato
			\item \textbf{textcrit}: apparato critico
			\item \textbf{transcr}:  trascrizione di fonti primarie (manoscritti)
			\item \textbf{verse}: elementi supplementari per testi poetici
		\end{itemize}

	\end{block}

\end{frame}

\begin{frame}
	\frametitle{Intro Text Encoding Initiative}
	\framesubtitle{TEI}
	\addtocounter{nframe}{1}

	\begin{block}{TEI Lite}

		una specifica \textbf{personalizzazione} della TEI versione P4/5

	\end{block}

	\textit{A simple demonstration of how the TEI encoding scheme
		might be adopted to meet 90\% of the needs of 90\% of the
		TEI user community (dalla Prefatory Note: \\url{http://www.tei-
		c.org/Lite/})}

\end{frame}


\begin{frame}
	\frametitle{Intro Text Encoding Initiative}
	\framesubtitle{TEI}
	\addtocounter{nframe}{1}

	\begin{block}{TEI Lite}

		La \textbf{versione P4} è stata tradotta anche in italiano: TEI Lite:
		introduzione alla codifica dei testi, a cura di \textit{F. Ciotti} (
		\url{http://www.tei-c.org/Lite/teiu5_it.html}).
    \end{block}
    \textit{Molto usata, ma presenta varie limitazioni, con la versione P5 è più semplice produrre una versione semplificata o personalizzata}
\end{frame}

\begin{frame}
	\frametitle{Intro Text Encoding Initiative}
	\framesubtitle{TEI}
	\addtocounter{nframe}{1}

	\begin{block}{TEI Pizza chef}

		la TEI P4 poteva essere modificata usando un programma su
		web chiamato Pizza chef
		\url{http://www.tei-c.org.uk/pizza.html}.
		\\Metafora della base e dei condimenti (toppings),
		corrispondenti ai moduli indispensabili e a quelli facoltativi.
		\\Meccanismo efficace, soprattutto considerando l’alternativa
		(modifica manuale delle DTD TEI).
    \end{block}
    
    \textit{Tool obsoleto, necessaria comunque modifica manuale per nuovi elementi}


\end{frame}

\begin{frame}
	\frametitle{Intro Text Encoding Initiative}
	\framesubtitle{TEI}
	\addtocounter{nframe}{1}

	\begin{block}{TEI Roma}

		Per la P5 si è deciso di proporre una modularizzazione più
		efficace, e uno strumento di personalizzazione più potente.
		Strumento basato sul nuovo formato ODD.


	\end{block}

	\begin{block}{TEI Roma}
		Il metodo da seguire per la versione P5 (quella che
		useremo) fino all’arrivo del successore (Byzantium)

	\end{block}

\end{frame}

\begin{frame}
	\frametitle{Intro Text Encoding Initiative}
	\framesubtitle{TEI}
	\addtocounter{nframe}{1}

	\begin{block}{TEI Roma}

		\begin{itemize}
			\item  possibilità di scegliere, escludere, modificare sia gli elementi (e le classi di elementi), sia gli attributi (e le classi di attributi)
			\item possibilità di aggiungere elementi (eventualmente inserendoli nelle classi preesistenti)
			\item possibilità di salvare lo schema in tre formati diversi: DTD  tradizionale, W3C e RelaxNG (anche in forma compatta)
		\end{itemize}

	\end{block}


\end{frame}


\begin{frame}
	\frametitle{Intro Text Encoding Initiative}
	\framesubtitle{TEI}
	\addtocounter{nframe}{1}

	\begin{block}{TEI Roma: Il formato ODD}
		le versioni P1 – P4 delle DTD TEI erano nel formato DTD language e dipendevano dalla sintassi SGML per molti aspetti
	\end{block}

	\begin{block}{TEI Roma: Il formato ODD}
		One Document Does it all (ODD): set di specifiche in base alle quali un semplice documento TEI XML \textit{“defines a schema in terms of the modules it requires, together with any possible modifications, such as the desired
        root element”}. 
    \end{block}
    
    \textbf{Risultato finale: lo schema nel linguaggio desiderato e la relativa documentazione}

\end{frame}


\begin{frame}
	\frametitle{Intro Text Encoding Initiative}
	\framesubtitle{TEI}
	\addtocounter{nframe}{1}

	\begin{block}{TEI Roma: Il formato ODD}

		tutorial: Getting Started with P5 ODDs\\
		\url{http://www.tei-c.org/Guidelines/Customization/odds.xml}

	\end{block}


\end{frame}


% 15La codifica di testi – Le norme TEI
% TEI Roma 2
% per motivi vari, la manutenzione di TEI Roma lascia a
% desiderare (per non parlare dello sviluppo di un successore)
% nel passato recente la componente di “controllo” dello
% schema (Sanity checker) non funzionava
% in tal caso è possibile andare sul sito TEI di Oxford dove è in
% esecuzione una versione più vecchia:
% http://tei.it.ox.ac.uk/Roma/
% Roma, in ogni caso, è solo un front-end per il linguaggio ODD
% (cfr. infra)




\begin{frame}
	\frametitle{Intro Text Encoding Initiative}
	\framesubtitle{TEI}
	\addtocounter{nframe}{1}

	\begin{block}{Il futuro della TEI}
		\textbf{negli ultimi anni gli schemi TEI sono stati oggetto di alcune critiche}
	\end{block}

	\begin{block}{Il futuro della TEI}
		\begin{itemize}
			\item la TEI è troppo grande / complicata / piccola
			\item la TEI è basata su XML e questo formato è in declino
			\item la TEI/XML non supporta le gerarchie multiple
			\item la TEI non supporta il markup di tipo stand-off
		\end{itemize}
	\end{block}

\end{frame}


\begin{frame}
	\frametitle{Intro Text Encoding Initiative}
	\framesubtitle{TEI}
	\addtocounter{nframe}{1}

	\begin{block}{Il futuro della TEI}
		la cosa importante da ricordare è che il formato XML non è
		la TEI (in passato SGML, in futuro chissà → abstraction level)
		al contrario, se c’è una discrepanza fra Guidelines e schemi,
		la precedenza va alle Guidelines
	\end{block}

\end{frame}


\begin{frame}
	\frametitle{Intro Text Encoding Initiative}
	\framesubtitle{TEI}
	\addtocounter{nframe}{1}

	\begin{block}{text encoding con la TEI}
		E' caldamente raccomandato usare direttamente la
		versione più recente della P5.\\
		La flessibilità della P5 permette di definire uno schema di
		codifica che corrisponda precisamente al modello
    \end{block}
    
    \textit{La comunità di utenti e sviluppatori TEI offre un buon supporto.}

\end{frame}




%%%%%%%%%%%%%%%%%%%%%%
% Codifica di testi
% I moduli base della TEI
% Roberto Rosselli Del Turco
% Dipartimento di Studi Umanistici
% Università di Torino
% roberto.rossellidelturco@fileli.unipi.it
% roberto.rossellidelturco@unito.itSchemi di codifica TEI – Moduli base
% Elementi disponibili per tutti i documenti TEI

\begin{frame}
	\frametitle{Intro Text Encoding Initiative}
	\framesubtitle{TEI}
	\addtocounter{nframe}{1}

	\begin{block}{Un documento TEI P5 ‘minimo’}
        \begin{itemize}
            \item prologo XML
            \item intestazione TEI
            \item elementi strutturali
            \item elementi semantici dei moduli base
        \end{itemize}
    \end{block}
    
\end{frame}

\begin{frame}
	\frametitle{Intro Text Encoding Initiative}
	\framesubtitle{TEI}
	\addtocounter{nframe}{1}

   \textbf{ Moduli di base: \textit{tei}, \textit{header}, \textit{textstructure}, \textit{core}}

	\begin{block}{un documento TEI P5 ‘minimo’}
        Anche usando soltanto i moduli essenziali si ha a disposizione
        uno schema adatto alla marcatura di numerosi tipi di testi.
    \end{block}
 
       \textit{Schemi ``leggeri'' consigliati: la TEI Lite, o se necessario una
        versione più ridotta della P5 (TEI
        Absolutely Bare)}
\end{frame}



\begin{frame}
	\frametitle{Intro Text Encoding Initiative}
	\framesubtitle{Schemi di codifica TEI – Moduli base}
	\addtocounter{nframe}{1}

	\begin{block}{Caratteristiche degli elementi illustrati}
        \begin{itemize}
            \item gli elementi TEI rientrano nelle categorie generali di
            elementi XML che abbiamo visto
            \item elementi che possono contenere solo altri elementi (=
            elementi strutturali)
            \item elementi che possono contenere altri elementi e testo
            \item elementi che possono contenere solo testo
            \item  elementi vuoti (es. \texttt{<pb/>})
            \item  gli elementi vuoti marcano una gerarchia differente
        \end{itemize}
    \end{block}
\end{frame}


\begin{frame}
	\frametitle{Intro Text Encoding Initiative}
	\framesubtitle{Schemi di codifica TEI – Moduli base}
	\addtocounter{nframe}{1}

	\begin{block}{Gerarchie multiple}
        
        \texttt{<?xml version="1.0" encoding="utf-8"?>
        <text>
        <titolo>Gli assassinii della Rue Morgue</titolo>
        <intestazione> I </intestazione>
        \emph{<pagina n=``5''>}
        <p>Le facoltà mentali che si sogliono chiamare analitiche sono, di
        per se stesse, poco suscettibili di analisi [...]</p>
        \emph{<p>}La facoltà di risolvere è probabilmente molto rinfor-
        \emph{</pagina>}
        <pagina n=``6''>
        zata dallo studio delle matematiche e in modo particolare
        dell’altissimo ramo di questa scienza che[...] \emph{</p>}
        </pagina>
        </text>}
    \end{block}

\end{frame}



\begin{frame}
	\frametitle{Intro Text Encoding Initiative}
	\framesubtitle{Schemi di codifica TEI – Moduli base}
	\addtocounter{nframe}{1}

	\begin{block}{Gerarchie multiple- elementi vuoti}
        \texttt{<?xml version="1.0" encoding="utf-8"?>
        <text>
        <titolo>Gli assassinii della Rue Morgue</titolo>
        <intestazione> I </intestazione>
        <pagina n="5"/>
        <p>Le facoltà mentali che si sogliono chiamare analitiche sono, di
        per se stesse [...]</p>
        <p>La facoltà di risolvere è probabilmente molto rinfor-
        <pagina n="6"/>
        zata dallo studio delle matematiche e in modo particolare
        dell’altissimo ramo di questa scienza che [...] </p> </text>}
       
    \end{block}
    
   

\end{frame}

\begin{frame}
	\frametitle{Intro Text Encoding Initiative}
	\framesubtitle{Schemi di codifica TEI – Moduli base}
	\addtocounter{nframe}{1}

	\begin{block}{Struttura di un documento TEI}
        \begin{itemize}
            \item \textit{struttura fondamentale all’interno della radice (\texttt{<TEI>})}
            \item una intestazione TEI (\texttt{<teiHeader>})
            \item un testo: \texttt{<text>} (o più testi, cfr. infra)
        \end{itemize}
    \end{block}
    
\end{frame}


\begin{frame}
	\frametitle{Intro Text Encoding Initiative}
	\framesubtitle{Schemi di codifica TEI – Moduli base}
	\addtocounter{nframe}{1}

    \begin{block}{Contenuto del TEI header}
        \begin{itemize}
            \item metadati relativi al documento (utili per collezioni di testi
            codificati)
            \item descrizione del file usando \texttt{<fileDesc>} (obbligatoria)
            \item descrizioni relative al tipo di codifica, al contenuto del
            documento, alle sue revisioni (facoltative)
        \end{itemize}
    \end{block}
\textit{E' possibile includere testi introduttivi e spiegazioni relative alla
codifica effettuata (preziosi per l’interscambio!)}

\end{frame}


\begin{frame}
	\frametitle{Intro Text Encoding Initiative}
	\framesubtitle{Schemi di codifica TEI:  Moduli base}
	\addtocounter{nframe}{1}

	\begin{block}{Esempio}
       \texttt{<?xml version="1.0" encoding="utf\-8"?>
         <!DOCTYPE TEI SYSTEM ``tei\_lite.dtd''>
         <TEI xmlns=``http://www.tei\-c.org/ns/1.0''>
         <teiHeader> </teiHeader>
         <text>
             <div>
             <p></p>
             </div>
        </text>
        </TEI>}
    \end{block}
\end{frame}


\begin{frame}
	\frametitle{Intro Text Encoding Initiative}
	\framesubtitle{Schemi di codifica TEI – Moduli base}
	\addtocounter{nframe}{1}

	\begin{block}{Esempio}
        \texttt{<?xml version="1.0" encoding="utf-8"?>
        <?xml-model href="tei-lite.rng"?>
        <TEI xmlns=``http://www.tei-c.org/ns/1.0''>
        <teiHeader>...</teiHeader>
        <text>
        <div>
        <p></p>
        </div>
        </text>
        </TEI>}
    \end{block}
    
   

\end{frame}



\begin{frame}
	\frametitle{Intro Text Encoding Initiative}
	\framesubtitle{Schemi di codifica TEI – Moduli base}
	\addtocounter{nframe}{1}

	\begin{block}{documento TEI - schema di intestazione TEI minima}
        \texttt{<teiHeader>
        <fileDesc>
        <titleStmt>...</titleStmt>
        <publicationStmt>...</publicationStmt>
        <sourceDesc>...</sourceDesc>
        </fileDesc>
        </teiHeader>}
    \end{block}
    

\end{frame}


\begin{frame}
	\frametitle{Intro Text Encoding Initiative}
	\framesubtitle{Schemi di codifica TEI – Moduli base}
	\addtocounter{nframe}{1}

	\begin{block}{documento TEI - schema di intestazione TEI minima}
        Metadati essenziali riguardo il titolo, la modalità di diffusione e
        la fonte originaria di un testo codificato.
        \\Permettono classificazione, archiviazione ed elaborazione
        bibliografica
    \end{block}
    
\end{frame}



\begin{frame}
	\frametitle{Intro Text Encoding Initiative}
	\framesubtitle{Schemi di codifica TEI – Moduli base}
	\addtocounter{nframe}{1}

        \texttt{<teiHeader>
        <fileDesc>
        <titleStmt>
        <title>La Divina Commedia: versione elettronica</title>
        <respStmt>
        <resp>Conversione TEI P5 a cura di</resp><name>M. Rossi</name>
        </respStmt>
        </titleStmt>
        <publicationStmt>
        <publisher>Università di Pisa</publisher>
        <date>2002-11-07</date>
        <availability status=``restricted''><p></p></availability>
        </publicationStmt>
        <sourceDesc>
        <bibl><title>La Divina Commedia</title><author>Dante Alighieri
        </author><publisher>Mondadori</publisher>
        <date>1988</date></bibl>
        </sourceDesc>
        </fileDesc>
        </teiHeader>}

\end{frame}



\begin{frame}
	\frametitle{Intro Text Encoding Initiative}
	\framesubtitle{Schemi di codifica TEI – Moduli base}
	\addtocounter{nframe}{1}

    \begin{block}{Le altre componenti dell’intestazione TEI}
        \begin{itemize}
            \item \texttt{<encodingDesc>} informazioni riguardo lo schema (e il
            modello di codifica) utilizzato
            \item  \texttt{<profileDesc>} descrizione del testo: quando è stato
            creato, da chi, usando quali lingue etc.
            \item \texttt{<revisionDesc>} informazioni sulle versioni del file
        \end{itemize}
    \end{block}
    \textit{I metadati sono una componente essenziale di qualsiasi
        progetto di digitalizzazione}
\end{frame}


\begin{frame}
	\frametitle{Intro Text Encoding Initiative}
	\framesubtitle{Schemi di codifica TEI – Moduli base}
	\addtocounter{nframe}{1}

	\begin{block}{Elementi strutturali}
        
       \begin{itemize}
           \item \texttt{<text>} un singolo testo di qualsiasi tipo (punto di partenza della gerarchia).
           \item \texttt{<facsimile>} riproduzione della fonte primaria, può affiancare o sostituire \texttt{<text>}
           \item \texttt{<front>} figlio di \texttt{<text>} materiale che precede il testo
           \item \texttt{<body>} figlio di \texttt{<text>} rappresenta il testo stesso
           \item \texttt{<back>} figlio di \texttt{<text>} materiale che segue il testo
           \item \texttt{<group>} figlio di \texttt{<text>} alternativo a \texttt{<body>}, raggruppa testi diversi
       \end{itemize}
        
    \end{block}
    
   

\end{frame}



\begin{frame}
	\frametitle{Intro Text Encoding Initiative}
	\framesubtitle{Schemi di codifica TEI – Moduli base}
	\addtocounter{nframe}{1}

	\begin{block}{Esempio schematico di documento TEI}
        \texttt{<TEI>
        <teiHeader> [informazioni del TEI Header]
        </teiHeader>
        <text>
        <front> [premessa, dedica ...] </front>
        <body> [corpo del testo ...] </body>
        <back> [postfazione, appendice ...]</back>
        </text>
        </TEI>}
    \end{block}
    
   

\end{frame}



\begin{frame}
	\frametitle{Intro Text Encoding Initiative}
	\framesubtitle{Schemi di codifica TEI – Moduli base}
	\addtocounter{nframe}{1}
    \begin{block}{Costruire documenti compositi}
        \begin{itemize}
            \item rimpiazzando il \texttt{<body>} con un gruppo (\texttt{<group>}) di testi si ottiene un documento composito
            \item ciascuno di questi testi è rappresentato secondo una struttura
            standard
            \item un’altra possibilità è creare un corpus con \texttt{<teiCorpus>}
            \item intestazioni (\texttt{<teiHeader>}) separate per il corpus e per
            ciascun gruppo di testi
            \item struttura più complessa, su più livelli
        \end{itemize}
    \end{block}
\end{frame}


\begin{frame}
	\frametitle{Intro Text Encoding Initiative}
	\framesubtitle{Schemi di codifica TEI – Moduli base}
	\addtocounter{nframe}{1}

        \texttt{<TEI>
        <teiHeader> [ intestazione del testo composito ] </teiHeader>
        <text>
        <front> [ front matter del composito ] </front>
        <group>
        <text>
        <front> [ front matter del primo testo ] </front>
        <body> [ body del primo testo ]
        </body>
        <back> [ back matter del primo testo ] </back>
        </text>
        <text>
        <front> [ front matter del secondo testo] </front>
        <body> [ body del secondo testo ]
        </body>
        <back> [ back matter del secondo testo ] </back>
        </text>
        ...
        [ altri testi o gruppi di testi ]
        ...
        </group>
        <back>
        [ back matter del composito ]
        </back>
        </text>
        </TEI>}

\end{frame}

\begin{frame}
	\frametitle{Intro Text Encoding Initiative}
	\framesubtitle{Schemi di codifica TEI – Moduli base}
	\addtocounter{nframe}{1}

    \begin{block}{Costruzione di corpora TEI}
        \texttt{<teiCorpus>
<teiHeader> [metadati per il corpus] </teiHeader>
<TEI>
<teiHeader> [metadati relativi al I testo]</teiHeader>
<text> [primo testo del corpus] </text>
</TEI>
<TEI>
<teiHeader>[metadati relativi al II testo]</teiHeader>
<text> [secondo testo del corpus] </text>
</TEI>
</teiCorpus>}
    \end{block}
\end{frame}


% \begin{frame}
% 	\frametitle{Intro Text Encoding Initiative}
% 	\framesubtitle{TEI}
% 	\addtocounter{nframe}{1}

% 	\begin{block}{TEI}
% 	 Schemi di codifica TEI – Moduli base
%   IMMAGINE
%     \end{block}
% \end{frame}


\begin{frame}
	\frametitle{Intro Text Encoding Initiative}
	\framesubtitle{Schemi di codifica TEI – Moduli base}
	\addtocounter{nframe}{1}

	\begin{block}{Altri elementi strutturali fondamentali}
        \begin{itemize}
            \item suddivisioni del testo, non numerati: \texttt{<div> }(nessun limite di nidificazione)
            \item suddivisioni del testo, numerati: \texttt{<div1> ... <div7> }(massimo 7 livelli)
            \item paragrafi: \texttt{<p>}
            \item testo riferito: \texttt{<q>} (discorso diretto, citazioni, etc.)
        \end{itemize}
        
    \end{block}
\end{frame}


\begin{frame}
	\frametitle{Intro Text Encoding Initiative}
	\framesubtitle{Schemi di codifica TEI – Moduli base}
	\addtocounter{nframe}{1}

	\begin{block}{Altri elementi strutturali fondamentali}
        \begin{itemize}
            \item versi: strofe \texttt{<lg>} e singoli versi \texttt{<l>}
            \item testi teatrali: discorsi \texttt{<sp>} che possono contenere paragrafi
            \texttt{<p>} o versi \texttt{<l>}, oltre a direzioni di scena \texttt{<stage>}
            \item milestone tags: \texttt{<pb/>, <lb/>, <cb/>, <milestone/>}
            \item notare che un \texttt{<div>} può contenere un \texttt{<floatingText>} (possibilità di introdurre gerarchie complesse).
        \end{itemize}
        
    \end{block}

\end{frame}

\begin{frame}
	\frametitle{Intro Text Encoding Initiative}
	\framesubtitle{Schemi di codifica TEI – Moduli base}
	\addtocounter{nframe}{1}

	\begin{block}{Apertura e chiusura di un \texttt{<div>}}
        \begin{itemize}
            \item \texttt{<head>}: qualunque tipo di intestazione: il titolo di un opera, l’intestazione di un paragrafo, di una sezione, ecc.
            \item l'attributo \textit{type} permette di classificare in base a una tipologia
            \item \texttt{<epigraph>} citazione all’inizio del testo, o nella pagina del titolo, eventualmente con riferimento bibliografico
            \item \texttt{<opener>} raggruppa un serie di elementi (data, luogo, saluti, ecc.) all’inizio del \texttt{<div>}, specie di una lettera
        \end{itemize}
    \end{block}

\end{frame}

\begin{frame}
	\frametitle{Intro Text Encoding Initiative}
	\framesubtitle{Schemi di codifica TEI – Moduli base}
	\addtocounter{nframe}{1}

	\begin{block}{Apertura e chiusura di un \texttt{<div>}}
        \begin{itemize}
            \item \texttt{<argument>}: lista degli argomenti trattati nel \texttt{<div>}
            \item \texttt{<trailer>} frase che compare alla fine del \texttt{<div>} (ad esempio ``Fine del capitolo 1'')
            \item \texttt{<closer>} raggruppa un serie di elementi (data, luogo, saluti, etc.) alla fine del \texttt{<div>}, specie di una lettera
        \end{itemize}
    \end{block}

\end{frame}


\begin{frame}
	\frametitle{Intro Text Encoding Initiative}
	\framesubtitle{Schemi di codifica TEI – Moduli base}
	\addtocounter{nframe}{1}        
        \texttt{<div type="lettera”>
        <opener>
        <dateline>
        <name type=``place''>Pisa</name>
        <date>20 marzo 2015</date>
        </dateline>
        <salute>Gentilissima Prof.ssa Scannagatti,</salute>
        </opener>
        <p>sono spiacente di doverle comunicare che un’invasione di cavallette si
        è abbattuta sui miei quaderni incautamente lasciati in giardino, e li ha
        divorati interamente.</p>
        <p>Questo purtroppo significa che non posso mostrare i compiti svolti,
        come sempre, con solerzia e assiduo impegno.</p>
        <closer>
        <salute>Certo di poter contare sulla sua comprensione le porgo i miei
        migliori saluti,</salute>
        <signed>Pierino Rossi</signed>
        </closer>
        </div>}

\end{frame}





\begin{frame}
	\frametitle{Intro Text Encoding Initiative}
	\framesubtitle{Schemi di codifica TEI – Moduli base}
	\addtocounter{nframe}{1}

	\begin{block}{Errori frequenti}
        \textit{Si fraintende il significato dell’elemento \texttt{<fileDesc>}}
        \begin{itemize}
            \item serve in primo luogo a dare informazioni sul file stesso, non sul testo originale
            \item il riferimento alla fonte dalla quale è tratto il testo codificato
            deve essere inserito nel \texttt{<sourceDesc>}
        \end{itemize}
    \end{block}

\end{frame}


\begin{frame}
	\frametitle{Intro Text Encoding Initiative}
	\framesubtitle{Schemi di codifica TEI – Moduli base}
	\addtocounter{nframe}{1}

	\begin{block}{Errori frequenti}
        I titoli sono codificati con \texttt{<title>} soltanto nel caso di riferimenti bibliografici i titoli del testo, dei capitoli etc. si marcano con \texttt{<head>}
    \end{block}
    
   

\end{frame}


\begin{frame}
	\frametitle{Intro Text Encoding Initiative}
	\framesubtitle{Schemi di codifica TEI – Moduli base}
	\addtocounter{nframe}{1}

	\begin{block}{Nota sugli errori possibili}
        \textbf{Tre categorie:}
        \begin{itemize}
            \item \textbf{errori sintattici}: un elemento inserito in un punto sbagliato
            della gerarchia, o che non può contenere testo etc.
            \item \textbf{errori di marcatura semantica}: usare un elemento inadatto
            allo scopo, ad esempio marcare un titolo con <emph>
            \item \textbf{errori di interpretazione} del testo (che portano al II tipo o
            all’assenza del markup che andrebbe inserito)
        \end{itemize}
       
    \end{block}
    \textit{Gli errori del primo tipo sono i più facili da individuare e
        correggere, quelli del terzo i più difficili}

\end{frame}


% \begin{frame}
% 	\frametitle{Intro Text Encoding Initiative}
% 	\framesubtitle{Schemi di codifica TEI – Moduli base}
% 	\addtocounter{nframe}{1}

% 	\begin{block}{ A proposito di <div>}
%         domanda che ricorre periodicamente: perché non usare nomi
%         più significativi invece di un contenitore generico come <div>
%         (= ‘suddivisione, parte di un testo’)?
%         perché non usare <book>, <chapter>, <section>, etc. come
%         fa, ad esempio, lo schema DOCBOOK?
%         troppa variabilità nell’uso comune, meglio usare un termine
%         generico che possa poi essere specificato usando l’attributo
%         type (es. <div type=”chapter”>)
%         i <div>, numerati o meno, possono essere ‘nidificati’, nessun
%         problema nel riproporre la struttura editoriale visibile
%     \end{block}

% \end{frame}


\begin{frame}
	\frametitle{Intro Text Encoding Initiative}
	\framesubtitle{Schemi di codifica TEI – Moduli base}
	\addtocounter{nframe}{1}

    \textbf{Alcuni attributi possono essere usati con qualsiasi elemento (v. la classe att.global)}

    \begin{block}{ Attributi globali}
        \begin{itemize}
            \item \textbf{n} un numero o un nome non univoco, possibilmente breve, per identificare un elemento
            \item \textbf{rend} informazioni relative all’aspetto (\textit{originale}!) del testo
            \item \textbf{rendition} simile a \textit{@rend}, ma fa riferimento a elementi
            \texttt{<rendition>} inseriti nell’\texttt{<encodingDesc>} (dentro \texttt{<tagsDecl>})
        \end{itemize}
    \end{block}
\end{frame}

\begin{frame}
	\frametitle{Intro Text Encoding Initiative}
	\framesubtitle{Schemi di codifica TEI – Moduli base}
	\addtocounter{nframe}{1}

    \begin{block}{ Attributi globali}
        \begin{itemize}
            \item \textbf{xml:lang} la lingua del testo contenuto da un elemento
            \item \textbf{xml:id} un identificatore univoco per l’elemento
        \end{itemize}
       \textit{NOTA: in base ai moduli usati nello schema sono disponibili ulteriori attributi globali}
    \end{block}
\end{frame}

\begin{frame}
	\frametitle{Intro Text Encoding Initiative}
	\framesubtitle{Schemi di codifica TEI – Moduli base}
	\addtocounter{nframe}{1}

	\begin{block}{Esempio}
        \texttt{<text>
        <body>
        \emph{<div n="ch1" type=``chapter''>}
        <pb n="1"/>[...]
        <p>[...] risulta chiaro se avete letto \emph{<title
        rend="underline" xml:lang=``fra''>}Les fleurs du
        mal</title> [...]</p>
        <p>[...] un grande esempio di <foreign
        xml:lang=``fra''>savoir faire</foreign> [...]</p>
        [...]
        </div>
        [ altri div ... ]
        </body>
        </text>}
    \end{block}

\end{frame}


\begin{frame}
	\frametitle{Intro Text Encoding Initiative}
	\framesubtitle{Schemi di codifica TEI: Moduli base}
	\addtocounter{nframe}{1}

	\begin{block}{Esempio}
       \texttt{
           <text>
        <body>
        <div n="ch1" type=``chapter''> <pb n="1"/> [...]
        <p n=``1''>[...] descritto altrove (si veda ad
        esempio \emph{<ref target=``\#Rossi94''>Rossi 1994</ref>})
        [...] </p> [...]
        </div>
        [ altri div ... ]
        <div n="bib" type=``bibliography''>
        [...]
        \emph{<bibl xml:id=``Rossi94''>
        <author>Rossi, M.</author>[...]</bibl>}
        [...]
        </div>
        </body>
        </text>}
    \end{block}
\end{frame}

\begin{frame}
	\frametitle{Intro Text Encoding Initiative}
	\framesubtitle{Schemi di codifica TEI – Moduli base}
	\addtocounter{nframe}{1}

    \begin{block}{Errori frequenti}
        \texttt{<div>} non può essere usato allo stesso livello gerarchico
        di \texttt{<p>}, in altre parole non si può alternare \texttt{<div>} con
        \texttt{<p>}
    \end{block}
    
    \begin{block}{Errore!}
        \texttt{<div> [...] </div>
        <p> [...] </p>
        <div> [...] </div>}
    \end{block}
\end{frame}


\begin{frame}
	\frametitle{Intro Text Encoding Initiative}
	\framesubtitle{Schemi di codifica TEI – Moduli base}
	\addtocounter{nframe}{1}

    \begin{block}{Errori frequenti}
        \texttt{<div>} e tutti gli altri elementi strutturali \textit{puri} non
        possono contenere testo.
       
    \end{block}
    
    \begin{block}{Errore!}
        \texttt{<div>Pippo</div>
        <person>Pippo</person>}
    \end{block}
    

\end{frame}



\begin{frame}
	\frametitle{Intro Text Encoding Initiative}
	\framesubtitle{Schemi di codifica TEI – Moduli base}
	\addtocounter{nframe}{1}

	\begin{block}{Enfasi e termini particolari}
       
        \begin{itemize}
            \item \texttt{<emph>} parole o frasi enfatizzate nel testo. (\texttt{Questo è il <emph>mio</emph> computer!})
            \item \texttt{<foreign>} parola o frase in una lingua diversa. (\texttt{In quel punto entrò il bidello a dare il <foreign xml:lang=``lat''>finis</foreign>}).
        \end{itemize}
    \end{block}

\end{frame}

\begin{frame}
	\frametitle{Intro Text Encoding Initiative}
	\framesubtitle{Schemi di codifica TEI – Moduli base}
	\addtocounter{nframe}{1}

	\begin{block}{Enfasi e termini particolari}
       
        \begin{itemize}
            \item \texttt{<distinct>} “diverso” dal testo perché arcaico, gergale, ecc. (\texttt{Saltò in groppa al <distinct>fido destriero</distinct>})
            \item \texttt{<hi>} elemento generico. (\texttt{<hi rend=``double''>N</hi>el mezzo del cammin di nostra vita.
            Il suo nome è <hi rend=``italic''>Mario Rossi</hi>})
        \end{itemize}
    \end{block}

\end{frame}

\begin{frame}
	\frametitle{Intro Text Encoding Initiative}
	\framesubtitle{Schemi di codifica TEI: Moduli base}
	\addtocounter{nframe}{1}

    \begin{block}{Enfasi e termini particolari}
        \begin{itemize}
            \item  \texttt{<mentioned>} parola o frase menzionata ma non usata.
            (\texttt{Il termine corretto è <mentioned>epigrafe</mentioned>})
            \item \texttt{<soCalled>} parola o espressione da cui ci si distanzia
            (\texttt{il cosiddetto <soCalled>darwinismo sociale</soCalled>})
        \end{itemize}
    \end{block}
    
\end{frame}

\begin{frame}
	\frametitle{Intro Text Encoding Initiative}
	\framesubtitle{Schemi di codifica TEI: Moduli base}
	\addtocounter{nframe}{1}

    \begin{block}{Enfasi e termini particolari}
        \begin{itemize}
            \item \texttt{<term>} una o più parole considerate termine tecnico.
            (\texttt{Possiamo definire il <term xml\:id="NPL" rend=``italic''>neopositivismo logico</term>})
            \item \texttt{<gloss>} una spiegazione o glossa riguardo il testo.
            (\texttt{<gloss target=``\#NPL''>una corrente filosofica basata
            sul principio che la filosofia debba aspirare al rigore
            proprio della scienza </gloss>})
        \end{itemize}
    \end{block}
    
\end{frame}

\begin{frame}
	\frametitle{Intro Text Encoding Initiative}
	\framesubtitle{Schemi di codifica TEI – Moduli base}
	\addtocounter{nframe}{1}

	\begin{block}{Esercizio}
       \textbf{ Marcare un testo plain text di circa 3000 caratteri a piacere.}
        \begin{itemize}
            \item inserire prologo XML
            \item marcare la struttura usando gli elementi fin qui descritti
            in particolare marcare tutti i paragrafi usando \texttt{<p>} e la struttura editoriale usando \text{<div>}
            \item verificare che sia ben formato con xmllint
            \item salvare il file XML su github
        \end{itemize}
    \end{block}
\end{frame}


%%%%%%%%%%%%%%%%%%%%%%%%%%%%%%%%%%%%%%%%%%%%%%%%%

% \begin{frame}
% 	\frametitle{Intro Text Encoding Initiative}
% 	\framesubtitle{TEI}
% 	\addtocounter{nframe}{1}

% 	\begin{block}{TEI}
%         Schemi di codifica TEI: Moduli base
%         Enfasi e termini particolari 2

%         \texttt{<mentioned> parola o frase menzionata ma non usata
%         il termine corretto è <mentioned>epigrafe</mentioned>
%         <soCalled>
%         parola o espressione da cui ci si distanzia
%         il cosiddetto <soCalled>darwinismo sociale</soCalled>
%         <term>
%         una o più parole considerate termine tecnico
%         possiamo definire il <term xml\:id="NPL"
%         rend=``italic''>neopositivismo logico</term>
%         <gloss>
%         una spiegazione o glossa riguardo il testo
%         come <gloss target=``\#NPL''>una corrente filosofica basata
%         sul principio che la filosofia debba aspirare al rigore
%         proprio della scienza</gloss>}
%     \end{block}
    
% \end{frame}


% \begin{frame}
% 	\frametitle{Intro Text Encoding Initiative}
% 	\framesubtitle{TEI}
% 	\addtocounter{nframe}{1}

% 	\begin{block}{TEI}
%         32Schemi di codifica TEI – Moduli base
%         Citazioni 1
%         <q>
%         testo citato da altre fonti: discorso diretto, esempi
%         (nei dizionari), etc.
%         La mia maestra della prima superiore mi salutò di
%         sulla porta della classe e mi disse: <q rend=``PRE
%         mdash''>Enrico, tu vai al piano di sopra, quest'anno;
%         non ti vedrò nemmen più passare!</q>
%         <quote>
%         frase o brano attribuito a fonte esterna
%         <p>E allora disse: <q rend=``PRE lsquo POST
%         rsquo''>Ecco come comincia la Divina Commedia:
%         <quote>Nel mezzo del cammin di nostra vita / Mi
%         ritrovai per una selva oscura</quote>.</q></p>
%     \end{block}
    
   

% \end{frame}





% \begin{frame}
% 	\frametitle{Intro Text Encoding Initiative}
% 	\framesubtitle{TEI}
% 	\addtocounter{nframe}{1}

% 	\begin{block}{TEI}
%         33Schemi di codifica TEI – Moduli base
%         Citazioni 2
%         <said> testo pronunciato ad alta voce o pensato
%         <cit> citazione con riferimento bibliografico
%         Lexicography has shown little sign of being affected by the work
%         of followers of J.R. Firth, probably best summarized in his
%         slogan, <cit>
%         <quote>You shall know a word by the company it keeps.</quote>
%         <ref>(Firth, 1957)</ref>
%         </cit>
%         semplice riferimento bibliografico nell’esempio, possibile
%         aggiungere un collegamento (a capitolo/sezione o a una
%         specifica entrata bibliografica) usando l’attributo @target
%     \end{block}
    
   

% \end{frame}






% \begin{frame}
% 	\frametitle{Intro Text Encoding Initiative}
% 	\framesubtitle{TEI}
% 	\addtocounter{nframe}{1}

% 	\begin{block}{TEI}
%         Schemi di codifica TEI: Moduli base

%         A proposito di \texttt{<q> e <quote>}
%         possono contenere non solo altri elementi simili (\texttt{<q> e
%         <quote>}) ma anche elementi come \texttt{<p>, <l>, ecc.}:
        
%     %    \texttt{ <p>
%     %     <q>The Lord! The Lord! It is Sakya Muni himself,</q> the lama half
%     %     sobbed; and under his breath began the wonderful Buddhist invocation:­<q>
%     %     <quote>
%     %     <l>To Him the Way \— the Law \— Apart \—</l>
%     %     <l>Whom Maya held beneath her heart</l>
%     %     <l>Ananda's Lord \— the Bodhisat</l>
%     %     </quote>
%     %     And He is here! The Most Excellent Law is here also. My
%     %     pilgrimage is well begun. And what work! What work!</q>
%     %     </p>}
%     %     \\Possibili problemi relativi alla gerarchia (tag overlap, gerarchie
%     %     sovrapposte):
%     %     \texttt{<p>Allora disse: <q>Sarò breve.</p><p>Ho finito.</q></p>}
%     \end{block}
% \end{frame}

% \begin{frame}
% 	\frametitle{Intro Text Encoding Initiative}
% 	\framesubtitle{TEI}
% 	\addtocounter{nframe}{1}

% 	\begin{block}{TEI}
%         Schemi di codifica TEI: Moduli base
%         Nomi, numeri e date 1
%         <rs>
%         lett. referring string, nome o etichetta generica
%         <q>Mio caro <rs type=``person''>Filippo</rs></q>,
%         gli disse <rs type=``person''>sua moglie</rs>...

%         <name>
%         nome proprio (di persona, luogo, ecc.)
%         <q>Mio caro <name type=``person''>Filippo</name></q>,

%         <num>
%         un numero in qualsiasi formato
%         <num value=``23''>XXIII</num>

%         % <date>
%         % una data in qualsiasi formato
%         % nato il <date when=``1868­02­10''>10 febb. 1868</date>

%         <time>
%         l'ora in qualsiasi formato
%         alle <time when=``8.00''>otto del mattino</time>
%     \end{block}   

% \end{frame}


% \begin{frame}
% 	\frametitle{Intro Text Encoding Initiative}
% 	\framesubtitle{TEI}
% 	\addtocounter{nframe}{1}

% 	\begin{block}{TEI}
%         36Schemi di codifica TEI – Moduli base
%         Nomi, numeri e date 2
%         se gli elementi dei moduli di base risultassero insufficienti per
%         la codifica è possibile usare un modulo specifico:
%         13 Names, Dates, People, and Places (
%         http://www.tei-c.org/release/doc/tei-p5-doc/en/html/ND.html )
%         questo modulo permette una granularità molto maggiore
%         grazie agli elementi specifici che mette a disposizione
%         (<persName>, <forename>, <surname>, <roleName>,
%         <addName>, etc.)
%         possibile creare sistema prosopografico
%         ricchezza degli attributi (anche nella versione base)
%     \end{block}
    
   

% \end{frame}


% \begin{frame}
% 	\frametitle{Intro Text Encoding Initiative}
% 	\framesubtitle{TEI}
% 	\addtocounter{nframe}{1}

% 	\begin{block}{TEI}
%         37Schemi di codifica TEI – Moduli base
%         La pagina del titolo
%         elemento <titlePage> può contenere:
%         <docTitle> e (o direttamente) <titlePart>: titolo anche in più
%         parti
%         <docEdition>: informazioni riguardo l’edizione
%         <byline> e (o direttamente) <docAuthor>: autore
%         <docImprint>: informazioni di stampa, a sua volta include
%         <publisher>: editore
%         <pubPlace>: luogo di stampa
%         <date>: data
%         <docDate>: data
%     \end{block}
    
   

% \end{frame}



% \begin{frame}
% 	\frametitle{Intro Text Encoding Initiative}
% 	\framesubtitle{TEI}
% 	\addtocounter{nframe}{1}

% 	\begin{block}{TEI}
%         38Schemi di codifica TEI – Moduli base
%         Esempio di pagina del titolo
%         <titlePage>
%         <titlePart>Cuore</titlePart>
%         <byline>di <docAuthor>E. de
%         Amicis</docAuthor></byline>
%         <docEdition>Edizione integrale</docEdition>
%         <docImprint>
%         <publisher>Newton Compton editori</publisher>
%         <pubPlace>Roma</pubPlace>
%         <date>1994</date>
%         </docImprint>
%         </titlePage>
%     \end{block}
    
   

% \end{frame}








% \begin{frame}
% 	\frametitle{Intro Text Encoding Initiative}
% 	\framesubtitle{TEI}
% 	\addtocounter{nframe}{1}

% 	\begin{block}{TEI}
%         39Schemi di codifica TEI – Moduli base
%         Esempio di pagina del titolo
%         <titlePage>
%         <docAuthor>E. DE AMICIS</docAuthor>
%         <docTitle>
%         <titlePart type= " main " >CUORE</titlePart>
%         <titlePart>Libro per i ragazzi</titlePart>
%         </docTitle>
%         <docEdition>98.a edizione</docEdition>
%         <graphic url= "publisher.png" >
%         <docImprint>
%         <pubPlace>MILANO</pubPlace>
%         <publisher>FRATELLI TREVES, EDITORI</publisher>
%         <date>1889</date>
%         </docImprint>
%         </titlePage>
%     \end{block}
    
   

% \end{frame}






% \begin{frame}
% 	\frametitle{Intro Text Encoding Initiative}
% 	\framesubtitle{TEI}
% 	\addtocounter{nframe}{1}

% 	\begin{block}{TEI}
%         40Schemi di codifica TEI – Moduli base
%         Associare uno schema di codifica
%         per qualsiasi progetto anche mediamente impegnativo è
%         preferibile creare un proprio schema TEI
%         cominciamo a validare sulla base di uno schema a partire
%         dall’esercizio ese02.txt
%         associazione di uno schema al documento TEI XML:
%         XML Copy Editor: XML → Associa → DTD di sistema
%         Oxygen: Document → Schema → Associate Schema...
%         altri editor: inserire manualmente il <!DOCTYPE> nel
%         caso si usi una DTD, o la processing instruction <?xml-
%         model> descritta nel capitolo A Gentle Introduction to XML
%     \end{block}
    
   

% \end{frame}






% \begin{frame}
% 	\frametitle{Intro Text Encoding Initiative}
% 	\framesubtitle{TEI}
% 	\addtocounter{nframe}{1}

% 	\begin{block}{TEI}
%         % Schemi di codifica TEI: Moduli base
%         % Nota su XML Copy Editor
%         % XCE non supporta ancora l’uso di schemi di codifica nel
%         % formato RelaxNG, necessario ricorrere alla vecchia DTD
%         % XCE controlla anche la coerenza interna dello schema
%         % un messaggio come quello che segue può essere
%         % ignorato, l’informazione essenziale è se il documento è
%         % valido oppure no (riportato in cima):
%         % ese02-*******.xml is valid
%         % Attenzione at line 2, column 37: element
%         % '_DUMMY_model.resourceLike' is referenced in a content
%         % model but was never declared
%         % Attenzione at line 2, column 37: element
%         % '_DUMMY_model.gLike' is referenced in a content model but
%         % was never declared
%     \end{block}
    
   

% \end{frame}


% \begin{frame}
% 	\frametitle{Intro Text Encoding Initiative}
% 	\framesubtitle{TEI}
% 	\addtocounter{nframe}{1}

% 	\begin{block}{TEI}
%         Schemi di codifica TEI – Moduli base
%         Esercizio 2
%         marcare il testo (file ese02.txt) come segue:
%         editor da usare: XML Copy Editor
%         aprire il sito delle norme TEI P5
%         marcare la struttura per prima cosa
%         marcare tutte le espressioni enfatizzate (underscore =
%         corsivo), i discorsi diretti, le citazioni, i nomi, etc.
%         verificare che sia ben formato premendo F2 e validare
%         salvare il file XML come ese02-Cognome.xml
%         testi tratti da http://www.gutenberg.org/
%     \end{block}
    
   

% \end{frame}






% \begin{frame}
% 	\frametitle{Intro Text Encoding Initiative}
% 	\framesubtitle{TEI}
% 	\addtocounter{nframe}{1}

% 	\begin{block}{TEI}
%         Schemi di codifica TEI: Moduli base
%         Collegamenti interni ed esterni
       
%         <ptr/>
%         specifica un puntatore a un altro \textit{luogo} (un altro
%         punto dello stesso testo o di un altro testo)
        
%         % \texttt{<p>Il sistema di puntatori è basato sul meccanismo W3C
%         % Xpointer. Per maggiori informazioni si veda il
%         % capitolo <ptr target="\#SA­id"/>; per le specifiche si
%         % veda <ptr target=``http://www.w3.org/TR/xptr­xpointer/
%         % ''/>; anche <ptr type="image" target="\#fig22"/>.</p>}
        
%         <ref>
%         specifica un puntatore a un altro \textit{luogo}, può
%         includere del testo.
%         Si veda il <ref>terzo capitolo, p. 24</ref>.
%         Si veda il <ref target=``\#cap3.24''>terzo capitolo,
%         par. 24</ref>
%     \end{block}
% \end{frame}






% \begin{frame}
% 	\frametitle{Intro Text Encoding Initiative}
% 	\framesubtitle{TEI}
% 	\addtocounter{nframe}{1}

% 	\begin{block}{TEI}
%         44Schemi di codifica TEI – Moduli base
%         Liste e tabelle
%         <list> qualsiasi tipo di lista (type per specificare)
%         <head> intestazione (titolo) della lista
%         <item> un elemento della lista
%         <label> numero o esponente associato all’<item>
%         <headLabel> intestazione per gli esponenti della lista
%         <headItem> intestazione per gli elementi della lista
%     \end{block}
    
   

% \end{frame}




% \begin{frame}
% 	\frametitle{Intro Text Encoding Initiative}
% 	\framesubtitle{TEI}
% 	\addtocounter{nframe}{1}

% 	\begin{block}{TEI}
%         45Schemi di codifica TEI – Moduli base
%         Liste e tabelle
%         agli item può essere associata o no una label, ma in caso
%         affermativo deve essere presente per tutti
%         il testo non deve essere necessariamente ordinato come lista
%         per essere marcato in quanto tale:
%         <list><head>Ingredienti:</head> <item>un cucchiaio di
%         zucchero;</item> <item>mezzo chilo di farina;</item>
%         <item>due uova.</item></list>
%         usare l’attributo type per specificare il tipo di lista
%         gli elementi per tabelle sono <table>, <row>, <cell> disponibili
%         con il modulo figures, v. il cap. 14 Tables, Formulae and
%         Graphics ( http://www.tei-c.org/release/doc/tei-p5-doc/en/html/FT.html )
%     \end{block}
    
   

% \end{frame}



% \begin{frame}
% 	\frametitle{Intro Text Encoding Initiative}
% 	\framesubtitle{TEI}
% 	\addtocounter{nframe}{1}

% 	\begin{block}{TEI}
%         46Schemi di codifica TEI – Moduli base
%         Lista semplice
%         <list type=``simple''>
%         <head>Lista della spesa:</head>
%         <item>pane;</item>
%         <item>frutta;</item>
%         <item>verdura;</item>
%         <item>latte;</item>
%         <item>farina;</item>
%         <item>uova;</item>
%         <item>tovaglioli;</item>
%         <item>bicchieri;</item>
%         <item>piatti.</item>
%         </list>
%     \end{block}
    
   

% \end{frame}




% \begin{frame}
% 	\frametitle{Intro Text Encoding Initiative}
% 	\framesubtitle{TEI}
% 	\addtocounter{nframe}{1}

% 	\begin{block}{TEI}
%         Schemi di codifica TEI: Moduli base
%         Glossario
        
%         \texttt{<list type=``gloss''>
%         <head>Tecnologie XML:</head>
%         <label>XSL</label>
%         <item>eXtensible Stylesheet Language</item>
%         <label>XSLT</label>
%         <item>XSL Transformations</item>
%         <label>XSLFO</label>
%         <item>XSL \- Formatting Objects</item>
%         <label>XQuery</label>
%         <item>XML Query Language</item>
%         <label>XPAth</label>
%         <item>XML Path Language</item>
%         </list>}
%     \end{block}
    
% \end{frame}




% \begin{frame}
% 	\frametitle{Intro Text Encoding Initiative}
% 	\framesubtitle{TEI}
% 	\addtocounter{nframe}{1}

% 	\begin{block}{TEI}
%         Schemi di codifica TEI: Moduli base
%         Tabella
        
%         \texttt{<table rows="3" cols=``2''>
%         <row>
%         <cell>HTML</cell>
%         <cell>Derivato da SGML (come applicazione), il
%         linguaggio del World Wide Web.</cell>
%         </row>
%         <row>
%         <cell>XML</cell>
%         <cell>Derivato da SGML (per semplificazione).</cell>
%         </row>
%         <row>
%         <cell>XSLT</cell>
%         <cell>Fogli di stile per XML.</cell>
%         </row>
%         </table>}
%     \end{block}
% \end{frame}



% \begin{frame}
% 	\frametitle{Intro Text Encoding Initiative}
% 	\framesubtitle{TEI}
% 	\addtocounter{nframe}{1}

% 	\begin{block}{TEI}
%         Schemi di codifica TEI – Moduli base
%         Note
        
%         l’elemento <note> permette di inserire una nota di
%         qualsiasi tipo (attributi type e place)
%         la nota può appartenere al testo codificato, o può essere
%         opera di chi lo codifica (attributo resp)
%         se non nel flusso del testo usare gli attributi target e
%         targetEnd per stabilire collegamento preciso
%         Le <title>Guidelines</title><note n="2" place="foot"
%         resp=``\#rrdt''>C.M. SperbergMcQueen and Lou Burnard,
%         <title>Guidelines for Electronic Text Encoding and
%         Interchange</title> (Chicago, Oxford: Text Encoding
%         initiative, 2002).</note> sono nate con lo scopo di [...]
%     \end{block}
    
   

% \end{frame}



% \begin{frame}
% 	\frametitle{Intro Text Encoding Initiative}
% 	\framesubtitle{TEI}
% 	\addtocounter{nframe}{1}

% 	\begin{block}{TEI}
%         50Schemi di codifica TEI – Moduli base
%         Esercizio 3
%         marcare il testo del file ese03.jpg come segue:
%         editor da usare: XML Copy Editor o Editix
%         marcare tutto quello che si vede nella pagina
%         marcare come glossario usando <list> oppure
%         in alternativa: marcare come glossario <term> + <gloss>
%         usare identificatori e puntatori/riferimenti
%         inserire un paio di note a piè di pagina
%         verificare costantemente che il documento sia valido
%         salvare il file XML come ese03-Cognome.xml
%     \end{block}
    
   

% \end{frame}




% \begin{frame}
% 	\frametitle{Intro Text Encoding Initiative}
% 	\framesubtitle{TEI}
% 	\addtocounter{nframe}{1}

% 	\begin{block}{TEI}
%         51Schemi di codifica TEI – Moduli base
%         Bibliografia
%         i riferimenti bibliografici, compresa la bibliografia vera e
%         propria, sono codificati usando i seguenti elementi:
%         <bibl>
%         citazione bibliografica di tipo “flessibile”: possono
%         essere presenti solo alcuni elementi; usata anche
%         nel corpo del testo (NB: stessi elementi del succ.)
%         <biblStruct>
%         citazione bibliografica di tipo strutturato: gli
%         elementi devono seguire uno schema fisso
%         <biblFull > citazione bibliografica di tipo strutturato completo:
%         gli elementi devono seguire uno schema fisso e
%         devono essere tutti presenti
%         <listBibl> lista di citazioni bibliografiche in uno qualsiasi
%         dei formati citati sopra
%     \end{block}
    
   

% \end{frame}





% \begin{frame}
% 	\frametitle{Intro Text Encoding Initiative}
% 	\framesubtitle{TEI}
% 	\addtocounter{nframe}{1}

% 	\begin{block}{TEI}
%         52Schemi di codifica TEI – Moduli base
%         Tipi di pubblicazione
%         ogni voce bibliografica può essere codificata sulla base di
%         tre possibili tipi di pubblicazione:
%         <analytic> “analitico”: informazioni su un titolo che
%         non costituisce una pubblicazione autonoma
%         (articolo di rivista o in una miscellanea)
%         <monogr> “monografico”: informazioni su una
%         pubblicazione, anche di tipo periodico (rivista)
%         <series> “serie”: informazioni su di una serie
%         editoriale
%         tipicamente impiegati con gli elementi <biblStruct> e
%         (meno di frequente) <biblFull>, più flessibilità con <bibl>
%     \end{block}
    
   

% \end{frame}





% \begin{frame}
% 	\frametitle{Intro Text Encoding Initiative}
% 	\framesubtitle{TEI}
% 	\addtocounter{nframe}{1}

% 	\begin{block}{TEI}
%         53Schemi di codifica TEI – Moduli base
%         Titolo, autore, curatore
%         <title> il titolo della pubblicazione; più livelli (attr. level):
%         level=”a”
%         articolo (article)
%         level=”j”
%         rivista (journal)
%         level=”m”
%         monografia (monograph)
%         level=”s”
%         serie (series)
%         level=”u”
%         non pubblicato (unpublished)
%         <author> autore della pubblicazione; può contenere
%         direttamente il nome dell’autore o elementi più
%         complessi (<persName> e sub-elementi)
%         <editor> curatore della pubblicazione; vedi sopra
%     \end{block}
    
   

% \end{frame}





% \begin{frame}
% 	\frametitle{Intro Text Encoding Initiative}
% 	\framesubtitle{TEI}
% 	\addtocounter{nframe}{1}

% 	\begin{block}{TEI}
% 	% 54Schemi di codifica TEI – Moduli base
% % Elementi descrittivi
% % <imprint> raggruppa informazioni relative alla stampa
% % o diffusione di una pubblicazione
% % <publisher> l’editore responsabile della produzione a
% % stampa o diffusione
% % <pubPlace> il luogo di pubblicazione
% % <date> la data di pubblicazione
% % <biblScope> l’estensione (in pagine, volumi) della
% % pubblicazione;
% % permette di specificare il tipo di estensione
% % ad es. se si tratta di pagine
% % type
% % <biblScope> può essere usato anche all’interno di <series>
%     \end{block}
    
   

% \end{frame}






% \begin{frame}
% 	\frametitle{Intro Text Encoding Initiative}
% 	\framesubtitle{TEI}
% 	\addtocounter{nframe}{1}

% 	\begin{block}{TEI}
%         55Schemi di codifica TEI – Moduli base
%         Esempio semplice
%         il caso di una monografia con autore unico:
%         <biblStruct xml:id=``Robinson1993d''>
%         <monogr>
%         <author>Robinson, P.</author>
%         <title level=``m''>The Digitization of Primary Textual
%         Sources</title>
%         <imprint>
%         <pubPlace>Oxford</pubPlace>
%         <publisher>Office for Humanities
%         Communication</publisher>
%         <date>1993</date>
%         </imprint>
%         </monogr>
%         </biblStruct>
%     \end{block}
    
   

% \end{frame}


% \begin{frame}
% 	\frametitle{Intro Text Encoding Initiative}
% 	\framesubtitle{TEI}
% 	\addtocounter{nframe}{1}

% 	\begin{block}{TEI}
%         Schemi di codifica TEI: Moduli base
%         un articolo di miscellanea con due autori:
        
% % \texttt{<biblStruct xml:id=``Mohler2001''>
% %         <analytic>
% %         <author>Mohler, Peter Ph.</author>
% %         <author>Zuell, Cornelia</author>
% %         <title level=``a''>Applied Text Theory: Qualitative Analysis of
% %         Answers to Open Ended Questions</title>
% %         </analytic>
% %         <monogr>
% %         <editor>Mark D. West</editor>
% %         <title level=``m''>Application of Computer Content Analysis</title>
% %         <imprint>
% %         <pubPlace>Westport CN</pubPlace>
% %         <publisher>Ablex</publisher>
% %         <date>2001</date>
% %         <biblScope type=``pages''>1­16</biblScope>
% %         </imprint>
% %         </monogr>
% %         </biblStruct>}
%     \end{block}
% \end{frame}


% \begin{frame}
% 	\frametitle{Intro Text Encoding Initiative}
% 	\framesubtitle{TEI}
% 	\addtocounter{nframe}{1}

% 	\begin{block}{TEI}
% 	% 57Schemi di codifica TEI – Moduli base
% % un articolo su rivista di un solo autore:
% % <biblStruct xml:id="Ester1994">
% % <analytic>
% % <author>Ester, Michael</author>
% % <title level="a">Digital Images in the Context of Visual
% % Collections and Scholarship</title>
% % </analytic>
% % <monogr>
% % <title level="j">Visual Resources</title>
% % <imprint>
% % <date>1994</date>
% % </imprint>
% % <biblScope type="vol">X, 1</biblScope>
% % <biblScope type="pages">23</biblScope>
% % </monogr>
% % <note type="key">iproc</note>
% % </biblStruct>
%     \end{block}
    
   

% \end{frame}





% \begin{frame}
% 	\frametitle{Intro Text Encoding Initiative}
% 	\framesubtitle{TEI}
% 	\addtocounter{nframe}{1}

% 	\begin{block}{TEI}
%         Schemi di codifica TEI: Moduli base
%         Componenti non testuali: immagini
%         possibile includere nella marcatura immagini in formato XML
%         (SVG, il formato TEI) e in formato binario (TIFF, JPEG, PNG)
%         <graphic/>
%         contiene un puntatore a una immagine
%         url
%         può essere locale o esterna
%         <binaryObject> dati binari che costituiscono una immagine
%         <media>
%         riferimento a file audio, video etc.
%         <figure>
%         raggruppa informazioni relative a un’immagine
%         <head>
%         titolo o didascalia relativa all’immagine
%         <figDesc>
%         descrizione dell'immagine
%         <graphic>
%         puntatore a un'immagine
%         NB <figure> nella P5 fa parte del modulo figures ( \url{http://www.tei-
%         c.org/release/doc/tei-p5-doc/en/html/FT.html\#FTGRA} )
%     \end{block}
    
   

% \end{frame}





% \begin{frame}
% 	\frametitle{Intro Text Encoding Initiative}
% 	\framesubtitle{TEI}
% 	\addtocounter{nframe}{1}

% 	\begin{block}{TEI}
% 	% 59Schemi di codifica TEI – Moduli base
% % Esempio TEI P4
% % NB: nella TEI P4 è indispensabile inserire dichiarazioni
% % relative alle immagini nell’intestazione del documento
% % <?xml version='1.0' encoding='utf­8'?>
% % ...
% % <!NOTATION JPEG PUBLIC
% % 'ISO DIS 10918//NOTATION JPEG Graphics Format//EN' >
% % <!ENTITY PDatabase SYSTEM "support/PDatabase.png" NDATA PNG>
% % ]>
% % ...
% % <p>The system is build around a relational database, called the
% % <soCalled>palaeographical database</soCalled>, that contains all
% % the information the application produces and processes:</p>
% % <figure entity="PDatabase">
% % <figDesc>Schematic of the SPI system.</figDesc>
% % </figure>
%     \end{block}
    
   

% \end{frame}





% \begin{frame}
% 	\frametitle{Intro Text Encoding Initiative}
% 	\framesubtitle{TEI}
% 	\addtocounter{nframe}{1}

% 	\begin{block}{TEI}
% 	% 60Schemi di codifica TEI – Moduli base
% % Esempi TEI P5
% % inserimento di un’immagine in un documento P5:
% % <p>Questo fenomeno è chiaramente visibile nella figura che
% % segue:
% % <graphic url="diagramma.png"/>
% % Altre variazioni delle crescite trimestrali sono
% % disponibili nel rapporto allegato alla relazione.</p>
% % immagine impiegata come intestazione in un documento P5:
% % <head>
% % <graphic
% % url="http://www.iath.virginia.edu/gants/Ornaments/Heads/hp­
% % ral02.gif"/>
% % </head>
%     \end{block}
    
   

% \end{frame}


