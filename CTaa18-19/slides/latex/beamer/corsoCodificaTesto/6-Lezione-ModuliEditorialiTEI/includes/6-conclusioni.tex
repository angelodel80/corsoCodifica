%%
\begin{frame}
    \frametitle{Edizioni Digitali}
    \addtocounter{nframe}{1}
    
    %\begin{center}
    %    \includegraphics[width=.2\textwidth]{../imgs/tei-r.pdf}
    %\end{center}

    \begin{block}{Moduli TEI P5}
        Le nuove caratteristiche degli schemi TEI P5 offrono un’ottima base per edizioni digitali complesse con collegamento testo-immagine
    \end{block}
    
\end{frame}



%%
\begin{frame}
    \frametitle{Conclusioni: Editioni Digitali}
    \addtocounter{nframe}{1}
    
    %\begin{center}
    %    \includegraphics[width=.2\textwidth]{../imgs/tei-r.pdf}
    %\end{center}

    \begin{block}{Edizioni digitali scientifiche}
        \begin{itemize}
            \item elementi contenuti nei \textit{moduli di base}
            \item elementi del modulo di \textit{descrizione dei manoscritti}
            \item elementi del modulo di \textit{trascrizione delle fonti primarie}
            \item elementi del modulo di \textit{apparato critico}
            \item elementi del modulo di \textit{gestione di caratteri non standard}
        \end{itemize}
    \end{block}
    
\end{frame}

%%

\begin{frame}
    \frametitle{Conclusioni: Edizioni Digitali}
    \addtocounter{nframe}{1}
    
    %\begin{center}
    %    \includegraphics[width=.2\textwidth]{../imgs/tei-r.pdf}
    %\end{center}

    \begin{block}{elementi per interventi editoriali:}
        \texttt{<abbr> <expan>, <orig> <reg>, <sic> <corr>, <subst>
        <gap/>, <supplied>, <unclear>, <damage>}
    \end{block}

    \begin{block}{strutturali specifici:}
        \texttt{<gb/>, <line>}
    \end{block}
    
\end{frame}


%%
\begin{frame}
    \frametitle{Conclusioni: Edizioni Digitali}
    \addtocounter{nframe}{1}
    
    %\begin{center}
    %    \includegraphics[width=.2\textwidth]{../imgs/tei-r.pdf}
    %\end{center}

    \begin{block}{Elementi di intervento editoriali}
        \begin{itemize}
            \item \texttt{<damage>} marca la parte di testo danneggiata
            \item[] non proprio “intervento editoriale” ma spesso usato contestualmente con \texttt{<gap/>,  <unclear> e <supplied>}
            \item \texttt{<supplied>} testo inserito dal curatore perché l’originale è mancante o illeggibile
        \end{itemize} 
    \end{block}
\end{frame}

\begin{frame}
    \frametitle{Conclusioni: Edizioni Digitali}
    \addtocounter{nframe}{1}
    
    %\begin{center}
    %    \includegraphics[width=.2\textwidth]{../imgs/tei-r.pdf}
    %\end{center}

    \begin{block}{Elementi di intervento editoriali: esempio}
        \texttt{<l n="1" >Nel mezzo del cammin di nostra vita</l>}
        \\\texttt{<l n="2" ><damage agent="fire" extent="1line" ><unclear>Mi ritrovai</unclear> <supplied reason="illegible" resp="amdg" >per una selva oscura,</supplied></damage></l>}
        \\\texttt{<l n="3" >Ché la diritta via era smarrita</l>}
    \end{block}
\end{frame}

%%
\begin{frame}
    \frametitle{Conclusioni: Edizioni Digitali}
    \addtocounter{nframe}{1}
    
    %\begin{center}
    %    \includegraphics[width=.2\textwidth]{../imgs/tei-r.pdf}
    %\end{center}

    \begin{block}{Elementi di intervento editoriale}
        \begin{itemize}
            \item \texttt{<subst>} raggruppa una cancellazione e un’aggiunta scribale per rendere evidente che si tratta di una sostituzione
            \item[] Stessa semantica funzionale di \texttt{<choice>}
        \end{itemize}
    \end{block}

\end{frame}

%%
\begin{frame}
    \frametitle{Conclusioni: Edizioni Digitali}
    \addtocounter{nframe}{1}
    
    %\begin{center}
    %    \includegraphics[width=.2\textwidth]{../imgs/tei-r.pdf}
    %\end{center}

    \begin{block}{Elementi di intervento editoriale}
        \texttt{<l n="1" >Nel mezzo del cammin di nostra vita</l> }
        \\\texttt{<l n="2" >Mi ritrovai }
        \\\texttt{<subst>}
        \\\texttt{<del>pir</del>}
        \\\texttt{<add>per</add>}
        \\\texttt{</subst>}
        \\\texttt{una selva oscura,</l>}
        \\\texttt{<l n="3" >Ché la diritta via era smarrita</l>}
    \end{block}
    
\end{frame}


%%

\begin{frame}
    \frametitle{Conclusioni: Edizioni Digitali}
    \addtocounter{nframe}{1}
    
    %\begin{center}
    %    \includegraphics[width=.2\textwidth]{../imgs/tei-r.pdf}
    %\end{center}

    \begin{block}{Edizione image-based: Elementi strutturali}
        \begin{itemize}
            \item \texttt{<gb/>} \textbf{gathering begins} 
            \item[] marca il punto in cui si presenta un nuovo fascicolo all’interno di un manoscritto
            \item \texttt{@type:} classificazione in base al tipo
            \item \texttt{@n:} numero progressivo
        \end{itemize}
    \end{block}
\end{frame}

\begin{frame}
    \frametitle{Conclusioni: Edizioni Digitali}
    \addtocounter{nframe}{1}
    
    %\begin{center}
    %    \includegraphics[width=.2\textwidth]{../imgs/tei-r.pdf}
    %\end{center}

    \begin{block}{Edizione image-based: Elementi strutturali}
        \begin{itemize}
            \item \texttt{<line> }
            \item[] trascrizione di una riga del foglio del manoscritto 
            \item[] \textbf{può essere contenuto solo da \texttt{<surface> e <zone>}!}
            \item per le righe di testo da inserire all’interno di \texttt{<text>} è sempre necessario usare \texttt{<lb/>}.
        \end{itemize}
    \end{block}
\end{frame}

%%
\begin{frame}
    \frametitle{Conclusioni: Edizioni Digitali}
    \addtocounter{nframe}{1}
    
    %\begin{center}
    %    \includegraphics[width=.2\textwidth]{../imgs/tei-r.pdf}
    %\end{center}

    \begin{block}{Elementi per interventi editoriali}
        A causa della relativa complessità di codifica di edizioni digitali image-based è preferibile usare strumenti software per facilitare la creazione di un facsimile digitale
    \end{block}

    \begin{block}{strutturali specifici}
        Manca però ancora uno strumento/funzione per collegare le immagini annotate al testo della trascrizione
    \end{block}
\end{frame}

